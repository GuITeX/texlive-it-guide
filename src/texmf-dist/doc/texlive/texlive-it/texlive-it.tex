% arara: closepdf
% arara: pdflatex
% arara: showfile
% $Id: texlive-it.tex 44357 2017-05-14 22:47:28Z karl $
% TeX Live documentation.  Originally written by Sebastian Rahtz and
% Michel Goossens, now maintained by Karl Berry and others.
% Public domain.
%
% Traduzione originaria a cura di Marco Pallante. Attualmente la traduzione
% è portata avanti in seno al progetto GuITeX del GuIT, Gruppo Utilizzatori
% Italiani di TeX (http://www.guit.sssup.it/).
%
% Chiunque può contribuire accedendo al repository Git diponibile
% all'indirizzo
%
%    https://github.com/GuITeX/texlive-it-guide
%
\documentclass{article}
%\let\tldocenglish=1  % for live4ht.cfg
%\def\Status{1}%CM istruzione inserita solo per poter compilare
\usepackage{tex-live}
\usepackage[italian]{babel}
\usepackage[utf8]{inputenc}
\usepackage[T1]{fontenc}

\title{%
  {\huge \textit{Guida a \TeX\ Live---2020}}
}

\author{Karl Berry\\[3mm]
        \url{https://tug.org/texlive/}
       }

\date{Marzo 2020}

\begin{document}
\maketitle

\begin{multicols}{2}
\tableofcontents
%\listoftables
\end{multicols}


\section*{Note all'edizione italiana}

Dopo tanti anni, finalmente abbiamo anche un'edizione italiana di questa
guida alla distribuzione \TL. Vorrei premettere che, per garantirne il
completamento entro l'uscita di \TL{} 2009, mi sono fatto in quattro
destinando praticamente ogni attimo libero a questo lavoro; ho tenuto
sempre acceso il computer, con l'editor aperto e, spesso, passandoci
davanti, mi fermavo a tradurre una sola frase o addirittura una sola
parola pur di non restare fermo.

Ovviamente il risultato è un lavoro fatto con i piedi, uno stile
incoerente, parti che nemmeno io riesco a capire, errori di battitura e di
grammatica. Di tutto questo chiedo scusa. Però, adesso, il grosso dello
sforzo è stato fatto e revisionare periodicamente questa guida per limarla
e migliorarla sarà un'opera di gran lunga più semplice.

Purtroppo, ho mancato l'appuntamento con \TL\ 2010 e la guida lì presente
era la stessa del 2009, con tutti i suoi difetti. Questa volta, per
l'edizione 2011, mi sono impegnato un po' di più, ho ripreso l'intero
lavoro, l'ho aggiornato prima con le modifiche del 2010 e quindi con quelle
del 2011, ho corretto una valanga di errori e ritradotto frasi che con
l'italiano avevano davvero poco a che vedere.

Ovviamente ci saranno ancora tanti errori, frasi incomprensibili, incoerenze
e chi più ne ha più ne metta, ma spero nuovamente che il mio piccolo
contributo a \TeX{} sia apprezzato. Potete contattarmi per qualunque cosa,
aiuti, suggerimenti, correzioni, all'indirizzo
\email{marco.pallante@gmail.com}.

Vorrei dedicare questo lavoro alla mia città.

\bigskip

\noindent Marco Pallante\\
\emph{L'Aquila, 20 giugno 2011}.


\section{Introduzione}
\label{sec:intro}

\subsection{\protect\TeX\protect\ Live e la \protect\TeX\protect\ Collection}

Questo documento descrive le caratteristiche principali della
distribuzione \TL{}\Dash \TeX{} e i programmi ad esso correlati per
i sistemi GNU/Linux e altre versioni di Unix, \MacOSX{} e Windows.

Potete ottenere \TL{} scaricandola, oppure sul \DVD{} \TK, che i gruppi di
utenti \TeX{} distribuiscono ai propri membri, o in altri modi. La sezione
\ref{sec:tl-coll-dists} descrive brevemente il contenuto del \DVD. Sia
\TL{} che \TK{} sono progetti cooperativi dei gruppi di utenti \TeX.
Questo descrive principalmente \TL.

\TL{} include gli eseguibili per \TeX, \LaTeXe, \ConTeXt, \MF, \MP,
\BibTeX{} e molti altri programmi, una vasta collezione di macro, font e
documentazione e il supporto per la composizione tipografica in molti
diversi alfabeti provenienti da tutte le parti del mondo.

Per un breve riassunto dei principali cambiamenti in questa edizione di
\TL, consultate la fine di questo documento, sezione~\ref{sec:history}
(\p.\pageref{sec:history}).



\htmlanchor{platforms}
\subsection{Supporto per i sistemi operativi}
\label{sec:os-support}

\TL{} contiene gli eseguibili per molte piattaforme basate su Unix,
inclusi \GNU/Linux, \MacOSX{} e Cygwin. I sorgenti inclusi possono essere
compilati per quelle piattaforme per le quali non forniamo i binari.

Per quanto riguarda Windows: Windows~7 e successivi sono supportati.
Windows Vista probabilmente continuerà a funzionare quasi del
tutto, ma \TL{} non sarà più neppure installato su Windows XP e
precedenti. \TL{} non include specifici eseguibili a 64-bit per Windows, ma
gli eseguibili a 32-bit dovrebbero funzionare sui sistemi a 64-bit.

Consulta la sezione~\ref{sec:tl-coll-dists} per scoprire soluzioni
alternative per Windows e \MacOSX.

\subsection{Installazione base di \protect\TL{}}
\label{sec:basic}

È possibile installare \TL{} dal \DVD{} oppure attraverso Internet
(\url{https://tug.org/texlive/acquire.html}). Il programma di installazione
via rete è piccolo e scarica tutto ciò che è necessario da Internet.

Il programma di installazione nel \DVD{} vi permette l'installazione su un
disco locale. Non è possibile eseguire \TL{} direttamente dal \DVD{} \TK{}
(o dalla sua immagine \code{.iso}), ma potete preparare un'installazione
pronta per l'uso su, per esempio, una pennetta \USB{} (consultate la
sezione~\ref{sec:portable-tl}). L'installazione è descritta nelle prossime
sezioni (\p.\pageref{sec:install}), ma eccone un rapido accenno:

\begin{itemize*}

\item Lo script di installazione è chiamato \filename{install-tl}.
  Può operare in ``modalità grafica'' dando l'opzione \code{-gui}
  (predefinita sotto Windows e \MacOSX), in modalità testuale
  dando l'opzione \code{-gui=text} (predefinito per gli altri sistemi).
  Nelle piattafforme Unix, le precedenti modalità Perl/Tk e guidata sono
  ancora disponibili e Perl/Tk installato; consultate la
  sezione~\ref{sec:wininst} per Windows.

\item Uno degli elementi installati è il programma ``\TL\ Manager'',
  chiamato \prog{tlmgr}. Così come l'installatore, può essere usato sia in
  modalità \GUI{} che in modalità testuale. Lo si può usare per installare
  e disinstallare i pacchetti e per compiere varie operazioni di
  configurazione.

\end{itemize*}


\htmlanchor{security}
\subsection{Considerazioni sulla sicurezza}
\label{sec:security}

Sulla base delle nostre conoscenze, i programmi centrali di \TeX\ sono (e
sono sempre stati) estremamente robusti. Tuttavia, gli altri programmi in
\TeX\ Live potrebbero non raggiungere lo stesso livello, nonostante il
massimo impegno di tutti. Come sempre, dovreste essere cauti
nell'eseguire i programmi su documenti di origine inaffidabile; per
aumentare la sicurezza, eseguiteli in una nuova sotto directory o in chroot.

Questo bisogno di attenzione è particolarmente pressante sotto Windows,
dato che generalmente Windows cerca i programmi nella directory attuale
prima che in qualunque altra posizione, indipendentemente dal percorso di
ricerca. Questo comportamento crea un'ampia varietà di possibili attacchi.
Abbiamo chiuso molte falle, ma indubbiamente alcune ne rimangono,
soprattutto con i programmi di terze parti. Perciò raccomandiamo di
controllare la presenza di file sospetti, soprattutto eseguibili 
(binari o script), nella directory attuale. Normalmente questi non
dovrebbero essere presenti e senza alcun dubbio normalmente non dovrebbero
essere creati semplicemente elaborando un documento.

Infine, \TeX\ (e i programmi che lo accompagnano) è capace di creare nuovi
file quando elabora i documento, una funzionalità che può anche essere
abusata in una grande quantità di modi. Di nuovo, elaborare documenti
sconosciuti in una nuova sotto directory è la scelta più sicura.

Un altro aspetto della sicurezza è assicurarsi che il materiale scaricato
non sia stato modificato rispetto a quanto creato. Il programma
\prog{tlmgr} (sezione~\ref{sec:tlmgr}) eseguirà automaticamente una
verificca criptografica sui file scaricati se il programma \prog{gpg} (GNU
Privacy Guard) è disponibile. Non è distribuito come parte di \TL, ma
visitate \url{https://texlive.info/tlpgp/} per informazioni su \prog{pgp}
se è necessario.


\subsection{Ottenere aiuto}
\label{sec:help}

La comunità \TeX{} è attiva ed amichevole e le domande più importanti
finiscono per ricevere una risposta. Tuttavia il supporto è informale,
offerto da volontari e utenti casuali, per cui è particolarmente
importante fare la propria parte prima di chiedere (se preferite un
supporto commerciale garantito, potete rinunciare del tutto a \TL{} e
ordinare il sistema di un fornitore; alla pagina
\url{https://tug.org/interest.html#vendors} trovate un elenco).

Ecco una lista di risorse, approssimativamente nell'ordine in cui noi
raccomandiamo di usarle:

\begin{description}
\item [Per cominciare] Se siete nuovi di \TeX, alla pagina web
  \url{https://tug.org/begin.html} troverete una breve introduzione al sistema.

\item [\TeX{} FAQ] Le \TeX{} FAQ sono un enorme compendio di
  risposte ad ogni genere di domanda, dalle più elementari alle più
  oscure. Sono incluse in \TL{} in
  \OnCD{texmf-dist/doc/generic/FAQ-en} e sono disponibili
  su Internet alla pagina \url{https://www.tex.ac.uk/faq}. Controllate
  prima qui.

\item [\TeX{} Catalogue] Se state cercando un pacchetto, un font, un
  programma o altro in particolare, il \TeX{} Catalogue è il luogo dove
  guardare. È un enorme catalogo con tutte le voci relative a \TeX. Visitate
  \url{https://ctan.org/pkg/catalogue}.

\item [Risorse Web per \TeX{}] La pagina web
  \url{https://tug.org/interest.html} ha molti collegamenti relativi a
  \TeX, in particolare a numerosi libri, manuali ed articoli su tutti
  gli aspetti del sistema.

\item [archivi di supporto] I principali forum di supporto per \TeX\
  includono
  il sito della comunità \LaTeX{} su \url{https://latex.org},
  il sito di domande e risposte \url{https://tex.stackexchange.com}, il
  gruppo Usenet \url{news:comp.text.tex} e la mailing list
  \email{texhax@tug.org}. I loro archivi raccolgono anni di domande e risposte
  per il vostro piacere di ricerca tramite, per gli ultimi due,
  \url{https://groups.google.com/group/comp.text.tex/topics} e
  \url{https://tug.org/mail-archives/texhax}. E una
  generica ricerca su web non fa mai male.

\item [porre domande] Se non riuscite a trovare una risposta, potete
  formulare la vostra domanda su
  \url{https://latex-community.org/} e \url{https://tex.stackexchange.com/}
  tramite le loro interfacce web, su \dirname{comp.text.tex} tramite Google o
  un programma per Usenet, oppure tramite posta elettronica su
  \email{texhax@tug.org}. Ma prima di scrivere in qualunque posto, leggete
  questa voce delle FAQ, al fine di massimizzare le probabilità di ottenere una
  risposta utile:
  \url{https://texfaq.org/FAQ-askquestion}.

\item [supporto \TL{}] Se volete segnalare un bug o avete dei suggerimenti o
  commenti sulla distribuzione \TL{}, sull'installazione o la
  documentazione, la mailing list è \email{tex-live@tug.org}. Tuttavia, se
  la vostra domanda è relativa all'uso di un particolare programma incluso in
  \TL{}, scrivete al mantenitore o alla mailing list di quel programma.
  Spesso eseguire un programma con l'opzione \code{-{}-help} fornisce un
  indirizzo al quale segnalare i bug.

\end{description}

L'altra faccia della medaglia è aiutare coloro che hanno domande. Tutte le
risorse indicate sopra sono aperte a chiunque, quindi sentevi liberi di
iscrivervi, di iniziare a leggere e dare una mano dove potete.

% don't use \TL so the \uppercase in the headline works.  Also so
% tex4ht ends up with the right TeX.  Likewise the \protect's.
\section{Panoramica su \protect\TeX\protect\ Live}
\label{sec:overview-tl}

Questa sezione descrive i contenuti di \TL{} e della \TK{} di cui è parte.

\subsection{La \protect\TeX\protect\ Collection: \protect\TL,
            pro\protect\TeX{}t, Mac\protect\TeX}
\label{sec:tl-coll-dists}

Il \DVD{} \TK{} include quanto segue:

\begin{description}

\item [\TL] Un sistema \TeX{} completo da installare sul proprio disco. Home
  page: \url{https://tug.org/texlive/}.

\item [Mac\TeX] per \MacOSX\ (chiamato attualmente macOS da Apple, ma noi
  continuiamo ad usare il vecchio nome in questo documento), aggiunge a \TL{}
  un installatore nativo per \MacOSX\ ed altre applicazioni Mac. Home page:
  \url{https://tug.org/mactex/}.

\item [pro\TeX{}t] Un miglioramento della distribuzione \MIKTEX\ per
  Windows, \ProTeXt\ aggiunge alcuni strumenti supplementari a \MIKTEX\ e
  semplifica l'installazione. È interamente indipendente da \TL{} e ha le
  proprie istruzioni per l'installazione. Home page:
  \url{https://tug.org/protext}.  

\item [CTAN] Un'instantanea dell'archivio \CTAN{}
  (\url{https://ctan.org/}).

\end{description}

\CTAN{} e \pkgname{protext} non seguono le stesse condizioni di licenza di
\TL{}, per cui fate attenzione nel ridistribuirli o modificarli.


\subsection{Directory primarie di \protect\TL{}}
\label{sec:tld}

Segue un breve elenco e una descrizione delle directory primarie di
un'installazione di \TL.

\begin{ttdescription}
\item[bin] I programmi del sistema \TeX{}, raggruppati per piattaforma.
%
\item[readme-*.dir] Una rapida panoramica ed alcuni collegamenti utili per 
  \TL{},
  in varie lingue, sia in \HTML{} che in formato testuale.
%
\item[source] I sorgenti di tutti i programmi inclusi, comprese le
  distribuzioni \TeX{} basate su \Webc{}.
%
\item[texmf-dist] Il percorso principale; vedi \dirname{TEXMFDIST} sotto.
%
\item[tlpkg] Script, programmi e dati per la gestione
  dell'installazione e per il supporto specifico per Windows.
\end{ttdescription}

Per quanto riguarda la documentazione, i collegamenti nel file
\OnCD{doc.html} possono risultare utili. La documentazione per quasi
tutto (pacchetti, formati, font, manuali dei programmi, 
pagine di manuali in linea, file Info)
si trova in \dirname{texmf-dist/doc}. Potete usare il programma
\cmdname{texdoc} per trovare la documentazione, ovunque sia
collocata.

Questa stessa documentazione di \TL\ si trova in
\dirname{texmf-dist/doc/texlive}, disponibile in varie lingue:

\begin{itemize*}
\item{Ceco/Slovacco:} \OnCD{texmf-dist/doc/texlive/texlive-cz}
\item{Cinese Semplificato:} \OnCD{texmf-dist/doc/texlive/texlive-zh-cn}
\item{Francese:} \OnCD{texmf-dist/doc/texlive/texlive-fr}
\item{Giapponese:} \OnCD{texmf-dist/doc/texlive/texlive-ja}
\item{Inglese:} \OnCD{texmf-dist/doc/texlive/texlive-en}
\item{Italiano:} \OnCD{texmf-dist/doc/texlive/texlive-it}
\item{Polacco:} \OnCD{texmf-dist/doc/texlive/texlive-pl}
\item{Russo:} \OnCD{texmf-dist/doc/texlive/texlive-ru}
\item{Serbo:} \OnCD{texmf-dist/doc/texlive/texlive-sr}
\item{Tedesco:} \OnCD{texmf-dist/doc/texlive/texlive-de}
\end{itemize*}

\subsection{Panoramica dei percorsi predefiniti in texmf}
\label{sec:texmftrees}

Questa sezione elenca le variabili predefinite che specificano i percorsi
all'interno di texmf usati dal sistema, il loro scopo e la strutturazione
predefinita di \TL{}. Il comando \texttt{tlmgr~conf} mostra i valori di
queste variabili, così che sia possibile scoprire con facilità come queste
mappino le specifiche directory nella propria installazione.

Tutti i percorsi, inclusi quelli personali, dovrebbero seguire la Struttura
delle Directory di \TeX\ (\TeX\ Directory Structure, \TDS,
\url{https://tug.org/tds}) con tutta la sua miriade di sotto directory,
altrimenti i file potrebbero non essere trovati. La sezione
\ref{sec:local-personal-macros} (\p.\pageref{sec:local-personal-macros})
descrive ciò con maggior dettaglio. L'ordine che trovate qui è l'ordine
opposto con cui sono esplorati i percorsi, ossia, gli ultimi percorsi
nella lista hanno priorità rispetto ai primi.

\begin{ttdescription}
\item [TEXMFDIST] Il percorso che contiene quasi tutti i file nella
  distribuzione originale --- file di configurazione, script, pacchetti,
  font, ecc (l'eccezione principale sono gli eseguibili relativi a ciascuna
  piattaforma, che sono contenuti in un percorso gemello \code{bin/}).
\item [TEXMFSYSVAR] Il percorso (a livello di sistema) usato da
  \verb+texconfig-sys+, \verb+updmap-sys+, \verb+fmtutil-sys+ ed anche da
  \verb+tlmgr+ per memorizzare (nella cache) i dati generati durante 
  l'esecuzione, come i file dei formati e le mappe per i font.
\item [TEXMFSYSCONFIG] Il percorso (a livello di sistema) usato dai
  programmi \verb+texconfig-sys+, \verb+updmap-sys+ e \verb+fmtutil-sys+
  per memorizzare i dati di configurazione modificati.
\item [TEXMFLOCAL] Il percorso che un amministratore può usare per
  l'installazione di macro, di font, ecc., aggiuntivi o aggiornati a livello
  di sistema, in modo che siano disponibili per tutti gli utenti.
\item [TEXMFHOME] Il percorso che ciascun utente può usare per la propria
  installazione personale di macro, font, ecc., aggiuntivi o aggiornati.
  L'espansione di questa variabile si adatta dinamicamente per ciascun
  utente alla directory individuale dell'utente stesso.
\item [TEXMFVAR] Il percorso (a livello di ciascun utente) usato da
  \verb+texconfig+, \verb+updmap-user+ e \verb+fmtutil-user+ per memorizzare (nella 
  cache) i dati generati durante l'esecuzione, come i file dei formati e le
  mappe per i font generati.
\item [TEXMFCONFIG] Il percorso (a livello di ciascun utente) usato dai
  programmi \verb+texconfig+, \verb+updmap-sys+ e \verb+fmtutil-sys+ per
  memorizzare i dati di configurazione modificati.
\item [TEXMFCACHE] I percorsi usati da \ConTeXt\ MkIV e Lua\LaTeX\
  per memorizzare (nella cache) i dati generati durante l'esecuzione; il valore 
  predefinito è \code{TEXMFSYSVAR}, oppure (se non si hanno i permessi di
  scrittura) \code{TEXMFVAR}.
\end{ttdescription}

\noindent
La strutturazione predefinita è:
\begin{description}
  \item[percorso a livello di sistema] può contenere diverse edizioni di \TL{}
  (in Unix di default è \texttt{/url/local/texlive}):
  \begin{ttdescription}
    \item[2019] Un'edizione precedente.
    \item[2020] L'attuale edizione.
    \begin{ttdescription}
      \item [bin] ~
      \begin{ttdescription}
        \item [i386-linux] Esebuibili per \GNU/Linux (32 bit)
        \item [...]
        \item [x86\_64-darwin] Eseguibili per \MacOSX
        \item [x86\_64-linux] Eseguibili per \GNU/Linux (64 bit)
        \item [win32] Eseguibili per Windows
      \end{ttdescription}
      \item [texmf-dist\ \ ]      \envname{TEXMFDIST} e \envname{TEXMFMAIN}
      \item [texmf-var \ \ ]      \envname{TEXMFSYSVAR}, \envname{TEXMFCACHE}
      \item [texmf-config]        \envname{TEXMFSYSCONFIG}
    \end{ttdescription}
    \item [texmf-local] \envname{TEXMFLOCAL}, pensato per essere mantenuto
      tra diverse edizioni.
  \end{ttdescription}
  \item[percorso home dell'utente] (\texttt{\$HOME} o
      \texttt{\%USERPROFILE\%})
    \begin{ttdescription}
      \item[.texlive2019] Dati privati generati e di configurazione relativi ad
        un'edizione precedente.
      \item[.texlive2020] Dati privati generati e di configurazione relativi
        all'attuale edizione.
      \begin{ttdescription}
        \item [texmf-var\ \ \ ] \envname{TEXMFVAR}, \envname{TEXMFCACHE}
        \item [texmf-config]    \envname{TEXMFCONFIG}
      \end{ttdescription}
    \item[texmf] \envname{TEXMFHOME} Macro personali, ecc.
  \end{ttdescription}
\end{description}


\subsection{Estensioni di \protect\TeX}
\label{sec:tex-extensions}

Lo sviluppo dell'originale \TeX{} di Knuth è congelato, se si escludono
rare correzioni di bug. È presente in \TL\ come \prog{tex} e si prevede
che non sarà modificato nel prossimo futuro. 
\TL{} contiene anche diverse versioni estese
di \TeX\ (note anche come motori \TeX):

\begin{description}

\item [\eTeX] aggiunge un insieme di nuove primitive \label{text:etex}
(riguardanti l'espansione delle macro, la scansione dei caratteri, le
classi di segnaposto, funzionalità di debug aggiuntive ed altro ancora)
e le estensioni \TeXXeT{} per la scrittura bidirezionale. Nella modalità
di base, \eTeX{} è compatibile al 100\% con il \TeX{} ordinario.
Consultate \OnCD{texmf-dist/doc/etex/base/etex_man.pdf}.

\item [pdf\TeX] parte dalle estensioni di \eTeX, aggiunge il supporto per
la generazione di file PDF oltre che dei \dvi{} e molte estensioni
che non sono legate alla generazione dell'output. Questo programma è
invocato dalla maggior parte dei formati, come \prog{etex}, \prog{latex},
\prog{pdflatex}. Il suo sito web è \url{http://www.pdftex.org/}.
Consultate \OnCD{texmf-dist/doc/pdftex/manual/pdftex-a.pdf} per il manuale
e \OnCD{texmf-dist/doc/pdftex/samplepdftex/samplepdf.tex} per gli
esempi d'uso di alcune delle sue funzionalità.

\item [Lua\TeX] è il successore designato di pdf\TeX\ ed è quasi del tutto
(ma non completamente) compatibile con i predecessori. È anche destinato a
sostituire le funzioni di Aleph (vedi sotto), per quanto non sia stata
prevista una compatibilità perfetta. L'interprete Lua incorportato
(\url{https://lua.org/}) permette soluzioni eleganti a molti problemi
spinosi di \TeX. Quando è invocato come \filename{texlua}, si comporta
come un interprete Lua autonomo ed è usato in questo modo all'interno
di \TL. Il suo sito web è \url{http://www.luatex.org} e il manuale di
riferimento è \OnCD{texmf-dist/doc/luatex/base/luatexref.pdf}.

\item [\XeTeX] aggiunge il supporto per l'input in Unicode e per i font
OpenType e di sistema, implementato usando librerie di terze parti
standard. Visitate \url{https://tug.org/xetex}.

\item [\OMEGA\ (Omega)] è basato sull'Unicode (caratteri a 16 bit) e
dunque consente di lavorare con quasi tutti gli alfabeti del mondo
contemporaneamente. Supporta anche i cosiddetti ``\OMEGA{} Translation
Process'' (OTP, Processi di Traduzione Omega), per compiere
trasformazioni complesse su input arbitrari. Omega non è più incluso in
\TL{} come programma separato; soltanto Aleph è fornito.

\item [Aleph] combina le estensioni \OMEGA\ ed \eTeX. Consultate
\OnCD{texmf-dist/doc/aleph/base}.

\end{description}


\subsection{Altri programmi rilevanti inclusi in \protect\TL}

Seguono alcuni ulteriori programmi di uso comune inclusi in \TL{}:
\begin{cmddescription}

\item [bibtex, biber] supporto per la bibliografia.

\item [makeindex, xindy] supporto per gli indici.

\item [dvips] converte i \dvi{} in \PS{}.

\item [xdvi] programma di anteprima dei \dvi{} per l'X Window System.

\item [dviconcat, dviselect] tagliano ed incollano le pagine contenute nei
file \dvi{}.

\item [dvipdfmx] converte i \dvi{} in PDF, un approccio alternativo
a pdf\TeX\ (citato in precedenza).

\item [psselect, psnup, \ldots] programmi per il trattamento dei \PS{}.

\item [pdfjam, pdfjoin, \ldots] programmi per il trattamento dei PDF.

\item [context, mtxrun] processore Con\TeX{}t e PDF.

\item [htlatex, \ldots] \cmdname{tex4ht}: convertitore da \AllTeX{} ad
  HTML (e ad XML ed altro ancora).

\end{cmddescription}


\htmlanchor{installation}
\section{Installazione}
\label{sec:install}

\subsection{Avviare l'installatore}
\label{sec:inst-start}

Per cominciare, procuratevi il \DVD{} \TK{} oppure scaricate
l'installatore di rete di \TL. Consulate
\url{https://tug.org/texlive/acquire.html} per maggiori informazioni ed
altri metodi per ottenere il software.

\begin{description} \item [Installatore di rete, .zip o .tar.gz:]
Scaricatelo da \CTAN, dal percorso \dirname{systems/texlive/tlnet};
l'indirizzo \url{http://mirror.ctan.org/systems/texlive/tlnet}
dovrebbe reindirizzare ad un vicino mirror aggiornato. Potete
scaricare sia \filename{install-tl.zip} che può essere usato sotto
Unix e Windows, sia il notevolmente più piccolo
\filename{install-unx.tar.gz} solo per Unix. Dopo averlo
decompresso, \filename{install-tl} e
\filename{install-tl-windows.bat} si troveranno nella sotto
directory \dirname{install-tl}.

\item [Installatore di rete, .exe per Windows:] Scaricatelo da \CTAN{}
come sopra e fate doppio clic su di esso. Questo avvierà una prima fase
di decompressione e installazione; fate riferimento alla
figura~\ref{fig:nsis}. Vi dà due scelte: ``Intalla'' e
``Decomprimi soltanto''.

\item [\DVD{} \TeX{} Collection:] aprite la sotto directory
\dirname{texlive} del \DVD. Sotto Windows, il programma di installazione
dovrebbe avviarsi automaticamente quando inserite il \DVD. Potete ottenere il
\DVD\ diventando membri di un gruppo utenti \TeX\ (caldamente consigliato,
\url{https://tug.org/usergroups.html}) oppure acquistandolo separatamente
(\url{https://tug.org/store}) o ancora masterizzandolo a partire
dall'immagine \ISO. Sulla maggior parte dei sistemi potete anche montare
direttamente l'\ISO. Dopo l'installazione dal \DVD\ o dall'immagine \ISO, se
volete ottenere gli aggiornamenti da Internet, consulate
\ref{sec:dvd-install-net-updates}.

\end{description}

\begin{figure}[tb]
\tlpng{nsis_installer}{.6\linewidth}
\caption{Prima fase dell'installatore \code{.exe} per Windows}\label{fig:nsis}
\end{figure}

Indipendentemente dall'origine, viene eseguito lo stesso programma di
installazione. La differenza più evidente tra i due è che con l'installatore
di rete riceverete i pacchetti che sono disponibili al momento. Invece
le immagini \DVD\ e \ISO\ non vengono aggiornate tra una versione 
pricipale pubblicata e l'altra.

Se avete la necessità di usare un proxy, usate un file
\filename{~/.wgetrc} o le variabili d'ambiente con le impostazioni proxy
per Wget
(\url{https://www.gnu.org/software/wget/manual/html_node/Proxies.html}),
o l'equivalente per qualsiasi programma di download usiate. Questo non
conta se state installando dal \DVD\ o da un'immagine \ISO.

Le sezioni seguenti spiegano l'avvio dell'installazione in maggiore
dettaglio.

\subsubsection{Unix}

Di seguito, \texttt{>} denota il prompt della shell; l'input dell'utente è
in \Ucom{\texttt{grassetto}}.
Il programma \filename{install-tl} è uno script Perl. Il modo più semplice
per avviarlo su un sistema compatibile Unix è il seguente:
\begin{alltt}
> \Ucom{perl /percorso/verso/il/programma/install-tl}
\end{alltt}
In alternativa potete invocare
\Ucom{/percorso/verso/il/programma/install-lt} se è rimasto eseguibile,
oppure spostarvi con \texttt{cd} nella directory, ecc.; non staremo a ripetere
tutte queste varianti. Potreste dover ingrandire la finestra di terminale
(prompt dei comandi, per gli utenti Windows)
affinché mostri l'intera schermata dell'installatore testuale
(figura~\ref{fig:text-main}).

Per installare in modalità \GUI\ (figura~\ref{fig:advanced-lnx}),
dovete avere installato Tcl/Tk. In questo caso, potete eseguire
\begin{alltt}
> \Ucom{perl install-tl -gui}
\end{alltt}

Le vecchie opzioni \code{wizard} e \code{perltk}/\code{expert} sono ancora
disponibili. Queste richiedono il modulo \dirname{Perl::Tk} compilato con
il supporto per XFT, che di solito non è un problema con \GNU/Linux, ma spesso
lo è con gli altri sistemi. Per un elenco completo delle varie opzioni:
\begin{alltt}
> \Ucom{perl install-tl -help}
\end{alltt}

\textbf{Sui permessi Unix:} Al momento dell'installazione,
la tua \code{umask} sarà rispettata dall'installatore di \TL{}.
Quindi, se vuoi che l'installazione sia usabile da altri utenti oltre che da
te, sii sicuro che le tue impostazioni siano permissive a sufficienza, per
esempio, \code{umask 002}. Per ulteriori informazioni riguardo
\code{umask}, consulta la documentazione del tuo sistema.

\textbf{Considerazioni speciali per Cygwin:} Diversamente da altri sistemi
Unix-compatibili, Cygwin non è preimpostato per includere tutti i
programmi di cui l'installatore di \TL{} ha bisogno. Consulta la
sezione~\ref{sec:cygwin}.


\subsubsection{MacOSX}
\label{sec:macosx}

Come accennato nella sezione~\ref{sec:tl-coll-dists}, abbiamo preparato
una distribuzione separata per \MacOSX{} chiamata Mac\TeX\
(\url{https://tug.org/mactex}). Su \MacOSX{} raccomandiamo di usare
l'installatore nativo di Mac\TeX\ al posto dell'installatore di \TL, in
quanto quello nativo esegue alcuni aggiustamenti specifici per il Mac, in
particolare per consentire con semplicità il passaggio tra le varie
distribuzioni \TeX\ per \MacOSX\ (Mac\TeX, Fink, MacPorts, \ldots)
usando la cosiddetta struttura dati \TeX{}Dist.

Mac\TeX\ è strettamente basato su \TL, le principali strutture delle
directory e gli eseguibili sono esattamente gli stessi, ma sono aggiunte
alcune ulteriori cartelle con documentazione e applicazioni specifici per
il Mac.


\subsubsection{Windows}\label{sec:wininst}

Se state usando il file zip scaricato e decompresso, oppure se
l'installatore su \DVD\ non si avvia automaticamente, fate doppio clic su
\filename{install-tl-windows.bat}.

Potete avviare l'installatore anche dal prompt dei comandi. Qui sotto,
\texttt{>} denota il prompt; l'input dell'utente è in
\Ucom{\texttt{grassetto}}. Se vi trovate nella cartella dell'installatore,
eseguite semplicemente:
\begin{alltt}
> \Ucom{install-tl-windows}
\end{alltt}

In alternativa potete lanciarlo con un percorso assoluto, come:
\begin{alltt}
> \Ucom{D:\bs{}texlive\bs{}install-tl-windows}
\end{alltt}
per il \DVD\ \TK, supponendo che \dirname{D:} sia l'unità del lettore \DVD. La
figura~\ref{fig:basic-w32} mostra la schermata base dell'installazione \GUI,
che è quella predefinita per Windows.

Per installare in modalità testuale, usate:
\begin{alltt}
> \Ucom{install-tl-windows -no-gui}
\end{alltt}

Per un elenco completo delle diverse opzioni:
\begin{alltt}
> \Ucom{install-tl-windows -help}
\end{alltt}

%%% TODO
\begin{figure}[tb]
\begin{boxedverbatim}
Installing TeX Live 2020 from: ...
Platform: x86_64-linux => 'GNU/Linux on Intel x86_64'
Distribution: inst (compressed)
Directory for temporary files: /tmp
...
 Detected platform: GNU/Linux on Intel x86_64

 <B> binary platforms: 1 out of 16

 <S> set installation scheme: scheme-full

 <C> customizing installation collections
     40 collections out of 41, disk space required: 6536 MB

 <D> directories:
   TEXDIR (the main TeX directory):
     /usr/local/texlive/2020
   ...

 <O> options:
   [ ] use letter size instead of A4 by default
   ...

 <V> set up for portable installation

Actions:
 <I> start installation to hard disk
 <P> save installation profile to 'texlive.profile' and exit
 <H> help
 <Q> quit
\end{boxedverbatim}
\vskip-\baselineskip
\caption{Schermata principale dell'installatore testuale
  (\GNU/Linux)}\label{fig:text-main}
\end{figure}

\begin{figure}[tb]
\tlpng{basic-w32}{.6\linewidth}
\caption{Schermata dell'installazione (Windows); il pulsante Avanzate
  darà qualcosa di simile alla figura~\ref{fig:advanced-lnx}}\label{fig:basic-w32}
\end{figure}

\begin{figure}[tb]
\tlpng{advanced-lnx}{\linewidth}
\caption{Schermata dell'installazione \GUI\ avanzata
  (\GNU/Linux)}\label{fig:advanced-lnx}
\end{figure}


\htmlanchor{cygwin}
\subsubsection{Cygwin}
\label{sec:cygwin}

Prima di iniziare l'installazione, usate il programma \filename{setup.exe}
di Cygwin per installare i pacchetti \filename{perl} e \filename{wget}, a
meno che non lo abbiate già fatto. I seguenti pacchetti aggiuntivi sono
raccomandati:
\begin{itemize*}
\item \filename{fontconfig} [richiesto da \XeTeX\ e Lua\TeX]
\item \filename{ghostscript} [richiesto da vari programmi]
\item \filename{libXaw7} [richiesto da \code{xdvi}]
\item \filename{ncurses} [fornisce il comando \code{clear} usato
  dall'installatore]
\end{itemize*}

\subsubsection{L'installatore testuale}

La figura~\ref{fig:text-main} mostra la schermata principale della
modalità testuale sotto Unix. L'installatore testuale è quello
predefinito, in ambiente Unix.

Questo è soltanto un installatore a riga di comando; non c'è alcun
supporto per il movimento del cursore. Ad esempio, non
potete muovervi tra le caselle di spunta o i campi di inserimento.
Voi semplicemente digitate qualcosa al prompt (MAIUSCOLE e minuscole sono
differenti), premete il tasto Invio e l'intera schermata del terminale
sarà aggiornata, con il contenuto modificato.

L'interfaccia di installazione testuale è così primitiva per farla
funzionare sul maggior numero possibile di piattaforme, anche con un Perl
minimale.

\subsubsection{L'installatore grafico}
\label{sec:graphical-inst}

L'installatore grafico predefinito parte semplice, giusto con poche opzioni;
vedi la figura~\ref{fig:basic-w32}. Può essere avviato con
\begin{alltt}
> \Ucom{install-tl -gui}
\end{alltt}
Il pulsante Avanzato dà accesso alla maggior parte delle opzioni
dell'installazione testuale; vedi la figura~\ref{fig:advanced-lnx}.

\subsubsection{Gli installatori obsoleti}

Le modalità \texttt{perltk}/\texttt{expert} e \texttt{wizard} sono ancora
disponibili per i sistemi che hanno Perl/Tk installato. Possono essere
specificate rispettivamente con gli argomenti \texttt{-gui=perltk} e
\texttt{-gui=wizard}.

\subsection{Eseguire l'installatore}
\label{sec:runinstall}

L'installatore è pensato per essere perlopiù autoesplicativo, comunque seguono
alcune note riguardo a varie opzioni e sottomenu.

\subsubsection{Menu delle architetture (solo Unix)}
\label{sec:binary}

%TODO figura!
\begin{figure}[tb]
\begin{boxedverbatim}
Available platforms:
===============================================================================
   a [ ] Cygwin on Intel x86 (i386-cygwin)
   b [ ] Cygwin on x86_64 (x86_64-cygwin)
   c [ ] MacOSX current (10.13-) on x86_64 (x86_64-darwin)
   d [ ] MacOSX legacy (10.6-) on x86_64 (x86_64-darwinlegacy)
   e [ ] FreeBSD on x86_64 (amd64-freebsd)
   f [ ] FreeBSD on Intel x86 (i386-freebsd)
   g [ ] GNU/Linux on ARM64 (aarch64-linux)
   h [ ] GNU/Linux on ARMv6/RPi (armhf-linux)
   i [ ] GNU/Linux on Intel x86 (i386-linux)
   j [X] GNU/Linux on x86_64 (x86_64-linux)
   k [ ] GNU/Linux on x86_64 with musl (x86_64-linuxmusl)
   l [ ] NetBSD on x86_64 (amd64-netbsd)
   m [ ] NetBSD on Intel x86 (i386-netbsd)
   o [ ] Solaris on Intel x86 (i386-solaris)
   p [ ] Solaris on x86_64 (x86_64-solaris)
   s [ ] Windows (win32)
\end{boxedverbatim}
\vskip-\baselineskip
\caption{Menu delle architetture}\label{fig:bin-text}
\end{figure}

La figura~\ref{fig:bin-text} mostra il menu delle architetture in modalità
testuale. Di base, saranno installati solo gli eseguibili per la propria
piattaforma attuale. Da questo menu si può selezionare l'installazione
dei binari anche per altre piattaforme. Questa opzione è utile se si
condivide un'installazione di \TeX\ in una rete di macchine eterogenee,
oppure per una macchina con due sistemi operativi.

\subsubsection{Selezionare cosa deve essere installato}
\label{sec:components}

% TODO
\begin{figure}[tbh]
\begin{boxedverbatim}
Select scheme:
===============================================================================
 a [X] full scheme (everything)
 b [ ] medium scheme (small + more packages and languages)
 c [ ] small scheme (basic + xetex, metapost, a few languages)
 d [ ] basic scheme (plain and latex)
 e [ ] minimal scheme (plain only)
 f [ ] ConTeXt scheme
 g [ ] GUST TeX Live scheme
 h [ ] infrastructure-only scheme (no TeX at all)
 i [ ] teTeX scheme (more than medium, but nowhere near full)
 j [ ] custom selection of collections
\end{boxedverbatim}
\vskip-\baselineskip
\caption{Menu degli schemi}\label{fig:scheme-text}
\end{figure}

La figura~\ref{fig:scheme-text} mostra il menu degli schemi di \TL; da qui
potete scegliere uno ``schema'', che è un insieme di collezioni di pacchetti.
Lo schema \optname{full} installa tutto quanto disponibile.
Questo è lo schema raccomandato, ma potete anche scegliere \optname{basic}
per avere solo plain e \LaTeX, \optname{small} per quache programma in più
(equvallente alla cosiddetta installazione Basic\TeX\ di Mac\TeX),
\optname{minimal} per scopi di prova e \optname{medium} o \optname{teTeX} per
una via di mezzo. Ci sono anche ulteriori schemi specializzati e specifici per
particolari paesi.

\begin{figure}[tbh]
\centering \tlpng{stdcoll}{.7\linewidth}
\caption{Menu delle collezioni}\label{fig:collections-gui}
\end{figure}

Potete rifinire la vostra scelta dello schema con i menu ``Collezioni''
(figura~\ref{fig:collections-gui}, mostrati, per cambiare, in modalità \GUI).

Le collezioni stanno ad un livello di dettaglio maggiore rispetto agli
schemi \Dash\ in pratica, uno schema consiste in varie collezioni, una
collezione consiste in uno o più pacchetti e un pacchetto (la più piccola
forma di raggruppamento in \TL) contiene i file con le macro \TeX, i font
e così via.

Se desiderate un controllo maggiore di quello fornito dal menu delle
collezioni, potete usare il programma \TeX\ Live Manager (\prog{tlmgr})
dopo l'installazione (cfr.\ la sezione~\ref{sec:tlmgr}); tramite tale
programma, potete controllare l'installazione al livello di dettaglio dei
singoli pacchetti.

\subsubsection{Directory}
\label{sec:directories}

La strutturazione predefinita è descritta nella
sezione~\ref{sec:texmftrees}, \p.\pageref{sec:texmftrees}. La directory
predefinita di installazione è
\dirname{/usr/local/texlive/2020} sotto
Unix e |%SystemDrive%\texlive\2020| sotto Windows. Questa configurazione
permette di avere molte installazioni parallele di \TL, ad esempio una per
ogni edizione (tipicamente per anno, come qui) e di passare tra di loro
semplicemente modificando il vostro percorso di ricerca.

La directory di installazione può essere modificata impostando la
cosiddetta \dirname{TEXDIR} nel programma di installazione. La schermata
\GUI\ per questa ed altre opzioni è mostrata nella
figura~\ref{fig:advanced-lnx}. Tra le ragioni più comuni per cambiarla ci sono
la mancanza di sufficiente spazio su disco in quella partizione (l'intera
\TL\ richiede diversi gigabyte) o la mancanza di permessi di scrittura per
la posizione predefinita (non è necessario essere root o amministratori
per installare \TL, ma è necessario disporre dei permessi di scrittura sulla
directory di destinazione).

Le directory di installazione possono essere modificate anche impostando
alcune variabili d'ambiente prima di eseguire il programma di
installazione (molto probabilmente \envname{TEXLIVE\_INSTALL\_PREFIX} o
\envname{TEXLIVE\_INSTALL\_TEXDIR}); consultate la documentazione tramite
|install-tl --help| (disponibile online su
\url{https://tug.org/texlive/doc/install-tl.html}) per la lista completa e
per maggiori dettagli.

Una ragionevole scelta alternativa è una directory all'interno della
vostra home, soprattutto se sarete gli unici utilizzatori. Usate `|~|'
per specificare la vostra home, come ad esempio in `|~/texlive/2020|'.

Raccomandiamo di includere l'anno nel percorso, così da poter tenere
diverse edizioni di \TL{} fianco a fianco (potete anche mantenere un nome
indipendente dall'edizione, come \dirname{/usr/local/texlive-cur}, sotto
forma di collegamento simbolico, di cui modificare la destinazione dopo
aver provato la nuova edizione).

Modificare \dirname{TEXDIR} durante l'installazione causerà anche la
modifica di \dirname{TEXMFLOCAL}, \dirname{TEXMFSYSVAR} e
\dirname{TEXMFSYSCONFIG}.

\dirname{TEXMFHOME} è la posizione raccomandata per le macro e i pacchetti
personali. Il suo valore predefinito è |~/texmf| (|~/Library/texmf| su Mac).
A differenza di \dirname{TEXDIR}, qui il carattere |~| viene mantenuto all'interno dei
file di configurazione creati, dato che è utile che si riferisca alla home
dell'utente che esegue \TeX. Si espande in \dirname{$HOME} sotto
Unix e \verb|%USERPROFILE%| sotto Windows. Nota speciale ridondante:
\envname{TEXMFHOME}, come tutti gli altri percorsi, deve essere organizzata
secondo il \TDS, o i file potrebbero non essere trovati. %stopzone

\dirname{TEXMFVAR} è la posizione in cui vengono memorizzati la maggior
parte dei dati generati durante l'esecuzione specifici per ciascun utente.
\dirname{TEXMFCACHE} è il nome della variabile usata per lo stesso scopo da
Lua\LaTeX\ e \ConTeXt\ MkIV (consultate la
sezione~\ref{sec:context-mkiv}, \p.\pageref{sec:context-mkiv}); il suo
valore predefinito è \dirname{TEXMFSYSVAR}, oppure (se non si hanno i
permessi di scrittura per quel percorso) \dirname{TEXMFVAR}.


\subsubsection{Opzioni}
\label{sec:options}

% TODO FIGURA!!!
\begin{figure}[tbh]
\begin{boxedverbatim}
Options setup:
===============================================================================
 <P> use letter size instead of A4 by default: [ ]
 <E> execution of restricted list of programs: [X]
 <F> create all format files:                  [X]
 <D> install font/macro doc tree:              [X]
 <S> install font/macro source tree:           [X]
 <L> create symlinks in standard directories:  [ ]
            binaries to:
            manpages to:
                info to:
 <Y> after install, set CTAN as source for package updates: [X]
\end{boxedverbatim}
\vskip-\baselineskip
\caption{Menu delle opzioni (Unix)}\label{fig:options-text}
\end{figure}

La figura~\ref{fig:options-text} mostra il menu delle opzioni in modalità
testuale. Ulteriori informazioni su ciascuna opzione:

\begin{description}
\item[use letter size instead of A4 by default:]
  Questa è la selezione del formato di carta predefinito. Ovviamente, se
  lo si desidera, si potrà e si dovrebbe specificare il formato della
  carta in ogni singolo documento.

\item[execution of restricted list of programs:] A partire da \TL\ 2010,
  è consentita l'esecuzione da parte di \TeX\ di pochi programmi esterni.
  La (molto breve) lista di questi programmi è data nel file
  \filename{texmf.cnf}. Consultate le novità 2010
  (sezione~\ref{sec:2010news}) per maggiori dettagli.

\item[create all format files:] Per quanto i file di formato non
  strettamente necessari richiedano tempo per essere generati e spazio su
  disco per essere memorizzati, è comunque raccomandato di lasciare questa
  opzione attiva: se non lo fate, allora i file di formato saranno
  generati, quando necessari, all'interno delle directory private
  \dirname{TEXMFVAR} di ciascun utente. In tali posizioni, non saranno
  aggiornati automaticamente se lo sono (ad esempio) gli eseguibili o i
  modelli di sillabazione nell'installazione, e così potreste trovarvi ad
  avere formati incompatibili.

\item[install font/macro \ldots\ tree:] Scaricano/installano 
  la documentazione e i file sorgenti inclusi nella
  maggior parte dei pacchetti. Non è raccomandato disattivarle.

\item[create symlinks in standard directories:] Questa opzione
  (solo Unix) evita di dover modificare le variabili di ambiente.
  Senza di essa, le directory di \TL{} devono solitamente essere aggiunte
  alle variabili \envname{PATH}, \envname{MANPATH} e \envname{INFOPATH}.
  Avete bisogno dei permessi di scrittura nelle directory di destinazione.
  Questa opzione serve ad accedere al sistema \TeX\ tramite directory che
  sono già note agli utenti, come \dirname{/usr/local/bin}, che già non
  contengono alcun file di \TeX. State attenti a non sovrascrivere i file 
  esistenti sul
  vostro sistema, ad esempio, non fate l'errore di indicare directory di 
  sistema. L'approccio
  più sicuro e raccomandato è quello di lasciare questa opzione
  disattivata.

\item[after install, set CTAN as source for package updates:]
  Quando si esegue l'installazione
  da \DVD, questa opzione è abilitata di default, dato che di solito si
  vuole prendere ogni successivo aggiornamento di pacchetti dall'area
  \CTAN\ che è aggiornata durante l'anno. L'unica ragione plausibile per
  disabilitarla è quando si vuole installare solo un sottoinsieme dal \DVD
  e si ha intenzione di estendere l'installazione successivamente. In
  ogni caso, l'archivio dei pacchetti per l'installatore e per gli
  aggiornamenti successivi all'installazione può essere impostato in modo
  indipendente secondo le necessità; consultate le sezioni~\ref{sec:location}
  e \ref{sec:dvd-install-net-updates}.
\end{description}
Opzioni specifiche di Windows, come mostrate dall'interfaccia avanzata
Perl/Tk:
\begin{description}
\item[regola l'impostazione del PATH nel registry] Questa assicura che tutti
  i programmi vedranno la directory degli esebuibili di \TL{} nel loro
  percorso di ricerca.

\item[aggiungi scorciatoie al menu] Se attiva, ci sarà un sottomenu \TL{}
   del menu Start. C'è una terza opzione `Voce nel launcher' oltre a
  `TeX Live menu' e `Nessuna scorciatoia'. Questa opzione è descritta nella
  sezione \ref{sec:sharedinstall}.

\item[cambia le associazioni dei file] Le opzioni sono solo `Solo nuovi'
  (crea le associazioni dei file, ma non sovrascrive quelle esistenti),
  `Tutti' e `Nessuno'.

\item[installa il programma \TeX{}works]
\end{description}
Quando tutte le impostazioni sono come le desiderate, potete digitare
``I'' nell'interfaccia testuale, oppure potete premere il pulsante
`Installa TeX Live' nella \GUI\ Perl/Tk, per avviare il processo di
installazione. Quando questo è completo, passate alla
sezione~\ref{sec:postinstall} per leggere cos'altro dovete fare, se
ce n'è bisogno.


\subsection{Opzioni della riga di comando di install-tl}
\label{sec:cmdline}

Digitate
\begin{alltt}
> \Ucom{install-tl -help}
\end{alltt}
per ottenere un elenco delle opzioni della riga di comando. Sia |-| che
|--| possono essere usati per introdurre i nomi delle opzioni. Ecco le più
comuni:

\begin{ttdescription}
\item[-gui] Usa l'installatore \GUI\ se è possibile. Questo richiede Tcl/Tk
  versione 8.5 o superiore. Vale per \MacOSX\ ed è distribuito con \TL{}
  su Windows. Le opzoni obsolete \texttt{-gui=perltk} e \texttt{-gui=wizard}
  sono ancora disponibili e richiedono il modulo Perl/Tk
  (\url{https://tug.org/texlive/distro.html#perltk}) compilato con il
  supporto per XFT; se Tcl/Tk e Perl/Tk non sono disponibili, l'installazione
  prosegue in modalità testuale.

\item[-no-gui] Costringe all'uso dell'installatore testuale.

\item[-lang {\sl LL}] Specifica la lingua dell'interfaccia
  dell'installatore sotto forma di un codice standard (di solito di due
  lettere). L'installatore cerca di determinare automaticamente la lingua
  corretta, ma se fallisce o se la lingua corretta non è disponibile,
  allora usa l'inglese come ripiego. Lanciate \verb|install-tl --help|
  per la lista delle lingue disponibili.

\item[-portable] Installa per un uso portabile su una chiavetta \USB{}.
  Questa opzione è anche selezionabile dall'installatore testuale con il 
  comando
  \code{V} e dall'installatore grafico. Consultate la
  sezione~\ref{sec:portable-tl}.

\item[-profile {\sl file}] Carica il profilo di installazione \var{file} ed
  esegue l'installazione senza interazione con l'utente. Il programma di
  installazione scrive sempre un file \filename{texlive.profile} nella sotto
  directory \dirname{tlpkg} della vostra installazione. Ad esempio, quel file 
  può essere
  passato come argomento per ripetere esattamente la stessa installazione su
  un sistema differente. In alternativa, potete usare un profilo
  personalizzato, creato facilmente partendo da uno generato automaticamente
  e modificandone i valori, oppure potete usare un file vuoto, che prenderà
  come valori tutti quelli predefiniti.

\item [-repository {\sl url-o-directory}] Specifica l'archivio dei
  pacchetti da cui installare; vedi più avanti.

\htmlanchor{opt-in-place}
\item[-in-place] (Documentato solo per completezza: non usate questa
  opzione a meno che non sappiate cosa state facendo.) Se avete già una
  copia di \TL{}, anche ottenuta tramite rsync o svn (consultate
  \url{https://tug.org/texlive/acquire-mirror.html}),
  allora questa opzione userà ciò che avete ottenuto, così com'è, ed
  eseguirà solo le necessarie procedure di post installazione. Siete
  avvisati che il file \filename{tlpkg/texlive.tlpdb} potrebbe venire
  sovrascritto; salvarne una copia è vostra responsabilità. Inoltre, la
  rimozione dei pacchetti deve essere eseguita manualmente. Questa
  opzione non può essere attivata dall'interfaccia dell'installatore.
\end{ttdescription}


\subsubsection{L'opzione \optname{-repository}}
\label{sec:location}

L'archivio dei pacchetti di rete predefinito è uno dei mirror di \CTAN{}
scelto automaticamente tramite \url{http://mirror.ctan.org}.

Per rimpiazzare questa scelta, potete specificare un url che comincia per
\texttt{ftp:}, \texttt{http:} o \texttt{file:/}, oppure il semplice nome
di una directory (quando specificate un indirizzo \texttt{http:} o
\texttt{ftp:}, l'eventuale carattere finale ``\texttt{/}'' e l'eventuale
``\texttt{/tlpkg}'' conclusivo saranno ignorati).

Per esempio, potete scegliere uno specifico mirror di \CTAN\ con qualcosa
del tipo:
\url{http://ctan.example.org/tex-archive/systems/texlive/tlnet/},
indicando un vero nome di un sito e il suo specifico
percorso che ospita \CTAN al posto di |ctan.example.org/tex-archive|. 
L'elenco dei mirror di \CTAN\ si trova alla
pagina \url{https://ctan.org/mirrors}.
 
Se il valore specificato è locale (sia tramite percorso che tramite un url
\texttt{file:/}), saranno usati i file compressi contenuti in una sotto
directory \dirname{archive} del percorso (anche se sono disponibili anche
file non compressi).


\subsection{Azioni successive all'installazione}
\label{sec:postinstall}

Potrebbero essere necessarie alcune azioni successivie all'installazione.

\subsubsection{Variabili d'ambiente per Unix}
\label{sec:env}

Se avete deciso di creare i collegamenti simbolici nelle directory standard
(come descritto nella sezione~\ref{sec:options}), allora non c'è bisogno di
modificare le variabili d'ambiente. Altrimenti, nei sistemi Unix, la
directory degli eseguibili per la vostra piattaforma deve essere aggiunta al
percorso di ricerca (sotto Windows, è l'installatore ad occuparsene).

Ogni piattaforma supportata ha la propria sotto directory all'interno di
\dirname{TEXDIR/bin}. Consultate la figura~\ref{fig:bin-text} per la lista
delle sotto directory e delle corrispondenti piattaforme.

Facoltativamente potete anche aggiungere le directory della documentazione
man e Info ai propri rispettivi percorsi di ricerca, se volete che gli
strumenti di sistema la trovino. Le pagine di manuale potrebbero essere
trovate magicamente dopo l'aggiunta al \envname{PATH}. 

Per le shell compatibili con la Bourne, come \prog{bash}, e usando
\GNU/Linux per Intel x86 e le directory predefinite come esempio, il file
da modificare dovrebbe essere \filename{$HOME/.profile} %stopzone
(o un qualunque altro file aperto da \filename{.profile}), e le linee da
aggiungere sarebbero simili alle seguenti:

\begin{sverbatim}
PATH=/usr/local/texlive/2020/bin/x86_64-linux:$PATH; export PATH
MANPATH=/usr/local/texlive/2020/texmf/doc/man:$MANPATH; export MANPATH
INFOPATH=/usr/local/texlive/2020/texmf/doc/info:$INFOPATH; export INFOPATH
\end{sverbatim}
%stopzone

Per csh o tcsh, il file da modificare tipicamente è
\filename{$HOME/.cshrc} e le linee da aggiungere sarebbero del
tipo:

\begin{sverbatim}
setenv PATH /usr/local/texlive/2020/bin/x86_64-linux:$PATH
setenv MANPATH /usr/local/texlive/2020/texmf/doc/man:$MANPATH
setenv INFOPATH /usr/local/texlive/2020/texmf/doc/info:$INFOPATH
\end{sverbatim}

Se già avete delle impostazioni del genere in uno dei file citati,
naturalmente dovete unirvi le directory di \TL\ come è più
opportuno.


\subsubsection{Variabili d'ambiente: configurazione globale}
\label{sec:envglobal}

Se volete che questi cambiamenti siano globali oppure che si applichino a
ciascun nuovo utente aggiunto nel sistema, allora dovete cavarvela da
soli; c'è semplicemente troppa varietà tra i diversi sistemi nel come e
nel dove queste cose siano configurate.

I nostri due suggerimenti sono: 1)~potete controllare il file
\filename{/etc/manpath.config} e, se esiste, aggiungere linee come

\begin{sverbatim}
MANPATH_MAP /usr/local/texlive/2020/bin/x86_64-linux \
            /usr/local/texlive/2020/texmf/doc/man
\end{sverbatim}

E 2)~verificate la presenza di un file \filename{/etc/environment} che
potrebbe definire il percorso di ricerca ed altre variabili di ambiente
predefinite.

In ciascuna directory contenente eseguibili (per Unix), noi creiamo anche
un collegamento simbolico chiamato \code{man} alla directory
\dirname{texmf-dist/doc/man}. Alcuni programmi \code{man}, come quello di
\MacOSX, lo individueranno automaticamente, ovviando alla necessità di
qualunque configurazione per le pagine di manuale.


\subsubsection{Aggiornamenti da Internet dopo l'installazione del \DVD}
\label{sec:dvd-install-net-updates}

Se avete installato \TL\ dal \DVD\ e successivamente volete ottenere gli
aggiornamenti da Internet, dovete eseguire questo comando---\emph{dopo} aver
aggiornato il vostro percorso di ricerca (come descritto nella sezione
precedente):

\begin{alltt}
> \Ucom{tlmgr option repository http://mirror.ctan.org/systems/texlive/tlnet}
\end{alltt}

Questo dice a \cmdname{tlmgr} di usare un mirror di \CTAN\ nelle vicinanze
per gli aggiornamenti futuri. È l'azione predefinita quando si installa dal
\DVD, tramite l'opzione descritta nella sezione~\ref{sec:options}.

Se ci sono problemi con la selezione automatica di un mirror, potete
specificarne uno in particolare dall'elenco alla pagina
\url{https://ctan.org/mirrors}. Usate il percorso esatto alla sotto directory
\dirname{tlnet} su quel mirror, come mostrato in precedenza.


\htmlanchor{xetexfontconfig}  % manteniamo il link storico funzionante
\htmlanchor{sysfontconfig}
\subsubsection{Configurazione dei font di sistema per \protect\XeTeX\protect\ e Lua\protect\TeX}
\label{sec:font-conf-sys}

\XeTeX\ e Lua\TeX\ possono usare qualsiasi font installato nel sistema,
non solo quelli nelle directory di \TeX. Fanno questo tramite metodi
correlati, ma non identici.

Su Windows, i font distribuiti con \TL\ sono resi automaticmanete
disponibili a \XeTeX\ tramite il loro nome. Su \MacOSX, il supporto
alla ricerca per nome dei font richiede del passi aggiuntivi; visitate
le pagine dedicate a Mac\TeX\ (\url{https://tug.org/mactex}). Per
altri sistemi Unix, segue la procedura per trovare i font forniti con
\TL\ tramite il loro nome.

Per facilitare il compito, quando il pacchetto \pkgname{xetex} è
installato (sia durante l'installazione iniziale che in seguito), viene 
creato il file di configurazione necessario in
\filename{TEXMFSYSVAR/fonts/conf/texlive-fontconfig.conf}.

Per impostare i font di \TL{} per l'uso nell'intero sistema (assumendo che
abbiate gli opportuni privilegi), procedete come segue:
\begin{enumerate*}
\item Copiate il file \filename{texlive-fontconfig.conf} in
  \dirname{/etc/fonts/conf.d/09-texlive.conf}.

\item Eseguite \Ucom{fc-cache -fsv}.
\end{enumerate*}

Se non avete privilegi sufficienti per completare i passi precedenti,
o se volete rendere disponibili i font di \TL{} ad un
solo utente, fate quello che segue:
\begin{enumerate*}
\item Copiate il file \filename{texlive-fontconfig.conf} in
  \filename{~/.fonts.conf}, dove \filename{~} è la vostra directory di
  home.

\item Eseguite \Ucom{fc-cache -fv}.
\end{enumerate*}

Potete eseguire \code{fc-list} per vedere i nomi dei font di sistema.
L'incantesimo \code{fc-list : family style file spacing} (dove tutti gli
argomenti sono esattamente come mostrati) generalmente fornisce alcune
informazioni interessanti.


\subsubsection{\protect\ConTeXt{} Mark IV}
\label{sec:context-mkiv}

Sia il ``vecchio'' \ConTeXt{} (Mark II) che il ``nuovo'' \ConTeXt{}
(Mark IV) dovrebbero funzionare così come sono dopo l'installazione
di \TL{} e non dovrebbero richiedere alcuna attenzione particolare finché
restate fedeli a \verb+tlmgr+ per gli aggiornamenti.

Comunque, dato che \ConTeXt{} MkIV non usa la libreria kpathsea,
sarà richiesta un po' di configurazione ogni qual volta installate nuovi
file manualmente (senza usare \verb+tlmgr+). Dopo ogni installazione del
genere, ogni utente di MkIV deve eseguire:
\begin{sverbatim}
context --generate
\end{sverbatim}
per aggiornare  i dati della cache di \ConTeXt{} sul disco.
I file risultanti sono memorizzati in \code{TEXMFCACHE}, il cui valore
predefinito in \TL\ è \verb+TEXMFSYSVAR;TEXMFVAR+.

\ConTeXt\ MkIV leggerà da tutti i percorsi menzionati in
\verb+TEXMFCACHE+ e scriverà nel primo percorso che sia scrivibile. Durante
la lettura, in caso di dati di cache duplicati avrà la precedenza l'ultima
corrispondenza trovata.

Per ulteriori informazioni, consultate le pagine
\url{https://wiki.contextgarden.net/Running_Mark_IV}.


\subsubsection{Integrare macro locali e personali}
\label{sec:local-personal-macros}

Questa questione è già stata citata implicitamente nella
sezione~\ref{sec:texmftrees}:
\dirname{TEXMFLOCAL} (\dirname{/usr/local/texlive/texmf-local} o
\verb|%SystemDrive%\texlive\texmf-local| su Windows, di default) è pensato
per i font e le macro locali, ma disponibili per l'intero sistema; mentre
\dirname{TEXMFHOME} (\dirname{$HOME/texmf} o \verb|%USERPROFILE%\texmf|,
come predefiniti) è per le macro e i font personali.
Queste directory sono pensate per rimanere fisse da un'edizione all'altra
ed il loro contenuto per essere visto automaticamente da una nuova versione di 
\TL.
Quindi, è meglio trattenersi dal modificare la definizione di
\dirname{TEXMFLOCAL} in modo che punti ad una directory troppo diversa da
quella principale di \TL, oppure potreste doverla modificare manualmente
nelle future edizioni.

In entrambe le posizioni, i file dovrebbero essere posti ognuno nella
propria sotto directory secondo la Struttura delle Directory di \TeX\
(\TDS); visitate \url{https://tug.org/tds} o consultate
\filename{texmf-dist/web2c/texmf.cnf}. Ad esempio, il file di una
classe o di un pacchetto \LaTeX{} andrebbero posizionati in
\dirname{TEXMFLOCAL/tex/latex} o in \dirname{TEXMFHOME/tex/latex}, oppure
in una sotto directory di una delle due.

\dirname{TEXMFLOCAL} richiede un database dei nomi dei file aggiornato,
altrimenti i file non saranno trovati. Potete aggiornarlo con il comando
\cmdname{mktexlsr} o utilizzando il pulsante ``Aggiorna il database dei
nomi dei file'' nel pannello di configurazione della \GUI\ di \TeX\ Live
Manager.

Di default, ciascuna di queste variabili è definita come una directory
singola, come mostrato. Questo non è un requisito rigido. Se dovete passare
facilmente tra diverse versioni di grandi pacchetti, ad esempio, potete
mantenere directory multiple per i vostri scopi. Questo è ottenuto
impostando \dirname{TEXMFHOME} alla lista delle directory, tra parentesi
graffe, separate da virgole:

\begin{verbatim}
  TEXMFHOME = {/mia/dir1,/miadir2,/una/terza/dir}
\end{verbatim}

La sezione~\ref{sec:brace-expansion} descrive ulteriormente l'espansione
delle graffe.


\subsubsection{Integrare font di terze parti}

Sfortunatamente, si tratta di un argomento ingarbugliato. Dimenticatevene
a meno che non vogliate addentrarvi nei numerosi dettagli
dell'installazione di \TeX{}. Molti font sono già inclusi in \TL, quindi
dategli uno sguardo se non sapete già che quello che volete non è lì.

Una possibile alternativa è quella di usare \XeTeX\ o Lua\TeX\ (consultate
la sezione~\ref{sec:tex-extensions}), che vi consentono di usare i font
del sistema operativo senza alcuna installazione in \TeX.

Se avete comunque bisogno di farlo, visitate la pagina
\url{https://tug.org/fonts/fontinstall.html} dove abbiamo fatto del nostro
meglio per descrivere la procedura.


\subsection{Collaudare l'installazione}
\label{sec:test-install}

Dopo aver installato \TL, naturalmente vorrete provarlo, così da poter
iniziare a creare bellissimi documenti e\slash o font.

Una cosa che vorrete immediatamente cercare è un programma con cui
elaborare i file. \TL{} installa \TeX{}works su Windows
(\url{https://tug.org/texworks}) e Mac\TeX\ installa TeXShop
(\url{https://pages.uoregon.edu/koch/texshop}). Su altri sistemi Unix
spetta a voi scegliere un editor. Ci sono molte scelte disponibili, alcune
delle quali sono elencate nella prossima sezione; consultate anche
\url{https://tug.org/interest.html#editors}. Qualsiasi editor di testo
funzionerà; non è necessario qualcosa specifico per \TeX.

Il resto di questa sezione fornisce alcune procedure elementari per verificare che il
nuovo sistema funzioni. Qui daremo i comandi per Unix; è più facile che
sotto \MacOSX e Windows eseguiate le prove tramite un'interfaccia grafica,
ma i principi sono gli stessi.

\begin{enumerate}

\item Assicuratevi per prima cosa di poter eseguire il programma
\cmdname{tex}:
\begin{alltt}
> \Ucom{tex -{}-version}
TeX 3.14159265 (TeX Live ...)
Copyright ... D.E. Knuth.
...
\end{alltt}
Se questo restituisce ``comando non trovato'' al posto delle informazioni
sulla versione e sul copyright, oppure una versione più vecchia,
verosimilmente non avete la giusta directory \dirname{bin} nel vostro
\envname{PATH}. Consultate le informazioni sull'impostazione delle
variabili d'ambiente a \p.\pageref{sec:env}.

\item Elaborate un file \LaTeX{} elementare:
\begin{alltt}
> \Ucom{latex sample2e.tex}
This is pdfTeX 3.14...
...
Output written on sample2e.dvi (3 pages, 7484 bytes).
Transcript written on sample2e.log.
\end{alltt}
Se questo non riesce a trovare \filename{sample2e.tex} o altri file,
verosimilmente avete qualche interferenza proveniente da vecchie variabili
d'ambiente o vecchi file di configurazione; noi raccomandiamo, tanto per
iniziare, di eliminare tutte le variabili d'ambiente relative a \TeX\ (per
un'analisi approfondita, potete chiedere a \TeX{} di fare un resoconto su
ciò che esattamente stia cercando e che trovi; consultate ``Risoluzione
dei problemi'' a pagina~\pageref{sec:debugging}).

\item Mostrate un'anteprima del risultato:
\begin{alltt}
> \Ucom{xdvi sample2e.dvi}    # Unix
> \Ucom{dviout sample2e.dvi}  # Windows
\end{alltt}
Dovreste vedere una nuova finestra con un bel documento che spiega alcuni
dei fondamenti di \LaTeX{} (che vale la pena di leggere, se siete nuovi
utilizzatori di \TeX). Dovete avere X in esecuzione affinché \cmdname{xdvi}
funzioni; se non è così, oppure se la vostra variabile d'ambiente
\envname{DISPLAY} è impostata non correttamente, otterrete un errore
\samp{Can't open display}.

\item Create un file \PS{} per la stampa o la visualizzazione:
\begin{alltt}
> \Ucom{dvips sample2e.dvi -o sample2e.ps}
\end{alltt}

\item Create un file PDF al posto di un \dvi{}; questo comando
elabora il file \filename{.tex} e scrive direttamente un PDF:
\begin{alltt}
> \Ucom{pdflatex sample2e.tex}
\end{alltt}

\item Mostrate un'anteprima del PDF:
\begin{alltt}
> \Ucom{gv sample2e.pdf}
\textrm{o:}
> \Ucom{xpdf sample2e.pdf}
\end{alltt}
Né \cmdname{gv}, né \cmdname{xpdf} sono inclusi in \TL{}, quindi dovete
installarli separatamente. Visitate \url{https://www.gnu.org/software/gv} e
\url{https://www.xpdfreader.com}, rispettivamente. Ci sono anche tantissimi
altri visualizzatori PDF. Per Windows, noi raccomandiamo di provare
Sumatra PDF (\url{https://www.sumatrapdfreader.org/free-pdf-reader.html}).

\item Potreste trovare utili, in aggiunta a \filename{sample2e.tex}, i
seguenti file di prova predefiniti:

\begin{ttdescription}
\item [small2e.tex] Un documento più semplice di \filename{sample2e}, per
ridurre la dimensione dell'input qualora si riscontrino problemi.
\item [testpage.tex] Verifica se la vostra stampante introduce qualche
margine.
\item [nfssfont.tex] Per stampare tabelle e prove riguardanti i font.
\item [testfont.tex] Anche questo per le tabelle dei font, ma usando plain
\TeX{}.
\item [story.tex] Il più canonico tra tutti file di prova per (plain) \TeX{}.
Dovete digitare \samp{\bs bye} al prompt \code{*} dopo \samp{tex story.tex}.
\end{ttdescription}

\item Se avete installato il pacchetto \filename{xetex}, potete verificare che
riesca ad accedere ai font di sistema in questo modo:
\begin{alltt}
> \Ucom{xetex opentype-info.tex}
This is XeTeX, Version 3.14\dots
...
Output written on opentype-info.pdf (1 page).
Transcript written on opentype-info.log.
\end{alltt}

Se ricevete un messaggio di errore che riporta ``Invalid fontname `Latin
Modern Roman/ICU\dots'', allora dovete configurare il vostro sistema in
modo che i font distribuiti con \TL\ possano essere trovati. Consultate la
sezione~\ref{sec:font-conf-sys}.

\end{enumerate}

\subsection{Collegamenti ad ulteriori software scaricabili}

Se siete nuovi utilizzatori di \TeX{} o comunque avete bisogno di aiuto
nella scrittura di documenti \TeX{} o \LaTeX, visitate la pagina
\url{https://tug.org/begin.html} per alcune risorse introduttive.

Ecco i collegamenti ad alcuni altri strumenti che potete prendere in
considerazione per l'installazione:
\begin{description}
\item[Ghostscript] \url{https://ghostscript.com/}
\item[Perl] \url{https://perl.org/} con i pacchetti supplementari da
      CPAN, \url{https://cpan.org/}
\item[ImageMagick] \url{https://imagemagick.org}, per l'elaborazione e
      la conversione di immagini
\item[NetPBM] \url{http://netpbm.sourceforge.net/}, anch'esso per le
      immagini.

\item[Editor orientati a \TeX] C'è un'ampia scelta ed è una questione di
      gusto dell'utente. Eccone una selezione in ordine alfabetico (alcuni
      sono solo per Windows).
  \begin{itemize*}
  \item \cmdname{GNU Emacs} è disponibile anche per Windows, visita
        \url{https://www.gnu.org/software/emacs/emacs.html}.
  \item \cmdname{Emacs con Auc\TeX} per Windows è disponibile su \CTAN. La
        home page di Auc\TeX\ è \url{https://www.gnu.org/software/auctex}.
  \item \cmdname{SciTE} è disponibile su
        \url{https://www.scintilla.org/SciTE.html}.
  \item \cmdname{Texmaker} è un software libero disponibile su
        \url{https://www.xm1math.net/texmaker/}.
  \item \cmdname{TeXstudio} è nato come un fork di \cmdname{Texmaker}
        con funzionalità aggiuntive; \url{https://texstudio.org/}.
  \item \cmdname{TeXnicCenter} è un software libero, disponibile su
        \url{https://www.texniccenter.org} e nella distribuzione
        pro\TeX{}t.
  \item \cmdname{TeXworks} è un software libero disponibile su
        \url{https://tug.org/texworks} e installato come parte di \TL\ solo
        su Windows.
  \item \cmdname{Vim} è un software libero disponibile su
        \url{https://www.vim.org}.
  \item \cmdname{WinEdt} è un software shareware disponibile su
        \url{https://tug.org/winedt} o \url{https://www.winedt.com}.
  \item \cmdname{WinShell} è disponibile su \url{https://www.winshell.de}.
  \end{itemize*}
\end{description}
Per un elenco molto più lungo di pacchetti e programmi, visitate
\url{https://tug.org/interest.html}.


\section{Installazione specializzate}

Le sezioni precedenti hanno descritto il processo di installazione di base.
Ora passiamo ad alcuni casi particolari.

\htmlanchor{tlsharedinstall}
\subsection{Installazioni condivise tra utenti o macchine}
\label{sec:sharedinstall}

\TL{} è stato progettato per essere condiviso tra diversi sistemi in una
rete. Con l'organizzazione predefinita delle directory, nessun percorso
assoluto viene configurato: le posizioni dei file necessari ai programmi di
\TL{} sono relative ai programmi stessi. Potete vederlo nel file di
configurazione principale \filename{$TEXMFDIST/web2c/texmf.cnf}, che
contiene linee come:
\begin{sverbatim}
TEXMFROOT = $SELFAUTOPARENT
...
TEXMFDIST = $TEXMFROOT/texmf-dist
...
TEXMFLOCAL = $SELFAUTOGRANDPARENT/../texmf-local
\end{sverbatim}
Questo significa che aggiungere la directory degli eseguibili di \TL{}
appropriati alla piattaforma al loro percorso di ricerca è sufficiente
per ottenere una configurazione funzionante.

Per la stessa ragione, potete anche installare \TL{} localmente e
successivamente muovere l'intera struttura verso un percorso di rete.

Per Windows, \TL{} include un programma \filename{tlaunch}. La sua finestra
principale contiene voci di menu e pulsanti per i vari programmi e
documentazione relativi a \TeX, personalizzabili tramite un file
\code{ini}. Al primo utilizzo, il programma replica le tipiche operazioni
post-installazione specifiche per Windows, ossia la modifica dei percorsi
e le associazioni dei file, ma soltanto per l'utente corrente Quindi,
i computer che accedono a \TL{} nella rete locale hanno solo bisogno di una
scorciatoia di menu per questo programma. Consultate il manuale di
\code{tlaunch} (\code{texdoc tlaunch}, o \url{https://ctan.org/pkg/tlaunch}).


\htmlanchor{tlportable}
\subsection{Installazioni portabili (\USB)} 
\label{sec:portable-tl}

L'opzione \code{-portable} dell'installazione (o il comando \code{V}
nell'installatore testuale o la corrispondente opzione della \GUI) crea
un'installazione di \TL{} contenuta completamente in sé stessa all'interno
di una radice comune e rinuncia all'integrazione con il sistema. Potete
creare una tale installazione direttamente su una penna \USB{} o copiarla su
una penna \USB{} successivamente.

Per eseguire \TeX\ usando l'installazione portabile, avete bisogno di
aggiungere la directory degli eseguibili appropriata al percorso di ricerca
durante la sessione di terminale, come al solito.

Su Windows, potete fare doppio click su \filename{tl-tray-menu} alla radice
dell'installazione e creare un `menu tray' temporaneo che permette di
scegliere tra alcuni compiti comuni, come indicato in questa schermata:
\medskip
\tlpng{tray-menu}{4cm}
\smallskip
\noindent La voce ``More\ldots'' spiega come potete personalizzare questo menu.

%\htmlanchor{tlisoinstall}
%\subsection{Installazioni degli \ISO\ (o del \DVD)}
%\label{sec:isoinstall}
%
%Se non avete bisogno di aggiornare o in qualunque altro modo modificare
%spesso la vostra installazione e\slash o avete diversi sistemi sui quali
%usare \TL, potrebbe essere conveniente creare un'\ISO\ della vostra
%installazione di \TL{} perché:
%
%\begin{itemize}
%\item Copiare una \ISO\ tra diversi computer è molto più rapido che copiare
%  un'installazione normale.
%\item Se avete due sistemi operativi sulla stessa macchina e volete che
%  condividano un'installazione di \TL, un'installazione \ISO\ non è
%  vincolata alle idiosincrasie e alle limitazioni di altri filesystem
%  reciprocamente supportati (FAT32, NTFS, HFS+).
%\item Le macchine virtuali possono semplicemente montare tale \ISO.
%\end{itemize}
%
%Ovviamente potete anche masterizzare un'\ISO\ su un \DVD, se lo ritenete
%utile per voi.
%
%I sistemi desktop \GNU/Linux/Unix, incluso \MacOSX, sono capaci di montare
%una \ISO. Windows 8 è la prima versione di Windows che può farlo. Al di là
%di questo, non cambia nulla confrontato con una normale
%installazione su hard disk, consulate la sezione \ref{sec:env}.
%
%Quando preparate tale installazione su \ISO, è meglio omettere la
%sottodirectory per l'anno dell'edizione e avere \filename{texmf-local} allo
%stesso livello degli altri alberi (\filename{texmf-dist},
%\filename{texmf-var}, ecc.). Potete fare ciò con le normali opzioni per le
%directory nell'installatore.
%
%Per un sistema Windows fisico (piuttosto che virtuale), potete masterizzare
%la \ISO\ su un DVD. Comunque, potrebbe valere la pena investigare le
%varie opzioni gratuite per il montaggio delle \ISO, come WinCDEmu da
%\url{http://wincdemu.sysprogs.org/}.
%
%Per l'integrazione con il sistema Windows, potete includere gli script
%\filename{w32client} descritti nella sezione~\ref{sec:sharedinstall} e su
%\url{http://tug.org/texlive/w32client.html}, che funzionano per
%un'installazione \ISO\ così come per una condivisa.
%
%Su \MacOSX, TeXShop sarà capace di usare l'installazione su
%DVD se un collegamento simbolico \filename{/usr/texbin} punta alla
%directory degli eseguibili appropriata, per esempio,
%\begin{verbatim}
%sudo ln -s /Volumes/MyTeXLive/bin/universal-darwin /usr/texbin
%\end{verbatim}
%
%Nota storica: \TL{} 2010 è stata la prima edizione di \TL{} che non è stata
%più distribuita ``live''. Comunque, richiedeva sempre alcune acrobazie per
%funzionare da \DVD\ o da \ISO; in particolare, non c'era alcun modo di
%evitare di impostare almeno un'ulteriore variabile d'ambiente. Se create la
%vostra \ISO\ da un'installazione esistente, allora non c'è bisogno di
%questo.

\htmlanchor{tlmgr}
\section{\cmdname{tlmgr}: gestire la vostra installazione}
\label{sec:tlmgr}

\begin{figure}[tb]
\tlpng{tlshell-macos}{\linewidth}
\caption{\GUI\ \prog{tlshell} che mostra il menu Azioni (\MacOSX)}
\label{fig:tlshell}
\end{figure}

\begin{figure}[tb]
\tlpng{tlcockpit-packages}{.8\linewidth}
\caption{\GUI\ \prog{tlcockpit} per \prog{tlmgr}}
\label{fig:tlcockpit}
\end{figure}

\begin{figure}[tb]
\tlpng{tlmgr-gui}{\linewidth}
\caption{Modalità \GUI\ obsoleta di \prog{tlmgr}: finestra principlae dopo aver
  premuto il pulsante `Carica'}
\label{fig:tlmgr-gui}
\end{figure}

\TL{} include un programma chiamato \prog{tlmgr} per gestire \TL{} dopo
l'installazione iniziale. Le sue funzionalità includono:

\begin{itemize*}
\item installare, aggiornare, archiviare, ripristinare e disinstallare i
  singoli pacchetti, eventualmente tenendo conto delle dipendenze;
\item cercare ed elencare pacchetti e le loro descrizioni;
\item elencare, aggiungere e rimuovere le piattaforme disponibili;
\item cambiare le opzioni di installazione come la dimensione del foglio e
  il percorso di installazione (consultate la sezione~\ref{sec:location}).
\end{itemize*}

Le funzionalità di \prog{tlmgr} incorporano completamente quelle del
programma \prog{texconfig}. Distribuiamo e manteniamo ancora
\prog{texconfig} per tutti quelli abituati alla sua interfaccia, ma
attualmente raccomandiamo di usare \prog{tlmgr}.

\subsection{Interfacce \GUI\ per \cmdname{tlmgr}}

\TL{} contiene diverse interfacce \GUI\ per \prog{tlmgr}. Due degne di nota:
(1)~la figura~\ref{fig:tlshell} mostra \cmdname{tlshell}, che è scritto in Tcl/Tk
e funziona così com'è sotto Windows e \MacOSX; (2)~la figura~\ref{fig:tlcockpit}
mostra \prog{tlcockpit}, che richiede Java versione~8 o superiore e JavaFX.
Entrambi sono pacchetti separati.

\prog{tlmgr} ha anche una \GUI{} nativa (mostrata in
figura~\ref{fig:tlmgr-gui}), che è avviata con:
\begin{alltt}
> \Ucom{tlmgr -gui}
\end{alltt}
Tuttavia questa estensione \GUI\ richiede Perl/Tk, il cui modulo non è più
incluso nella distribuzione Perl di \TL\ per Windows.

\subsection{Esempi di invocazioni di \cmdname{tlmgr} dalla riga di comando}

Dopo l'installazione iniziale, potete aggiornare il vostro sistema alle
ultime versioni disponibili con:
\begin{alltt}
> \Ucom{tlmgr update -all}
\end{alltt}
Se questo vi rende ansiosi, provate prima
%If this makes you nervous, first try
\begin{alltt}
> \Ucom{tlmgr update -all -dry-run}
\end{alltt}
o (meno prolisso):
%or (less verbose):
\begin{alltt}
> \Ucom{tlmgr update -list}
\end{alltt}

Questo esempio più complesso aggiunge una collezione, per il motore
Xe\TeX, da una directory locale:

\begin{alltt}
> \Ucom{tlmgr -repository /local/mirror/tlnet install collection-xetex}
\end{alltt}
Genera il seguente risultato (ridotto):
\begin{fverbatim}
install: collection-xetex
install: arabxetex
...
install: xetex
install: xetexconfig
install: xetex.i386-linux
running post install action for xetex
install: xetex-def
...
running mktexlsr
mktexlsr: Updating /usr/local/texlive/2020/texmf-dist/ls-R...
...
running fmtutil-sys --missing
...
Transcript written on xelatex.log.
fmtutil: /usr/local/texlive/2020/texmf-var/web2c/xetex/xelatex.fmt installed.
\end{fverbatim}

Come potete vedere, \prog{tlmgr} installa le dipendenze e si occupa di ogni
azione necessaria dopo l'installazione, incluso l'aggiornamento del
database dei nomi dei file e la generazione (o rigenerazione) dei formati.
Nell'esempio precedente, abbiamo generato nuovi formati per \XeTeX.

Per ottenere la descrizione di un pacchetto (o di una collezione o di uno
schema):
\begin{alltt}
> \Ucom{tlmgr show collection-latexextra}
\end{alltt}
che produce un output simile a questo:
\begin{fverbatim}
package:    collection-latexextra
category:   Collection
shortdesc:  LaTeX supplementary packages
longdesc:   A very large collection of add-on packages for LaTeX.
installed:  Yes
revision:   46963
sizes:      657941k
\end{fverbatim}

Infine, cosa più importante, per ottenere la documentazione completa,
visitate \url{https://tug.org/texlive/tlmgr.html} oppure eseguite:
\begin{alltt}
> \Ucom{tlmgr -help}
\end{alltt}


\section{Note relative a Windows}
\label{sec:windows}


\subsection{Funzionalità specifiche per Windows}
\label{sec:winfeatures}

Sotto Windows, l'installatore fa alcune cose in più:
\begin{description}
\item[Menu e collegamenti.] Viene creato nel menu Start un nuovo sotto
  menu ``\TL'' che contiene le voci per alcuni programmi grafici
  (\prog{tlmgr}, \prog{texdoctk}) e per un po' di documentazione. 
\item[Associazioni dei file.] Se abilitati, \prog{TeXworks}, \prog{Dviout} e
  \prog{PS\_view} diventano ciascuno il programma predefinito per i loro
  rispettivi tipi di file, oppure ottengono una voce nei menu contestuali
  ``Apri con'' per quei tipi di file.
\item[Convertitore da bitmap ad eps.] I vari formati bitmap ricevono una
  voce \cmdname{bitmap2eps} nei loro menu contestuali ``Apri con''.
  Bitmap2eps è un semplice script che permette a \cmdname{sam2p} o
  \cmdname{bmeps} di compiere il vero lavoro.
\item[Impostazione automatica delle variabili di ambiente.] Non sono
  richiesti interventi manuali per la loro configurazione.
\item[Disinstallatore.] Il programma di installazione crea una voce per
  \TL\ in ``Installazione Applicazioni''. La pagina di disinstallazione
  della \GUI\ \TeX\ Live Manager fa riferimento a questo. Nel caso di
  installazione per singolo utente, il programma crea anche una voce di
  disinstallazione nel menu Start.
\item[Protezione dalla scrittura.] Nel caso dell'installazione da parte
  di un amministratore, le directory di \TL\ sono protette dalla
  scrittura, per lo meno se \TL\ è installato su un normale disco non
  removibile formattato con NTFS.
\end{description}

Inoltre, per un approccio diverso, date un'occhiata al comando 
\cmdname{tlaunch}, descritto nella sezione~\ref{sec:sharedinstall}.

\subsection{Software aggiuntivo incluso sotto Windows}

Per essere completa, un'installazione di \TL\ ha bisogno di alcuni
pacchetti di supporto che di solito non sono presenti su una macchina
Windows. \TL{} fornisce questi pezzi mancanti. Questi programmi sono
installati come parte di \TL{} solo su Windows.

\begin{description}
\item[Perl e Ghostscript.] A causa dell'importanza di Perl e Ghostscript,
  \TL{} include una copia ``nascosta'' di questi programmi. I programmi di
  \TL{} che ne hanno bisogno sanno dove trovarli, ma non ne tradiscono la
  presenza tramite variabili d'ambiente o impostazioni sul registro di
  sistema. Non si tratta di installazioni complete e non dovrebbero
  interferire con installazioni nel sistema di Perl o Ghostscript.

\item[dviout.] Viene installato anche \prog{dviout}, un visualizzatore di
  DVI. All'inizio, quando visualizzate i file con \cmdname{dviout},
  questo creerà i font in quanto quelli per la visione su schermo non sono
  installati. Dopo un po', avrete creato la maggior parte dei font che
  usate e raramente vedrete la finestra di creazione dei font. Maggiori
  informazioni possono essere trovate nell'aiuto del programma (altamente
  raccomandato).

\item[\TeX{}works.] \TeX{}works è un editor orientato a \TeX\ con un
  visualizzatore per PDF incorporato.

\item[Strumenti a riga di comando.] Assieme ai soliti programmi di \TL{}
  vengono installate alcune versioni per Windows di tipici programmi Unix
  a riga di comando. Questi programmi includono \cmdname{gzip},
  \cmdname{zip}, \cmdname{unzip} e gli strumenti della suite \cmdname{poppler}
  (\cmdname{pdfinfo}, \cmdname{pdffonts}, \ldots); non è incluso nessun
  visualizzatore di PDF per Windows. Una possibilità è
  Sumatra PDF, disponibile su
  \url{https://www.sumatrapdfreader.org/}.

\item[fc-list, fc-cache, \ldots] Gli strumenti dalla libreria fontconfig
  consentono a \XeTeX{} di gestire i font di sistema sotto Windows. Potete
  usare \prog{fc-list} per trovare i nomi dei font da passare al comando
  esteso di \XeTeX\ \cs{font}. Se è necessario, eseguite prima
  \prog{fc-cache} per aggiornare le informazioni sui font.

\end{description}


\subsection{Il profilo utente è ``home''}
\label{sec:winhome}

La controparte Windows di una directory di home di Unix è la directory
\verb|%USERPROFILE%|. Sotto Windows Vista e successivi è
\verb|C:\Users\<nomeutente>|. Nel file
\filename{texmf.cnf} e in
generale in \KPS{}, \verb|~| verrà interpretato in modo appropriato sia
sotto Windows che sotto Unix.


\subsection{Il registro di configurazione di Windows}
\label{sec:registry}

Windows memorizza quasi tutti i dati di configurazione nel suo registro.
Il registro contiene un insieme di chiavi organizzate gerarchicamente,
con varie gerarchie. Le più importanti per i programmi di installazione
sono \path{HKEY_CURRENT_USER} e \path{HKEY_LOCAL_MACHINE}, abbreviate in
\path{HKCU} e \path{HKLM}. La porzione \path{HKCU} del registro si trova
nella directory di home dell'utente (vedi sezione~\ref{sec:winhome}).
\path{HKLM} si trova di solito in una sotto directory della cartella
Windows.

In alcuni casi, le informazioni di sistema possono essere ottenute dalle
variabili d'ambiente, ma per altre informazioni, come la posizione dei
collegamenti, è necessario consultare il registro. Impostare
permanentemente le variabili d'ambiente richiede ugualmente l'accesso al
registro.


\subsection{Permessi di Windows}
\label{sec:winpermissions}

Nelle ultime versioni di Windows si fa distinzione tra utenti normali ed
amministratori e solo questi ultimi hanno libero accesso all'intero
sistema operativo. Ci siamo sforzati di rendere \TL{} installabile senza i
privilegi di amministratore.

Se il programma di installazione è eseguito con i privilegi di
amministratore, c'è un'opzione per rendere l'installazione disponibile a
tutti gli utenti. Se questa opzione è selezionata, i collegamenti sono
creati per tutti gli utenti e il percorso di ricerca di sistema è modificato.
Altrimenti, i collegamenti e le voci nel menu sono creati solo per
l'utente corrente e viene modificato il percorso di ricerca dell'utente.

Indipendentemente dallo stato di amministratore, la cartella predefinita
di \TL{} proposta dal programma di installazione si trova sempre sotto
\verb|%SystemDrive%|. L'installatore verifica sempre se l'utente corrente
ha i permessi di scrittura in tale cartella.

Possono sorgere dei problemi se l'utente non è un amministratore e \TeX{}
già esiste nel percorso di ricerca. Dato che nel percorso di ricerca
vengono prima le impostazioni di sistema e poi quelle dell'utente, la
nuova \TL{} non avrebbe mai la precedenza. Come precauzione, il programma
di installazione crea un collegamento ad un prompt dei comandi in cui la
nuova directory degli eseguibili di \TL{} è anteposta al percorso di
ricerca locale. La nuova \TL{} sarà sempre accessibile usando questo
prompt. Anche il collegamento a \TeX{}works, se è installato, antepone
\TL{} al percorso di ricerca, quindi dovrebbe essere immune a questo
problema.

Siate consapevoli che anche se avete eseguito l'accesso come amministratori,
dovete chiedere esplicitamente i privilegi di amministrazione. In pratica, non
serve a molto entrare come amministratore. Invece, facendo clic con il tasto
destro sul programma o sul collegamento che volete eseguire, solitamente appare
anche la voce ``Esegui come amministratore''.


\subsection{Incrementare la memoria massima sotto Windows e Cygwin}
\label{sec:cygwin-maxmem}

Gli utenti di Windows e Cygwin (consulta la sezione~\ref{sec:cygwin} per i
dettagli sull'installazione con Cygwin) possono ritrovarsi con la memoria
esaurita durante l'esecuzione di alcuni dei programmi forniti con \TL. Ad
esempio, \prog{asy} potrebbe esaurire la memoria se tentate di allocare un
array di 25.000.000 numeri reali e Lua\TeX\ potrebbe esaurirla se tentate di
elaborare un documento con un gran numero di font grandi.

Per Cygwin, potete incrementare la quantità di memoria disponibile seguendo
le istruzioni nella Guida dell'Utente di Cygwin
(\url{https://cygwin.com/cygwin-ug-net/setup-maxmem.html}).

Per Windows, dovete creare un file, ad esempio \code{moremem.reg}, con le
seguenti quattro righe:

\begin{sverbatim}
Windows Registry Editor Version 5.00

[HKEY_LOCAL_MACHINE\Software\Cygwin]
"heap_chunk_in_mb"=dword:ffffff00
\end{sverbatim}

\noindent e poi eseguire il comando \code{regedit /s moremem.reg} come
amministratore (se volete cambiare la memoria unicamente per l'utente
corrente invece che per l'intero sistema, usate \code{HKEY\_CURRENT\_USER}).


\section{Una guida a Web2C}

\Webc{} è una collezione integrata di programmi legati a \TeX: \TeX{}
stesso, \MF{}, \MP, \BibTeX{}, ecc. È il cuore di \TL{}. Il sito web di
\Webc{}, con il manuale più recente e molto altro, è
\url{https://tug.org/web2c}.

Ecco un po' di storia: l'implementazione originale fu realizzata da Tomas
Rokicki il quale, nel 1987, sviluppò un primo sistema \TeX{}-to-C
cambiando dei file sotto Unix, che erano principalmente un lavoro
originale di Howard Trickey e Pavel Curtis. Tim Morgan divenne il
manutentore del sistema e durante questo periodo il nome cambiò in
Web-to-C\@. Nel 1990, Karl Berry prese in mano il lavoro, con il
contributo di molte persone, e nel 1997 passò il testimone a Olaf Weber,
il quale lo restituì a Karl nel 2006.

Il sistema \Webc{} gira sotto Unix, sotto i sistemi Windows a 32 bit,
sotto MacOSX{} e sotto altri sistemi operativi. Usa i sorgenti originali
di Knuth per \TeX{} e per altri programmi scritti nel sistema di
programmazione letterata \web{} e li traduce in codice C. I programmi
basilari di \TeX{} gestiti in questo modo sono:

\begin{cmddescription}
\item[bibtex]    Gestisce le bibliografie.
\item[dvicopy]   Espande i riferimenti ai font virtuali nei file \dvi{}.
\item[dvitomp]   Converte da \dvi{} a MPX (immagini MetaPost).
\item[dvitype]   Converte i \dvi{} in testo leggibile
\item[gftodvi]   Provini dei font generici.
\item[gftopk]    Converte da font generici ad impacchettati.
\item[gftype]    Converte i font generici in testo leggibile.
\item[mf]        Crea famiglie di caratteri tipografici.
\item[mft]       Formatta i sorgenti \MF{}.
\item[mpost]     Crea diagrammi tecnici.
\item[patgen]    Crea modelli di sillabazione.
\item[pktogf]    Converte da font impacchettati a generici.
\item[pktype]    Converte i font impacchettati in testo leggibile.
\item[pltotf]    Converte le liste di proprietà da testo leggibile in TFM.
\item[pooltype]  Mostra i file delle riserve di \web{}.
\item[tangle]    Converte da \web{} al Pascal.
\item[tex]       Compone tipograficamente i documenti.
\item[tftopl]    Converte le liste di proprietà da TFM in testo leggibile
\item[vftovp]    Converte da font virtuali a liste di proprietà virtuali.
\item[vptovf]    Converte da liste di proprietà virtuali a font virtuali.
\item[weave]     Converte da \web{} a \TeX.
\end{cmddescription}

\noindent Le esatte funzioni e la sintassi di questi programmi sono
descritte nella documentazione dei singoli pacchetti e di \Webc{} stesso.
Tuttavia, conoscere alcuni principi che governano l'intera famiglia di
programmi vi aiuterà a trarre vantaggio dalla vostra installazione \Webc{}.

Tutti i programmi rispettano queste opzioni \GNU{} standard:
\begin{ttdescription}
\item[-{}-help] visualizza una guida sintetica all'utilizzo.
\item[-{}-verbose] visualizza un rapporto dettagliato sull'avanzamento.
\item[-{}-version] visualizza le informazioni sulla versione, poi esce.
\end{ttdescription}

E molti rispettano anche:
\begin{ttdescription}
\item[-{}-verbose] stampa un rapporto dettagliato dell'avanzamento
\end{ttdescription}

Per individuare i file, i programmi \Webc{} usano la libreria di ricerca
dei percorsi \KPS{} (\url{https://tug.org/kpathsea}). Questa libreria usa
una combinazione di variabili d'ambiente e un file di configurazione per
ottimizzare la ricerca nella (enorme) raccolta di file di \TeX{}. \Webc{}
può controllare simultaneamente molte directory, cosa utile per mantenere
la distribuzione standard \TeX{}, le estensioni locali e quelle personali
in directory distinte. Per velocizzare le ricerche dei file, alla radice
di ogni albero di directory c'è un file \file{ls-R} contenente una voce
con il nome ed il percorso relativo per ciascun file all'interno di
quell'albero.


\subsection{Ricerca dei percorsi con Kpathsea}
\label{sec:kpathsea}

Per prima cosa descriviamo il generico meccanismo di ricerca dei percorsi
della libreria \KPS.

Chiameremo \emph{percorso di ricerca} una lista separata da un due punti o
da un punto e virgola di \emph{elementi di percorso}, che sono
fondamentalmente nomi di directory. Un percorso di ricerca può provenire
da (una combinazione di) molte fonti. Per cercare un file \samp{my-file}
attraverso un percorso \samp{.:/dir}, \KPS{} controlla a turno ogni
elemento del percorso: prima \file{./my-file}, poi \file{/dir/my-file},
restituendo la prima corrispondenza (o, eventualmente, tutte).

Al fine di adattarsi al meglio alle convenzioni di tutti i sistemi
operativi, sui sistemi non Unix \KPS{} può usare separatori di nomi di
file diversi dai due punti (\samp{:}) e dalla barra obliqua (slash, \samp{/}).

Per controllare un particolare elemento \var{p} di un percorso, \KPS{} per
prima cosa verifica se può applicare a \var{p} un database precostruito
(consulta ``Database di file'' a pagina~\pageref{sec:filename-database}),
cioè se il database è in una directory che è prefisso di \var{p}. Se è
così, la specifica del percorso è abbinata al contenuto del
database.

Per quanto il più semplice e comune elemento di un percorso sia il nome di
una directory, \KPS{} supporta caratteristiche aggiuntive nei percorsi di
ricerca: valori predefiniti stratificati, nomi di variabili
d'ambiente, valori di file di configurazione, directory personali degli
utenti e ricerca ricorsiva di sotto directory. Dunque, diciamo che \KPS{}
\emph{espande} un elemento di percorso, intendendo che trasforma tutte le
specifiche in semplici nomi di directory. Questa trasformazione è
descritta nelle seguenti sezioni nello stesso ordine in cui avviene.

Notate che se il nome del file cercato è assoluto o esplicitamente relativo,
ossia se comincia con \samp{/}, \samp{./} o \samp{../}, \KPS{} controlla
semplicemente se quel file esiste.

\ifSingleColumn
\else
\begin{figure*}
\verbatiminput{examples/ex5.tex}
\setlength{\abovecaptionskip}{0pt}
  \caption{Un pezzo illustrativo di file di configurazione}
  \label{fig:config-sample}
\end{figure*}
\fi

\subsubsection{Fonti dei percorsi}
\label{sec:path-sources}

Un percorso di ricerca può provenire da molte fonti. Nell'ordine in cui
\KPS{} le usa:

\begin{enumerate}
\item
  Una variabile d'ambiente impostata dall'utente, per esempio
  \envname{TEXINPUTS}\@. Le variabili d'ambiente a cui è stato aggiunto un
  punto ed il nome di un programma hanno la precedenza; per esempio, se
  \samp{latex} è il nome del programma in esecuzione, allora
  \envname{TEXINPUTS.latex} avrà la precedenza su \envname{TEXINPUTS}.
\item
  Il file di configurazione specifico per un programma, per esempio una
  riga \samp{S /a:/b} nel file di configurazione \file{config.ps} di
  \cmdname{dvips}.
\item
  Un file di configurazione \file{texmf.cnf} di \KPS, contenente una riga
  del tipo \samp{TEXINPUTS=/c:/d} (vedi più avanti).
\item Il valore predefinito in essere al momento della compilazione del 
  programma.
\end{enumerate}
\noindent Potete vedere ciascuno di questi valori per un dato percorso di
ricerca usando le opzioni di debug (consultate ``Risoluzione dei
problemi'' a pagina~\pageref{sec:debugging}).

\subsubsection{File di configurazione}

\KPS{} legge i percorsi di ricerca ed altre definizioni nei \emph{file di
configurazione al momento dell'esecuzione} chiamati \file{texmf.cnf}. Il
percorso di ricerca \envname{TEXMFCNF} è usato per cercare questi file,
ma non raccomandiamo di impostare questa (o qualunque altra) variabile
d'ambiente per scavalcare le directory di sistema.

Invece, l'installazione normale dà luogo ad un file
\file{.../2020/texmf.cnf}. Se dovete apportare delle modifiche ai valori
predefiniti (di norma non è necessario), questo è il posto in cui
piazzarle. Il file di configurazione principale è
\file{.../2020/texmf-dist/web2c/texmf.cnf}. Non dovreste modificare
quest'ultimo file, dato che le vostre modifiche andrebbero perdute quando
la versione distribuita viene aggiornata.

Per inciso, se avete semplicemente bisogno di aggiungere una directory
personale ad un particolare percorso di ricerca, impostare una variabile
d'ambiente è un metodo ragionevole:
\begin{verbatim}
  TEXINPUTS=.:/my/macro/dir:
\end{verbatim}
Affinché l'impostazione resti mantenibile e portable negli anni, usate un
\samp{:} come terminatore (\samp{;} su Windows) per inserire i percorsi di
sistema, invece di provare a scriverli tutti esplicitamente (consultate la
sezione~\ref{sec:default-expansion}). Un'altra possibilità è quella di
usare l'alberatura \envname{TREEMFHOME} (cfr.
sezione~\ref{sec:directories}).

\emph{Tutti} i file \file{texmf.cnf} nel percorso di ricerca saranno letti
e le definizioni in quelli che appaiono per primi sostituiranno quelle nei
file che appaiono per ultimi. Ad esempio, con un percorso di ricerca di
\verb|.:$TEXMF|, i valori provenienti da \file{./texmf.cnf} sostituiranno
quelli provenienti da \verb|$TEXMF/texmf.cnf|.

\begin{itemize*}
\item
  I commenti iniziano con il carattere \code{\%}, sia all'inizio di una
  riga che preceduto da uno spazio bianco, e continuano fino alla fine
  della riga.
\item
  Le righe vuote sono ignorate.
\item
  Una \bs{} alla fine di una riga funziona come un carattere di
  continuazione, cioè la linea successiva viene aggiunta di seguito alla
  precedente. Gli spazi bianchi
  all'inizio delle righe di continuazione non sono ignorati.
\item
  Ogni riga rimanente ha la forma:\\
  \hspace*{2em}\texttt{\var{variabile} \textrm{[}.\var{nomeprogramma}\textrm{]}
  \textrm{[}=\textrm{]} \var{valore}}\\[1pt]
  dove il carattere \samp{=} e gli spazi che lo circondano sono opzionali
  (ma se \var{valore} inizia con \samp{.}, è più semplice usare il \samp{=}
  per evitare che il punto sia interpretato come qualificatore del nome
  del programma).
\item
  Il nome \ttvar{variabile} può contenere un qualunque carattere diverso
  dallo spazio, da \samp{=} e da \samp{.}, ma usare solo i caratteri
  \samp{A-Za-z\_} è più sicuro.
\item
  Se la parte \samp{.\var{nomeprogramma}} è presente, la definizione si
  applica solo se il programma che è in esecuzione ha il nome
  \texttt{\var{nomeprogramma}} oppure \texttt{\var{nomeprogramma}.exe}.
  Questo consente, ad esempio, alle diverse varianti di \TeX{} di avere
  diversi percorsi di ricerca.
\item Trattato come stringa, \var{valore} può contentere qualunque
  carattere. Tuttavia in pratica la maggior parte dei valori in
  \file{texmf.cnf} sono correlati all'espansione dei percorsi e dato che
  diversi caratteri speciali sono usati nell'espansione (cfr.
  sezione~\ref{sec:cnf-special-chars}), come parentesi e virgole, questi
  non possono essere usati nei nomi delle directory.

  Un \samp{;}\ in \var{valore} è tradotto in \samp{:} se state usando
  Unix, in modo da avere un singolo \file{texmf.cnf} che può supportare
  sia i sistemi Unix che Windows. Questa traduzione avviene con qualsiasi
  valore, non solo con i percorsi di ricerca, ma fortunatamente in pratica
  \samp{;} non è necessario in altri valori.

  La forma \code{\$\var{variabile}.\var{programma}} non è disponibile sul
  lato destro dell'espressione; al suo posto dovete usare una variabile
  aggiuntiva.

\item
  Tutte le definizioni vengono lette prima che qualunque cosa sia espansa,
  quindi si può fare riferimento alle variabili prima che siano definite.
\end{itemize*}
Un frammento di file di configurazione che mostra la maggior parte di
questi punti è
\ifSingleColumn
riportato sotto:

\verbatiminput{examples/ex5.tex}
\else
mostrato nella figura~\ref{fig:config-sample}.
\fi

\subsubsection{Espansione dei percorsi}
\label{sec:path-expansion}

\KPS{} riconosce nei percorsi di ricerca alcuni caratteri e costrutti
speciali, simili a quelli disponibili nelle shell di Unix. Come esempio
generale, il percorso \verb+~$USER/{foo,bar}//baz+ viene espanso
in tutte le directory sotto \file{foo} e \file{bar}, nella home dell'utente
\texttt{\$USER}, che contengono a loro volta una directory o un file di
nome \file{baz}. Queste espansioni sono spiegate nelle prossime sezioni.
%$
\subsubsection{Espansione predefinita}
\label{sec:default-expansion}

Se il percorso di ricerca a più alta priorità (consultate ``Fonti dei
percorsi'' a pagina~\pageref{sec:path-sources}) contiene un carattere \emph{due
punti in più} (all'inizio, alla fine oppure una coppia di due
punti), \KPS{} inserisce in quel punto il percorso di ricerca con la
seconda più alta priorità che sia stato definito. Se questo percorso
inserito ha a sua vuolta un due punti aggiuntivo, accade lo stesso con il
successivo in ordine di priorità. Per esempio, data l'impostazione di una
variabile d'ambiente

\begin{alltt}
> \Ucom{setenv TEXINPUTS /home/karl:}
\end{alltt}
e un valore di \code{TEXINPUTS} proveniente da \file{texmf.cnf} di

\begin{alltt}
  .:\$TEXMF//tex
\end{alltt}
allora il valore finale usato nella ricerca sarà:

\begin{alltt}
  /home/karl:.:\$TEXMF//tex
\end{alltt}

Dato che sarebbe inutile inserire lo stesso valore predefinito in più di
un posto, \KPS{} cambia solo uno dei \samp{:}\ di troppo e lascia gli
altri al loro posto. Per prima cosa cerca un \samp{:} all'inizio, quindi
cerca un \samp{:} alla fine, per ultimo cerca un \samp{:} doppio.

\subsubsection{Espansione delle parentesi graffe}
\label{sec:brace-expansion}

Una caratteristica utile è l'espansione delle parentesi graffe, che
significa che, per esempio, \verb+v{a,b}w+ viene espanso in
\verb+vaw:vbw+. L'annidamento delle parentesi è permesso. Tale
caratteristica è usata per implementare gerarchie \TeX{} multiple,
assegnando una lista di nomi tra graffe a \code{\$TEXMF}.
Nel \file{texmf.cnf} distribuito, viene fatta una definizione
(semplificata per questo esempio) di questo tipo:
\begin{verbatim}
  TEXMF = {$TEXMFVAR,$TEXMFHOME,!!$TEXMFLOCAL,!!$TEXMFDIST}
\end{verbatim}
Quindi la usiamo per definire, ad esempio, il percorso di input per \TeX:
\begin{verbatim}
  TEXINPUTS = .;$TEXMF/tex//
\end{verbatim}
%$
che vuol dire che, dopo aver guardato nella directory attuale, saranno
esaminati i percorsi \code{\$TEXMFVAR/tex}, \code{\$TEXMFHOME/tex},
\code{\$TEXMFLOCAL/tex} e \code{\$TEXMFDIST/tex} (gli ultimi due usando i
database \file{ls-R}).

\subsubsection{Espansione delle sotto directory}
\label{sec:subdirectory-expansion}

In un elemento di percorso, due o più barre (slash) consecutive alla fine
di una directory \var{d} sono sostituite da tutte le sotto directory di
\var{d\/}: prima quelle che si trovano direttamente sotto \var{d}, quindi
quelle all'interno delle prime e così via. Ad ogni livello, l'ordine in
cui le directory sono cercate \emph{non è specificato}.

Se specificate il nome di un file dopo il \samp{//}, saranno incluse solo
le sotto directory che lo contengono. Ad esempio, \samp{/a//b} si
espande nelle directory \file{/a/1/b}, \file{/a/2/b}, \file{/a/1/1/b} e
così via, ma non in \file{/a/b/c} o \file{/a/1}.

È possibile usare più volte il costrutto \samp{//} in un percorso, ma ogni
\samp{//} all'inizio di un percorso viene ignorato.

\subsubsection{Riepilogo dei caratteri speciali in \file{texmf.cnf}}
\label{sec:cnf-special-chars}

Il seguente elenco riassume i caratteri speciali e i costrutti nei file di
configurazione di \KPS{}.

% need a wider space for the item labels here.
\newcommand{\CODE}[1]{\makebox[3em][l]{\code{#1}}}
\begin{ttdescription}
\item[\CODE{:}] Separatore nella specifica di un percorso; all'inizio o
  alla fine di un percorso, o doppio nel mezzo, sostituisce l'espansione
  di percorso predefinita.\par
\item[\CODE{;}] Separatore nei sistemi non Unix (si comporta come
  \code{:}).
\item[\CODE{\$}] Espansione di una variabile.
\item[\CODE{\string~}] Rappresenta la directory di home di un utente.
\item[\CODE{\char`\{...\char`\}}] Espansione di parentesi graffe.
\item[\CODE{//}] Espansione di sotto directory (può trovarsi ovunque in un
  percorso tranne che al suo inizio).
\item[\CODE{\%}] Inizio di un commento.
\item[\CODE{\bs}] Alla fine di una linea, carattere di continuazione per
  permettere voci di più linee.
\item[\CODE{!!}] Cerca \emph{solo} nei database per individuare un file,
    \emph{non} cerca sul disco.
\end{ttdescription}

Quando esattamente un carattere sarà considerato speciale o si comporterà
come sé stesso dipende dal contesto in cui è usato. Le regole sono
intrinseche nei livelli multipli di interpretazione della configurazione
(parsing, espansione, ricerca, \ldots)\ e quindi, sfortunatamente, non
possono essere espresse in modo conciso. Non esiste un meccanismo di
\emph{escape} generale; in particolare, \samp{\bs} non è un ``carattere di
escape'' nei file \file{texmf.cnf}.

Quando è il momento di scegliere i nomi delle directory per
l'istallazione, è più sicuro saltarli completamente.

\subsection{Database di nomi di file}
\label{sec:filename-database}

\KPS{} si sforza di minimizzare gli accessi al disco per le ricerche.
Nonostante ciò, nella \TL\ standard o in qualunque installazione con un
numero sufficiente di directory, esplorare qualunque possibile directory
per un dato file richiederebbe un tempo eccessivamente lungo. Per questo
\KPS{} può usare un file di ``database'' testuale costruito esternamente,
chiamato \file{ls-R}, che mappa file a directory, evitando così la
necessità di esplorare esaustivamente il disco.

Un secondo file di database, \file{aliases}, permette di dare nomi
aggiuntivi ai file elencati in \file{ls-R}.

\subsubsection{Il database di nomi}
\label{sec:ls-R}

Come spiegato in precedenza, il nome del database principale di nomi di
file deve essere \file{ls-R}. Potete metterne uno alla radice di ogni
ramo \TeX{} della vostra installazione nel quale volete che si effettuino
le ricerche (\code{\$TEXMF} di base). \KPS{} cerca i file \file{ls-R} nel
percorso \code{TEXMFDBS}.

Il metodo raccomandato per creare e mantenere \samp{ls-R} è quello di
eseguire lo script \code{mktexlsr} incluso nella distribuzione. Esso è
invocato dai vari script \samp{mktex}\dots. In teoria, questo script esegue
semplicemente il comando
\begin{alltt}
cd \var{/la/vostra/radice/texmf} && \path|\|ls -1LAR ./ >ls-R
\end{alltt}
assumendo che il comando \code{ls} del vostro sistema produca il giusto
formato di output (\code{ls} del progetto \GNU lo fa). Per garantire che
l'archivio sia sempre aggiornato, la cosa più semplice è rigenerarlo
regolarmente tramite \code{cron}, così che sia automaticamente ricostruito
quando i file installati cambiano, ad esempio dopo aver installato o
aggiornato un pacchetto \LaTeX.

Se un file non è trovato nel database, per default \KPS{} va avanti e
cerca sul disco. Se un particolare elemento di percorso comincia con
\samp{!!}, però, la ricerca per quell'elemento sarà effettuata \emph{solo}
nel database, mai sul disco.


\subsubsection{kpsewhich: ricerca indipendente di percorsi}
\label{sec:invoking-kpsewhich}

Il programma \texttt{kpsewhich} compie la ricerca di percorsi
indipendentemente da qualunque particolare applicazione. Esso può essere utile
come una sorta di programma \code{find} per individuare i file nelle
gerarchie \TeX{} (è ampiamente usato negli script \samp{mktex}\dots\
distribuiti).

\begin{alltt}
> \Ucom{kpsewhich \var{opzioni}\dots{} \var{nomefile}\dots{}}
\end{alltt}
Le opzioni specificate in \ttvar{opzioni} cominciano con \samp{-} oppure
con \samp{-{}-} ed è accettata qualunque abbreviazione non ambigua.

\KPS{} cerca ogni argomento sulla riga di comando che non sia un'opzione
come se fosse il nome di un file e restituisce il primo trovato. Non ci
sono opzioni per restituire tutti i file con un particolare nome (per fare
questo, potete eseguire il programma Unix \samp{find}).

Le opzioni più comuni sono descritte in seguito.

\begin{ttdescription}
\item[\texttt{-{}-dpi=\var{num}}]\mbox{}
  Imposta la risoluzione a \ttvar{num}; questa opzione ha effetto solo
  nelle ricerche dei file \samp{gf} e \samp{pk}. \samp{-D} è un sinonimo,
  per compatibilità con \cmdname{dvips}. Il valore preimpostato è 600.

\item[\texttt{-{}-format=\var{nome}}]\mbox{}\\
  Imposta il formato da cercare a \ttvar{nome}. Di base, il formato è
  ipotizzato a partire dal nome del file. Per i formati che non hanno
  associato un suffisso non ambiguo, come i file di supporto di \MP{} e i
  file di configurazione di \cmdname{dvips}, dovete specificare il nome
  così come è conosciuto da \KPS{}, come \texttt{tex} o \texttt{enc
  files}. Eseguite \texttt{kpsewhich -{}-help-formats} per un elenco.

\item[\texttt{-{}-mode=\var{stringa}}]\mbox{}\\
  Imposta il nome della modalità a \ttvar{stringa}; questa opzione ha
  effetto solo sulle ricerche dei file \samp{gf} e \samp{pk}. Non esiste
  un valore preimpostato: sarà trovata qualunque tipo di modalità.
\item[\texttt{-{}-must-exist}]\mbox{}\\
  Fa tutto ciò che è possibile per trovare il file, inclusa nello
  specifico la ricerca sul disco. Di base, per questioni di efficienza, è
  controllato solo l'archivio \file{ls-R}.
\item[\texttt{-{}-path=\var{stringa}}]\mbox{}\\
  Cerca nel percorso di ricerca \ttvar{stringa} (separata da due punti,
  come al solito), invece di ricavarlo dal nome del file. \samp{//} e
  tutte le solite espansioni sono supportate. Le opzioni \samp{-{}-path} e
  \samp{-{}-format} si escludono a vicenda.
\item[\texttt{-{}-progname=\var{nome}}]\mbox{}\\
  Imposta il nome del programma a \texttt{\var{nome}}. Questa opzione può
  modificare i percorsi di ricerca per mezzo della funzionalità
  \texttt{.\var{nomeprogramma}}. Il valore predefinito è
  \cmdname{kpsewhich}.
\item[\texttt{-{}-show-path=\var{nome}}]\mbox{}\\
  Mostra il percorso usato per la ricerca dei file il cui tipo sia
  \texttt{\var{nome}}. Può essere usata sia un'estensione (\code{.pk},
  \code{.vf}, ecc.) che un nome, proprio come per l'opzione
  \samp{-{}-format}.
\item[\texttt{-{}-debug=\var{num}}]\mbox{}\\
  Imposta le opzioni per la ricerca degli errori a \texttt{\var{num}}.
\end{ttdescription}


\subsubsection{Esempi d'uso}
\label{sec:examples-of-use}

Diamo uno sguardo a \KPS{} in azione. Ecco una ricerca semplice:

\begin{alltt}
> \Ucom{kpsewhich article.cls}
   /usr/local/texmf-dist/tex/latex/base/article.cls
\end{alltt}
Stiamo cercando il file \file{article.cls}. Dato che il suffisso
\samp{.cls} non è ambiguo, non abbiamo bisogno di specificare che vogliamo
cercare un file di tipo \optname{tex} (directory dei file sorgente \TeX). Lo 
troviamo
nella sotto directory \file{tex/latex/base} sotto la directory di \TL\
\samp{texmf-dist}. In modo simile, tutti i file seguenti sono trovati
senza problemi grazie alla non ambiguità dei loro suffissi.
\begin{alltt}
> \Ucom{kpsewhich array.sty}
   /usr/local/texmf-dist/tex/latex/tools/array.sty
> \Ucom{kpsewhich latin1.def}
   /usr/local/texmf-dist/tex/latex/base/latin1.def
> \Ucom{kpsewhich size10.clo}
   /usr/local/texmf-dist/tex/latex/base/size10.clo
> \Ucom{kpsewhich small2e.tex}
   /usr/local/texmf-dist/tex/latex/base/small2e.tex
> \Ucom{kpsewhich tugboat.bib}
   /usr/local/texmf-dist/bibtex/bib/beebe/tugboat.bib
\end{alltt}

A proposito, quest'ultimo è un database bibliografico in formato
\BibTeX{} per gli articoli di \textsl{TUGboat}.

\begin{alltt}
> \Ucom{kpsewhich cmr10.pk}
\end{alltt}
I file di glifi dei font bitmat di tipo \file{.pk} sono usati dai
programmi di visualizzazione come \cmdname{dvips} e \cmdname{xdvi}. Non
viene restituito nulla in questo caso dato che in \TL\ non ci sono file
\samp{.pk} pregenerati per il carattere tipografico Computer Modern \Dash
come predefinite sono usate le varianti Type~1.
\begin{alltt}
> \Ucom{kpsewhich wsuipa10.pk}
\ifSingleColumn   /usr/local/texmf-var/fonts/pk/ljfour/public/wsuipa/wsuipa10.600pk
\else /usr/local/texmf-var/fonts/pk/ljfour/public/
...                         wsuipa/wsuipa10.600pk
\fi\end{alltt}
Per questi font (un alfabeto fonetico creato dall'Università di
Washington) dobbiamo generare i file \samp{.pk} e dato che la modalità
predefinita di \MF{} nella nostra installazione è \texttt{ljfour} con una
risoluzione di base di 600\dpi{} (dots per inch, punti per pollice), viene
restituito questo valore.
\begin{alltt}
> \Ucom{kpsewhich -dpi=300 wsuipa10.pk}
\end{alltt}
In questo caso, quando specifichiamo di essere interessati ad una
risoluzione di 300\dpi{} (\texttt{-dpi=300}) osserviamo che questo font
non è disponibile nel sistema. Un programma come \cmdname{dvips} o
\cmdname{xdvi} andrebbe avanti e genererebbe i file \texttt{.pk} richiesti
usando lo script \cmdname{mktexpk}.

Adesso spostiamo la nostra attenzione sui file di intestazione e di
configurazione di \cmdname{dvips}. Cercheremo innanzitutto un file tra
quelli usati comunemente, il prologo generale \file{tex.pro} per il
supporto a \TeX, prima di spostare l'attenzione sul generico file di
configurazione (\file{config.ps}) e la mappa dei font \PS{}
\file{psfonts.map} \Dash\ a partire dal 2004, i file di mappatura e di
codifica hanno i propri percorsi di ricerca e una nuova posizione negli
alberi \dirname{texmf}. Dato che il suffisso \samp{.ps} è ambiguo,
dobbiamo specificare esplicitamente quale tipo stiamo considerando
(\optname{dvips config}) per il file \texttt{config.ps}.
\begin{alltt}
> \Ucom{kpsewhich tex.pro}
   /usr/local/texmf/dvips/base/tex.pro
> \Ucom{kpsewhich --format="dvips config" config.ps}
   /usr/local/texmf/dvips/config/config.ps
> \Ucom{kpsewhich psfonts.map}
   /usr/local/texmf/fonts/map/dvips/updmap/psfonts.map
\end{alltt}

Diamo ora uno sguardo ravvicinato ai file di supporto per il carattere
\PS{} URW Times. Il prefisso per questi file nello schema standard dei nomi
dei font è \samp{utm}. Il primo file che cerchiamo è quello di
configurazione, che contiene il nome del file di mappatura:
\begin{alltt}
> \Ucom{kpsewhich --format="dvips config" config.utm}
   /usr/local/texmf-dist/dvips/psnfss/config.utm
\end{alltt}
Il contenuto di questo file è
\begin{alltt}
  p +utm.map
\end{alltt}
che punta al file \file{utm.map}, che sarà il prossimo che cercheremo.
\begin{alltt}
> \Ucom{kpsewhich utm.map}
   /usr/local/texmf-dist/fonts/map/dvips/times/utm.map
\end{alltt}
Questo file di mappatura definisce i nomi dei file dei font \PS{} Type~1
nella collezione URW. Il suo contenuto è simile al seguente (mostriamo
solo una parte delle righe):
\begin{alltt}
utmb8r  NimbusRomNo9L-Medi    ... <utmb8a.pfb
utmbi8r NimbusRomNo9L-MediItal... <utmbi8a.pfb
utmr8r  NimbusRomNo9L-Regu    ... <utmr8a.pfb
utmri8r NimbusRomNo9L-ReguItal... <utmri8a.pfb
utmbo8r NimbusRomNo9L-Medi    ... <utmb8a.pfb
utmro8r NimbusRomNo9L-Regu    ... <utmr8a.pfb
\end{alltt}
Prendiamo, ad esempio, il file per il Times Roman \file{utmr8a.pfb} e
cerchiamo la sua posizione nell'albero delle directory \file{texmf} con
una ricerca dei file di font Type~1:
\begin{alltt}
> \Ucom{kpsewhich utmr8a.pfb}
\ifSingleColumn   /usr/local/texmf-dist/fonts/type1/urw/times/utmr8a.pfb
\else   /usr/local/texmf-dist/fonts/type1/
... urw/utm/utmr8a.pfb
\fi\end{alltt}

Dovrebbe essere chiaro da questi esempi come potete individuare facilmente
dove si trovi un dato file. Tutto ciò è particolarmente importante se
sospettate che in qualche modo venga prelevata la versione sbagliata di un
file, in quanto \cmdname{kpsewhich} mostrerà il primo file incontrato.

\subsubsection{Risoluzione dei problemi}
\label{sec:debugging}

A volte è necessario investigare sul come un programma determina i
riferimenti ad un file. Per rendere pratico ciò, \KPS{} offre vari livelli
di messaggi diagnostici:

\begin{ttdescription}
\item[\texttt{\ 1}] Chiamate a \texttt{stat} (accessi al disco). Quando si
  compie una ricerca avendo a disposizione un database \file{ls-R}
  aggiornato, questo livello non dovrebbe mostrare quasi nessun messaggio.
\item[\texttt{\ 2}] Riferimenti alle tabelle dei dati (come i database
  \file{ls-R}, i file di mappatura, quelli di configurazione).
\item[\texttt{\ 4}] Operazioni di apertura e chiusura dei file.
\item[\texttt{\ 8}] Informazioni generali sui percorsi per i tipi di file
  cercati da \KPS. Utile per scoprire dove è stato definito un particolare
  percorso per un file.
\item[\texttt{16}] Elenco delle directory per ogni elemento di un percorso
  (rilevante solo per le ricerche su disco).
\item[\texttt{32}] Ricerche di file.
\item[\texttt{64}] Valori delle variabili.
\end{ttdescription}
Il valore \texttt{-1} attiverà tutte le opzioni precedenti; in
pratica, questo valore è di solito il più comodo.

Analogamente, con il programma \cmdname{dvips}, impostando una
combinazione di opzioni di diagnostica, è possibile seguire in dettaglio
le posizioni da cui i file sono prelevati. In alternativa, quando un file
non viene trovato, la traccia dei messaggi mostra in quali directory il
programma ha cercato il file, così che si possa ottenere un'indicazione
sull'origine del problema.

In termini generali, dato che la maggior parte dei programmi invoca la
libreria \KPS{} internamente, è possibile selezionare il livello di
diagnostica usando la variabile d'ambiente \envname{KPATHSEA\_DEBUG} ed
impostandola ad una combinazione dei valori descritti nell'elenco di cui
sopra.

(Nota per gli utenti Windows: in questo sistema non è semplice redirigere
tutti i messaggi verso un file. Per scopi di diagnostica potete impostare
temporaneamente \texttt{SET KPATHSEA\_DEBUG\_OUTPUT=err.log}).

Consideriamo, come esempio, un piccolo file sorgente di \LaTeX,
\file{hello-world.tex}, che contiene il seguente testo.
\begin{verbatim}
  \documentclass{article}
  \begin{document}
  Ciao Mondo!
  \end{document}
\end{verbatim}
Questo piccolo file usa solo il font \file{cmr10}, quindi vediamo come
\cmdname{dvips} prepara il file \PS{} (vogliamo usare la versione Type~1
dei font Computer Modern, da qui l'opzione \texttt{-Pcms}).
\begin{alltt}
> \Ucom{dvips -d4100 hello-world -Pcms -o}
\end{alltt}
In questo caso abbiamo combinato la classe di diagnostica 4 di
\cmdname{dvips} (percorsi dei font) con l'espansione degli elementi di
percorso di \KPS{} (vedi il manuale di riferimento di \cmdname{dvips}).
Il risultato (leggermente riorganizzato) appare in
figura~\ref{fig:dvipsdbga}.
\begin{figure*}[tp]
\centering
\input{examples/ex6a.tex}
\caption{Ricerca dei file di configurazione}\label{fig:dvipsdbga}
\end{figure*}

\cmdname{dvips} inizia individuando i propri file di lavoro. Per primo
viene trovato \file{texmf.cnf}, che fornisce le definizioni dei percorsi
di ricerca per gli altri file, quindi il database di file \file{ls-R} (per
ottimizzare la ricerca) e il file \file{aliases}, che rende possibile
dichiarare diversi nomi (ad esempio, una versione breve in formato
DOS 8.3 e una più lunga e più naturale) per lo stesso file. Quindi
\cmdname{dvips} prosegue nel cercare il file di configurazione generico
\file{config.ps} prima di guardare al file per le personalizzazioni 
\file{.dvipsrc}
(che in questo caso \emph{non viene trovato}). Infine,
\cmdname{dvips} individua il file di configurazione \file{config.cms} per
il font \PS{} Computer Modern (questo passo è stato attivato con l'opzione
\texttt{-Pcms} data al comando \cmdname{dvips}). Questo file contiene
la lista delle mappature che definiscono la relazione tra il modo di
chiamare i font in \TeX{}, in \PS{} e sul disco.
\begin{alltt}
> \Ucom{more /usr/local/texmf/dvips/cms/config.cms}
   p +ams.map
   p +cms.map
   p +cmbkm.map
   p +amsbkm.map
\end{alltt}
\cmdname{dvips}, quindi, procede nel trovare tutti questi file, più il
generico file di mappatura \file{psfonts.map}, che è caricato sempre
(contiene le dichiarazioni per i font \PS{} usati comunemente; consultate
la parte finale della sezione~\ref{sec:examples-of-use} per ulteriori
dettagli sulla gestione delle mappature \PS).

A questo punto, \cmdname{dvips} si presenta all'utente:
\begin{alltt}
This is dvips(k) 5.92b Copyright 2002 Radical Eye Software (www.radicaleye.com)
\end{alltt}
\ifSingleColumn
Quindi prosegue nel cercare il file di prologo \file{texc.pro}:
\begin{alltt}\small
kdebug:start search(file=texc.pro, must\_exist=0, find\_all=0,
  path=.:~/tex/dvips//:!!/usr/local/texmf/dvips//:
       ~/tex/fonts/type1//:!!/usr/local/texmf/fonts/type1//).
kdebug:search(texc.pro) => /usr/local/texmf/dvips/base/texc.pro
\end{alltt}
\else
Quindi prosegue nel cercare il file di prologo \file{texc.pro} (vedi la
figura~\ref{fig:dvipsdbgb}).
\fi

Dopo aver trovato il file in questione, \cmdname{dvips} mostra la data e
l'ora, ci informa che genererà il file \file{hello-world.ps}, poi
che avrà bisogno del file del font \file{cmr10} e che quest'ultimo è
dichiarato ``residente'' (non sono necessarie bitmap):
\begin{alltt}\small
TeX output 1998.02.26:1204' -> hello-world.ps
Defining font () cmr10 at 10.0pt
Font cmr10 <CMR10> is resident.
\end{alltt}
Adesso la ricerca prosegue con il file \file{cmr10.tfm}, che viene
trovato, quindi viene fatto riferimento ad alcuni ulteriori file di
prologo (non mostrati) e, infine, l'istanza Type~1 del font,
\file{cmr10.pfb}, è individuata ed inclusa del file in uscita (vedi
l'ultima riga).
\begin{alltt}\small
kdebug:start search(file=cmr10.tfm, must\_exist=1, find\_all=0,
  path=.:~/tex/fonts/tfm//:!!/usr/local/texmf/fonts/tfm//:
       /var/tex/fonts/tfm//).
kdebug:search(cmr10.tfm) => /usr/local/texmf/fonts/tfm/public/cm/cmr10.tfm
kdebug:start search(file=texps.pro, must\_exist=0, find\_all=0,
   ...
<texps.pro>
kdebug:start search(file=cmr10.pfb, must\_exist=0, find\_all=0,
  path=.:~/tex/dvips//:!!/usr/local/texmf/dvips//:
       ~/tex/fonts/type1//:!!/usr/local/texmf/fonts/type1//).
kdebug:search(cmr10.pfb) => /usr/local/texmf/fonts/type1/public/cm/cmr10.pfb
<cmr10.pfb>[1]
\end{alltt}

\subsection{Opzioni di esecuzione}

Un'altra funzionalità utile di \Webc{} è la sua possibilità di controllare
un certo numero di parametri relativi alla memoria (nello specifico la
dimensione degli array) tramite il file \file{texmf.cnf} letto da \KPS{}
durante l'esecuzione. Le impostazioni della memoria possono essere trovate
nella Parte 3 di quel file nella distribuzione \TL. Le più importanti
sono:

\begin{ttdescription}
\item[\texttt{main\_memory}]
  La quantità complessiva di memoria disponibile per \TeX, \MF{} e \MP.
  Dovete creare un nuovo file di formato per ogni impostazione differente.
  Per esempio, potreste generare una versione ``enorme'' di \TeX{} e
  chiamare il file di formato \texttt{hugetex.fmt}. Usando il modo normale
  di indicare il nome del programma usato da \KPS{}, lo specifico valore
  della variabile \texttt{main\_memory} sarà letto da \file{texmf.cnf}.
\item[\texttt{extra\_mem\_bot}]
  Spazio aggiuntivo per le strutture dati ``grandi'' di \TeX: scatole,
  colle, interruzioni, ecc. Utile specialmente se usate \PiCTeX.
\item[\texttt{font\_mem\_size}]
  Numero di registri per le informazioni sui font disponibili in \TeX.
  Questo valore è più o meno pari alla dimensione totale di tutti i file
  TFM che vengono letti.
\item[\texttt{hash\_extra}]
  Spazio aggiuntivo per la tabella con i nomi delle sequenze di
  controllo. Nella tabella principale possono essere memorizzate solo
  approssimativamente 10.000 sequenze di controllo; se lavorate su un
  libro di grandi dimensioni con numerosi riferimenti incrociati, questo
  valore potrebbe non essere sufficiente. Il valore predefinito per
  \texttt{hash\_extra} è \texttt{50000}.
\end{ttdescription}

\noindent Questa funzionalità non è un sostituto per una vera
allocazione dinamica di memoria e array, ma dato che questi sono
estremamente difficili da implementare negli attuali sorgenti di \TeX,
questi parametri di esecuzione forniscono un pratico compromesso per
offrire  un minimo di flessibilità.

\htmlanchor{texmfdotdir}
\subsection{\texttt{\$TEXMFDOTDIR}}
\label{sec:texmfdotdir}

Nei vari posti qui sopra, abbiamo dato diversi percorsi di ricerca che
iniziano per \code{.} (per cercarre prima nella directory corrente), come
in
\begin{alltt}\small
TEXINPUTS=.;$TEXMF/tex//
\end{alltt}

Questa è una semplificazione. Il file \code{texmf.cnf} che distribuiamo in
\TL{} usa \filename{$TEXMFDOTDIR} al posto del semplice \samp{.}, come in:
\begin{alltt}\small
TEXINPUTS=$TEXMFDOTDIR;$TEXMF/tex//
\end{alltt}
(nel file distribuito, anche il secondo elemento del percorso è
leggermente più complesso di \filename{$TEXMF/tex//}, ma è un dettaglio;
qui vogliamo parlare di \filename{$TEXMFDOTDIR}).

Il motivo per usare la variabile \filename{$TEXMFDOTDIR} nelle definizioni
dei percorsi invece del semplice \samp{.} è soltanto per poterla
ridefinire. Ad esempio, un documento commplesso potrebbe avere molti file
sorgente organizzati in molte sotto directory. Per gestire questo, potete
impostare \filename{TEXMFDOTDIR} a \filename{.//} (ad esempio
nell'ambiente in cui compilate il documento) e saranno esplorate tutte
(attenzione: non usate \filename{.//} automaticamente; di solito è
fortemente indesiderato e potenzialmente insicuro cercare tra tutte le
sotto directory per un documento qualunque).

Come altro esempio, potreste non voler cercare per nulla nella directory
attuale, per esempio se avete organizzato i vostri file per essere
individuati tramite percorsi espliciti. Potete impostare
\filename{$TEXMFDOTDIR} a, per dire, \filename{/nonesuch} o qualunque
altra directory non esistente per questo.

Il valore predefinito di \filename{$TEXMFDOTDIR} è semplicemente \samp{.},
come impostato nel nostro \filename{texmf.cnf}.

\htmlanchor{ack}
\section{Ringraziamenti}

\TL{} è il risultato dello sforzo congiunto di praticamente tutti i gruppi
utenti \TeX.  Questa edizione di \TL{} è stata supervisionata da Karl
Berry. Gli altri contributori principali, passati e presenti, sono
elencati qui sotto.

\begin{itemize*}

\item I gruppi utenti \TeX{} Inglese, Tedesco, Olandese e Polacco
(TUG, DANTE e.V., NTG e GUST,
rispettivamente), che forniscono la necessaria infrastruttura tecnica ed
amministrativa. Unitevi al gruppo utenti \TeX\ più vicino a voi
(consultate la pagina \url{https://tug.org/usergroups.html})!

\item Il team di CTAN (\url{https://ctan.org}), che distribuisce le
immagini di \TL{} e fornisce l'infrastruttura comune per gli aggiornamenti
dei pacchetti, dalla quale dipende \TL.

\item Nelson Beebe per aver reso disponibili molte piattaforme agli
sviluppatori di \TL{}, per i suoi esaustivi collaudi e per gli
impareggiabili sforzi bibliografici.

\item John Bowman per aver fatto molti cambiamenti al suo programma
di grafica avanzata Asymptote affinché funzionasse in \TL.

\item Peter Breitenlohner ed il team di \eTeX\ per le stabili
fondamenta del \TeX{} del futuro e Peter in particolare per gli anni
di aiuto stellare con gli autotool di \GNU e per aver mantenuto i sorgenti 
aggiornati. Peter è venuto a mancare
nell'ottobre 2015 e noi dedichiamo il lavoro che va avanti alla sua
memoria.

\item Jin-Hwan Cho e tutti i membri del team di DVIPDFM$x$ per il loro
eccellente driver e la velocità di risposta ai problemi di configurazione.

\item Thomas Esser, senza il cui meraviglioso pacchetto \teTeX{} \TL{} non
sarebbe mai esistito.

\item Michel Goossens, che è stato coautore della documentazione
originale.

\item Eitan Gurari, il cui \TeX4ht è stato usato per creare la versione
\HTML{} di questa documentazione e che ha lavorato instancabilmente per
migliorarlo con brevi preavvisi ogni anno. Eitan è scomparso
prematuramente nel giugno 2009 e dedichiamo questa documentazione alla sua
memoria.

\item Hans Hagen per la grande quantità di collaudi e per aver permesso al
suo pacchetto \ConTeXt{} (\url{https://pragma-ade.com}) di lavorare
nell'infrastruttura di \TL\ e per guidare continuamente lo sviluppo di
\TeX.

\item \Thanh, Martin Schr\"oder e il team di pdf\TeX{}
(\url{http://pdftex.prg}) per i continui miglioramenti delle capacità di
\TeX.

\item Hartmut Henkel per i significativi contributi allo sviluppo di
pdf\TeX, Lua\TeX{} e molto ancora.

\item Shunshaku Hirata, per i contributi originalissimi e per continuare
su DVIPDFM$x$.

\item Taco Hoekwater per gli importanti forzi nel rinnovato sviluppo di
MetaPost e (Lua)\TeX{} (\url{http://luatex.org}) stesso, per aver
incorporato \ConTeXt{} in \TL, per aver dato a Kpathsea le funzionalità
multi-thread e per molto altro ancora.

\item Khaled Hosny, per il considerevole lavoro su \XeTeX, DVIPDFM$x$, e
per gli sforzi con i font, arabi ed altri.

\item Pawe{\l} Jackowski per l'installatore Windows \cmdname{tlpm} e
Tomasz {\L}uczak per \cmdname{tlpmgui}, usati nelle edizioni passate.

\item Akira Kakuto per aver fornito gli eseguibili per Windows a partire
dalla sua distribuzione per il \TeX{} giapponese W32TEX 
(\url{http://w32tex.org}) e per molti altri
contributi allo sviluppo.

\item Jonathan Kew per aver sviluppato il notevole motore \XeTeX{} e per
aver investito tempo e fatica per integrarlo in \TL, così come per la
versione iniziale dell'installatore di Mac\TeX{} e per l'editor
\TeX{}works, che raccomandiamo.

\item Hironori Kitagawa, per il tanto lavoro su p\TeX\ e il relativo
supporto.

\item Dick Koch per mantenere Mac\TeX\ (\url{https://tug.org/mactex}) a
distanza molto ravvicinata da \TL{} e per il suo gran buon umore nel
farlo.

\item Reinhard Kotucha per i suoi considerevoli contributi 
all'infrastruttura e
all'installatore di \TL{} 2008, così come per gli sforzi di ricerca sotto
Windows, lo script \texttt{getnonfreefonts} e molto altro.

\item Siep Kroonenberg, anche lui per importanti contributi
all'infrastruttura e all'installatore di \TL{} 2008, specialmente sotto
Windows, e il grosso del lavoro di aggiornamento di questo manuale per
descrivere queste funzionalità.

\item Clerk Ma, per la correzione dei bug e le estensioni nel motore.

\item Mojca Miklavec, per il grande aiuto con \ConTeXt, per aver fornito gli
eseguibili per molte piattaforme e per tanto altro ancora.

\item Heiko Oberdiek per il pacchetto \pkgname{epstopdf} e molti altri,
per aver compresso gli enormi file di dati di \pkgname{pst-geo} così che
potessimo includerli e, soprattutto, per il suo notevole lavoro su
\pkgname{hyperref}.

\item Phelype Oleinik, per l'\cs{input} confinato nel gruppo tra i motori
nel 2020, e molto altro.

\item Petr Ok\v{s}ak che ha coordinato e controllato con grande attenzione
tutto il materiale in ceco e in slovacco.

\item Toshio Oshima per il suo visualizzatore per Windows
\cmdname{dviout}.

\item Manuel Pégourié-Gonnard per aver aiutato con gli aggiornamenti dei
pacchetti, con i miglioramenti alla documentazione e con lo sviluppo di
\cmdname{texdoc}.

\item Fabrice Popineau per l'originale supporto per Windows in \TL{} e per
il lavoro sulla documentazione in francese.

\item Norbert Preining, l'architetto principale dell'attuale infrastruttura e
installatore di \TL{}, anche per aver coordinato la versione Debian di \TL{}
(insieme con Frank Küster) e per l'enorme lavoro svolto lungo il percorso.

\item Sebastian Rahtz per aver originariamente creato \TL{} ed averne
curato la manutenzione per molti anni. Sebastian è venuto a mancare nel
marzo 2016 e noi dedichiamo il lavoro che va avanti alla sua memoria.

\item Luigi Scarso, per aver continuato lo sviluppo di Metapost, Lua\TeX\
e molto altro ancora.

\item Andreas Scherer per \texttt{cwebbit}, l'implementazione di CWEB
usata in \TL{}.

\item Tomasz Trzeciak per aiuti su vasta scala con Windows.

\item Vladimir Volovich per il sostanziale aiuto nel porting e in altre
problemi di manutenzione e specialmente per aver reso possibile includere
\cmdname{xindy}.

\item Staszek Wawrykiewicz, un collaudatore principale di tutta \TL{} e
coordinatore di molti dei più importanti contributi polacchi: font,
installazione sotto Windows e molto altro. Staszek è venuto a mancare
nel febbraio 2018 e noi dedichiamo il lavoro che va avanti alla sua memoria.

\item Olaf Weber per la sua paziente manutenzione di \Webc\ negli anni
passati.

\item Gerben Wierda per aver creato e fatto manutenzione all'originale
supporto per \MacOSX{}.

\item Graham Williams, l'artefice del Catalogo \TeX{} dei pacchetti.

\item Joseph Wright, per il tanto lavoro sul rendere le stesse
funzionalità primitive disponibili su tutti i motori.

\item Hironobu Yamashita, per il tanto lavoro su p\TeX\ e il relativo
supporto.

\end{itemize*}

Preparatori degli eseguibili:
Marc Baudoin (\pkgname{amd64-netbsd}, \pkgname{i386-netbsd}),
Ken Brown (\pkgname{i386-cygwin}, \pkgname{x86\_64-cygwin}),
Simon Dales (\pkgname{armhf-linux}),
Johannes Hielscher (\pkgname{aarch64-linux}),
Akira Kakuto (\pkgname{win32}),
Dick Koch (\pkgname{x86\_64-darwin}),
Nikola Le\v{c}i\'c (\pkgname{amd64-freebsd}, \pkgname{i386-freebsd}),
Henri Menke (\pkgname{x86\_64-linuxmusl}),
Mojca Miklavec (\pkgname{i386-linux},
                \pkgname{x86\_64-darwinlegacy},
                \pkgname{i386-solaris}, \pkgname{x86\_64-solaris},
                \pkgname{sparc-solaris}),
Norbert Preining (\pkgname{x86\_64-linux}).
Per informazioni sul processo di compilazione di \TL, visitate
\url{https://tug.org/texlive/build.html}.

Attuali traduttori della documentazione:
Denis Bitouz\'e \& Patrick Bideault (francese),
Carlos Enriquez Figueras (spagnolo),
Jjgod Jiang, Jinsong Zhao, Yue Wang \& Helin Gai (cinese),
Nikola Le\v{c}ić (serbo),
Marco Pallante \& Carla Maggi (italiano),
Petr Sojka \& Jan Busa (ceco\slash slovacco),
Boris Veytsman (russo),
Zofia Walczak (polacco),
Uwe Ziegenhagen (tedesco), La pagina web della documentazione di
\TL{} è \url{https://tug.org/texlive/doc.html}.

Ovviamente il più importante ringraziamento deve andare a Donald Knuth,
innanzitutto per aver inventato \TeX{} e poi per averlo donato al mondo.


\section{Storia delle edizioni}
\label{sec:history}

\subsection{Passato}

La discussione iniziò nel tardo 1993 quando il gruppo utenti \TeX{}
olandese stava iniziando a lavorare al proprio \CD{} 4All\TeX{} per gli
utenti MS-DOS e si sperava a quel tempo di rilasciare un solo,
razionale, \CD{} per tutti i sistemi. Questo era un obiettivo troppo
ambizioso a quel tempo, ma non solo diede origine al \CD{} di grande successo
4All\TeX{}, ma altresì spinse il gruppo di lavoro del TUG Technical
Council verso una \emph{Struttura delle Directory \TeX{}} (\TDS{}, \TeX{}
Directory Structure, \url{https://tug.org/tds}), che specificò come creare
collezioni coerenti e gestibili di file di supporto a \TeX. Una bozza
completa della \TDS{} fu pubblicata nel numero di dicembre 1995 di
\textsl{TUGboat} e fu chiaro dalle fasi iniziali che un prodotto
desiderabile sarebbe stata una struttura di modello su \CD{}. La
distribuzione che hai ora è il risultato diretto delle decisioni del
gruppo di lavoro. Fu anche chiaro dal successo del \CD{} 4All\TeX{} che
gli utenti Unix avrebbero beneficiato da un simile semplice sistema e
questo è l'altro filone principale di \TL.

Per prima cosa ci mettemmo all'opera per realizzare un nuovo \CD{} della
\TDS{} basato su Unix nell'autunno del 1995 e rapidamente identificammo il
\teTeX{} di Thomas Esser come l'impianto ideale, dato che già aveva il
supporto per più piattaforme ed era costruito con la portabilità tra
diversi file system in mente. Thomas acconsentì ad aiutarci e il lavoro
cominciò seriamente all'inizio del 1996. La prima edizione fu rilasciata
nel maggio 1996. All'inizio del 1996, Karl Berry completò una nuova
versione di Web2c, che includeva praticamente tutte le funzionalità che
Thomas Esser aveva aggiunto in \teTeX{} e decidemmo di basare la seconda
edizione del \CD{} sul \Webc{} standard, con l'inclusione dello script
\texttt{texconfig} proveniente da \teTeX. La terza edizione del \CD{} fu
basata su una nuova importante revisione di \Webc, la 7.2, realizzata da Olaf
Weber; allo stesso tempo, era stata fatta una nuova revisione di \teTeX{}
e \TL{} incluse quasi tutte le sue funzionalità. La quarta edizione seguì
lo stesso modello, usando una nuova versione di \teTeX{} e di \Webc{}
(7.3). Il sistema adesso includeva anche un completo allestimento per
Windows, grazie a Fabrice Popineau.

Per la quinta edizione (marzo 2000) furono riviste e controllate molte
parti del \CD, aggiornando centinaia di pacchetti. I dettagli sui
pacchetti furono memorizzati in file XML. Ma il cambiamento principale per
\TeX\ Live 5 fu che tutto il software non libero fu rimosso. Tutto in
\TL{} era pensato per essere compatibile con le Debian Free Software
Guidelines (linee guida Debian sul software libero,
\url{https://debian.org/intro/free}); abbiamo fatto del nostro meglio
per controllare le condizioni di licenza di tutti i pacchetti, ma
apprezzeremo tantissimo ogni segnalazione di errori.

La sesta edizione (luglio 2001) aveva aggiornato ancora più materiale. Il
cambiamento più importante fu un nuovo concetto di installazione: l'utente
poteva selezionare un insieme più preciso delle collezioni desiderate. Le
collezioni relative alle lingue furono completamente riorganizzate, così
che, selezionandone una qualunque, non solo venissero installati le macro, i
font, ecc., ma fosse anche preparato un opportuno file
\texttt{language.dat}.

La settima edizione del 2002 ebbe la notevole aggiunta del supporto per
\MacOSX{} e la solita miriade di aggiornamenti ad ogni genere di pacchetto
e programma. Un traguardo importante fu l'integrazione dei sorgenti con
quelli di \teTeX{} per correggere l'allontanamento l'uno dall'altro
avvenuto nelle versioni~5 e~6.

\subsubsection{2003}

Nel 2003, con il continuo flusso di aggiornamenti ed aggiunte, scoprimmo
che \TL{} era cresciuta così tanto che non poteva più essere contenuta in
un singolo \CD, quindi la dividemmo in tre diverse distribuzioni (consultate
la sezione~\ref{sec:tl-coll-dists}, \p.\pageref{sec:tl-coll-dists}). In
più:

\begin{itemize*}
\item Su richiesta del team di \LaTeX, cambiammo i comandi \cmdname{latex} e
      \cmdname{pdflatex} affinché usassero \eTeX{} (consultate 
      \p.\pageref{text:etex}).
\item I nuovi font Latin Modern furono inclusi (e sono raccomandati).
\item Il supporto per Alpha OSF fu rimosso (il supporto per
      HPUX era già stato rimosso in precedenza) dato che nessuno
      aveva (o voleva offrire) l'hardware su cui compilare i nuovi
      eseguibili.
\item L'allestimento per Windows fu cambiato sostanzialmente; per la prima
      volta fu introdotto un ambiente integrato basato su XEmacs.
\item Importanti programmi aggiuntivi per Windows (Perl, Ghost\-script,
      Image\-Magic, Ispell) sono ora installati nelle directory \TL{}.
\item I file di mappatura per i font usati da \cmdname{dvips},
      \cmdname{dvipdfm} e \cmdname{pdftex} sono ora generati da un nuovo
      programma \cmdname{updmap} ed installati in
      \dirname{texmf/fonts/map}.    
\item \TeX, \MF{} e \MP{} adesso, come impostazione predefinita, mostrano la 
      maggior parte dei caratteri
      in ingresso (dal numero 32 \acro{ASCII} in su) come sé stessi nei
      file che vengono generati (ad esempio, con \verb|\write|), nei file
      di registro e sul terminale, ossia \emph{non sono più} tradotti
      usando la notazione \verb|^^|. In \TL{}~7 questa traduzione
      dipendeva dalle impostazioni sulla lingua del sistema; adesso,
      queste impostazioni locali non influenzano più il comportamento dei 
      programmi
      \TeX. Se per qualche ragione avete bisogno della notazione \verb|^^|,
      rinominate il file \verb|texmf/web2c/cp8bit.tcx| (le edizioni future
      avranno un modo più pulito per controllare questa opzione).
\item Questa documentazione fu revisionata sostanzialmente.
\item Infine, dato che i numeri delle edizioni erano cresciuti in modo
      poco agevole, da adesso la versione è identificata semplicemente
      dall'anno: \TL{} 2003.
\end{itemize*}


\subsubsection{2004}

Il 2004 vide molti cambiamenti:

\begin{itemize}

\item Se avete font installati localmente che usano i propri file di
supporto \filename{.map} o (molto meno probabilmente) \filename{.enc},
potreste aver bisogno di spostare questi file.

I file \filename{.map} adesso sono cercati soltanto in sotto directory di
\dirname{fonts/map} (per ciascuna gerarchia \filename{texmf}, lungo il
percorso \envname{TEXFONTMAPS}). In modo simile, i file \filename{.enc}
sono cercati soltanto nelle sottodirectory di \dirname{font/enc}, lungo il
percorso \envname{ENCFONTS}. \cmdname{updmap} tenterà di avvisarvi su file
che possono provocare problemi.

Per i metodi per gestire questa ed altre informazioni, visitate la pagina
\url{https://tug.org/texlive/mapenc.html}.

\item \TK{} è stata espansa con l'aggiunta di un \CD{} installabile basato
su \MIKTEX, per coloro che preferiscono quell'implementazione a Web2C.
Consultate la sezione~\ref{sec:overview-tl} (\p.\pageref{sec:overview-tl}).

\item All'interno di \TL, la singola grande directory \dirname{texmf}
delle edizioni precedenti è stata sostituita da: \dirname{texmf},
\dirname{texmf-dist} e \dirname{texmf-doc}. Consultate la
sezione~\ref{sec:tld} (\p.\pageref{sec:tld}) e il file \filename{README}
contenuto in ciascuna di esse.

\item Tutti file in ingresso relativi a \TeX{} sono adesso raccolti nella
sotto directory \dirname{tex} delle varie \dirname{texmf*}, piuttosto che
avere le diverse posizioni \dirname{tex}, \dirname{etex},
\dirname{pdftex}, \dirname{pdfetex}, ecc. Consultate
\CDref{texmf-dist/doc/generic/tds/tds.html\#Extensions}
{\texttt{texmf-dist/doc/generic/tds/tds.html\#Extensions}}.

\item Gli script di supporto (pensati per non essere invocati dagli
utenti) sono adesso posizionati in una nuova sotto directory delle varie
\dirname{texmf*} chiamata \dirname{scripts} e possono essere cercati
usando \verb|kpsewhich -format=texmfscripts|. Se quindi avete dei programmi
che richiamano questi script, dovranno essere corretti. Consultate
\CDref{texmf-dist/doc/generic/tds/tds.html\#Scripts}
{\texttt{texmf-dist/doc/generic/tds/tds.html\#Scripts}}.

\item Quasi tutti i formati lasciano la maggior parte dei caratteri
stampabili uguali a se stessi tramite il ``file di traduzione''
\filename{cp227.tcx}, piuttosto che trasformarli nella notazione
\verb|^^|. Nello specifico, i caratteri alle posizioni ASCII
32--256, la tabulazione orizzontale, quella verticale e il ``form feed''
sono considerati stampabili e non vengono trasformati. Le eccezioni sono
plain \TeX{} (solo i caratteri 32--126 sono stampabili), \ConTeXt{}
(caratteri 0--255) e i formati legati ad \OMEGA. Questo comportamento
predefinito è quasi lo stesso che in \TL\,2003, ma è implementato in
maniera più pulita, con maggiori possibilità di personalizzazione.
Consultate \CDref{texmf-dist/doc/web2c/web2c.html\#TCX-files}
{\texttt{texmf-dist/doc/web2c/web2c.html\#TCX-files}} (ad ogni modo, con
l'input in formato Unicode, \TeX{} potrebbe stampare sequenze parziali di
caratteri quando viene mostrato il contesto degli errori dato che legge
l'input come una sequenza di byte, non di caratteri).

\item \textsf{pdfetex} è adesso il motore predefinito per tutti i formati
tranne (plain) \textsf{tex} stesso (ovviamente genera file DVI
quando è eseguito come \textsf{latex}, ecc.). Questo significa, tra le
altre cose, che le caratteristiche di microtipografia di \textsf{pdftex}
sono disponibili in \LaTeX, \ConTeXt, ecc., così come le funzionalità di
\eTeX{} (\OnCD{texmf-dist/doc/etex/base/}).

Significa anche che è \emph{più importante che mai} usare il pacchetto
\pkgname{ifpdf} (funziona sia con plain \TeX, che con \LaTeX) o del codice
equivalente, perché verificare semplicemente se \cs{pdfoutput} o qualche
altra primitiva sono definiti non è un modo affidabile per determinare se
si sta generando un file PDF. Per quest'anno abbiamo cercato di
rendere questo aspetto compatibile con le edizioni passate nel miglior modo
possibile, ma a partire dal prossimo anno \cs{pdfoutput} potrebbe
essere definito anche se il file generato è un DVI.

\item pdf\TeX\ (\url{http://pdftex.org}) ha molte nuove funzionalità:

  \begin{itemize*}

  \item \cs{pdfmapfile} e \cs{pdfmapline} forniscono il supporto alle
  mappature dei font direttamente all'interno di un documento.

  \item L'espansione microtipografica dei font può essere usata più
  facilmente.\\
  \url{https://www.ntg.nl/pipermail/ntg-pdftex/2004-May/000504.html}

  \item Tutti i parametri che prima erano impostati tramite lo speciale
  file di configurazione \filename{pdftex.cfg} devono essere adesso
  impostati tramite primitive, tipicamente in \filename{pdftexconfig.tex};
  \filename{pdftex.cfg} non è più supportato. Ogni file \filename{.fmt}
  esistente
  deve essere rigenerato quando \filename{pdftexconfig.tex} viene
  modificato.

  \item Per saperne di più, consultate il manuale di pdf\TeX{}:
  \OnCD{texmf-dist/doc/pdftex/manual/pdftex-a.pdf}.

  \end{itemize*}

\item La primitiva \cs{input} in \cmdname{tex} (e in \cmdname{mf} e
\cmdname{mpost}) adesso accetta nomi con spazi ed altri caratteri speciali
racchiusi tra doppi apici. Esempi tipici:
\begin{verbatim}
\input "file con spazi"   % plain
\input{"file con spazi"}  % latex
\end{verbatim}
Consultate il manuale di Web2C per saperne di più:
\OnCD{texmf-dist/doc/web2c}.

\item Il supporto per enc\TeX{} è ora incluso in Web2C e, di conseguenza,
in tutti i programmi \TeX, per mezzo dell'opzione \optname{-enc} \Dash\
\emph{solo quando i formati sono stati generati}. enc\TeX{} supporta la
ricodifica generale dell'input e dell'output, permettendo il supporto
completo per l'Unicode (in UTF-8). Consultate
\OnCD{texmf-dist/doc/generic/enctex/} e
\url{http://olsak.net/enctex.html}.

\item Aleph, un nuovo motore che combina \eTeX\ ed \OMEGA, è disponibile.
Alcune informazioni sono disponibili in \OnCD{texmf-dist/doc/aleph/base} e
su \url{https://texfaq.org/FAQ-enginedev}. Il formato
basato su \LaTeX{} per Aleph è chiamato \textsf{lamed}.

\item L'ultimo aggiornamento di \LaTeX\ ha una nuova versione della
LPPL \Dash\ ora una licenza ufficialmente approvata da Debian.
Per altri svariati aggiornamenti, consultate i file \filename{ltnews} in
\OnCD{texmf-dist/doc/latex/base}.

\item \cmdname{dvipng}, un nuovo programma per convertire i DVI in
immagini PNG, è incluso. Consultate
\url{https://ctan.org/pkg/dvipng}.

\item Abbiamo ridotto il pacchetto \pkgname{cbgreek} ad un insieme di font
di ``medie'' dimensioni, con il consenso e i suggerimenti dell'autore
(Claudio Beccari). I font omessi sono quelli invisibili, quelli profilati
e i trasparenti, che sono usati abbastanza raramente, mentre noi avevamo
bisogno di spazio. L'insieme completo è ovviamente disponibile su
CTAN (\url{https://ctan.org/pkg/cbgreek-complete}).

\item \cmdname{oxdvi} è stato rimosso; usate semplicemente \cmdname{xdvi}.

\item I comandi (collegamenti) \cmdname{ini} e \cmdname{vir} per
\cmdname{tex}, \cmdname{mf} e \cmdname{mpost} non sono più creati, così
come \cmdname{initex}. Le funzionalità di \cmdname{ini} sono disponibili
oramai da anni tramite l'opzione su riga di comando \optname{-ini}.

\item Il supporto per la piattaforma \textsf{i386-openbsd} è stato
rimosso. Dato che è disponibile il pacchetto \pkgname{tetex} nel sistema
BSD Ports e gli eseguibili per GNU/Linux e FreeBSD
erano disponibili, ci è sembrato che il tempo dedicato dai volontari
potesse essere meglio speso in altre attività.

\item Su \textsf{sparc-solaris} (almeno), potreste dover impostare la
variabile d'ambiente \envname{LD\_LIBRARY\_PATH} per eseguire i programmi
delle \pkgname{t1utils}. La ragione è che questi sono compilati con il C++
e non esiste una posizione comune per le librerie (questo problema non è
nuovo dell'edizione 2004, ma non era stato documentato in precedenza). In
modo simile, su \textsf{mips-irix}, sono richieste le librerie di runtime
del MIPSpro 7.4.

\end{itemize}

\subsubsection{2005}

L'edizione del 2005 ha visto il solito enorme numero di aggiornamenti ai
pacchetti ed ai programmi. L'infrastruttura è rimasta sostanzialmente
invariata dal 2004, ma inevitabilmente ci sono stati comunque dei
cambiamenti:

\begin{itemize}

\item Sono stati introdotti i nuovi script \cmdname{texconfig-sys},
      \cmdname{updmap-sys} e \cmdname{fmtutil-sys} che modificano la
      configurazione nei percorsi di sistema. Gli script
      \cmdname{texconfig}, \cmdname{updmap} e \cmdname{fmtutil} ora
      modificano i file specifici per i singoli utenti, sotto
      \dirname{$HOME/.texlive2005}.

\item Sono state introdotte le corrispondenti nuove variabili
      \envname{TEXMFCONFIG} e \envname{TEXMFSYSCONFIG} per specificare i
      percorsi dove trovare i file di configurazione (per il singolo
      utente e per l'intero sistema, rispettivamente). Quindi, potreste
      dover spostare le versioni personali di \filename{fmtutil.cnf} e
      \filename{updmap.cfg} in questi percorsi; un'altra possibilità è quella
      di ridefinire \envname{TEXMFCONFIG} o \envname{TEXMFSYSCONFIG} in
      \filename{texmf.cnf}. In ogni caso la posizione reale di questi file
      e i valori di \envname{TEXMFCONFIG} e \envname{TEXMFSYSCONFIG}
      devono concordare. Consultate la sezione~\ref{sec:texmftrees},
      \p.\pageref{sec:texmftrees}.

\item L'anno scorso abbiamo tenuto \verb|\pdfoutput| ed altre
      primitive non definite per l'output in \dvi, anche quando veniva
      usato il programma \cmdname{pdfetex}. Quest'anno, come promesso,
      abbiamo annullato quella misura di compatibilità. Per cui, se il
      vostro documento usa \verb|\ifx\pdfoutput\undefined| per verificare
      se viene generato un PDF, dovrà essere cambiato. Potete usare il
      pacchetto \pkgname{ifpdf.sty} (che funziona sia sotto plain \TeX{}
      che sotto \LaTeX) per fare ciò, oppure rubarne la logica.

\item L'anno scorso abbiamo cambiato la maggior parte dei formati per
      stampare i caratteri (a 8 bit) come sé stessi (consultate la sezione
      precedente). Il nuovo file TCX \filename{empty.tcx} adesso offre un
      modo più semplice per ottenere l'originaria notazione \verb|^^| se
      la desiderate, come in:
\begin{verbatim}
latex --translate-file=empty.tcx vostrofile.tex
\end{verbatim}

\item È incluso il nuovo programma \cmdname{dvipdfmx} per la
      trasformazione dei DVI in PDF; si tratta di un aggiornamento
      attivamente mantenuto di \cmdname{dvipdfm} (che per ora è ancora
      disponibile, anche se non più raccomandato).

\item Sono inclusi i nuovi programmi \cmdname{pdfopen} e
      \cmdname{pdfclose} per consentire di ricaricare i file PDF nel
      lettore Adobe Acrobat Reader senza riavviare il programma
      (altri lettori PDF, in particolare \cmdname{xpdf}, \cmdname{gv} e
      \cmdname{gsview}, non hanno mai sofferto di questo problema).

\item Per coerenza, le variabili \envname{HOMETEXMF} e
      \envname{VARTEXMF} sono state rinominate \envname{TEXMFHOME} e
      \envname{TEXMFVAR}, rispettivamente. C'è anche \envname{TEXMFVAR},
      che è, di default, specifica per ogni utente. Consultate il primo punto
      dell'elenco.

\end{itemize}


\subsubsection{2006--2007}

Nell'edizione 2006--2007, la nuova aggiunta più rilevante a \TL{} è stato il
programma \XeTeX, disponibile con i comandi \texttt{xetex} e
\texttt{xelatex}; visitate il sito \url{https://scripts.sil.org/xetex}.

Anche MetaPost ha ricevuto un aggiornamento degno di nota, mentre altri ne
sono stati pianificati per il futuro
(\url{https://tug.org/metapost/articles}), così come per pdf\TeX{}
(\url{https://tug.org/applications/pdftex}).

I \filename{.fmt} di \TeX{} (formati ad alta velocità) e i file simili per
MetaPost e \MF{} adesso sono posizionati in sotto directory di
\dirname{texmf/web2c}, invece che nella directory stessa (sebbene la
directory sia ancora visitata durante la ricerca, a beneficio dei
\filename{.fmt} esisitenti). Le sotto directory sono chiamate in base al
``motore'' usato, come \filename{tex} o \filename{pdftex} o
\filename{xetex}. Questo cambiamento dovrebbe essere invisibile nell'uso
normale.

Il programma (plain) \texttt{tex} non legge più le prime linee
identificate da \texttt{\%\&} per determinare quale formato adoperare; è
il puro \TeX{} Knuthiano (\LaTeX{} e tutto il resto leggono ancora le
linee \texttt{\%\&}).

Ovviamente anche quest'anno ha visto (i soliti) centinaia di altri
aggiornamenti ai pacchetti ed ai programmi. Come sempre, visitate
CTAN (\url{https://ctan.org}) per tutti gli aggiornamenti.

Internamente, i sorgenti sono ora memorizzati tramite Subversion, con
un'interfaccia web standard per visitarli; il collegamento è sulla nostra
pagina home. Sebbene invisibile nella distribuzione finale, ci aspettiamo
che questo fornisca un fondamento stabile per lo sviluppo negli anni a
venire.

Infine, nel maggio 2006 Thomas Esser ha annunciato che non avrebbe più
aggiornato te\TeX{} (\url{https://tug.org/tetex}). Come risultato, c'è
stata un'impennata nell'interesse verso \TL, specialmente tra i distributori
di \GNU/Linux (c'è un nuovo schema di installazione in \TL{} chiamato
\texttt{tetex}, che fornisce un equivalente approssimativo).  Speriamo che
alla fine questo si traduca in miglioramenti all'ambiente \TeX{} per tutti.

\subsubsection{2008}

Nell'edizione del 2008, l'intera infrastruttura di \TL{} è stata
riprogettata e reimplementata. Le informazioni complete su
un'installazione sono adesso memorizzate in un file di testo puro
\filename{tlpkg/texlive.tlpdb}.

Tra le altre cose, ciò rende finalmente possibile aggiornare
un'installazione \TL{} tramite Internet dopo l'installazione iniziale, una
caratteristica che Mik\TeX{} ha offerto per anni. Prevediamo di
aggiornare regolarmente i nuovi pacchetti così come diventano disponibili
su \CTAN.

Il nuovo motore Lua\TeX{} (\url{http://luatex.org}) è ora incluso;
accanto ad un nuovo livello di flessibilità nella composizione
tipografica, questo fornisce un eccellente linguaggio di scripting da
usare sia dentro che fuori i documenti \TeX.

Il supporto tra Windows e le piattaforme basate su Unix è ora molto più
uniforme. In particolare, la maggior parte degli script Perl e Lua sono
ora disponibili sotto Windows tramite l'interprete Perl distribuito
internamente con \TL.

Il nuovo script \cmdname{tlmgr} (sezione~\ref{sec:tlmgr}) è l'interfaccia
generale per amministrare \TL{} dopo l'installazione iniziale. Esso
gestice l'aggiornamento dei pacchetti e la conseguente rigenerazione dei
formati, dei file di mappatura e dei file delle lingue, incluse
opzionalmente le aggiunte locali.

Con l'avvento di \cmdname{tlmgr}, è ora disabilitata l'azione di modifica
dei file di configurazione dei formati e delle sillabazioni operata da
\cmdname{texconfig}.

Il programma per la creazione degli indici \cmdname{xindy}
(\url{http://xindy.sourceforge.net/}) è ora incluso nella maggior parte
delle piattaforme.

Lo strumento \cmdname{kpsewhich} può ora riportare tutte le corrispondenze
per un dato file (opzione \optname{--all}) e limitarle ad una data sotto
directory (opzione \optname{--subdir}).

Il programma \cmdname{dvipdfmx} adesso include la funzionalità di estrarre
le informazioni sulla bounding box attraverso il comando
\cmdname{extractbb}; si tratta di una delle ultime funzionalità che
erano fornite da \cmdname{dvipdfm} ma non ancora incluse in
\cmdname{dvipdfmx}.

Gli alias dei font \filename{Times-Roman}, \filename{Helvetica} e così via
sono stati rimossi. Diversi pacchetti si aspettavano che funzionassero in
modo diverso (in particolare, che avessero codifiche differenti) e non
c'era un buon modo per risolvere questo problema.

Il formato \pkgname{platex} è stato rimosso per risolvere un conflitto di
nome con un completamente diverso \pkgname{platex} giapponese; il
pacchetto \pkgname{polski} contiene ora il principale supporto per la
lingua polacca.

Internamente, i file con le riserve di stringhe di \web{} sono compilati
dentro gli eseguibili, per semplificare gli aggiornamenti.

Infine, in questa edizione sono stati inclusi i cambiamenti fatti da
Donald Knuth nella sua ``messa a punto di \TeX\ del 2008''. Consultate
\url{https://tug.org/TUGboat/Articles/tb29-2/tb92knut.pdf}.


\subsubsection{2009}

Nell'edizione 2009, il formato di output predefinito di Lua\AllTeX\ è ora
il PDF per avvalersi del supporto OpenType di Lua\TeX, ecc. I nuovi
eseguibili chiamati \code{dviluatex} e \code{dvilualatex} eseguono
Lua\TeX{} con l'output in DVI. La pagina web di Lua\TeX{} è
\url{http://luatex.org}.

Il motore Omega e il formato Lambda originari sono stati rimossi dopo
averne discusso con gli autori di Omega. Restano le versioni aggiornate
Aleph e Lamed, così come le utilità di Omega.

È stata inclusa una nuova versione dei font AMS \TypeI, compreso il
Computer Modern: sono stati integrati alcuni cambiamenti nelle forme
fatti da Knuth nei sorgenti Metafont nel corso degli anni e
l'\emph{hinting} è stato aggiornato. I font Euler sono stati ridisegnati 
meticolosamente da Hermann Zapf (visitate
\url{https://tug.org/TUGboat/Articles/tb29-2/tb92hagen-euler.pdf}). In ogni
caso, le metriche non hanno subito modifiche. La pagina dei font AMS è
\url{https://ams.org/tex/amsfonts.html}.

Il nuovo editor \GUI{} \TeX{}works è incluso per Windows così come in
Mac\TeX. Per le altre piattaforme e per ulteriori informazioni, consultate
il sito di \TeX{}works, \url{https://tug.org/texworks}. Si tratta di un
programma multi piattaforma ispirato all'editor TeXShop per \MacOSX\ e che
ha come obiettivo la facilità d'uso.

Il programma per la grafica Asymptote è stato incluso per diverse
piattaforme. Esso implementa un linguaggio descrittivo per la grafica
basato su testo vagamente simile a MetaPost, ma con un supporto avanzato
al 3D ed altre caratteristiche. Il suo sito web è
\url{https://asymptote.sourceforge.io}.

Il programma \code{dvipdfm} è stato sostituito da \code{dvipdfmx}, che
opera in una speciale modalità di compatibilità quando viene invocato con
il primo nome. \code{dvipdfmx} include il supporto per CJK
(Chinese-Japanese-Korean, Cinese-Giapponese-Coreano) ed ha accumulato
molte correzioni negli anni trascorsi dall'ultima versione di
\code{dvipdfm}.

Gli eseguibili per le piattaforme \pkgname{cygwin} e
\pkgname{i386-netbsd} sono ora inclusi, mentre siamo stati avvisati che
gli utenti di OpenBSD ottengono \TeX\ tramite il loro sistema di
pacchetti, e in più c'erano difficoltà nel produrre eseguibili che
avessero una qualche possibilità di funzionare su più di una versione.

Una varietà di cambiamenti minori: adesso usiamo la compressione
\pkgname{xz}, il rimpiazzo stabile per \pkgname{lzma}
(\url{https://tukaani.org/xz/}); il carattere |$| è ammesso nei nomi dei
file quando non introduce il nome di variabile noto; la libreria Kpathsea
è ora multi-thread (lo sfrutta MetaPost); l'intera compilazione di
\TL{} è ora basata su Automake.%stopzone

Nota finale sul passato: tutte le edizioni di \TL{}, assieme al materiale
ausiliario come le etichette dei \CD, sono disponibili all'indirizzo
\url{ftp://tug.org/historic/systems/texlive}.


\subsubsection{2010}
\label{sec:2010news} % keep with 2010

Nell'edizione 2010, la versione predefinita per l'output PDF è ora la 1.5,
che consente una maggiore compressione. Questo si applica a tutti i motori
\TeX\ quando vengono usati per produrre PDF e a \code{dvipdfmx}. Caricando
il pacchetto \LaTeX\ \pkgname{pdf14} o impostando |\pdfminorversion=4|, si
si ritorna al PDF~1.4.

pdf\AllTeX\ ora converte \emph{automaticamente} un dato file di tipo 
Encapsulated
Postscript (EPS) in PDF tramite il pacchetto \pkgname{epstopdf}
quando e se viene caricato il file di configurazione per \LaTeX\
\code{graphics.cfg} e viene richiesto l'output in PDF. Le opzioni
predefinite sono tali da eliminare ogni possibilità che siano sovrascritti i
file PDF creati a mano, ma potete del tutto evitare che \code{epstopdf} sia
caricato mettendo |\newcommand{\DoNotLoadEpstopdf}{}| (o |\def...|) prima
della dichiarazione \cs{documentclass}. Non viene caricato neppure se viene
usato il pacchetto \pkgname{pst-pdf}. Per maggiori dettagli, consultate la
documentazione del pacchetto \pkgname{epstopdf}
(\url{https://ctan.org/pkg/epstopdf-pkg}).

Un cambiamento correlato è che adesso è abilitata per default l'esecuzione
di alcuni comandi esterni a \TeX, tramite la funzionalità \cs{write18}.
Questi comandi sono \code{repstopdf}, \code{makeindex}, \code{kpsewhich},
\code{bibtex} e \code{bibtex8}; la lista è definita in \code{texmf.cnf}. Gli
ambienti che devono disabilitare tutti questi comandi esterni possono
deselezionare questa opzione nell'installatore (consultate la
sezione~\ref{sec:options}), oppure rimpiazzare il valore dopo
l'installazione eseguendo |tlmgr conf texmf shell_escape 0|.

Ancora un altro cambiamento correlato è che ora \BibTeX\ e Makeindex si
rifiutano di scrivere il proprio output in una directory arbitraria (come lo
stesso \TeX) per default. In questo modo diventa ora possibile consentirne
l'uso tramite la funzionalità ristretta \cs{write18}. Per cambiare
questo, può essere impostata la variabile d'ambiente \envname{TEXMFOUTPUT}
oppure può essere modificata l'opzione |openout_any|.

\XeTeX\ ora supporta la crenatura al margine lungo le stesse righe come
pdf\TeX\ (l'espansione dei caratteri non è supportata al momento).

Per default, \prog{tlmgr} ora salva un backup di ciascun pacchetto
aggiornato (\code{tlmgr option autobackup 1}), così che gli aggiornamenti
con pacchetti danneggiati possano essere facilmente ripristinati con
\code{tlmgr restore}. Se eseguite aggiornamenti successivi all'installazione
e non avete spazio sul disco per i backup, eseguite \code{tlmgr option
autobackup 0}.

Nuovi programmi inclusi: il motore p\TeX\ e le utilità correlate per la
composizione in giapponese; il programma \BibTeX{}U per un \BibTeX\
compatibile con Unicode; l'utilità \prog{chktex}
(originariamente da \url{http://baruch.ev-en.org/proj/chktex}) per
controllare i documenti \AllTeX; il traduttore da DVI a SVG \prog{dvisvgm}
(\url{https://dvisvgm.de}).

Ora sono inclusi gli eseguibili per queste nuove piattaforme:
\code{amd64-freebsd}, \code{amd64-kfreebsd}, \code{i386-freebsd},
\code{i386-kfreebsd}, \code{x86\_64-darwin}, \code{x86\_64-solaris}.

Un cambiamento in \TL{} 2009 che ci siamo dimenticati di evidenziare: numerosi
eseguibili legati a \TeX4ht (\url{https://tug.org/tex4ht}) sono stati rimossi
dalle directory degli eseguibili. Il programma generico \code{mk4ht} può
essere usato per eseguire una qualunque tra le varie combinazioni di
\code{tex4ht}.

Infine, la versione di \TL{} sul \DVD\ \TK\ non può più essere eseguita
``live'' (curiosamente). Un singolo \DVD\ non ha più abbastanza spazio. Un
effetto collaterale positivo è che l'installazione da un \DVD\ fisico è ora
molto più rapida.

\subsubsection{2011}

Gli eseguibili per \MacOSX\ (\code{universal-darwin} e \code{x86\_64-darwin}
ora funzionano solo su Leopard e succesivi; Panther e Tiger non sono più
supportati.

Il programma \code{biber} per l'elaborazione delle bibliografie è incluso
per le piattaforme comuni. Il suo sviluppo è strettamente accoppiato con il
pacchetto \code{biblatex}, che reimplementa completamente l'infrastruttura
bibliografica fornita da LaTeX.

Il programma MetaPost (\code{mpost}) non crea né usa più i file \code{.mem}.
I file necessari, come \code{plain.mp}, sono semplicemente letti ad ogni
esecuzione. Questo è correlato al supporto di MetaPost come libreria, che è
un'altra modifica significativa, sebbene non visibile dall'utente.

L'implementazione di \code{updmap} in Perl, precedentemente usata solo su
Windows, è stata rimodernata ed è ora usata su tutte le piattaforme. Non
dovrebbe esserci alcun cambiamento visibile dall'utente, tranne che ora gira
molto più velocemente.

I programmi \cmdname{initex} e \cmdname{inimf} sono stati ripristinati
(ma non le altre varianti \cmdname{ini*}).

\subsubsection{2012}

\code{tlmgr} supporta gli aggiornamenti da archivi di rete multipli. La
sezione sugli archivi multipli nel messaggio di output di \code{tlmgr help} ha
ulteriori informazioni.

Il parametro \cs{XeTeXdashbreakstate} ha il valore predefinito di~1 sia per
\code{xetex} che per \code{xelatex}. Questo permette l'interruzione delle
linee dopo i trattini lunghi e medi, che è stato sempre il comportamento di
\TeX, \LaTeX, Lua\TeX, ecc. I documenti \XeTeX\ esistenti che devono
mantenere la perfetta compatibilità con le interruzioni di linee, devono
impostare \cs{XeTeXdashbreakstate} a~0 esplicitamente.

I file di output generati, tra gli altri, da \code{pdftex} e \code{dvips}
possono ora superare i 2 gigabyte.

I 35 font standard di PostScript sono ora inclusi nell'output di \code{dvips}
per default, dato che ne esistono così tante versioni differenti.

Nella modalità di esecuzione \cs{write18} ristretta, che è impostata per
default, \code{mpost} è diventato un programma consentito.

Un file \code{texmf.cnf} viene trovato anche in \filename{../texmf-local},
se esiste (ad esempio, \filename{/usr/local/texlive/texmf-local/web2c/texmf.cnf}).

Lo script \code{updmap} legge un file \code{updmap.cfg} in ciascuna directory
invece di una configurazione globale. Questo cambiamento dovrebbe essere 
trasparente per l'utente,
a meno che non abbiate modificato il vostro updmap.cfg direttamente. Il messaggio
mostrato da \verb|updmap --help| dà ulteriori informazioni.

Piattaforme: aggiunte \pkgname{armel-linux} e \pkgname{mipsel-linux};
\pkgname{sparc-linux} e \pkgname{i386-netbsd} non fanno più parte della
distribuzione principale.

\subsubsection{2013}

Struttura della distribuzione: per semplicità, la directory \code{texmf/} è stata
fusa all'interno di \code{texmf-dist}. Entrambe le variabili
di Kpathsea \code{TEXMFMAIN} e \code{TEXMFDIST} adesso puntano a
\code{texmf-dist}.

Molte piccole collezioni di lingue sono state fuse insieme, per
semplificare l'instllazione.

\MP: sono stati aggiunti il supporto nativo per l'output in PNG e per
i numeri in virgola mobile (\em{double} secondo lo standard IEEE 754).

Lua\TeX: aggiornato a Lua 5.2 e include una nuova libreria
(\code{pdfscanner}) per processare il contenuto di pagine PDF esterne,
tra le altre cose (visitate il suo sito).

\XeTeX\ (anche in questo caso, visitate il sito per altri dettagli):
\begin{itemize*}
\item Ora, per il layout dei font, è usata la libreria HarfBuzz al posto di
ICU (ICU è ancora usata per la codifica dell'input, la bidirezionalità
e interruzione di riga Unicode opzionale).
\item Graphite2 e HarfBuzz sono usate al posto di SilGraphite per il
layout di Graphite.
\item Sul Mac, Core Text è usato al posto del deprecato ATSUI.
\item Preferisce i font TrueType/OpenType ai Type1 quando i nomi sono
gli stessi.
\item Corregge le occasionali discrepanze tra \XeTeX\ e \code{xdvipdfmx}
nella ricerca dei font.
\item Supporta i cut-in per i font OpenType Math.
\end{itemize*}

\cmdname{xdvi}: ora usa FreeType al posto di \code{t1lib} per il
rendering.

\pkgname{microtype.sty}: un qualche supporto per \XeTeX\ (protrusione)
e Lua\TeX\ (protrusione, espansione dei font, tracking), tra i vari
miglioramenti.

\cmdname{tlmgr}: nuova azione di \emph{affissione} (\code{pinning}) per
facilitare la configurazione di repository multipli; la relativa sezione
in \verb|tlmgr --help|, visibile online su
\url{https://tug.org/texlive/doc/tlmgr.html#MULTIPLE-REPOSITORIES},
ne parla più approfonditamente.

Piattaforme: \pkgname{armhf-linux}, \pkgname{mips-irix},
\pkgname{i386-netbsd} e \pkgname{amd64-netbsd} aggiunte o ripristinate;
\pkgname{powerpc-aix} rimossa.

\subsubsection{2014}

Il 2014 ha visto un altro aggiustamento a \TeX\ da parte di Knuth; questo ha
riguardato tutti i motori, ma l'unico cambiamento visibile è probabilmente il
ripristino della stringa \code{preloaded format} nella linea di
intestazione. Dal punto di vista di Knuth, questo ora riflette il formato che
\emph{dovrebbe} essere caricato di default, piuttosto che un formato 
incorporato nell'eseguibile; può essere sostituito in vari modi.

pdf\TeX: nuovo parametro per la soppressione dei warning
\cs{pdfsuppresswarningpagegroup}; nuove primitive per falsi spazi tra
parole per facilitare la ridisposizione del testo PDF:
\cs{pdfinterwordspaceon}, \cs{pdfinterwordspaceoff}, \cs{pdffakespace}.

Lua\TeX: sono stati fatti notevoli cambiamenti e correzioni al caricamento dei
font e alla sillabazione. L'aggiunta più grande è una nuova variante del
motore, \code{luajittex} e i suoi fratelli \code{texluajit} e
\code{texluajitc}. Questo motore usa un compilatore Lua
just-in-time (articolo \textsl{TUGboat} dettagliato su
\url{https://tug.org/TUGboat/tb34-1/tb106scarso.pdf}). \code{luajittex} è
ancora in fase di sviluppo, non è disponibile su tutte le piattaforme ed è
notevolmente meno stabile di \code{luatex}. Né noi né i suoi sviluppatori
lo raccomandano eccetto che per lo scopo di sperimentare la
compilazione jit del codice Lua.

\XeTeX: gli stessi formati di immagini sono ora supportati su tutte le
piattaforme (incluso il Mac); evita il fallback di decomposizione di
compatibilità di Unicode (ma non altre varianti); preferisce i font
OpenType ai Graphite, per compatibilità con le versioni precedenti di
\XeTeX.

\MP: un nuovo sistema di numerazione \code{decimal} è supportato, insieme
ad uno che lo accompagna, \code{numberprecision}, ad uso interno; una nuova
definizione di \code{drawdot} in \filename{plain.mp}, secondo Knuth;
correzioni di bug nell'output in SVG e PNG, tra le altre
cose.

L'utilità Con\TeX{}t \cmdname{pstopdf} sarà rimossa come comando a sé
stante ad un certo punto dopo questa versione, a causa dei conflitti con
le utilità del sistema operativo con lo stesso nome. Può essere ancora (e
tuttora) invocato con \code{mtxrun --script pstopdf}.

\cmdname{psutils} è stato notevolmente revisionato da un nuovo
manutentore. Come risultato, molte utilità usate raramente (\code{fix*},
\code{getafm}, \code{psmerge}, \code{showchar}) ora si trovano solo nella
directory \dirname{scripts/} invece che essere eseguibili accessibili
direttamente (si può tornare indietro, se si rivela problematico). Un
nuovo script, \code{psjoin}, è stato aggiunto.

La distribuzione Mac\TeX\ di \TeX\ Live (sezione~\ref{sec:macosx}) non
include più i pacchetti opzionali solo per Mac dei font Latin Modern e
\TeX\ Gyre, dato che è abbastanza semplice per i singoli utenti renderli
disponibili al sistema. Anche il programma \cmdname{convert} da ImageMagick
è stato rimosso, dato che \TeX4ht (nello specifico \code{tex4ht.env}) ora
usa Ghostscript direttamente.

La collezione \pkgname{langcjk} per il supporto al cinese, giapponese e
coreano è stata suddivisa in collezioni per le singole lingue per avere
dimensioni più piccole.

Piattaforme: \pkgname{x86\_64-cygwin} aggiunta, \pkgname{mips-irix}
rimossa; Microsoft ha cessato il supporto di Windows XP, quindi i
nostri programmi potrebbero iniziare a funzionare male su quel sistema in
qualunque momento.


\subsubsection{2015}

\LaTeXe\ ora incorpora, per default, i cambiamenti precedentemente inclusi
solo caricando esplicitamente il pacchetto \pkgname{fixltx2e}, che ora non
fa nulla. Un nuovo pacchetto \pkgname{latexrelease} ed altri meccanismi
permettono di controllare ciò che viene fatto. I documenti inclusi \LaTeX\
News \#22 e ``\LaTeX\ changes'' contengono i dettagli. Incidentalmente, i
pacchetti \pkgname{babel} e \pkgname{psnfss}, nonostante siano parti
centrali di \LaTeX, sono mantenuti separatamente e non sono toccati da
questi cambiamenti (e dovrebbero ancora funzionare).

Internamente, \LaTeXe\ ora include una configurazione del motore relativa
a Unicode (quali caratteri sono lettere, nome delle primitive, ecc.) che
prima era parte di \TeX\ Live. Questo cambiamento dovrebbe essere
trasparente per gli utenti; alcune sequenze di controllo interne di basso
livello sono state
rinominate o rimosse, ma il comportamento dovrebbe essere lo stesso.

pdf\TeX: supporto per gli Exif nei JPEG e per i JFIF; non
emette più un warning se \cs{pdfinclusionerrorlevel} è negativo;
sincronizzazione con \prog{xpdf}~3.04.

Lua\TeX: nuova libreria \pkgname{newtokenlib} per la scansione dei token;
correzione di errori nel generatore di numeri casuali \code{normal} e in
altri punti.

\XeTeX: corretta la gestione delle immagini; l'eseguibile \prog{xdvipdfmx}
viene cercato per primo, come gemello di \prog{xetex}; cambiati gli opcode 
interni
\code{XDV}.

MetaPost: nuovo sistema di numerazione \code{binary}; nuovi programmi per
il giapponese \prog{upmpost} e \prog{updvitomp}, analogamente a
\prog{up*tex}.

Mac\TeX: aggiornamenti al pacchetto Ghostscript per il supporto a
CJK. Il Pannello delle Preferenze della Distribuzione \TeX\ ora
funziona in Yosemite (\MacOSX~10.10). I resource fork dei font (di solito
senza un'estensione) non sono più supportati da \XeTeX; i data fork
(\code{.dfont}) restano supportati.

Infrastruttura: lo script \prog{fmtutil} è stato reimplementato per
leggere i \filename{fmtutil.cnf} rispettando le gerarchie delle directory, 
analogamente a
\prog{updmap}. Gli script Web2c \prog{mktex*} (inclusi \prog{mktexlsr},
\prog{mktextfm}, \prog{mktexpk}) ora preferiscono i programmi nelle loro
directory, invece di usare sempre il \envname{PATH} esistente.

Piattaforme: \pkgname{*-kfreebsd} rimossa, dato che \TeX\ Live è ora
facilmente disponibile tramite il meccanismo di piattaforma del sistema.
Il supporto per alcune piattaforme aggiuntive è disponibile come
eseguibili personalizzati (\url{https://tug.org/texlive/custom-bin.html}).
In più, alcune piattaforme adesso sono omesse dal \DVD\ (semplicemente per
risparmiare spazio), ma possono essere installate normalmente via
internet.

% 
\subsubsection{2016}

Lua\TeX: cambiamenti radicali alle primitive, che sono state sia
rinominate che rimosse, assieme ad alcune riorganizzazioni della
struttura dei nodi. Questi cambiamenti sono ricapitolati in un articolo
di Hans Hagen, ``Lua\TeX\ 0.90 backend changes for PDF and more''
(\url{https://tug.org/TUGboat/tb37-1/tb115hagen-pdf.pdf}); per tutti
i dettagli, consultate il manuale di Lua\TeX,
\OnCD{texmf-dist/doc/luatex/base/luatex.pdf}.

Metafont: nuovi programmi fratelli altamente sperimentali MFlua e
MFluajit che integrano Lua con \MF, per scopi di test.

MetaPost: correzione di bug e preparazioni interne per MetaPost 2.0.

pdf\TeX: nuove primitive \cs{pdfinfoomitdate}, \cs{pdftrailerid},
\cs{pdfsuppressptexinfo}, per controllare quei valori che appaiono
nell'output e che normalmente cambiano ad ogni esecuzione. Queste
funzionalità sono solo per l'output in PDF, non per il DVI.

Xe\TeX: nuove primitive \cs{XeTeXhypenatablelength},
\cs{XeTeXgenerateactualtext},\\ \cs{XeTeXinterwordspaceshaping},
\cs{mdfivesum}; limite delle classi di caratteri innalzato a 4096;
incremetati i byte di id nel DVI.

Altri strumenti:
\begin{itemize*}
\item \code{gregorio} è un nuovo programma che fa parte del pacchetto
\code{gregoriotex} per impaginare le partiture di canto gregoriano; è
incluso di default in \code{shell\_escape\_commads}.

\item \code{upmendex} è un programma di creazione dell'indice, quasi
del tutto compatibile con \code{makeindex}, con il supporto tra le
altre cose per l'ordinamento basato sull'Unicode.

\item \code{afm2tfm} ora esegue solo aggiustamenti dell'altezza basati
sull'accento verso l'alto; una nuova opzione \code{-a} tralascia tutti
gli aggiustamenti.

\item \code{ps2pk} può gestire i font PK/GF estesi.
\end{itemize*}

Mac\TeX: il pannello \TeX\ Distribution Preference è stato rimosso; le
sue funzionalità ora si trovano nella TeX Live Utility; aggiornate le
applicazioni grafiche a corredo; il nuovo script \code{cjk-gs-integrate}
deve essere eseguito dagli utenti che vogliono incorporare i font CJK in
Ghostscript.

Infrastruttura: supporto per il file di configurazione di \code{tlmgr}
a livello di sistema; verifica delle checksum dei pacchetti; se GPG è
disponibile, verifica la firma degli aggiornamenti di rete. Queste
verifiche avvengono sia nell'installatore che in \code{tlmgr}. Se
GPG non è disponibile, gli aggiornamenti avvengono come di consueto.

Piattaforme: rimosse \code{alpha-linux} e \code{mipsel-linux}.


\subsubsection{2017}

Lua\TeX: più callback, più controllo per la composizione tipografica, più 
accesso
ai dati inteni; libreria \code{ffi} per il caricamento dinamico del
codice in alcune piattaforme.

pdf\TeX: variabile d'ambiente |SOURCE_DATE_EPOCH_TEX_PRIMITIVES|
dell'anno precedente rinominata |FORCE_SOURCE_DATE|, con nessuna modifica
alle funzionalità; se la lista di token \cs{pdfpageattr} contiene la
stringa \code{/MediaBox}, omette l'output del \code{/MediaBox} predefinito.

Xe\TeX: font Unicode/OpenType Math ora basati sulla tabella di supporto
MATH di HarfBuzz; correzione di alcuni bug.

%CM non so come tradurre "special" in questo contesto, ma non penso sia "speciale"
%MP il dizionario https://www.wordreference.com/enit/special ha come
% possibile traduzione di "special", "offerta", "offerta speciale"; penso
% qui "special" possa avere un significato del genere, ma in italiano
% suona un po' male, quindi ho liberamente sostituito "offerta" con
% "proposta"
Dvips: fa sì che l'ultima proposta di dimensione carta vinca, per
coerenza con \code{dvipdfmx} e con le attese dei pacchetti; l'opzione
\code{-L0} (impostazione di configurazione \code{L0}) ripristina il
comportamento precedente in cui vince la prima proposta di dimensione.

ep\TeX, eup\TeX: nuove primitive \cs{pdfuniformdeviate},
\cs{pdfnormaldeviate}, \cs{pdfrandomseed}, \cs{pdfsetrandomseed},
\cs{pdfelapsedtime}, \cs{pdfresettimer}, da pdf\TeX.

Mac\TeX: a partire da quest'anno, solo le versioni di \MacOSX\ per
cui Apple ancora rilascia aggiornamenti di sicurezza saranno supportate
in Mac\TeX, sotto il nome di piattaforma |x86_64-darwin|; attualmente
ciò significa Yosemite, El~Capitan e Sierra (10.10 e successivi). Gli
eseguibili per le versioni precedenti di \MacOSX\ non sono incluse in
Mac\TeX, ma sono ancora disponibili in \TeX\ Live (|x86_64-darwinlegacy|,
\code{i386-darwin}, \code{powerpc-darwin}).

Infrastruttura: la directory \envname{TEXMFLOCAL} viene ora esplorata prima
di \envname{TEXMFSYSCONFIG} e \envname{TEXMFSYSVAR} (di default); la
speranza è che questo corrisponda meglio alle attese dei file locali che
sostituiscono quelli di sistema. Inoltre, \code{tlmgr} ha una nuova
modalità \code{shell} per l'uso interattivo e in script e una nuova
azione \code{conf auxtrees} per aggiungere e rimovere con facilità
ulteriori directory.

\code{updmap} e \code{fmtutil}: questi script ora avvisano quando sono
chiamati senza specificare esplicitamente la cosiddetta modalità di
sistema (\code{updmap-sys}, \code{fmtutil-sys} o l'opzione \code{-sys}),
o la modalità utente (\code{updmap-user}, \code{fmtutil-user} o
l'opzione \code{-user}).
La speranza è che questo riduca il costante problema di invocare la
modalità utente per sbaglio e quindi perdere i futuri aggiornamenti di
sistema. Consultate
\url{https://tug.org/texlive/scripts-sys-user.html} per i dettagli.

\code{install-tl}: i percorsi personali come \envname{TEXMFHOME} su Mac sono
ora impostati di default agli stessi valori di Mac\TeX\ (|~/Library/...|).
Nuova opzione \code{-init-from-profile} per iniziare un'installazione con i
valori da un dato profilo; nuovo comando \code{P} per salvare esplicitamente un
profilo; nuovi nomi di variabili di profilo (ma i precedenti sono ancora
accettati).

Sync\TeX: il nome del file temporaneo ora somiglia a 
\code{foo.synctex(busy)} e non a \code{foo.synctex.gz(busy)}
(niente~\code{.gz}). Le applicazioni e i sistemi di compilazione che
vogliono rimuovere i file temporanei potrebbero richiedere un adattamento.

Altri strumenti: \code{texosquery-jre8} è un nuovo programma
multi-piattaforma per recuperare alcune informazioni di ambiente e del
sistema operativo da un documento \TeX; di default è incluso in
|shell_escape_command| per essere eseguito in modalità ristretta (versioni
precedenti della JRE sono supportate da texosquery, ma non possono essere
attivate in modalità ristretta perché non sono più supportate da Oracle,
neppure per gli aggiornamenti di sicurezza).

Piattaforme: consultate il paragrafo su Mac\TeX\ qui sopra; nessun altro
cambiamento.


\subsection{2018}

Kpathsea: la ricerca di file nelle directory non di sistema è ora per default
insensibile alle maiuscole; per disattivare questo comportamento, impostate il
parametro \code{texmf\_casefold\_search} a~\code{0} in \code{texmf.cnf} o nella
variabile d'ambiente. Tutti i dettagli nel manuale di Kpathsea
(\url{https://tug.org/kpathsea}).

ep\TeX, eup\TeX: nuova primitiva \cs{epTeXversion}.

Lua\TeX: preparazione per migrare a Lua 5.3 nel 2019: un eseguibile
\code{luatex53} è disponibile nella maggior parte delle piattaforme, ma deve
essere rinominato in \code{luatex} per funzionare. Altrimenti usate i file di
\ConTeXt\ Garden (\url{https://wiki.contextgarden.net}); visitate il sito per
maggiori informazioni.

MetaPost: correzioni per direzioni di path sbagliate,  per output di TFM e di PNG.

pdf\TeX: consente i vettori di codifica per i font bitmap; la directory attuale
non è usata per generare l'ID del PDF; fix di bug per \cs{pdfprimitive} e
correlati.

Xe\TeX: supporto per \code{/Rotate} nell'inclusione di immagini nel PDF;
termina con un codice di uscita diverso da zero se il driver in uscita
fallisce; varie correzioni a primitive per UTF-8 e altro.

Mac\TeX: consultate qui sotto i cambiante sul supporto alle versioni. Oltre a
questo, i file installati in \code{/Applications/TeX/} da Mac\TeX\ sono stati
riorganizzati per maggiore chiarezza; ora questa directory contiene quattro
programmi grafici (BibDesk, LaTeXiT, TeX Live Utility e TeXShop) e ha varie
sottodirectory con utility e documentazione aggiuntive.

\code{tlmgr}: nuove interfacce grafiche \code{tlshell} (Tcl/Tk) e
\code{tlcockpit} (Java); output in JSON; \code{uninstall} è ora un sinonimo
di \code{remove}; nuova azione/opzione \code{print-platform-info}.

Piattaforme:
\begin{itemize*}
\item
Rimosse: \code{armel-linux}, \code{powerpc-linux}.

\item \code{x86\_64-darwin} supporta le versioni 10.10--10.13
(Yosemite, El~Capitan, Sierra e High~Sierra).

\item \code{x86\_64-darwinlegacy} supporta le versioni 10.6--10.10 (sebbene
\code{x86\_64-darwin} sia preferito per la 10.10). Tutto il supporto per la
versione 10.5 (Leopard) è andato via, nel senso che sia \code{powerpc-darwin}
che \code{i386-darwin platforms} sono stati rimossi.

\item Windows: XP non è più supportato.
\end{itemize*}


\subsection{2019}

Kpathsea: suddivisione dei path ed espansione delle graffe più consistenti;
nuova variabile \code{TEXMFDOTDIR} al posto del \code{.}\ esplicito nei path
consente di cercare più facilmente in sotto directory aggiuntive (consultate
i commenti in \code{texmf.cnf}).

ep\TeX, eup\TeX: nuove primitive \cs{readpapersizespecial} e \cs{expanded}.

Lua\TeX: adesso si usa Lua 5.3, con i concomitanti cambiamenti all'aritmetica
e all'interfaccia. La libreria pplib fatta in casa è usata per leggere i file
pdf, così eliminando la dipendenza da poppler (e la necessità del C++);
l'interfaccia Lua è cambiata di conseguenza.

MetaPost: il nome del comando \code{r-mpost} è riconosciuto come un alias
per l'invocazione con l'opzione \code{--restricted} ed è stato aggiunto alla
lista dei comandi riservati disponibili di default.
La precisione minima ora è di 2 per la modalità binaria e decimale.
La modalità binaria non è più disponibile in MPlib ma è ancora disponibile
nel MetaPost a sé stante.

pdf\TeX: nuova primitiva \cs{expanded}; se il nuovo parametro primitivo
\cs{pdfomitcharset} è impostato a 1, la stringa \code{/CharSet} è omessa
dall'output PDF, dato che non può essere garantita corretta in modo fattibile,
come richiesto da PDF/A-2 e PDF/A-3.

Xe\TeX: nuove primitive \cs{expanded},
\cs{creationdate},
\cs{elapsedtime},
\cs{filedump}, 
\cs{filemoddate}, 
\cs{filesize}, 
\cs{resettimer}, 
\cs{normaldeviate}, 
\cs{uniformdeviate}, 
\cs{randomseed}; esteso \cs{Ucharcat} per produrre i caratteri attivi.

\code{tlmgr}: supporto per \code{curl} come programma di download; uso di
  \code{lz4} e gzip prima di \code{xz} per i backup locali, se disponibile;
  preferisce i binari forniti dal sistema a quelli forniti da \TL\ per i
  programmi di compressione e download, a meno che la variabile d'ambiente
  \code{TEXLIVE\_PREFER\_OWN} è impostata.

\code{install-tl}: la nuova opzione \code{-gui} (senza argomenti) è predefinita
su Windows e Macs e lancia una nuova GUI in Tcl/Tk (vedi le
sezioni~\ref{sec:basic} e~\ref{sec:graphical-inst}).

Utilità:
\begin{itemize*}
\item \code{cwebbin} (\url{https://ctan.org/pkg/cwebbin}) è ora l'implementazione
di CWEB in \TeX\ Live, con il supporto per più dialetti di lingue e l'inclusione
del programma \code{ctwill} per creare mini-indici.

\item \code{chkdvifont}: riporta le informazioni sui font dai file \dvi{} e
anche da tfm/ofm, vf, gf, pk.

\item \code{dvispc}: rende un file DVI indipendente dalla pagina rispetto agli
specials.
\end{itemize*}

Mac\TeX: \code{x86\_64-darwin} ora supporta la versione 10.12 e successive
(Sierra, High Sierra, Mojave); \code{x86\_86-darwinlegacy} supporta ancora
la versione 10.6 e successive. Il controllore ortografico Excalibur non è
più incluso in quanto richiede il supporto a 32 bit.

Piattaforme: rimossa \code{sparc-solaris}.


\htmlanchor{news}
\subsection{Presente: 2020}
\label{sec:tlcurrent}

Generale: \begin{itemize}
\item La primitiva \cs{input} in tutti i motori \TeX, incluso \texttt{tex},
ora accetta anche come argomento un nome di file delimitato da gruppo, come
estensione dipendente dal sistema. L'uso con il solito filename delimitato da
spazi/token non è cambiato. L'argomento delimitato da gruppo era implementato
precedentemente in Lua\TeX; ora è disponibile per tutti i motori. I caratteri
ASCII dei doppi apici (\texttt{"}) sono rimossi dal nome del file, ma oltre
questo resta invariato dopo la divisione in token. Questo non riguarda
attualmente il comando \cs{input} di \LaTeX\ in quanto è una ridefinizione
macro della primitiva standard \cs{input}.

\item Nuova opzione \texttt{--cnf-lline} per \texttt{kpsewhich}, \texttt{tex},
\texttt{mf} e tutti gli altri motori per supportare impostazioni di
configurazione arbitrarie sulla riga di comando.

\item L'aggiunta delle varie primitive ai diversi motori in questo e nei
precedenti anni è pensata per concludersi con un insieme comune di funzionlità
disponibili tra tutti i motori (\textsl{\LaTeX\ News \#31},
\url{https://latex-project.org/news}).

\end{itemize}

ep\TeX, eup\TeX: nuove primitive \cs{Uchar}, \cs{Ucharcat},
\cs{current(x)spacingmode}, \cs{ifincsname}; rivedere \cs{fontchar??} e
\cs{iffontchar}. Per il solo eup\TeX: \cs{currentcjktoken}.

Lua\TeX: integrazione con la libreria HarfBuzz disponibile come nuovo motore
\texttt{luahbtex} (usato per \texttt{lualatex}) e \texttt{luajithbtex}. Nuove
primitive: \cs{eTeXgluestretchorder}, \cs{eTeXglueshrinkorder}.

pdf\TeX: nuova primitiva \cs{pdfmajorversion}; questa semplicemente cambia il
numero di versione del PDF prodotto; non ha effetto su alcun contenuto del
PDF. \cs{pdfximage} e simili ora cercano i file di immagine nello stesso modo
di \cs{openin}.

p\TeX: nuove primitive \cs{ifjfont}, \cs{iftfont}. Anche in ep\TeX, up\TeX,
eup\TeX.

Xe\TeX: correzioni per \cs{Umathchardef}, \cs{XeTeXinterchartoks},
\cs{pdfsavepos}.

Dvips: codifiche in uscita per i font bitmap, per migliori capacità di
copia/incolla
(\url{https://tug.org/TUGboat/tb40-2/tb125rokicki-type3search.pdf}).

Mac\TeX: Mac\TeX\ e \texttt{x86\_64-darwin} ora richiedono la 10.13 o
successive (High~Sierra, Mojave e Catalina); \texttt{x86\_64-darwinlegacy}
supporta la 10.6 e successive. Mac\TeX\ è certificato e i programmi a riga di
comando hanno runtime irrobustiti come richiesto ora da Apple per i pacchetti
di installazione. BibDesk e \TeX\ Live Utility non fanno parte di Mac\TeX\
perché non sono certificati, ma un file \filename{README} elenca gli indirizzi
da dove possono essere ottenuti.

\code{tlmgr} e infrastruttura: \begin{itemize*}
\item prova di nuovo automaticamente (una sola volta) i pacchetti il cui
download è fallito;
\item nuova opzione \texttt{tlmgr check texmfdbs} per verificare la
consistenza dei file \texttt{ls-R} e delle specifiche \texttt{!!} in ogni
sotto directory;
\item usa nomi di file versionati per i container dei pacchetti, come in
\texttt{tlnet/archive/\textsl{pkgname}.rNNN.tar.xz}; dovrebbe essere
invisibile agli utenti, ma è un'importante cambiamento nella distribuzione;
\item le informazioni \texttt{catalogue-date} non sono più propagate dal
\TeX~Catalogue, dato che erano spesso non collegate agli aggiornamenti dei
pacchetti.
\end{itemize*}



\subsection{Futuro}

\TL{} non è perfetto, e mai lo sarà. Abbiamo intenzione di continuare a
rilasciare nuove versioni e vorremmo fornire più documentazione, più programmi
e un insieme sempre migliore e meglio controllato di font, di macro e di tutto
quanto legato a \TeX.  Questo lavoro è fatto completamente da volontari nel
loro tempo libero e quindi c'è sempre molto da fare. Visitate il sito
\url{https://tug.org/texlive/contribute.html}.

Potete inviare correzioni, suggerimenti e offerte d'aiuto a:
\begin{quote}
\email{tex-live@tug.org} \\
\url{https://tug.org/texlive}
\end{quote}

\medskip
\noindent \textsl{Buon lavoro con \TeX!}

\end{document}
