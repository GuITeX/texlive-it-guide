% $Id$
% TeX Live documentation.  Originally written by Sebastian Rahtz and
% Michel Goossens, now maintained by Karl Berry and others.
% Public domain.
%
% Traduzione italiana a cura di Marco Pallante.
%
\documentclass{article}
%\let\tldocenglish=1  % for live4ht.cfg
\usepackage{tex-live}
\usepackage[italian]{babel}
\usepackage[utf8]{inputenc}
%\usepackage[latin1]{inputenc} % translators: use your preferred encodings.
\usepackage[T1]{fontenc}

\begin{document}

\title{%
  {\huge \textit{Guida a \TeX\ Live}\\\smallskip}%
  {\LARGE \textsf{\TL{} 2009}}%
%  {\huge \textit{The \TeX\ Live Guide}\\\smallskip}%
%  {\LARGE \textsf{\TL{} 2009}}%
}

\author{Karl Berry\\[3mm]
        \url{http://tug.org/texlive/}
       }
%\author{Karl Berry, editor \\[3mm]
%        \url{http://tug.org/texlive/}
%       }

\date{Settembre 2009}
%\date{September 2009}

\maketitle

\begin{multicols}{2}
\tableofcontents
%\listoftables
\end{multicols}


\section*{Note all'edizione italiana}

Dopo tanti anni, finalmente abbiamo anche un'edizione italiana di questa
guida alla distribuzione \TL. Vorrei premettere che, per garantirne il
completamento entro l'uscita di \TL{} 2009, mi sono fatto in quattro
destinando praticamente ogni attimo libero a questo lavoro; ho tenuto
sempre acceso il computer, con l'editor aperto e, spesso, passandoci
davanti, mi fermavo a tradurre una sola frase o addirittura una sola
parola pur di non restare fermo.

Ovviamente il risultato è un lavoro fatto con i piedi, uno stile
incoerente, parti che nemmeno io riesco a capire, errori di battitura e di
grammatica. Di tutto questo chiedo scusa. Però, adesso, il grosso dello
sforzo è stato fatto e revisionare periodicamente questa guida per limarla
e migliorarla sarà un'opera di gran lunga più semplice.

Per quanto riguarda lo stile che ho adottato, ho deciso di essere diretto
e colloquiale con il lettore, dandogli del ``tu''. Sapendo che in media
l'utente italiano ha una conoscenza informatica non elevata, ho deciso di
tradurre, spesso con perifrasi, concetti quali ``file system'',
``database'', ``download''.

Spero che, nonostante tutti i difetti, il mio piccolo contributo a \TeX{}
sia apprezzato. Potete contattarmi per qualunque cosa, aiuti,
suggerimenti, correzioni, all'indirizzo \email{marco.pallante@gmail.com}.

Vorrei dedicare questo lavoro alla mia città.

\bigskip

\noindent Marco Pallante\\
\emph{L'Aquila, 5 ottobre 2009}.


\section{Introduzione}
%\section{Introduction}
\label{sec:intro}

\subsection{\TeX\ Live e la \TeX\ Collection}
%\subsection{\TeX\ Live and the \TeX\ Collection}

Questo documento descrive le caratteristiche principali della
distribuzione \TL{}\Dash \TeX{} e i programmi ad esso correlati
per i sistemi \GNU/Linux ed altre versioni di Unix, per \MacOSX
e Windows.
%This document describes the main features of the \TL{} software
%distribution\Dash \TeX{} and related programs for \GNU/Linux
%and other Unix flavors, \MacOSX, and Windows systems.

Potete ottenere \TL{} scaricandola, oppure sul \DVD{} \TK, che
i gruppi di utenti \TeX{} distribuiscono tra i propri membri.
La sezione \ref{sec:tl-coll-dists} descrive brevemente il
contenuto di questo \DVD. Sia \TL{} che \TK{} sono progetti
cooperativi dei gruppi di utenti \TeX. Questo documento descrive
principalmente \TL.
%You may have acquired \TL{} by downloading, or on the \TK{} \DVD,
%which \TeX{} usergroups distribute among their members. Section
%\ref{sec:tl-coll-dists} briefly describes the contents of this \DVD.
%Both \TL{} and the \TK{} are cooperative efforts by the \TeX{} user
%groups. This document mainly describes \TL{} itself.

\TL{} include gli eseguibili per \TeX, \LaTeXe, \ConTeXt, \MF,
\MP, \BibTeX{} e molti altri programmi, una vasta collezione di macro,
font e documentazione, e il supporto per la composizione tipografica
in molti diversi alfabeti da tutto il mondo.
%\TL{} includes executables for \TeX{}, \LaTeXe{}, \ConTeXt,
%\MF, \MP, \BibTeX{} and many other programs; an extensive collection
%of macros, fonts and documentation; and support for typesetting in
%many different scripts from around the world.

Per un breve sommario dei principali cambiamenti in questa edizione
di \TL, consulta la fine del documento, sezione~\ref{sec:history}
(\p.\pageref{sec:history}).
%For a brief summary of the major changes in this edition of \TL{},
%see the end of the document, section~\ref{sec:history}
%(\p.\pageref{sec:history}).

\subsection{Supporto per i sistemi operativi}
%\subsection{Operating system support}
\label{sec:os-support}

\TL{} contiene gli eseguibili per molte architetture basate su Unix,
incusi \GNU/Linux e \MacOSX. Ci sono anche gli eseguibili per Cygwin.
I sorgenti inclusi possono essere compilati per quelle piattaforme
per le quali non abbiamo gli eseguibili.
%\TL{} contains binaries for many Unix-based architectures, including
%\GNU/Linux and \MacOSX. There are also Cygwin binaries.  The
%included sources can be compiled on platforms for which we do not
%have binaries.

Per quanto riguarda Windows: solo Windows 2000 e i successivi sono
supportati. Windows~9\textit{x}, \acro{ME} e \acro{NT} sono stati
abbandonati. A causa di questo cambiamento, Windows richiede un
trattamento molto meno speciale rispetto ai sistemi Unix. Non ci
sono eseguibili speciali a 64 bit per Windows, ma quelli a 32 bit
dovrebbero funzionare sui sistemi a 64 bit.
%As to Windows: only Windows 2000 and later are
%supported. Windows~9\textit{x}, \acro{ME} and \acro{NT} have been
%dropped. Because of this change, Windows requires much less special
%treatment compared to the Unix systems. There are no special 64-bit
%executables for Windows, but the 32-bit executables should run on 64-bit
%systems.

Consulta la sezione~\ref{sec:tl-coll-dists} per soluzioni alternative
per Windows e \MacOSX.
%See section~\ref{sec:tl-coll-dists} for alternate solutions
%for Windows and \MacOSX.

\subsection{Installazione base di \protect\TL{}}
%\subsection{Basic installation of \protect\TL{}}
\label{sec:basic}

Puoi installare \TL{} sia dal \DVD{} che attraverso Internet. Il
programma di installazione di rete è piccolo e scarica tutto il
necessario da Internet. L'installatore di rete è un'opzione
attraente se hai bisogno solo di una frazione di \TL{} completo.
%You can install \TL{} either from \DVD{} or over the Internet. The net
%installer itself is small, and downloads everything requested from the
%Internet. The net installer is an attractive option if you need only a
%fraction of the complete \TL.

Il programma di installazione nel \DVD{} permette l'installazione su
disco, ma anche di eseguire \TL{} direttamente dal \DVD{} (o da
un'immagine, se il tuo sistema lo supporta). L'installazione è 
descritta nelle prossime sezioni, ma eccone un rapido accenno:
%The \DVD{} installer lets you install to a local disk, but you can also
%run \TL{} directly from the \DVD{} (or from a \DVD{} image, if your
%system supports that).  Installation is described in later sections, but
%here is a quick start:

\begin{itemize*}

\item Lo script di installazione è chiamato \filename{install-tl}.
  Può operare in ``modalità guidata'' (\emph{wizard}) dando l'opzione
  \code{-gui=wizard} (predefinita sotto Windows), in modalità testuale
  dando l'opzione \code{-gui=text} (predefinito per gli altri sistemi) e
  in modalità \GUI{} esperta dando \code{-gui=perltk}.
%\item The installation script is named \filename{install-tl}.  It can
%  operate in a ``wizard mode'' given the option \code{-gui=wizard}
%  (default for Windows), a text mode given \code{-gui=text} (default for
%  everything else), and an expert \GUI{} mode given \code{-gui=perltk}.

\item Uno degli elementi installati è il programma ``\TL\ Manager'',
  chiamato \prog{tlmgr}. Così come l'installatore, può essere usato si in
  modalità \GUI{} che in modalità testuale. Puoi usarlo per installare e
  disinstallare i pacchetti ed eseguire vari compiti di configurazione.
%\item One of the installed items is the `\TL\ Manager' program,
%  named \prog{tlmgr}.  Like the installer, it can be used in both \GUI{}
%  mode and in text mode. You can use it to install and uninstall
%  packages and do various configuration tasks.

\end{itemize*}


\subsection{Ottenere aiuto}
%\subsection{Getting help}
\label{sec:help}

La comunità \TeX{} è attiva ed amichevole e le domande più importanti
finiscono per essere soddisfatte. Tuttavia il supporto è informale, dato
da volontari e lettori casuali, per cui è particolarmente importante che
tu faccia i tuoi compiti a casa prima di chiedere (se preferisci un
supporto commerciale garantito, puoi rinunciare del tutto a \TL{} e
acquistare il sistema di un fornitore; alla pagina
\url{http://tug.org/interest.html#vendors} ne trovi una lista).
%The \TeX{} community is active and friendly, and most serious questions
%end up getting answered.  However, the support is informal, done by
%volunteers and casual readers, so it's especially important that you do
%your homework before asking.  (If you prefer guaranteed commercial
%support, you can forgo \TL{} completely and purchase a vendor's system;
%\url{http://tug.org/interest.html#vendors} has a list.)

Ecco una lista di risorse, approssimativamente nell'ordine in cui noi
raccomandiamo di usarle:
%Here is a list of resources, approximately in the order we recommend
%using them:

\begin{description}
\item [Per Cominciare] Se sei nuovo di \TeX, alla pagina web
  \url{http://tug.org/begin.html} avrai una breve introduzione al sistema.
%\item [Getting Started] If you are new to \TeX, the web page
%\url{http://tug.org/begin.html} gives a brief introduction to the system.

\item [\TeX{} \acro{FAQ}] Le \TeX{} \acro{FAQ} sono un vasto compendio di
  risposte ad ogni genere di domanda, dalle più elementari alle più
  oscure. Sono incluse in \TL{} in
  \OnCD{texmf-dist/doc/generic/FAQ-en/html/index.html} e sono disponibili
  su Internet alla pagina \url{http://www.tex.ac.uk/faq}. Visitale per
  prime.
%\item [\TeX{} \acro{FAQ}] The \TeX{} \acro{FAQ} is a huge compendium
%  of answers to all sorts of questions, from the most basic to the
%  most arcane.  It is included on \TL{} in
%  \OnCD{texmf-dist/doc/generic/FAQ-en/html/index.html}, and is available
%  on the Internet through \url{http://www.tex.ac.uk/faq}.  Please
%  check here first.

\item [\TeX{} Catalogue] Se stai cercando uno specifico pacchetto, font,
  programma, ecc., il \TeX{} Catalogue è il posto da visitare. È una
  enorme catalogo di tutte le voci connesse a \TeX. Visita
  \url{http://www.ctan.org/help/Catalogue/}.
%\item [\TeX{} Catalogue] If you are looking for a specific package,
%font, program, etc., the \TeX{} Catalogue is the place to look.  It is a
%huge collection of all \TeX{}-related items.  See
%\url{http://www.ctan.org/help/Catalogue/}.

\item [Risorse Web per \TeX{}] La pagina web
  \url{http://tug.org/interest.html} ha molti collegamenti legati a
  \TeX{}, in particolare a numerosi libri, manuali ed articoli su tutti
  gli aspetti del sistema.
%\item [\TeX{} Web Resources] The web page
%\url{http://tug.org/interest.html} has many \TeX{}-related links, in
%particular for numerous books, manuals, and articles on all aspects of
%the system.

\item [archivi di supporto] I due principali forum di supporto sono il
  gruppo Usenet \url{news:comp.text.tex} e la mailing list
  \email{texhax@tug.org}. I loro archivi hanno anni di domande e risposte
  per il vostro piacere di ricerca tramite
  \url{http://groups.google.com/group/comp.text.tex/topics} e
  \url{http://tug.org/mail-archives/texhax}, rispettivamente. E una
  generica ricerca su web, ad esempio da \url{http://google.com}, non fa
  mai male.
%\item [support archives] The two principal support forums are the
%Usenet newsgroup \url{news:comp.text.tex} and the mailing list
%\email{texhax@tug.org}.  Their archives have years of past
%questions and answers for your searching pleasure, via
%\url{http://groups.google.com/group/comp.text.tex/topics} and
%\url{http://tug.org/mail-archives/texhax}, respectively.  And a general web
%search, for example on \url{http://google.com}, never hurts.

\item [porre domande] Se non riesci a trovare una risposta, puoi inviare
  una domanda a \dirname{comp.text.tex} tramite Google o un programma per
  Usenet, oppure per posta elettronica a \email{texhax@tug.org}. Ma prima
  di scrivere, \emph{leggi} questa voce delle \acro{FAQ}, al fine di
  massimizzare le probabilità di ottenere una risposta utile:
  \url{http://www.tex.ac.uk/cgi-bin/texfaq2html?label=askquestion}.
%\item [asking questions] If you cannot find an answer, you can post to
%\dirname{comp.text.tex} through Google or your newsreader, or to
%\email{texhax@tug.org} through email.  But before you post,
%\emph{please} read this \acro{FAQ} entry, to maximize
%your chances of getting a useful answer:
%\url{http://www.tex.ac.uk/cgi-bin/texfaq2html?label=askquestion}.

\item [supporto \TL{}] Se vuoi segnalare un bug o hai suggerimenti o
  commenti sulla distribuzione \TL{}, sull'installazione o la
  documentazione, la mailing list è \email{tex-live@tug.org}. Tuttavia, se
  la tua domanda è relativa all'uso di un particolare programma incluso in
  \TL{}, scrivi al mantenitore o alla mailing list di quel programma.
  Spesso eseguire un programma con l'opzione \code{-{}-help} fornisce un
  indirizzo al quale segnalare i bug.
%\item [\TL{} support] If you want to report a bug or have
%suggestions or comments on the \TL{} distribution, installation, or
%documentation, the mailing list is \email{tex-live@tug.org}.  However,
%if your question is about how to use a particular program included in
%\TL{}, please write to that program's maintainer or
%mailing list.  Often running a program with the \code{-{}-help} option
%will provide a bug reporting address.

\end{description}

L'altra faccia della medaglia è aiutare coloro che hanno domande. Sia
\dirname{comp.text.tex} che \code{texhax} sono aperte a chiunque, quindi
sentiti libero di unirti, iniziare a leggere e dare una mano dove puoi. 
%The other side of the coin is helping others who have questions.  Both
%\dirname{comp.text.tex} and \code{texhax} are open to anyone, so feel
%free to join, start reading, and help out where you can.


% don't use \TL so the \uppercase in the headline works.  Also so
% tex4ht ends up with the right TeX.  Likewise the \protect's.
\section{Panoramica di \protect\TeX\protect\ Live}
%\section{Overview of \protect\TeX\protect\ Live}
\label{sec:overview-tl}

Questa sezione descrive i contenuti di \TL{} e della \TK{} di cui è parte.
%This section describes the contents of \TL{} and the \TK{} of which it
%is a part.

\subsection{La \protect\TeX\protect\ Collection: \TL, pro\TeX{}t, Mac\TeX}
%\subsection{The \protect\TeX\protect\ Collection: \TL, pro\TeX{}t, Mac\TeX}
\label{sec:tl-coll-dists}

Il \DVD{} \TK{} include quanto segue:
%The \TK{} \DVD{} comprises the following:

\begin{description}

\item [\TL] Un sistema \TeX{} completo che può essere eseguito senza
  installazione, oppure può essere installato su disco. La sua pagina web
  è \url{http://tug.org/texlive/}.
%\item [\TL] A complete \TeX{} system which can be run live or
%  installed to disk.  Its home page is \url{http://tug.org/texlive/}.

\item [Mac\TeX] per \MacOSX, aggiunge un installatore per \MacOSX\ nativo
  ed altre applicazioni Mac a \TL{}. La sua pagina web è
  \url{http://tug.org/mactex/}.
%\item [Mac\TeX] for \MacOSX, this adds a native \MacOSX\ installer and other
%Mac applications to \TL{}.  Its home page is
%\url{http://tug.org/mactex/}.

\item [pro\TeX{}t] Un miglioramento della distribuzione \MIKTEX\ per
  Windows, \ProTeXt\ aggiunge alcuni strumenti aggiuntivi a \MIKTEX\ e
  semplifica l'installazione. È interamente indipendente da \TL e ha le
  proprie istruzioni per l'installazione. La pagina web di \ProTeXt\ è
  \url{http://tug.org/protext}.  
%\item [pro\TeX{}t] An enhancement of the \MIKTEX\ distribution for Windows,
%\ProTeXt\ adds a few extra tools to \MIKTEX, and simplifies
%installation.  It is entirely independent of \TL{}, and has its own
%installation instructions.  The \ProTeXt\ home page is
%\url{http://tug.org/protext}.

\item [CTAN] Un'instantanea dell'archivio \CTAN{}
  (\url{http://www.ctan.org}).
%\item [CTAN] A snapshot of the \CTAN{} repository (\url{http://www.ctan.org}).

\item [\texttt{texmf-extra}] Una directory con pacchetti aggiuntivi
  assortiti.
%\item [\texttt{texmf-extra}] A directory with assorted additional packages.

\end{description}

\CTAN{}, \pkgname{protext} e \dirname{texmf-extra} non seguono
necessariamente le stesse condizioni di copia di \TL{}, per cui fai
attento nel ridistribuirli o modificarli.
%\CTAN{}, \pkgname{protext}, and \dirname{texmf-extra} do not
%necessarily follow the same copying conditions as \TL{}, so be careful
%when redistributing or modifying.


\subsection{Directory primarie di \TL{}}
%\subsection{Top level \TL{} directories}
\label{sec:tld}

Segue un breve elenco e una descrizione delle directory primarie nella
distribuzione \TL. Sul \DVD{} \TK, l'intera gerarchia di directory di
\TL{} si trova all'interno di \dirname{texlive} e non direttamente
all'apertura del disco.
%Here is a brief listing and description of the top level directories in the
%\TL{} distribution.  On the \TK{} \DVD, the entire \TL{} hierarchy is in
%a subdirectory \dirname{texlive}, not at the top level of the disc.

\begin{ttdescription}
\item[bin] I programmi del sistema \TeX{}, organizzati per piattaforma.
%\item[bin] The \TeX{} system programs, arranged by platform.
%
\item[readme-*.dir] Rapida panoramica e collegamenti utili per \TL{}, in
  varie lingue, sia in \HTML{} che in formato testo.
%\item[readme-*.dir] Quick overview and useful links for \TL{},
%in various languages, in both \HTML{} and plain text.
%
\item[source] I sorgenti di tutti i programmi inclusi, comprese le
  distribuzioni \Webc{}, \TeX{} e \MF{}.
%\item[source] The source to all included programs, including the main \Webc{}
%  \TeX{} and \MF{} distributions.
%
\item[texmf] Vedi \dirname{TEXMFMAIN} sotto.
%\item[texmf] See \dirname{TEXMFMAIN} below.
%
\item[texmf-dist] Vedi \dirname{TEXMFDIST} sotto.
%\item[texmf-dist] See \dirname{TEXMFDIST} below.
%
\item[tlpkg] Script, programmi e dati per l'amministrazione
  dell'installazione e del software di supporto specifico per Windows.
%\item[tlpkg] Scripts, programs and data for managing the
%  installation, and some special support for Windows.
\end{ttdescription}

In aggiunta alle precedenti directory, gli script di installazione e i
file \filename{README} (in varie lingue) si trovano nella directory della
distribuzione.
%In addition to the directories above, the installation scripts and
%\filename{README} files (in various languages) are at the top level of
%the distribution.

Per quanto riguarda la documentazione, i collegamenti nel file
\OnCD{doc.html} possono risultare utili. La documentazione per i programmi
(manuali, pagine di manuale, file Info) si trova in \dirname{texmf/doc}.
La documentazione per i pacchetti ed i formati \TeX\ si trova nella
directory \dirname{texmf-dist/doc}. Puoi usare il programma
\cmdname{texdoc} per trovare una qualsiasi documentazione, ovunque sia
collocata.
%For documentation, the comprehensive links in the top-level file
%\OnCD{doc.html} may be helpful.  The documentation for the programs
%(manuals, man pages, Info files) is in \dirname{texmf/doc}.  The
%documentation for \TeX\ packages and formats is in
%\dirname{texmf-dist/doc}.  You can use the \cmdname{texdoc} program to
%find any documentation wherever it is located.

Questa stessa documentazione di \TL\ si trova in
\dirname{texmf/doc/texlive}, disponibile in varie lingue:
%This \TL\ documentation itself is in \dirname{texmf/doc/texlive},
%available in several languages:

\begin{itemize*}
\item{Ceco/Slovacco:} \OnCD{texmf/doc/texlive/texlive-cz}
\item{Cinese Semplificato:} \OnCD{texmf/doc/texlive/texlive-zh-cn}
\item{Francese:} \OnCD{texmf/doc/texlive/texlive-fr}
\item{Inglese:} \OnCD{texmf/doc/texlive/texlive-en}
\item{Italiano:} \OnCD{texmf/doc/texlive/texlive-it}
\item{Polacco:} \OnCD{texmf/doc/texlive/texlive-pl}
\item{Russo:} \OnCD{texmf/doc/texlive/texlive-ru}
\item{Tedesco:} \OnCD{texmf/doc/texlive/texlive-de}
%\item{Simplified Chinese:} \OnCD{texmf/doc/texlive/texlive-zh-cn}
%\item{Czech/Slovak:} \OnCD{texmf/doc/texlive/texlive-cz}
%\item{English:} \OnCD{texmf/doc/texlive/texlive-en}
%\item{French:} \OnCD{texmf/doc/texlive/texlive-fr}
%\item{German:} \OnCD{texmf/doc/texlive/texlive-de}
%\item{Italian:} \OnCD{texmf/doc/texlive/texlive-it}
%\item{Polish:} \OnCD{texmf/doc/texlive/texlive-pl}
%\item{Russian:} \OnCD{texmf/doc/texlive/texlive-ru}
\end{itemize*}

\subsection{Panoramica dei percorsi texmf predefiniti}
%\subsection{Overview of the predefined texmf trees}
\label{sec:texmftrees}

Questa sezione elenca le variabili predefinite che specificano i percorsi
texmf usati dal sistema, il loro scopo e la strutturazione predefinita di
\TL{}.  Il comando \texttt{tlmgr~conf} mostra i valori di queste
variabili, così che tu possa scoprire facilmente come sono riferite a
specifiche directory nella tua installazione.
%This section lists the predefined variables specifying the texmf trees
%used by the system, and their intended purpose, and the default layout
%of \TL{}. The command \texttt{tlmgr~conf} shows the values of these
%variables, so that you can easily find out how they map to particular
%directories in your installation.

\begin{ttdescription}
\item [TEXMFMAIN] Il percorso che contiene le parti vitali del sistema
  come i file di configurazione, gli script di aiuto e la documentazione
  dei programmi.
%\item [TEXMFMAIN] The tree which holds vital parts of the
%  system such as configuration files, helper scripts, and program
%  documentation.
\item [TEXMFDIST] Il percorso che contiene il gruppo principale di
  pacchetti di macro, font, ecc.
%\item [TEXMFDIST] The tree which holds the main set of macro packages,
%  fonts, etc.
\item [TEXMFLOCAL] Il percorso che un amministratore può usare per
  l'installazione nell'intero sistema di macro, font, ecc., aggiuntivi o
  aggiornati.
%\item [TEXMFLOCAL] The tree which administrators can use for system-wide
%  installation of additional or updated macros, fonts, etc.
\item [TEXMFHOME] Il percorso che ciascun utente può usare per la propria
  installazione personale di macro, font, ecc., aggiuntivi o aggiornati.
  Quando viene usata, questa variabile si modifica per ogni utente al fine
  di riferirsi alla directory dello specifico utente.% FIXME
%\item [TEXMFHOME] The tree which users can use for their own individual
%  installations of additional or updated macros, fonts, etc.
%  The expansion of this variable dynamically adjusts for each user to
%  their own individual directory.
\item [TEXMFCONFIG] Il percorso usato dai programmi \verb+texconfig+,
  \verb+updmap+ e \verb+fmtutil+ per memorizzare i dati di configurazione
  modificati. Normalmente si trova sotto \envname{TEXMFHOME}.
%\item [TEXMFCONFIG] The tree used by the utilities
%  \verb+texconfig+, \verb+updmap+, and \verb+fmtutil+ to store modified
%  configuration data.  Under \envname{TEXMFHOME} by default.
\item [TEXMFSYSCONFIG] Il percoros usato dai programmi
  \verb+texconfig-sys+, \verb+updmap-sys+ e \verb+fmtutil-sys+ per
  memorizzare i dati di configurazione modificati.
%\item [TEXMFSYSCONFIG] The tree used by the utilities
%  \verb+texconfig-sys+, \verb+updmap-sys+, and \verb+fmtutil-sys+ to
%  store modified configuration data.
\item [TEXMFVAR] Il percorso usato da \verb+texconfig+, \verb+updmap+ e
  \verb+fmtutil+ per memorizzare i dati generati durante l'esecuzione,
  come i file dei formati e le mappe per i font. Normalmente si trova
  sotto \envname{TEXMFHOME}.
%\item [TEXMFVAR] The tree used by \verb+texconfig+, \verb+updmap+ and
%  \verb+fmtutil+ to store (cached) runtime data such as format files and
%  generated map files.  Under \envname{TEXMFHOME} by default.
\item [TEXMFSYSVAR] Il percorso usato da \verb+texconfig-sys+,
  \verb+updmap-sys+, \verb+fmtutil-sys+ ed anche da \verb+tlmgr+ per
  memorizzare i dati generati durante l'esecuzione, come i file dei
  formati e le mappe per i font.
%\item [TEXMFSYSVAR] The tree used by \verb+texconfig-sys+,
%  \verb+updmap-sys+ and \verb+fmtutil-sys+, and also by \verb+tlmgr+, to
%  store (cached) runtime data such as format files and generated map files.
\end{ttdescription}

\noindent
La strutturazione predefinita è:
%The default layout is:
\begin{description}
  \item[percorso del sistema] può contenere diverse edizioni di \TL{}:
%  \item[system-wide root] can span multiple \TL{} releases:
  \begin{ttdescription}
    \item[2008] Un'edizione precedente.
%    \item[2008] A previous release.
    \item[2009] L'attuale edizione.
%    \item[2009] The current release.
    \begin{ttdescription}
      \item [bin] ~
      \begin{ttdescription}
        \item [i386-linux] Esebuibili per \GNU/Linux
%        \item [i386-linux] \GNU/Linux binaries
        \item [...]
        \item [universal-darwin] Eseguibili per \MacOSX
%        \item [universal-darwin] \MacOSX binaries
        \item [win32] Eseguibili per Windows
%        \item [win32] Windows binaries
      \end{ttdescription}
      \item [texmf\ \ \ \ \ \ \ ] This is \envname{TEXMFMAIN}.
      \item [texmf-dist\ \ ]      \envname{TEXMFDIST}
      \item [texmf-var \ \ ]      \envname{TEXMFSYSVAR}
      \item [texmf-config]        \envname{TEXMFSYSCONFIG}
    \end{ttdescription}
    \item [texmf-local] \envname{TEXMFLOCAL}, pensato per essere mantenuto
      tra diverse edizioni.
%    \item [texmf-local] \envname{TEXMFLOCAL}, intended to be
%      retained from release to release.
  \end{ttdescription}
  \item[home dell'utente] (\texttt{\$HOME} o \texttt{\%USERPROFILE\%})
%  \item[user's home] (\texttt{\$HOME} or
%      \texttt{\%USERPROFILE\%})
    \begin{ttdescription}
      \item[.texlive2008] Dati privati generati e di configurazione per
        un'edizione precedente.
%      \item[.texlive2008] Privately generated and configuration data
%        for a previous release.
      \item[.texlive2009] Dati privati generati e di configurazione per
        l'attuale edizione.
%      \item[.texlive2009] Privately generated and configuration data
%        for the current release.
      \begin{ttdescription}
        \item [texmf-var\ \ \ ] \envname{TEXMFVAR}
        \item [texmf-config]    \envname{TEXMFCONFIG}
      \end{ttdescription}
    \item[texmf] \envname{TEXMFHOME} Macro personali, ecc.
%    \item[texmf] \envname{TEXMFHOME} Personal macros, etc.
  \end{ttdescription}
\end{description}


\subsection{Estensioni di \protect\TeX}
%\subsection{Extensions to \protect\TeX}
\label{sec:tex-extensions}

Lo sviluppo dell'originale \TeX{} di Knuth è congelato, se si escludono
rare correzioni di bug. È ancora presente in \TL\ con il nome del
programma \prog{tex} e vi rimarrà nel futuro. \TL{} contiene diverse
versioni estese di \TeX:
%Knuth's original \TeX{} itself is frozen, apart from rare bug fixes. It
%is still present in \TL\ as the program \prog{tex}, and will remain so
%for the foreseeable future.  \TL{} contains several extended versions of
%\TeX:

\begin{description}

\item [\eTeX] aggiunge un insieme di nuove primitive \label{text:etex}
(riguardanti l'espansione delle macro, la scansione dei caratteri, le
classi di segnaposti, caratteristiche di debug aggiuntive ed altro ancora)
e le estensioni \TeXXeT{} per la composizione tipografica bidirezionale.
Di base, \eTeX{} è compatibile al 100\% con l'ordinario \TeX. Consulta
\OnCD{texmf-dist/doc/etex/base/etex_man.pdf}.
%\item [\eTeX] adds a set of new primitives
%\label{text:etex} (related to macro expansion, character scanning,
%classes of marks, additional debugging features, and more) and the
%\TeXXeT{} extensions for bidirectional typesetting.  In default mode,
%\eTeX{} is 100\% compatible with ordinary \TeX. See
%\OnCD{texmf-dist/doc/etex/base/etex_man.pdf}.

\item [pdf\TeX] parte dalle estensioni di \eTeX, aggiunge il supporto per
la generazione di file \acro{PDF} oltre che dei \dvi{} e molte estensioni
che non sono legate alla generazione dell'output. Questo programma è
invocato dalla maggior parte dei formati, come \prog{etex}, \prog{latex},
\prog{pdflatex}. Il suo sito web è \url{http://www.pdftex.org/}. Consulta
\OnCD{texmf-dist/doc/pdftex/manual/pdftex-a.pdf} per il manuale e
\OnCD{texmf-dist/doc/pdftex/manual/samplepdf/samplepdf.tex} per gli
esempi d'uso di alcune delle sue caratteristiche.
%\item [pdf\TeX] builds on the \eTeX\ extensions, adding support for
%writing \acro{PDF} output as well as \dvi{}, and many non-output-related
%extensions.  This is the program invoked for most formats, e.g.,
%\prog{etex}, \prog{latex}, \prog{pdflatex}.  Its web site is
%\url{http://www.pdftex.org/}.  See
%\OnCD{texmf-dist/doc/pdftex/manual/pdftex-a.pdf} for the manual, and
%\OnCD{texmf-dist/doc/pdftex/manual/samplepdf/samplepdf.tex} for example
%usage of some of its features.

% FIXME
\item [Lua\TeX] è il successore prescelto di pdf\TeX ed è quasi del tutto
(ma non completamente) compatibile con i predecessori. È anche pensato per
sostituire Aleph (vedi sotto), per quanto non sia ricercata la
compatibilità tecnica. L'interprete Lua incorportato
(\url{http://www.lua.org/}) permette soluzioni eleganti per molti problemi
spinosi di \TeX. Quando è invocato come \filename{texlua}, si comporta
come un interprete Lua autonomo ed è già usato in questo modo all'interno
di \TL. Il suo sito web è \url{http://www.luatex.org} e il manuale di
riferimento è \OnCD{texmf-dist/doc/luatex/luatexref-t.pdf}.
%\item [Lua\TeX] is the designated successor of pdf\TeX,
%and is mostly (but not entirely) backward-compatible.  It is also
%intended to be a functional superset of Aleph (see below), though
%technical compatibility is not intended. The incorporated Lua
%interpreter (\url{http://www.lua.org/}) enables elegant solutions for
%many thorny \TeX{} problems. When called as \filename{texlua}, it
%functions as a standalone Lua interpreter, and is already used as such
%within \TL.  Its web site is \url{http://www.luatex.org/}, and the
%reference manual is \OnCD{texmf-dist/doc/luatex/luatexref-t.pdf}.

\item [\XeTeX] aggiunge il supporto per l'input in Unicode e per i font
OpenType e di sistema, implementato usando librerie di terze parti
standard. Visita \url{http://tug.org/xetex}.
%\item [\XeTeX] adds support for Unicode input and OpenType- and system
%fonts, implemented using standard third-party libraries.  See
%\url{http://tug.org/xetex}.

\item [\OMEGA\ (Omega)] è basato sull'Unicode (caratteri a 16 bit) e
dunque consente di lavorare con quasi tutti gli alfabeti del mondo
contemporaneamente. Supporta anche i cosiddetti ``\OMEGA{} Translation
Process'' (\acro{OTP}, Processi di Traduzione Omega), per compiere
trasformazioni complesse su input arbitrari. Omega non è più incluso in
\TL{} come programma separato; è fornito soltanto Aleph:
%\item [\OMEGA\ (Omega)] is based on Unicode (16-bit characters), thus
%supports working with almost all the world's scripts simultaneously. It
%also supports so-called `\OMEGA{} Translation Processes' (\acro{OTP}s),
%for performing complex transformations on arbitrary input.  Omega is no
%longer included in \TL{} as a separate program; only Aleph is provided:

\item [Aleph] combina le estensioni \OMEGA\ e \eTeX. Consulta
\OnCD{texmf-dist/doc/aleph/base} per una documentazione minima.
%\item [Aleph] combines the \OMEGA\ and \eTeX\ extensions.
%See \OnCD{texmf-dist/doc/aleph/base} for some minimal documentation.

\end{description}


\subsection{Altri rilevanti programmi in \protect\TL}
%\subsection{Other notable programs in \protect\TL}

Seguono alcuni altri programmi di uso comune inclusi in \TL{}:
%Here are a few other commonly-used programs included in \TL{}:

\begin{cmddescription}

\item [bibtex] supporto per la bibliografia.
%\item [bibtex] bibliography support.

\item [makeindex, xindy] supporto per gli indici.
%\item [makeindex, xindy] index support.

\item [dvips] converte i \dvi{} in \PS{}.
%\item [dvips] convert \dvi{} to \PS{}.

\item [xdvi] programma di anteprima dei \dvi{} per l'X Window System.
%\item [xdvi] \dvi{} previewer for the X Window System.

\item [dvilj] convertitore di \dvi{} per la famiglia di stampanti HP
LaserJet.
%\item [dvilj] \dvi{} drive for the HP LaserJet family.

\item [dviconcat, dviselect] tagliano ed incollano le pagine contenute nei
file \dvi{}.
%\item [dviconcat, dviselect] cut and paste pages
%from \dvi{} files.

\item [dvipdfmx] converte i \dvi{} in \acro{PDF}, un approccio alternativo
a pdf\TeX\ (citato in precedenza). Consulta i pacchetti \pkgname{ps4pdf} e
\pkgname{pdftricks} per ulteriori alternative.
%\item [dvipdfmx] convert \dvi{} to \acro{PDF}, an alternative approach
%to pdf\TeX\ (mentioned above).  See the \pkgname{ps4pdf} and
%\pkgname{pdftricks} packages for still more alternatives.

\item [psselect, psnup, \ldots] programmi di manipolazione dei \PS{}.
%\item [psselect, psnup, \ldots] \PS{} utilities.

\item [texexec, texmfstart] processore Con\TeX{}t e \acro{PDF}.
%\item [texexec, texmfstart] Con\TeX{}t and \acro{PDF} processor.

\item [tex4ht] convertitore da \TeX{} a \acro{HTML} (e a \acro{XML} ed
altro ancora).
%\item [tex4ht] \TeX{} to \acro{HTML} (and \acro{XML} and more) converter.

\end{cmddescription}


\subsection{Font in \protect\TL}
%\subsection{Fonts in \protect\TL}
\label{sec:tl-fonts}

\TL{} è fornito con molti caratteri tipografici scalabili di alta qualità.
Visita \url{http://tug.org/fonts} e
\OnCD{texmf-dist/doc/fonts/free-math-font-survey}.
%\TL{} comes with many high-quality scalable fonts.  See
%\url{http://tug.org/fonts} and
%\OnCD{texmf-dist/doc/fonts/free-math-font-survey}.


\section{Installazione}
%\section{Installation}
\label{sec:install}

\subsection{Avviare l'installatore}
%\subsection{Starting the installer}
\label{sec:inst-start}

Per cominciare, procurati il \DVD{} \TL{} oppure scarica l'installatore di
rete di \TL{} ed individua il programma di installazione:
\filename{install-tl} per Unix, \filename{install-tl.bat} per Windows.
%To begin, get the \TK{} \DVD{} or download the \TL{} net installer,
%and locate the installer script: \filename{install-tl} for Unix,
%\filename{install-tl.bat} for Windows.

\begin{description}
\item [Installatore di rete:] Scaricalo da \CTAN, dal percorso
\dirname{systems/texlive/tlnet}; l'indirizzo
\url{http://mirror.ctan.org/systems/texlive/tlnet} ti reindirizzerà
automaticamente al mirror aggiornato più vicino. Puoi scaricare sia
\filename{install-tl.zip} che può essere usato sotto Unix e Windows, sia
il notevolmente più piccolo \filename{install-unx.tar.gz} solo per Unix.
Dopo averlo decompresso, \filename{install-tl} e \filename{install-tl.bat}
si troveranno nella sotto directory \dirname{install-tl}.
%\item [Net installer:] Download from \CTAN, under
%\dirname{systems/texlive/tlnet}; the url
%\url{http://mirror.ctan.org/systems/texlive/tlnet} will automatically
%redirect to a nearby, up-to-date, mirror.  You can retrieve either
%\filename{install-tl.zip} which can be used under Unix and Windows, or
%the considerably smaller \filename{install-unx.tar.gz} for Unix
%only. After unpacking, \filename{install-tl} and
%\filename{install-tl.bat} will be in the \dirname{install-tl}
%subdirectory.

\item [\DVD{} \TeX{} Collection:] vai nella sua sotto directory
\dirname{texlive}. Sotto Windows, il programma di installazione dovrebbe
partire automaticamente quando inserisci il \DVD. Puoi ottenere il \DVD\
diventando un membro di un gruppo utenti \TeX\ (caldamente consigliato,
\url{http://tug.org/usergroups.html}) oppure acquistandolo separatamente
(\url{http://tug.org/store}) o ancora masterizzandolo da te a partire
dall'immagine \ISO.
%\item [\TeX{} Collection \DVD:] go to its \dirname{texlive}
%subdirectory. Under Windows, the installer normally starts automatically
%when you insert the \DVD.  You can get the \DVD\ by becoming a member of
%a \TeX\ user group (highly recommended,
%\url{http://tug.org/usergroups.html}), or purchasing it separately
%(\url{http://tug.org/store}), or burning your own from the \ISO\
%image.

\end{description}

Visita \url{http://tug.org/texlive/acquire.html} per ulteriori
informazioni ed altri metodi per ottenere il software.
%See \url{http://tug.org/texlive/acquire.html} for more information and
%other methods of getting the software.

Le sezioni seguenti spiegano l'avvio dell'installazione in maggiore
dettaglio.
%The following sections explain installer start-up in more detail.

\subsubsection{Unix}
%\subsubsection{Unix}

\noindent
Di seguito, \texttt{>} denota il prompt della shell; l'input dell'utente è
in \Ucom{\texttt{grassetto}}.
Il programma \filename{install-tl} è uno script Perl. Il modo più semplice
per avviarlo su un sistema compatibile Unix è il seguente:
\begin{alltt}
> \Ucom{cd /percorso/verso/il/programma}
> \Ucom{perl install-tl}
\end{alltt}
In alternativa puoi invocare
\Ucom{perl /percorso/verso/il/programma/install-lt}, oppure
\Ucom{./install-tl} se hai i permessi di esecuzione, ecc.; non ripeteremo
tutte queste varianti. Potresti dover ingrandire la finestra di terminale
affinché mostri l'intera schermata dell'installatore testuale
(Figure~\ref{fig:text-main}).
%\noindent
%Below, \texttt{>} denotes the shell prompt; user input is
%\Ucom{\texttt{bold}}.
%The script \filename{install-tl} is a Perl script.  The simplest way
%to start it on a Unix-compatible system is as follows:
%\begin{alltt}
%> \Ucom{cd /path/to/installer}
%> \Ucom{perl install-tl}
%\end{alltt}
%(Or you can invoke \Ucom{perl /path/to/installer/install-tl}, or
%\Ucom{./install-tl} if it stayed executable, etc.; we won't repeat all
%these variations.)  You may have to enlarge your terminal window so
%that it shows the full text installer screen (Figure~\ref{fig:text-main}).

Per eseguire l'installazione nella modalità \GUI\ avanzata
(figura~\ref{fig:gui-main}; hai bisogno del modulo \dirname{Perl/TK}),
usa:
\begin{alltt}
> \Ucom{perl install-tl -gui}
\end{alltt}
%To install in expert \GUI\ mode (figure~\ref{fig:gui-main}; you'll
%need the \dirname{Perl/TK} module), use:
%\begin{alltt}
%> \Ucom{perl install-tl -gui}
%\end{alltt}

Per un elenco completo delle diverse opzioni:
\begin{alltt}
> \Ucom{perl install-tl -help}
\end{alltt}
%For a complete listing of the various options:
%\begin{alltt}
%> \Ucom{perl install-tl -help}
%\end{alltt}

\textbf{Attenzione riguardo i permessi Unix:} La tua \code{umask} nel
momento dell'installazione sarà rispettata dall'installatore di \TL{}.
Quindi se vuoi che l'installazione sia usabile da altri utenti oltre che
te, sii sicuro che le tue impostazioni siano permissive a sufficienza, per
esempio, \code{umask 002}. Per ulteriori informazioni riguardo
\code{umask}, consulta la documentazione del tuo sistema.
%\textbf{Warning about Unix permissions:} Your \code{umask} at the time
%of installation will be respected by the \TL{} installer.  Therefore, if
%you want your installation to be usable by users other than you, make
%sure your setting is sufficiently permissive, for instance, \code{umask
%002}.  For more information about \code{umask}, consult your system
%documentation.

\textbf{Considerazioni speciali per Cygwin:} Diversamente da altri sistemi
Unix-compatibili, Cygwin non è preimpostato per includere tutti i
programmi di cui l'installatore di \TL{} ha bisogno. Consulta la
sezione~\ref{sec:cygwin} per i dettagli.
%\textbf{Special considerations for Cygwin:} Unlike other
%Unix-compatible systems, Cygwin does not by default include all of the
%prerequisite programs needed by the \TL{} installer.  See
%Section~\ref{sec:cygwin} for details.


\subsubsection{MacOSX}

Come abbiamo accennato nella sezione~\ref{sec:tl-coll-dists}, abbiamo
preparato una distribuzione separata per \MacOSX chiamata Mac\TeX\
(\url{http://tug.org/mactex}). Su \MacOSX raccomandiamo di usare
l'installatore nativo di Mac\TeX\ al posto dell'installatore di \TL\ in
quanto quello nativo esegue alcuni aggiustamenti specifici per il Mac, in
particolare per consentire con semplicità il passaggio tra le varie
distribuzioni \TeX\ per \MacOSX\ (Mac\TeX, gw\TeX, Fink, MacPorts,
\ldots).

%As mentioned in section~\ref{sec:tl-coll-dists}, a separate distribution
%is prepared for \MacOSX, named Mac\TeX\ (\url{http://tug.org/mactex}).
%We recommend using the native Mac\TeX\ installer instead of the \TL\
%installer on \MacOSX, because the native installer makes a few
%Mac-specific adjustments, in particular to allow easily switching
%between the various \TeX\ distributions for \MacOSX\ (Mac\TeX, gw\TeX,
%Fink, MacPorts, \ldots).

Mac\TeX\ è strettamente basato su \TL\ e le rispettive strutture delle
directory sono esattamente le stessa. Il primo aggiunge alcune ulteriori
cartelle con documentazione e applicazioni specifiche per il Mac.
%Mac\TeX\ is firmly based on \TL, and the main \TeX\ trees are precisely
%the same.  It does add a few extra folders with Mac-specific
%documentation and applications.


\subsubsection{Windows}

Se stai usando l'installatore di rete oppure se l'installatore su \DVD\
non parte automaticamente, fai doppio click su \filename{install-tl.bat}.
Per avere maggiori opzioni di configurazione, come la selezione di
specifiche collezioni di pacchetti, esegui
\filename{install-tl-advanced.bat}.
%If you are using the net installer, or the \DVD\ installer failed to
%start automatically, double-click \filename{install-tl.bat}.
%For more customization options, e.g. selection of specific package
%collections, run \filename{install-tl-advanced.bat} instead.

Puoi anche avviare l'installatore dal prompt dei comandi. Qui sotto,
\texttt{>} denota il prompt; l'input dell'utente è in
\Ucom{\texttt{grassetto}}. Se ti trovi nella cartella dell'installatore,
esegui semplicemente:
\begin{alltt}
> \Ucom{install-tl}
\end{alltt}
%You can also start the installer from the command-prompt.  Below,
%\texttt{>} denotes the prompt; user input is \Ucom{\texttt{bold}}.  If
%you are in the installer directory, run just:
%\begin{alltt}
%> \Ucom{install-tl}
%\end{alltt}

In aternativa puoi invocarlo con un percorso assoluto, come:
\begin{alltt}
> \Ucom{D:\bs{}texlive\bs{}install-tl}
\end{alltt}
per il \DVD\ \TK, supponendo che \dirname{D:} sia il lettore \DVD. La
figura~\ref{fig:wizard} mostra l'installazione guidata, che è quella
preimpostata per Windows.
%Or you can invoke it with an absolute location, such as:
%\begin{alltt}
%> \Ucom{D:\bs{}texlive\bs{}install-tl}
%\end{alltt}
%for the \TK\ \DVD, supposing that \dirname{D:} is the optical
%drive. Figure~\ref{fig:wizard} displays the wizard installer, which
%is the default for Windows.

Per installare in modalità testuale, usa:
\begin{alltt}
> \Ucom{install-tl -no-gui}
\end{alltt}
%To install in text mode, use:
%\begin{alltt}
%> \Ucom{install-tl -no-gui}
%\end{alltt}

Per un elenco completo delle varie opzioni:
\begin{alltt}
> \Ucom{install-tl -help}
\end{alltt}
%For a complete listing of the various options:
%\begin{alltt}
%> \Ucom{install-tl -help}
%\end{alltt}

%%% TODO la figura!!
\begin{figure}[tb]
\begin{boxedverbatim}
Installing TeX Live 2009 from: ...
Platform: i386-linux => 'Intel x86 with GNU/Linux'
Distribution: live (uncompressed)
...
 Detected platform: Intel x86 with GNU/Linux

 <B> binary systems: 1 out of 14

 <S> Installation scheme (scheme-full)
     83 collections out of 84, disk space required: 1882 MB

 Customizing installation scheme:
   <C> standard collections
   <L> language collections

 <D> directories:
   TEXDIR (the main TeX directory):
     /usr/local/texlive/2009
   TEXMFLOCAL (directory for site-wide local files):
     /usr/local/texlive/texmf-local
   TEXMFSYSVAR (directory for variable and automatically generated data):
     /usr/local/texlive/2009/texmf-var
   TEXMFSYSCONFIG (directory for local config):
     /usr/local/texlive/2009/texmf-config
   TEXMFHOME (directory for user-specific files):
     ~/texmf

 <O> options:
   [ ] use letter size instead of A4 by default
   [X] create all format files
   [X] install macro/font doc tree
   [X] install macro/font source tree
   [ ] create symlinks to standard directories

 <V> set up for running from DVD

Other actions:
 <I> start installation to hard disk
 <H> help
 <Q> quit
\end{boxedverbatim}
\caption{Schermata principale dell'installatore testuale
  (\GNU/Linux)}\label{fig:text-main}
%\caption{Main text installer screen (\GNU/Linux)}\label{fig:text-main}
\end{figure}

\begin{figure}[tb]
\tlpng{install-lnx-main}{\linewidth}
\caption{Schermata dell'installatore avanzato \GUI{}
  (\GNU/Linux)}\label{fig:gui-main}
%\caption{Expert \GUI{} installer screen (\GNU/Linux)}\label{fig:gui-main}
\end{figure}

\begin{figure}[tb]
\tlpng{wizard-w32}{\linewidth}
\caption{Schermata dell'installazione guidata (Windows)}\label{fig:wizard}
%\caption{Wizard installer screen (Windows)}\label{fig:wizard}
\end{figure}


\htmlanchor{cygwin}
\subsubsection{Cygwin}
\label{sec:cygwin}

L'installatore \TL{} supporta soltanto Cygwin 1.7.
%The \TL{} installer supports only Cygwin 1.7.
% Add a note about Angelo's workaround for 1.5 if he gets it ready for
% public use.
Prima di iniziare l'installazione, usa il programma \filename{setup.exe}
di Cygwinper installare i pacchetti \filename{perl} e \filename{wget}, a
meno che tu non lo abbia già fatto. I seguenti pacchetti aggiuntivi sono
raccomandati:
%Before beginning the installation, use Cygwin's \filename{setup.exe} program to
%install the \filename{perl} and \filename{wget} packages if you have
%not already done so.  The following additional packages are
%recommended:
\begin{itemize*}
\item \filename{fontconfig} [richiesto da \XeTeX]
%\item \filename{fontconfig} [needed by \XeTeX]
\item \filename{ghostscript} [richiesto da vari programmi]
%\item \filename{ghostscript} [needed by various utilities]
\item \filename{libXaw7} [richiesto da xdvi]
%\item \filename{libXaw7} [needed by xdvi]
\item \filename{ncurses} [fornisce il comando 'clear' usato
  dall'installatore]
%\item \filename{ncurses} [provides the 'clear' command used by the installer]
\end{itemize*}

\subsubsection{L'installatore testuale}
%\subsubsection{The text installer}

La figura~\ref{fig:text-main} mostra la schermata principale della
modalità testuale sotto Unix. L'installatore testuale è quello
preimpostato sotto Unix.
%Figure~\ref{fig:text-main} displays the main text mode screen under
%Unix.  The text installer is the default on Unix.

Questo è l'unico installatore a riga di comando; non c'è alcun supporto
per il movimento del cursore di inserimento. Ad esempio, non puoi muoverti
tra le caselle di spunta o i campi di inserimento. Semplicemente digiti
qualcosa (MAIUSCOLE e minuscole sono differenti) al prompt, premi il
tasto Invio e l'intera schermata del terminale sarà aggiornata, con il
contenuto alterato.
%This is only a command-line installer; there is no cursor support at
%all.  For instance, you cannot tab around checkboxes or input fields.
%You just type something (case-sensitive) at the prompt and press the
%Enter key, and then the entire terminal screen will be rewritten, with
%adjusted content.

L'interfaccia dell'installatore testuale è così primitiva per una ragione:
è progettata per funzionare sul maggior numero di piattaforme possibile,
anche con una versione minimale di Perl.
%The text installer interface is this primitive for a reason: it
%is designed to run on as many platforms as possible, even with a
%very barebones Perl.

\subsubsection{L'installatore grafico avanzato}
%\subsubsection{The expert graphical installer}

La figura~\ref{fig:gui-main} mostra l'installatore grafico avanzato sotto
\GNU/Linux. Al di là dell'uso di pulsanti e menu, questo installatore non
differisce di molto da quello testuale (figura~\ref{fig:text-main}).
%Figure~\ref{fig:gui-main} displays the expert graphical installer under
%\GNU/Linux.  Other than using buttons and menus, this installer does
%not differ much from the text one (Figure~\ref{fig:text-main}). 

Questa modalità può essere invocata esplicitamente tramite
%This mode can be invoked explicitly with
\begin{alltt}
> \Ucom{install-tl -gui=perltk}
\end{alltt}


\subsubsection{La semplicistica installazione guidata}
%\subsubsection{The simplistic wizard installer}

Sotto Windows, è implicito eseguire il più semplice metodo di
installazione che potessimo escogitare, chiamato ``installazione
guidata''. Installa tutto e pone quasi nessuna domanda. Se vuoi
personalizzare la tua installazione, dovresti lanciare uno degli altri
installatori.
%Under Windows, the default is to run the simplest installation method we
%could devise, named the ``wizard'' installer.  It installs everything
%and asks almost no questions.  If you want to customize your setup, you
%should run one of the other installers.

Questa modalità può essere invocata esplicitamente con
%This mode can be invoked explicitly with
\begin{alltt}
> \Ucom{install-tl -gui=wizard}
\end{alltt}


\subsection{Eseguire l'installatore}
%\subsection{Running the installer}
\label{sec:runinstall}

L'installatore è pensato per essere quasi del tutto ovvio, ma seguono
alcune note riguardo varie opzioni e sottomenu.
%The installer is intended to be mostly self-explanatory, but following are a
%few notes about the various options and submenus.

\subsubsection{Menu delle architetture (solo Unix)}
%\subsubsection{Binary systems menu (Unix only)}
\label{sec:binary}

%TODO figura!
\begin{figure}[tbh]
\begin{boxedverbatim}
Available sets of binaries:
===============================================================================

   a [ ] alpha-linux      DEC Alpha with GNU/Linux
   b [ ] i386-cygwin      Intel x86 with Cygwin
   c [X] i386-linux       Intel x86 with GNU/Linux
   d [ ] i386-netbsd      Intel x86 with NetBSD
   e [ ] i386-solaris     Intel x86 with Sun Solaris
   f [ ] mips-irix        SGI IRIX
   g [ ] powerpc-aix      PowerPC with AIX
   h [ ] powerpc-linux    PowerPC with GNU/Linux
   i [ ] sparc-linux      Sparc with GNU/Linux
   j [ ] sparc-solaris    Sparc with Solaris
   k [ ] universal-darwin universal binaries for MacOSX/Darwin
   l [ ] win32            Windows
   m [ ] x86_64-linux     x86_64 with GNU/Linux
\end{boxedverbatim}
\caption{Menu delle architetture}\label{fig:bin-text}
%\caption{Binaries menu}\label{fig:bin-text}
\end{figure}

La figura~\ref{fig:bin-text} mostra il menu delle architetture in modalità
testuale. Di base, saranno installati solo gli eseguibili per la tua
piattaforma. Da questo menu puoi selezionare l'installazione degli
eseguibili anche per altre architetture. Questa opzione è utile se
condividi una struttura \TeX\ in una rete di macchine eterogenee, oppure
per una macchina con due sistemi operativi.
%Figure~\ref{fig:bin-text} displays the text mode binaries menu.  By
%default, only the binaries for your current platform will be installed.
%From this menu, you can select installation of binaries for other
%architectures as well.  This can be useful if you are sharing a \TeX\
%tree across a network of heterogenous machines, or for a dual-boot
%system.

\subsubsection{Selezionare cosa sarà installato}
%\subsubsection{Selecting what is going to be installed}
\label{sec:components}

% TODO FIGURAAA
\begin{figure}[tbh]
\begin{boxedverbatim}
Select a scheme:
===============================================================================
 a [ ] basic scheme (plain and LaTeX)
 b [ ] ConTeXt scheme
 c [X] full scheme (everything)
 d [ ] GUST TeX Live scheme
 e [ ] GUTenberg TeX Live scheme
 f [ ] medium scheme (plain, latex, recommended packages, some languages)
 g [ ] minimal scheme (plain only)
 h [ ] Omega scheme
 i [ ] teTeX scheme (more than medium, but nowhere near full)
 j [ ] XML scheme
 k [ ] custom selection of collections
\end{boxedverbatim}
\caption{Menu degli schemi}\label{fig:scheme-text}
%\caption{Scheme menu}\label{fig:scheme-text}
\end{figure}

La figura~\ref{fig:scheme-text} mostra il menu degli schemi di \TL; da qui
scegli uno ``schema'', che è un insieme di collezioni di pacchetti. Lo
schema predefinito \optname{full} installa tutto ciò che è disponibile, ma
puoi anche scegliere lo schema \optname{basic} per un sistema piccolo,
\optname{minimal} per scopi di test e \optname{medium} o \optname{teTeX}
per ottenere una via di mezzo. Ci sono anche ulteriori schemi
specializzati e specifici per un particolare paese.
%Figure~\ref{fig:scheme-text} displays the \TL\ scheme menu; from here,
%you choose a ``scheme'', which is an overall set of package collections.
%The default \optname{full} scheme installs everything available, but you
%can also choose the \optname{basic} scheme for a small system,
%\optname{minimal} for testing purposes, and \optname{medium} or
%\optname{teTeX} to get something in between.  There are also various
%specialized and country-specific schemes.

\begin{figure}[tbh]
\tlpng{stdcoll}{.7\linewidth}
\caption{Menu delle collezioni}\label{fig:collections-gui}
%\caption{Collections menu}\label{fig:collections-gui}
\end{figure}

Puoi raffinare la tua scelta dello schema con i menu ``Collezioni di
base'' e ``Collezioni di lingue'' (figura~\ref{fig:collections-gui},
mostrati, per cambiare, in modalità \GUI).
%You can refine your scheme selection with the `standard collections' and
%`language collections' menus (figure~\ref{fig:collections-gui}, shown in
%\GUI\ mode for a change).

Le collezioni stanno ad un livello di dettaglio successivo rispetto agli
schemi \Dash\ in pratica, uno schema consiste in svariate collezioni, una
collezione consta di uno o più pacchetti e un pacchetto (il più basso
livello di raggruppamento in \TL) contiene gli effettivi file di macro
\TeX, i file dei font e così via.
%Collections are one level more detailed than schemes\Dash in essence, a
%scheme consists of several collections, a collection consists of one or
%more packages, and a package (the lowest level grouping in \TL) contains
%the actual \TeX\ macro files, font files, and so on.

Se desideri un maggiore controllo di quanto il menu delle collezioni
fornisca, puoi usare il programma \prog{tlmgr} dopo l'installazione
(consulta la sezione~\ref{sec:tlmgr}); usando tale programma, puoi
controllare l'installazione al livello dei singoli pacchetti.
%If you want more control than the collection menus provide, you can use
%the \prog{tlmgr} program after installation (see
%section~\ref{sec:tlmgr}); using that, you can control the installation
%at the package level.

\subsubsection{Percorsi di destinazione}
%\subsubsection{Directories}
\label{sec:directories}

La strutturazione predefinita è descritta nella
sezione~\ref{sec:texmftrees}, \p.\pageref{sec:texmftrees}. La posizione
predefinita di \dirname{TEXDIR} è diversa sotto Windows
(|%SystemDrive%\texlive\2009|) e Unix
(\dirname{/usr/local/texlive/2009}). 
%The default layout is described in section~\ref{sec:texmftrees},
%\p.\pageref{sec:texmftrees}. The default location of
%\dirname{TEXDIR} is different between Windows
%(|%SystemDrive%\texlive\2009|) and Unix
%(\dirname{/usr/local/texlive/2009}).

La motivazione principale per cambiare questa impostazione è la mancanza
dei permessi di scrittura su tale percorso. Non devi essere root oppure un
amministratore per installare \TL, ma hai bisogno dell'accesso in
scrittura sulla directory di destinazione.
%The main reason to change this default is if you lack write permission
%for the default location. You don't have to be root or administrator to
%install \TL, but you do need write access to the target directory.

Una scelta alternativa ragionevole è quella di una directory al di sotto
della tua directory di home, specialmente se sara il solo utilizzatore.
Usa ``|~|'' per indicare la tua home, come in ``|~/texlive/2009|''.
%A reasonable alternative choice is a directory under your home directory,
%especially if you will be the sole user. Use
%`|~|' to indicate this, as in `|~/texlive/2009|'.

Raccomandiamo di includere l'anno nel nome del percorso, così da
consentire il mantenimento di diverse edizioni di \TL{} fianco a fianco
(potresti voler creare un nome indipendente dalla versione, come
\dirname{/usr/local/texlive-cur}, attraverso un collegamento simbolico,
che puoi successivamente aggiornare dopo aver testato la nuova edizione).
%We recommend including the year in the name, to enable keeping different
%releases of \TL{} side by side.  (You may wish to make a
%version-independent name such as \dirname{/usr/local/texlive-cur} via a
%symbolic link, which you can then update after testing the new release.)

Cambiare \dirname{TEXDIR} nell'installatore provocherà anche il
cambiamento di \dirname{TEXMFLOCAL}, \dirname{TEXMFSYSVAR} e
\dirname{TEXMFSYSCONFIG}.
%Changing \dirname{TEXDIR} in the installer will also change
%\dirname{TEXMFLOCAL}, \dirname{TEXMFSYSVAR} and
%\dirname{TEXMFSYSCONFIG}.

\dirname{TEXMFHOME} è la posizione che raccomandiamo per i file di macro e
i pacchetti personali. Il suo valore predefinito è |~/texmf|. A differenza
di quanto accade in \dirname{TEXDIR}, qui un |~| è mantenuto all'interno
dei file di configurazione creati, in quanto è utile far riferimento alla
directory di home di chiunque esegua \TeX (e non di chi installi \TL). Si
traduce in \dirname{$HOME} sotto Unix e \verb|%USERPROFILE%| sotto
Windows.
%\dirname{TEXMFHOME} is the recommended location for personal
%macro files or packages.  The default value is |~/texmf|.  In
%contrast to \dirname{TEXDIR}, here a |~| is preserved in the
%newly-written configuration files, since it usefully refers to the home
%directory of any individual running \TeX.  It expands to
%\dirname{$HOME} on Unix and \verb|%USERPROFILE%| on Windows.

\subsubsection{Opzioni}
%\subsubsection{Options}
\label{sec:options}

% TODO FIGURA!!!
\begin{figure}[tbh]
\begin{boxedverbatim}
 <P> use letter size instead of A4 by default: [ ]
 <F> create format files:                      [X]
 <D> install font/macro doc tree:              [X]
 <S> install font/macro source tree:           [X]
 <L> create symlinks in standard directories:  [ ]
            binaries to:
            manpages to:
                info to:
\end{boxedverbatim}
\caption{Menu delle opzioni (Unix)}\label{fig:options-text}
%\caption{Options menu (Unix)}\label{fig:options-text}
\end{figure}

La figura~\ref{fig:options-text} mostra il menu delle opzioni in modalità
testuale. Ulteriori informazioni:
%Figure~\ref{fig:options-text} shows the text mode options menu.
%More info:

\begin{description}
\item[Usa come predefinito il formato di pagina lettera al posto dell'A4:]
  La selezione del formato di carta predefinito. Ovviamente, i singoli
  documenti possono e dovrebbero specificare una particolare dimensione
  del foglio, se desiderato.
%\item[use letter size instead of A4 by default:] The default paper
%  size selection.  Of course, individual documents can and should
%  specify a specific paper size, if desired.

\item[Crea i file di formato:] Per quanto i file di formato non necessari
  richiedono tempo per essere generati e spazio su disco per essere
  memorizzati, è comunque raccomandato di lasciare questa opzione attiva:
  se la disattivi, allora i file dei formati saranno generati nella
  directory \dirname{TEXMFVAR} privata di ciascun utente a seconda delle
  necessità. In quella posizione, i formati non saranno aggiornati
  automaticamente se (supponiamo) gli eseguibili o i modelli di
  sillabazione fossero aggiornati nell'installazione e così si potrebbe
  arrivare ad avere file incompatibili.
%\item[create format files:] Although unnecessary format files
%  take time to generate and disk space to store, it is still recommended
%  to leave this option checked: if you don't, then format files will be
%  generated in people's private \dirname{TEXMFVAR} tree as they are
%  needed.  In that location, they will not be updated automatically if
%  (say) binaries or hyphenation patterns are updated in the
%  installation, and thus could end up with incompatible format files.

\begin{comment}
\item[Esecuzione di una ristretta lista di programmi:] A partire da \TL\
  2009, l'esecuzione di alcuni programmi esterni è consentita di base. La
  lista di tali programmi è fornita in \filename{texmf.cnf}; è molto
  piccola, ma comunque molto utile. Consulta le novità dell'edizione 2009
  (section~\ref{sec:2009news}) per maggiori dettagli.
%\item[execution of restricted list of programs:] As of \TL\ 2009,
%  execution of a few external programs is allowed by default.  The list
%  of allowed programs is given in the \filename{texmf.cnf}; it is very
%  small, but still very useful.  See the 2009 news
%  (section~\ref{sec:2009news}) for more details.
\end{comment}

\item[Installa \ldots\ per font e macro:] Queste opzioni consentono di
  omettere lo scaricamento e l'installazione della documentazione e dei
  file sorgenti presenti nella maggior parte dei pacchetti. Non è
  raccomandata.
%\item[install font/macro \ldots\ tree:] These options allow you to omit
%  downloading/installing the documentation and source files present in
%  most packages.  Not recommended.

\item[Crea i collegamenti nelle directory di sistema] (solo Unix): Questa
  opzione scavalca la necessità di cambiare le variabili di ambiente.
  Senza di essa, le directory di \TL{} devono solitamente essere aggiunte
  alle variabili \envname{PATH}, \envname{MANPATH} e \envname{INFOPATH}.
  Hai bisogno dei permessi di scrittura sulle directory di destinazione. È
  fortemente suggerito di \emph{non} sovrascrivere un sistema \TeX\
  fornito con il tuo sistema adoperando questa opzione. È pensata
  principalmente per accedere al sistema \TeX\ attraverso directory che
  sono già note agli utenti, come \dirname{/usr/local/bin}, che già non
  contengano alcun file di \TeX.
%\item[create symlinks in standard directories] (Unix only):
%  This option bypasses the need to change environment variables. Without
%  this option, \TL{} directories usually have to be added to
%  \envname{PATH}, \envname{MANPATH} and \envname{INFOPATH}. You will
%  need write permissions to the target directories.  It is strongly
%  advised \emph{not} to overwrite a \TeX\ system that came with your
%  system with this option.  It is intended primarily for accessing the
%  \TeX\ system through directories that are already known to users, such
%  as \dirname{/usr/local/bin}, which don't already contain any \TeX\
%  files.
\end{description}

Quando tutte le impostazioni sono come desideri, puoi digitare ``I'' per
avviare il processo di installazione. Quando è completo, passa alla
sezione~\ref{sec:postinstall} per leggere cos'altro devi fare, se ce n'è
bisogno.
%When all the settings are to your liking, you can type `I' to start the
%installation process. When it is done, skip to
%section~\ref{sec:postinstall} to read what else needs to be done, if
%anything.

\htmlanchor{runfromdvd}
\subsubsection{Configurazione per l'esecuzione diretta dal \DVD{} (solo
  modalità testuale)}
%\subsubsection{Set up for running from DVD{} (text mode only)}
\label{sec:fromdvd}

Digita ``|V|'' per selezionare questa opzione. Il menu principale cambia
in qualcosa di simile alla figura~\ref{fig:main-fromdvd}.
%Type `|V|' to select this option. This changes the main menu into something
%as in figure~\ref{fig:main-fromdvd}.

% TODO Figura
\begin{figure}[tbh]
\begin{boxedverbatim}
======================> TeX Live installation procedure <=====================
...
 <D> directories:
   TEXDIRW (Writable root):
     !! default location: /usr/local/texlive/2009
     !! is not writable, please select a different one!
   TEXMFLOCAL (directory for site-wide local files):
     /usr/local/texlive/texmf-local
   TEXMFSYSVAR (directory for variable and automatically generated data):
     /usr/local/texlive/2009/texmf-var
   TEXMFSYSCONFIG (directory for local config):
     /usr/local/texlive/2009/texmf-config
   TEXMFHOME (directory for user-specific files):
     ~/texmf

 <O> options:
   [ ] use letter size instead of A4 by default
   [X] create all format files

 <V> set up for installing to hard disk

Other actions:
 <I> start installation for running from DVD
 <H> help
 <Q> quit
\end{boxedverbatim}
\caption{Il menu principale con l'opzione ``\optname{dal DVD}'' impostata}
\label{fig:main-fromdvd}
%\caption{The main menu with `\optname{from DVD}' set}\label{fig:main-fromdvd}
\end{figure}

% FIXME: da verificare con il programma se esce writable root
Nota i cambiamenti: tutte le opzioni riguardanti le cose da installare
sono scomparse e la sezione sulle directory ora parla di \dirname{TEXDIRW}
o di ``destinazione scrivibile''. Anche l'opzione sui collegamenti
simbolici è scomparsa.
%Note the changes: all options about what to install have
%disappeared, and the directories section now talks about
%\dirname{TEXDIRW} or writable root. The symlinks option has also
%disappeared.

L'installatore creerà ancora le varie directory e i file di
configurazione, ma non copierà \dirname{texmf} o \dirname{texmf-dist}
sull'hard disk.
%The installer will still create various directories and configuration
%files, but won't copy \dirname{texmf} or \dirname{texmf-dist} to hard
%disk.

La configurazione post installazione per Unix sarà leggermente più
complessa, perché adesso la strutturazione delle directory devia da quella
predefinita; consulta la sezione~\ref{sec:postinstall}.
%Post-install configuration for Unix will be slightly more complicated,
%because now the directory layout deviates from the default; see
%section~\ref{sec:postinstall}.

Questa opzione non è presente nell'installatore grafico, ma è disponibile
sia sotto Unix che sotto Windows. Gli utenti di Windows devono avviare
l'installatore da un prompt dei comandi, consulta la
sezione~\ref{sec:cmdline}.
%This option is not in the \GUI{} installer, but it is available both
%for Unix and for Windows. Windows users have to start the installer
%from a command prompt, see section~\ref{sec:cmdline}.

La sezione~\ref{sec:portable-tl} descrive un modo ancora più portabile per
eseguire \TL, che non effettua né richiede alcun cambiamento alla
configurazione del sistema, ma non consente neppure alcuna configurazione.
%Section \ref{sec:portable-tl} describes a more strictly portable
%way to run \TL, which doesn't make or require any changes in the
%system's configuration, but doesn't allow any configuration either.

\subsection{Opzioni della riga di comando di install-tl}
%\subsection{Command-line install-tl options}
\label{sec:cmdline}

Digita
%Type
\begin{alltt}
> \Ucom{install-tl -help}
\end{alltt}
per ottenere un elenco delle opzioni della riga di comando. Sia |-| che
|--| possono essere usati per introdurre i nomi delle opzioni. Ecco le più
comuni:
%for a listing of command-line options.  Either |-| or |--| can be used
%to introduce option names.  These are the most common ones:

\begin{ttdescription}
\item[-gui] Se possibile, usa l'installatore grafico. Questa opzione
  richiede il modulo Perl/Tk
  (\url{http://tug.org/texlive/distro.html#perltk}); se Perl/Tk non è
  disponibile, l'installazione prosegue in modalità testuale.
%\item[-gui] If possible, use the \GUI{} installer. This requires the
%  Perl/Tk module (\url{http://tug.org/texlive/distro.html#perltk});
%  if Perl/Tk is not available, installation continues in text mode.

\item[-no-gui] Forza l'uso dell'installatore testuale, anche sotto
  Windows; hai bisogno di questa opzione se desideri un'installazione
  ``dal \DVD'', dato che non è disponibile nella \GUI.
%\item[-no-gui] Force using the text mode installer, even under
%  Windows; you'll need this if you want a `from \DVD' installation,
%  since that is not available in \GUI{} mode.

\item[-lang {\sl LL}] Specifica la lingua dell'interfaccia
  dell'installatore sotto forma del codice standard a due lettere
  \textsl{LL}. Le lingue al momento supportate sono: Inglese
  (\texttt{en}, predefinito), Tedesco (\texttt{de}), Francese
  (\texttt{fr}), Olandese (\texttt{nl}), Polacco (\texttt{pl}), Sloveno
  (\texttt{sl}), Vietnamese (\texttt{vi}) e Italiano (\texttt{it}).
  L'installatore cerca di determinare da solo la lingua corretta, ma se
  fallisce, oppure se la lingua corretta non è disponibile, allora passa
  all'uso dell'Inglese.
%\item[-lang {\sl LL}] Specify the installer interface
%  language as its standard two-letter code \textsl{LL}. Currently
%  supported languages: English (\texttt{en}, default), German
%  (\texttt{de}), French (\texttt{fr}), Dutch (\texttt{nl}), Polish
%  (\texttt{pl}), Slovenian (\texttt{sl}), Vietnamese (\texttt{vi})
%  and Italian (\texttt{it}). The installer tries to determine the right
%  language itself but if it fails, or if the right language is not
%  available, then it uses English as a fallback.

\item[-profile {\sl file}] L'installatore cerca sempre di scrivere un file
  \filename{texlive.profile} nella sotto directory \dirname{tlpkg} della
  tua installazione. Questa opzione dice all'installatore di riusare tale
  file di profilo, così che tu possa installare in sequenza su sistemi
  successivi, riproducendo le scelte fatte in una precedente
  installazione.
%\item[-profile {\sl file}] The installer always writes a file
%  \filename{texlive.profile} to the \dirname{tlpkg} subdirectory of your
%  installation.  This option tells the installer to re-use such a
%  profile file, so you can install in batch mode on subsequent systems,
%  reproducing the choices you made in a prior installation.

\item [-repository {\sl url-o-directory}] Specifica l'archivio dei
  pacchetti da cui installare; vedi più avanti.
%\item [-repository {\sl url-or-directory}] Specify package
%  repository from which to install; see following.
\end{ttdescription}

\subsubsection{L'opzione \optname{-repository}}
%\subsubsection{The \optname{-repository} option}
\label{sec:location}

L'archivio dei pacchetti predefinito è una delle copie di \CTAN{}
scelto automaticamente tramite \url{http://mirror.ctan.org}.
%The default package repository is a \CTAN{} mirror chosen automatically
%via \url{http://mirror.ctan.org}.

Se vuoi modificare questa scelta, il valore di questa opzione può essere
un url che comincia per \texttt{ftp:}, \texttt{http:} o \texttt{file:/},
oppure può essere un semplice percorso ad una directory (quando specifichi
un indirizzo \texttt{http:}\ o \texttt{ftp:}, il carattere finale
``\texttt{/}'' e l'eventuale ``\texttt{/tlpkg}'' conclusivo sono
ignorati).
%If you want to override that, the location value can be a url starting
%with \texttt{ftp:}, \texttt{http:}, or \texttt{file:/}, or a plain
%directory path.  (When giving an \texttt{http:}\ or \texttt{ftp:}\
%location, trailing `\texttt{/}' characters and/or a trailing
%`\texttt{/tlpkg}' component are ignored.)

Per esempio, puoi scegliere una particolare copia di \CTAN\ con qualcosa
del tipo:
\url{http://ctan.example.org/tex-archive/systems/texlive/tlnet/},
sostituendo il vero nome del sito e il suo percorso iniziale verso la
copia di \CTAN\ a |ctan.example.org|. L'elenco delle copie di \CTAN\ è
mantenuto all'indirizzo \url{http://ctan.org/mirrors}.
%For example, you could choose a particular \CTAN\ mirror with something
%like: \url{http://ctan.example.org/tex-archive/systems/texlive/tlnet/},
%substituting a real hostname and its particular top-level
%\CTAN\ path for |ctan.example.org|.  The list of \CTAN\ mirrors is
%maintained at \url{http://ctan.org/mirrors}.

Se l'argomento fornito è un file (sia tramite un percorso che tramite
l'url \texttt{file:/}), sono usati i file compressi in una sotto directory
\dirname{archive} del percorso (i file sono disponibili anche se non
compressi).
%If the given argument is local (either a path or a \texttt{file:/} url),
%compressed files in an \dirname{archive} subdirectory of the repository
%path are used (even if uncompressed files are available as well).


\subsection{Azioni successive all'installazione}
%\subsection{Post-install actions}
\label{sec:postinstall}

Potrebbe essere necessario compiere alcune azioni dopo l'installazione.
%Some post-install may be required.

\subsubsection{Windows}

Ma sotto Windows, l'installatore si prende cura di tutto.
%But on Windows, the installer takes care of everything.

\subsubsection{Se i collegamenti simbolici sono stati creati}
%\subsubsection{If symlinks were created}

Se hai scelto di creare i collegamenti simbolici nelle directory di
sistema (accennato nella sezione~\ref{sec:options}), allora non c'è
bisogno di modificare le variabili d'ambiente.
%If you elected to create symlinks in standard directories (mentioned in
%section~\ref{sec:options}), then there is no need to edit environment
%variables.

\subsubsection{Variabili d'ambiente per Unix}
%\subsubsection{Environment variables for Unix}
\label{sec:env}

La directory degli eseguibili per la tua piattaforma deve essere aggiunta
al percorso di ricerca. Ogni piattaforma supportata ha la propria sotto
directory all'interno di \dirname{TEXDIR/bin}. Vedi la
figura~\ref{fig:bin-text} per la lista delle sotto directory e le
piattaforme corrispondenti.
%The directory of the binaries for your platform must be added to
%the search path. Each supported platform has its own subdirectory
%under \dirname{TEXDIR/bin}. See figure~\ref{fig:bin-text} for the
%list of subdirectories and corresponding platforms.

Puoi anche aggiungere le directory della documentazione in pagine di
manuale e Info ai loro rispettivi percorsi di ricerca, se vuoi che gli
strumenti di sistema le trovino. Le pagine di manuale potrebbero essere
trovate automaticamente dopo l'aggiunta al \envname{PATH}.
%You can also add the documentation man and Info directories to their
%respective search paths, if you want the system tools to find them.
%The man pages might be found automatically after the addition to
%\envname{PATH}.

Per le shell compatibili con la Bourne, come \prog{bash}, e usando
\GNU/Linux per Intel x86 e le directory predefinite come esempio, il file
da modificare potrebbe essere \filename{$HOME/.profile} (o un qualunque
file che venga letto da \filename{.profile}) e le linee da aggiungere
diventerebbero simili alle seguenti:
%For Bourne-compatible shells such as \prog{bash}, and using Intel x86
%GNU/Linux and a default directory setup as an example, the file to edit
%might be \filename{$HOME/.profile} (or another file sourced by
%\filename{.profile}, and the lines to add would look like this:

\begin{sverbatim}
PATH=/usr/local/texlive/2009/bin/i386-linux:$PATH; export PATH
MANPATH=/usr/local/texlive/2009/texmf/doc/man:$MANPATH; export MANPATH
INFOPATH=/usr/local/texlive/2009/texmf/doc/info:$INFOPATH; export INFOPATH
\end{sverbatim}

Per csh o tcsh, il file da modificare tipicamente è
\filename{$HOME/.cshrc} e le linee da aggiungere potrebbero essere come:
%For csh or tcsh, the file to edit is typically \filename{$HOME/.cshrc}, and
%the lines to add might look like:

\begin{sverbatim}
setenv PATH /usr/local/texlive/2009/bin/i386-linux:$PATH
setenv MANPATH /usr/local/texlive/2009/texmf/doc/man:$MANPATH
setenv INFOPATH /usr/local/texlive/2009/texmf/doc/info:$INFOPATH
\end{sverbatim}

Se già hai delle linee così fatte da qualche parte in uno dei file citati,
naturalmente devi semplicemente unirci le directory di \TL\ come è più
opportuno.
%If you already have settings somewhere in your ``dot'' files, naturally
%the \TL\ directories should simply be merged in as appropriate.

\subsubsection{Variabili d'ambiente: configurazione globale}
%\subsubsection{Environment variables: global configuration}
\label{sec:envglobal}

Se vuoi che queste modifiche siano globali oppure vuoi che si applichino
per ciascun nuovo utente aggiunto al sistema, allora devi cavartela da
solo; c'è semplicemente troppa varietà tra i diversi sistemi nel come e
dove queste cose siano modificate.
%If you want to make these changes globally, or for a user newly added to
%the system, then you are on your own; there is just too much variation
%between systems in how and where these things are configured.

I nostri due suggerimenti sono i seguenti: 1)~potresti controllare il file
\filename{/etc/manpath.config} e, se presente, aggiungere linee del tipo
%Our two hints are: 1)~you may want to check for a file
%\filename{/etc/manpath.config} and, if present, add lines such as

\begin{sverbatim}
MANPATH_MAP /usr/local/texlive/2009/bin/i386-linux \
            /usr/local/texlive/2009/texmf/doc/man
\end{sverbatim}

E 2)~controllare il file \filename{/etc/environment} che potrebbe definire
il percorso di ricerca ed altre variabili di ambiente predefinite.
%And 2)~check for a file \filename{/etc/environment} which may define the
%search path and other default environment variables.

Noi creiamo anche un collegamento simbolico chiamato \code{man} in ogni
directory degli eseguibili (sotto Unix). Alcuni programmi \code{man}, come
lo standard \code{man} di \MacOSX, lo individueranno automaticamente,
ovviando alla necessità di una qualsiasi impostazione per le pagine di
manuale.
%We also create a symbolic link named \code{man} in each (Unix) binary
%directory.  Some \code{man} programs, such as the standard \MacOSX\
%\code{man}, will automatically find that, obviating the need for any man
%page setup.


\subsubsection{Configurazione dei font per \XeTeX}
%\subsubsection{Font configuration for \XeTeX}
\label{sec:font-conf-xetex}

Se hai installato il pacchetto \filename{xetex} su un sistema compatibile
Unix, hai bisogno di configurare il tuo sistema se desideri che \XeTeX\
sia in grado di trovare i font distribuiti con \TL. Per facilitare questo
compito, quando il pacchetto \pkgname{xetex} è installato (sia durante
l'installazione iniziale che successivamente), il file di configurazione
necessario è creato in
\filename{TEXMFSYSVAR/fonts/conf/texlive-fontconfig.conf}.
%If you have installed the \filename{xetex} package on a Unix-compatible
%system, you need to configure your system if you want \XeTeX\ to be able
%to find the fonts shipped with \TL.  To facilitate this, when the
%\pkgname{xetex} package is installed (either at initial installation or
%later), the necessary configuration file is created in
%\filename{TEXMFSYSVAR/fonts/conf/texlive-fontconfig.conf}.

Per impostare i font di \TL{} per l'uso nell'intero sistema (assumendo che
tu abbia gli opportuni privilegi), procedi come segue:
%To set up the \TL{} fonts for system-wide use (assuming you have
%suitable privileges), proceed as follows:
\begin{enumerate*}
\item Copia il file \filename{texlive-fontconfig.conf} in
\dirname{/etc/fonts/conf.d/09-texlive.conf}.
%\item Copy the \filename{texlive-fontconfig.conf} file to
%\dirname{/etc/fonts/conf.d/09-texlive.conf}.
\item Esegui \Ucom{fc-cache -fsv}.
%\item Run \Ucom{fc-cache -fsv}.
\end{enumerate*}

Se non hai i privilegi sufficienti per completare i passi precedenti, in
alternativa puoi fare quanto segue per rendere i font di \TL{} disponibili
solo a te stesso come utente di \XeTeX{}:
%If you do not have sufficient privileges to carry out the steps above,
%you can instead do the following to make the \TL{} fonts available
%to you as an individual \XeTeX{} user:
\begin{enumerate*}
\item Copia il file \filename{texlive-fontconfig.conf} in
      \filename{~/.fonts.conf}, dove \filename{~} è la tua directory di
      home.
%\item Copy the \filename{texlive-fontconfig.conf} file to
%      \filename{~/.fonts.conf}, where \filename{~} is your home directory.
\item Esegui \Ucom{fc-cache -fv}.
%\item Run \Ucom{fc-cache -fv}.
\end{enumerate*}


\subsubsection{Esecuzione dal \DVD}
%\subsubsection{When running from DVD}

Normalmente, un programma di \TL{} consulta il file \filename{texmf.cnf}
per trovare la posizione delle varie directory. La ricerca di questo file
avviene in alcuni percorsi relativi alla posizione del programma stesso.
Tuttavia, questo schema si spezza quando un programma è eseguito dal
\DVD{}: il \DVD{} è di sola lettura. Alcuni dei percorsi che devono essere
registrati in \filename{texmf.cnf} sono noti solo al momento
dell'installazione, quindi questo file non può trovarsi sul \DVD{} e deve
essere posizionato da qualche altra parte. Tutto ciò rende necessario
definire una variabile d'ambiente \envname{TEXMFCNF} che dica ai programmi
di \TL{} in quale directory trovare \filename{texmf.cnf}. È ancora
necessario modificare la variabile d'ambiente \envname{PATH}, come
descritto precedentemente.
%Normally, a \TL{} program consults a file \filename{texmf.cnf} for the
%location of the various trees. It looks for this file in a series of
%locations relative to its own location. However, this scheme breaks down
%when a program is run from \DVD{}: the \DVD{} is read-only. Some of the
%paths to be recorded in \filename{texmf.cnf} are only known at
%installation time, so this file cannot be on the \DVD{} and must be
%placed somewhere else. This makes it necessary to define an environment
%\envname{TEXMFCNF} variable which tells \TL{} programs in what directory
%to find this \filename{texmf.cnf}.  It is also still necessary to modify
%the \envname{PATH} environment variable, as described before.

Alla fine dell'installazione, l'installatore dovrebbe aver scritto un
messaggio in cui dà il valore a cui la variable \envname{TEXMFCNF}
dovrebbe essere impostata. Nel caso tu lo abbia perso: questo valore è
\dirname{$TEXMFSYSVAR/web2c}. Per l'impostazione predefinita,
\dirname{/usr/local/texlive/2009/texmf-var/web2c}, hai bisogno delle
seguenti righe
%At the end of the installation, the installer should have printed a
%message giving the value to which \envname{TEXMFCNF} should be set. In
%case you missed it: this value is \dirname{$TEXMFSYSVAR/web2c}. For the
%default, \dirname{/usr/local/texlive/2009/texmf-var/web2c}, you need the
%lines
\begin{sverbatim}
TEXMFCNF=/usr/local/texlive/2009/texmf-var/web2c; export TEXMFCNF
\end{sverbatim}
oppure, per csh/tcsh:
%or, for [t]csh:
\begin{sverbatim}
setenv TEXMFCNF /usr/local/texlive/2009/texmf-var/web2c
\end{sverbatim}

Questa opzione è molto utile quando vuoi eseguire \TL{} sul tuo sistema,
ma non hai abbastanza spazio su disco per installarlo. Se vuoi una \TL{}
davvero ``portatile'' che sia auto contenuta, ad esempio per una penna
USB, consulta la sezione~\ref{sec:portable-tl}.
%This option is most useful when you want to run \TL{} on your own
%system, but don't have enough disk space to install it.  If you want a
%truly `portable' \TL{} that is self-contained, e.g., for a USB stick,
%see section~\ref{sec:portable-tl}.

\subsubsection{\ConTeXt{} Mark IV}

Il ``vecchio'' \ConTeXt{} dovrebbe funzionare senza alcuna variazione. Il
nuovo \ConTeXt{} ``Mark IV'' richiederà un'impostazione manuale; consulta
la pagina \url{http://wiki.contextgarden.net/Running_Mark_IV}.
%The `old' \ConTeXt{} should run out of the box. The new `Mark IV'
%\ConTeXt{} will require manual setup; see
%\url{http://wiki.contextgarden.net/Running_Mark_IV}.


\subsubsection{Integrare le macro locali e personali}
%\subsubsection{Integrating local and personal macros}
\label{sec:local-personal-macros}

Questo aspetto è stato già menzionato implicitamente nella
sezione~\ref{sec:texmftrees}: \dirname{TEXMFLOCAL}
(\dirname{/usr/local/texlive/texmf-local} o
\verb|%SystemDrive%\texlive\texmf-local|, come predefiniti) è pensata per
i font e le macro installati localmente nell'intero sistema; e
\dirname{TEXMFHOME} (\dirname{$HOME/texmf} o \verb|%USERPROFILE%\texmf|,
come predefiniti) è per i font e le macro personali. Queste directory sono
pensate per restare fisse da una edizione all'altra e il loro contenuto è
visto automaticamente da ogni nuova edizione di \TL{}. Quindi, è meglio
evitare di modificare la definizione di \dirname{TEXMFLOCAL} dall'essere
troppo lontana dalla directory principale di \TL{}, altrimenti dovrai
modificare manualmente le future edizioni. % FIXME: che diavolo vuol dire?
%This is already mentioned implicitly in section~\ref{sec:texmftrees}:
%\dirname{TEXMFLOCAL} (by default,
%\dirname{/usr/local/texlive/texmf-local} or
%\verb|%SystemDrive%\texlive\texmf-local|)
%is intended for system-wide local fonts and macros; and
%\dirname{TEXMFHOME} (by default, \dirname{$HOME/texmf} or
%\verb|%USERPROFILE%\texmf|), is for personal fonts and macros.  These
%directories are intended to stick around from release to release, and
%have their content seen automatically by a new \TL{} release.
%Therefore, it is best to refrain from changing the definition of
%\dirname{TEXMFLOCAL} to be too far away from the main \TL{} directory,
%or you will need to manually change future releases.

In entrambe le locazioni, i file dovrebbero essere posizionati nelle
proprie opportune sotto directory; visita \url{http://tug.org/tds} o
consulta \filename{texmf/web2c/texmf.cnf}. Ad esempio, il file di una
classe o un pacchetto \LaTeX{} andrebbero posizionati in
\dirname{TEXMFLOCAL/tex/latex} o \dirname{TEXMFHOME/tex/latex}, oppure in
una sotto directory di uno di questi.
%For both trees, files should be placed in their proper subdirectories;
%see \url{http://tug.org/tds} or consult
%\filename{texmf/web2c/texmf.cnf}. For instance, a \LaTeX{} class file or
%package should be placed in \dirname{TEXMFLOCAL/tex/latex} or
%\dirname{TEXMFHOME/tex/latex}, or a subdirectory thereof.

\dirname{TEXMFLOCAL} richiede un database dei nomi dei file aggiornato,
altrimenti i file non saranno trovati. Puoi aggiornarlo con il comando
\cmdname{mktexlsr} o utilizzando il pulsante ``Inizializza nuovamente il
database dei file'' dalla pagina di configurazione di \prog{tlmgr} in
modalità \GUI.
%\dirname{TEXMFLOCAL} requires an up-to-date filename database, or files
%will not be found.  You can update it with the command
%\cmdname{mktexlsr} or use the `Reinit file database' button on the
%configuration tab of \prog{tlmgr} in \GUI\ mode.

\subsubsection{Integrare font di terze parti}
%\subsubsection{Integrating third-party fonts}

Sfortunatamente, si tratta di un argomento ingarbugliato. Dimenticatene a
meno che tu non voglia scavare nei numerosi dettagli dell'installazione di
\TeX{}. Non dimenticare per prima cosa di controllare cosa ricevi
liberamente: consulta la sezione~\ref{sec:tl-fonts}.
%This is unfortunately a messy topic.  Forget about it unless you want to
%delve into many details of the \TeX{} installation.  Don't forget to
%check first what you get for free: see section~\ref{sec:tl-fonts}.

Una possibile alternativa è quella di usare \XeTeX (consulta la
sezione~\ref{sec:tex-extensions}), che consente di usare i font del
sistema operativo senza alcuna installazione in \TeX.
%A possible alternative is to use \XeTeX (see
%section~\ref{sec:tex-extensions}), which lets you use operating system
%fonts without any installation in \TeX.

Se comunque hai bisogni di integrare font di terze parti, visita la pagina
\url{http://tug.org/fonts/fontinstall.html} dove descriviamo la procedura
al meglio delle nostre possibilità.
%If you do need to do this, see
%\url{http://tug.org/fonts/fontinstall.html} for our best effort at
%describing the procedure.

\subsection{Collaudare l'installazione}
%\subsection{Testing the installation}
\label{sec:test-install}

Dopo aver installato \TL{} nel modo migliore, vorrai naturalmente
collaudarla, così da poter iniziare a creare bellissimi documenti e\slash
o font.
%After installing \TL{} as best you can, you naturally want to test
%it out, so you can start creating beautiful documents and\slash or fonts.

Questa sezione fornisce alcune procedure elementari per verificare che il
nuovo sistema funzioni. Qui forniamo i comandi per Unix; sotto \MacOSX e
Windows, è più probabile che eseguirai le prove tramite un'interfaccia
grafica, ma i principi sono gli stessi.
%This section gives some basic procedures for testing that the new system
%is functional.  We give Unix commands here; under \MacOSX{} and Windows,
%you're more likely to run the tests through a graphical interface, but
%the principles are the same.

\begin{enumerate}

\item Assicurati per prima cosa di poter eseguire il programma
\cmdname{tex}:
%\item Make sure that you can run the \cmdname{tex} program in the first
%place:
\begin{alltt}
> \Ucom{tex -{}-version}
TeX 3.1415926 (TeX Live 2009)
kpathsea version 5.0.0
Copyright 2009 D.E. Knuth.
...
\end{alltt}
Se la risposta è ``comando non trovato'' al posto delle informazioni sulla
versione e sul copyright, oppure se la versione riportata è vecchia, molto
probabilmente nel tuo \envname{PATH} non hai la giusta sotto directory
\dirname{bin}. Consulta le informazioni sulle impostazioni dell'ambiente a
\p.\pageref{sec:env}.
%If this comes back with `command not found' instead of version and
%copyright information, or with an older version, most likely you don't
%have the correct \dirname{bin} subdirectory in your \envname{PATH}.  See
%the environment-setting information on \p.\pageref{sec:env}.

\item Componi un file \LaTeX{} elementare:
%\item Process a basic \LaTeX{} file:
\begin{alltt}
> \Ucom{latex sample2e.tex}
This is pdfTeX, Version 3.1415926-1.40.10 (TeX Live 2009)
...
Output written on sample2e.dvi (3 pages, 7484 bytes).
Transcript written on sample2e.log.
\end{alltt}
Se il comando fallisce nel trovare \filename{sample2e.tex} oppure altri
file, molto probabilmente hai qualche interferenza proveniente da vecchie
variabili o vecchi file di configurazione: noi raccomandiamo di cominciare
eliminando tutte le variabili d'ambiente legate a \TeX\ (per una analisi
approfondita, puoi chiedere a \TeX{} di produrre un rapporto su cosa
esattamente cerchi e trovi; consulta la sezione ``Risoluzione dei
problemi'' a pagina~\pageref{sec:debugging}).
%If this fails to find \filename{sample2e.tex} or other files, most
%likely you have interference from old environment variables or
%configuration files; we recommend unsetting all \TeX-related environment
%variables for a start.  (For a deep analysis, you can ask \TeX{} to
%report on exactly what it is searching for, and finding; see ``Debugging
%actions'' on page~\pageref{sec:debugging}.)

\item Visualizza un'anteprima del risultato:
%\item Preview the result online:
\begin{alltt}
> \Ucom{xdvi sample2e.dvi}    # Unix
> \Ucom{dviout sample2e.dvi}  # Windows
\end{alltt}
Dovresti vedere una nuova finestra con un bel documento che spiega alcuni
fondamenti di \LaTeX{} (vale la pena di leggerlo, comunque, se sei nuovo
di \TeX). Devi avere X in esecuzione affinché \cmdname{xdvi} funzioni; se
non è così oppure se la tua variabile d'ambiente \envname{DISPLAY} è
male impostata, otterrai un errore \samp{Can't open display}.
%You should see a new window with a nice document explaining some of the
%basics of \LaTeX{}.  (Well worth reading, by the way, if you're new to
%\TeX.)  You do have to be running under X for \cmdname{xdvi} to work; if
%you're not, or your \envname{DISPLAY} environment variable is set
%incorrectly, you'll get an error \samp{Can't open display}.

\item Crea un file \PS{} per la stampa o la visualizzazione:
%\item Create a \PS{} file for printing or display:
\begin{alltt}
> \Ucom{dvips sample2e.dvi -o sample2e.ps}
\end{alltt}

\item Crea un file \acro{PDF} al posto di un \dvi{}; questo comando
elabora il file \filename{.tex} e scrive direttamente un \acro{PDF}:
%\item Create a \acro{PDF} file instead of \dvi{}; this processes the
%\filename{.tex} file and writes \acro{PDF} directly:
\begin{alltt}
> \Ucom{pdflatex sample2e.tex}
\end{alltt}

\item Visualizza un'anteprima del \acro{PDF}:
%\item Preview the \acro{PDF} file:
\begin{alltt}
> \Ucom{gv sample2e.pdf}
\textrm{o:}
%\textrm{or:}
> \Ucom{xpdf sample2e.pdf}
\end{alltt}
Né \cmdname{gv}, né \cmdname{xpdf} sono inclusi in \TL{}, quindi devi
installarli separatamente. Visita, rispettivamente,
\url{http://www.gnu.org/software/gv} e \url{http://www.foolabs.com/xpdf}.
C'è un'infinità di altri visualizzatori \acro{PDF}. Per Windows, noi
raccomandiamo di provare Sumatra PDF
(\url{http://blog.kowalczyk.info/software/sumatrapdf}).
%Neither \cmdname{gv} nor \cmdname{xpdf} are included in \TL{}, so you
%must install them separately.  See \url{http://www.gnu.org/software/gv}
%and \url{http://www.foolabs.com/xpdf}, respectively.  There are plenty
%of other \acro{PDF} viewers, too.  For Windows, we recommend trying
%Sumatra PDF (\url{http://blog.kowalczyk.info/software/sumatrapdf}).

\item Ulteriori file di prova che potresti trovare utili, in aggiunta a
\filename{sample2e.tex}:
%\item Standard test files you may find useful in addition to
%\filename{sample2e.tex}:

\begin{ttdescription}
\item [small2e.tex] Un documento più semplice di \filename{sample2e}, per
ridurre la dimensione dell'input qualora si riscontrino problemi.
%\item [small2e.tex] A simpler document than \filename{sample2e}, to
%reduce the input size if you're having troubles.
\item [testpage.tex] Verifica se la tua stampante introduce un qualche
margine.
%\item [testpage.tex] Test if your printer introduces any offsets.
\item [nfssfont.tex] Per stampare tabelle e prove di font.
%\item [nfssfont.tex] For printing font tables and tests.
\item [testfont.tex] Ancora per le tabelle dei font, ma usando plain
\TeX{}.
%\item [testfont.tex] Also for font tables, but using plain \TeX{}.
\item [story.tex] Il più canonico dei file di prova per (plain) \TeX{}.
Devi digitale \samp{\bs bye} al prompt \code{*} dopo aver lanciato
\samp{tex story.tex}.
%\item [story.tex] The most canonical (plain) \TeX{} test file of all.
%You must type \samp{\bs bye} to the \code{*} prompt after \samp{tex
%story.tex}.
\end{ttdescription}

\item Se hai installato il pacchetto \filename{xetex}, puoi verificare che
riesca ad accedere ai font di sistema in questo modo:
%\item If you have installed the \filename{xetex} package, you can test
%its access to system fonts as follows:
\begin{alltt}
> \Ucom{xetex opentype-info.tex}
This is XeTeX, Version 3.1415926\dots
...
Output written on opentype-info.pdf (1 page).
Transcript written on opentype-info.log.
\end{alltt}
Se ottieni un messaggio di errore che riporta ``Invalid fontname `Latin
Modern/ICU\dots'', allora devi configurare il tuo sistema in modo tale che
\XeTeX{} possa trovare i font distribuiti con \TL. Consulta la
sezione~\ref{sec:font-conf-xetex}.
%If you get an error message saying ``Invalid fontname `Latin Modern
%Roman/ICU'\dots'', then you need to configure your system so that
%\XeTeX{} can find the fonts shipped with \TL.  See
%Section~\ref{sec:font-conf-xetex}.

\end{enumerate}

\subsection{Collegamenti ad ulteriori software scaricabili}
%\subsection{Links for additional downloadable software}

Se sei nuovo di \TeX{} oppure in alternativa hai bisogno di aiuto nella
scrittura di documenti \TeX{} o \LaTeX{}, visita la pagina
\url{http://tug.org/begin.html} per alcune risorse introduttive.
%If you are new to \TeX{}, or otherwise need help with actually writing
%\TeX{} or \LaTeX{} documents, please visit
%\url{http://tug.org/begin.html} for some introductory resources.

Ecco i collegamenti ad alcuni altri strumenti che puoi pensare di
installare:
%Links for some other tools you may consider installing:
\begin{description}
\item[Ghostscript] \url{http://www.cs.wisc.edu/~ghost/}
\item[Perl] \url{http://www.perl.org/} con i pacchetti supplementari da
      \acro{CPAN}, \url{http://www.cpan.org/}
%\item[Perl] \url{http://www.perl.org/} with
%      supplementary packages from \acro{CPAN}, \url{http://www.cpan.org/}
\item[ImageMagick] \url{http://www.imagemagick.com}, per l'elaborazione e
      la conversione di immagini
%\item[ImageMagick] \url{http://www.imagemagick.com}, for graphics
%      processing and conversion
\item[NetPBM] \url{http://netpbm.sourceforge.net/}, anche questo per le
      immagini.
%\item[NetPBM] \url{http://netpbm.sourceforge.net/}, also for graphics.

\item[Editor orientati a \TeX] C'è un'ampia scelta ed è una questione di
      gusto dell'utente. Eccone una selezione (alcuni sono solo per
      Windows).
%\item[\TeX-oriented editors] There is a wide choice, and it is a matter of the
%      user's taste. Here is a selection (a few here are for Windows only).
  \begin{itemize*}
  \item \cmdname{GNU Emacs} è disponibile per Windows, visita
        \url{http://www.gnu.org/software/emacs/windows/ntemacs.html}.
%  \item \cmdname{GNU Emacs} is available natively under Windows, see
%        \url{http://www.gnu.org/software/emacs/windows/ntemacs.html}.
  \item \cmdname{Emacs} con Auc\TeX\ per Windows è disponibile nella
        directory \path{tlpkg/support} sul \DVD\ di \TL; il suo sito è
        \url{http://www.gnu.org/software/auctex}.
%  \item \cmdname{Emacs} with Auc\TeX\ for Windows is available in
%        the directory \path{tlpkg/support} on the \TL \DVD; its home
%        page is \url{http://www.gnu.org/software/auctex}.
  \item \cmdname{LEd} è disponibile su \url{http://www.ctan.org/support/LEd}.
%  \item \cmdname{LEd} is available from \url{http://www.ctan.org/support/LEd}.
  \item \cmdname{SciTE} è disponibile su
        \url{http://www.scintilla.org/SciTE.html}.
%  \item \cmdname{SciTE} is available from
%        \url{http://www.scintilla.org/SciTE.html}.
  \item \cmdname{Texmaker} è un software libero disponibile su
        \url{http://www.xmlmath.net/texmaker/}.
%  \item \cmdname{Texmaker} is free software, available from
%        \url{http://www.xmlmath.net/texmaker/}.
  \item \cmdname{TeXnicCenter} è un software libero, disponibile su
        \url{http://www.texniccenter.org} e nella distribuzione
        pro\TeX{}t.
%  \item \cmdname{TeXnicCenter} is free software, available from
%        \url{http://www.texniccenter.org} and in the pro\TeX{}t distribution.
  \item \cmdname{TeXworks} è un software libero disponibile su
        \url{http://tug.org/texworks} e installato per Windows e \MacOSX\
        come parte di \TL.
%  \item \cmdname{TeXworks} is free software, available from
%        \url{http://tug.org/texworks} and installed for Windows and
%        \MacOSX\ as part of \TL.
  \item \cmdname{Vim} è un software libero, disponibile su
        \url{http://www.vim.org}.
%  \item \cmdname{Vim} is free software, available from
%        \url{http://www.vim.org}.
  \item \cmdname{WinShell} è disponibile su \url{http://www.winshell.de}.
%  \item \cmdname{WinShell} is available from \url{http://www.winshell.de}.
  \item \cmdname{WinEdt} è un software shareware disponibile su
        \url{http://tug.org/winedt} o \url{http://www.winedt.com}.
%  \item \cmdname{WinEdt} is shareware available though
%        \url{http://tug.org/winedt} or \url{http://www.winedt.com}.
  \end{itemize*}
\end{description}
Per un elenco molto più lungo di pacchetti e programmi, visita
\url{http://tug.org/interest.html}.
%For a much longer list of packages and programs, see
%\url{http://tug.org/interest.html}.


\section{Installazioni di rete}
%\section{Network installations}
\label{sec:netinstall}

\TL{} è stato progettato per essere condiviso tra diversi utenti ed anche
tra diversi sistemi attraverso una rete. Con una strutturazione delle
directory prestabilita, non sono configurati percorsi rigidi: le posizioni
dei file necessari ai programmi di \TL{} sono individuate in relazione a
quelle dei programmi. Puoi trovare la conferma di questo comportamento nel
file di configurazione principale \filename{$TEXMFMAIN/web2c/texmf.cnf},
che contiene linee come
%\TL{} has been designed to be shared between different users, and even
%different systems on a network. With a standard directory layout, no
%hard paths are configured: the locations for files needed by \TL{}
%programs are found relative to the programs.  You can see this in the
%principal configuration file \filename{$TEXMFMAIN/web2c/texmf.cnf},
%which contains lines such as
\begin{sverbatim}
TEXMFMAIN = $SELFAUTOPARENT/texmf
...
TEXMFLOCAL = $SELFAUTOPARENT/../texmf-local
\end{sverbatim}
Ciò significa che, per ottenere una configurazione funzionante, è
sufficiente aggiungere la directory degli eseguibili di \TL{} adatti
alla propria piattaforma nel percorso di ricerca.
%This means that adding the directory for \TL{} executables for their
%platform to their search path is sufficient to get a working setup.

Per la stessa ragione, puoi anche installare \TL{} localmente e muovere in
un secondo momento l'intera gerarchia verso una locazione di rete.
%By the same token, you can also install \TL{} locally and then move
%the entire hierarchy afterwards to a network location.

Per Windows, uno script di installazione di rete esemplificativo chiamato
\filename{w32client} può essere scaricato da
\url{http://tug.org/texlive/w32client.html}. Questo script crea le
impostazioni e i collegamenti nei menu per usare una installazione di
\TL{} esistente su una \acro{LAN}. Registra anche un disinstallatore
\filename{w32unclient}, disponibile nello stesso file zip. Visita la
pagina web indicata per maggiori informazioni.
%For Windows, a sample network installation script named
%\filename{w32client} can be downloaded through
%\url{http://tug.org/texlive/w32client.html}.  It creates settings and
%menu shortcuts for using an existing \TL{} installation on a \acro{LAN}.
%It also registers an uninstaller \filename{w32unclient}, available
%in the same zip file.  See the web page for more information.


\section{\protect\TL{} su DVD e USB per la massima portabilità}
%\section{Maximally portable \protect\TL{} on DVD and USB}
\label{sec:portable-tl}

L'opzione ``esecuzione dal \DVD{}'' descritta nella
sezione~\ref{sec:fromdvd} va bene se il sistema ti appartiene, ma se sei
solo un utente ospite del sistema di qualcun altro allora è probabile che
tu voglia qualcosa con il minor numero di effetti collaterali.
%The `running from \DVD{}' option described in
%section~\ref{sec:fromdvd} is fine for your own system, but if you
%are a guest on somebody else's system then you would probably like
%something with minimal side effects.

All'apertura del \DVD{} \TL{} oppure nella sotto directory
\dirname{texlive} del \DVD{} \TK{} ci sono uno script
\filename{tl-portable} (Unix) e un file batch \filename{tl-portable.bat}
(Windows) che avviano una seconda shell\slash prompt dei comandi in cui
l'ambiente è stato impostato per accedere direttamente a \TL{} sul \DVD.
%In the root of the \TL{} \DVD, or the \dirname{texlive} subdirectory of
%the \TK{} \DVD, are a \filename{tl-portable} script (Unix) and a
%\filename{tl-portable.bat} batch file (Windows) which start up a
%secondary shell\slash command prompt with environment settings
%for directly accessing the \TL{} on the \DVD.

Alla prima esecuzione, alcuni file saranno generati in una directory
\dirname{~/.tlportable2009}, dunque sarà necessario attendere un po' di
tempo. Alle esecuzioni successive, comunque, l'avvio sarà quasi
istantaneo.
%When it runs for the first time, some files will be generated in a
%directory \dirname{~/.tlportable2009}, which will take a little time.
%On subsequent runs, though, it will start almost instantaneously.

Il resto del sistema non avrà alcuna informazione su \TL. Se vuoi che il
tuo editor sappia della presenza di \TL, puoi avviarlo in parallelo da una
seconda sessione di \filename{tl-portable}. 
%The rest of the system will be unaware of \TL. If you want your editor
%to be aware of this \TL, then you can start it from a second, parallel,
%such \filename{tl-portable} session.

Puoi usare \filename{tl-portable} anche per eseguire \TL{} su una penna
\acro{USB}. In questo caso copia tutti i file che si trovano direttamente
nella directory di \TL{} e il contenuto (almeno) delle directory
\dirname{bin}, \dirname{texmf}, \dirname{texmf-dist} e \dirname{tlpkg}
sulla penna. Questa operazione potrebbe richiedere del tempo! Se stai
copiando su una penna \acro{USB} formattata come \acro{FAT}32, assicurati
di dereferenziare i collegamenti simbolici (\code{cp -L}). Se sulla penna
è presente una directory \dirname{texmf-local}, questa sarà individuata e
usata.
%You can also use \filename{tl-portable} to run \TL{} on a
%\acro{USB} stick.  In this case, copy all the top-level files and
%the contents of the \dirname{bin}, \dirname{texmf},
%\dirname{texmf-dist}, and \dirname{tlpkg} directories (at least) to
%the stick. This may take quite a while! If you are copying to a
%\acro{FAT}32-formatted \acro{USB} stick, make sure to dereference
%symbolic links (\code{cp -L}).  A \dirname{texmf-local} directory on
%the stick will also be found and used.

A questo punto esegui \filename{tl-portable} dalla penna, come descritto
sopra. In questo caso, lo script si accorgerà che la penna è scrivibile e
la userà per i file generati. Puoi anche masterizzare i contenuti della
penna, comprensivi dei file generati, su un \DVD{} se ritieni che sia più
conventiente (ad esempio) per consegnarlo ad altri.
%Then run \filename{tl-portable} from the root of the stick, as above.
%In this case, the script will notice that the stick is writable and use
%it for generated files.  You could burn the resulting contents of the
%stick back to \DVD{} if that is more convenient to (for example) give to
%others.


\section{\cmdname{tlmgr}: gestire la tua installazione}
%\section{\cmdname{tlmgr}: Managing your installation}
\label{sec:tlmgr}

\begin{figure}[tb]
\tlpng{tlmgr-gui}{\linewidth}
\caption{\prog{tlmgr} in modalità \GUI. L'elenco dei
  pacchetti/collezioni/schemi è mostrato solo dopo aver premuto il
  pulsante ``Carica''.}\label{fig:tlmgr-gui}
%\caption{\prog{tlmgr} in \GUI\ mode. The list of
%  packages/collections/schemes only shows up after clicking the
%  `Load' button.}\label{fig:tlmgr-gui}
\end{figure}

\begin{figure}[tb]
\tlpng{tlmgr-config}{.8\linewidth}
\caption{\texttt{tlmgr} in modalità \GUI: pagina di configurazione}
\label{fig:tlmgr-config}
%\caption{\texttt{tlmgr} in GUI mode: Configuration tab}\label{fig:tlmgr-config}
\end{figure}

\TL{} include un programma chiamato \prog{tlmgr} per gestire \TL{} dopo
l'installazione iniziale. I programmi \prog{updmap}, \prog{fmtutil} e
\prog{texconfig} sono ancora inclusi e saranno mantenuti nel futuro, ma
\prog{tlmgr} è ora l'interfaccia preferita. Le sue funzionalità includono:
%\TL{} includes a program named \prog{tlmgr} for managing \TL{} after the
%initial installation.  The programs \prog{updmap}, \prog{fmtutil} and
%\prog{texconfig} are still included and will be retained in the future,
%but \prog{tlmgr} is now the preferred interface.  Its capabilities
%include:

\begin{itemize*}
\item installare, aggiornare, archiviare, ripristinare e disinstallare i
  singoli pacchetti, eventualmente tenendo conto delle dipendenze;
%\item installing, updating, backing up, restoring, and uninstalling
%  individual packages, optionally taking dependencies into account;
\item cercare ed elencare pacchetti, ecc.;
%\item searching for and listing packages, etc.;
\item elencare, aggiungere e rimuovere architetture;
%\item listing, adding, and removing architectures;
\item cambiare le opzioni di installazione come la dimensione dei fogli e
  il percorso di installazione (consulta la sezione~\ref{sec:location}).
%\item changing installation options such as paper size and source
%  location (see section~\ref{sec:location}).
\end{itemize*}
\textit{Attenzione:} \prog{tlmgr} non è stato progettato o collaudato con
le installazioni che eseguono i programmi dal \DVD.
%\textit{Warning:} \prog{tlmgr} has not been designed for or tested
%with installations which run from \DVD.

\subsection{Modalità GUI di \cmdname{tlmgr}}
%\subsection{\cmdname{tlmgr} GUI mode}
\prog{tlmgr} può essere avviato in modalità \GUI{} con
%\prog{tlmgr} can be started in \GUI{} mode with
\begin{alltt}
> \Ucom{tlmgr -gui}
\end{alltt}
oppure sotto Windows tramite il menu Start: \texttt{Start},
\texttt{Programmi}, \texttt{TeX Live 2009}, \texttt{TeX Live Manager}.
Dopo aver premuto il pulsante ``Carica'' il programma mostra un elenco dei
pacchetti disponibili ed installati \Dash\ questi ultimi hanno un ``(i)''
anteposto. Ovviamente assumiamo che il percorso di installazione attuale
sia valido e accessibile.
%or in Windows via the Start menu: \texttt{Start}, \texttt{Programs},
%\texttt{TeX Live 2009}, \texttt{TeX Live Manager}. After clicking `Load'
%it displays a list of available and installed packages\Dash the latter
%prepended with `(i)'. This assumes of course that the installation
%source is valid and reachable.

La figura~\ref{fig:tlmgr-config} mostra la pagina di configurazione.
%Figure~\ref{fig:tlmgr-config} shows the configuration tab.

\subsection{Esempi di esecuzioni di \cmdname{tlmgr} dalla riga di comando}
%\subsection{Sample \cmdname{tlmgr} command-line invocations}

Dopo l'installazione iniziale, puoi aggiornare il tuo sistema alle ultime
versioni disponibili con:
%After the initial installation, you can update your system to the latest
%versions available with:
\begin{alltt}
> \Ucom{tlmgr update -all}
\end{alltt}
Se l'operazione ti rende nervoso, prova prima
%If this makes you nervous, first try
\begin{alltt}
> \Ucom{tlmgr update -all -dry-run}
\end{alltt}
o (less prolisso):
%or (less verbose):
\begin{alltt}
> \Ucom{tlmgr update -list}
\end{alltt}

Questo esempio più complesso aggiunge una collezione, per il motore
Xe\TeX, da una directory locale:
%This more complex example adds a collection, for the engine Xe\TeX, from
%a local directory:

\begin{alltt}
> \Ucom{tlmgr -repository /local/mirror/tlnet install collection-xetex}
\end{alltt}
Genera il seguente risultato (ridotto):
%It generates the following output (abridged):
\begin{fverbatim}
install: collection-xetex
install: arabxetex
...
install: xetex
install: xetexconfig
install: xetex.i386-linux
running post install action for xetex
install: xetex-def
...
running mktexlsr
mktexlsr: Updating /usr/local/texlive/2009/texmf/ls-R...
...
running fmtutil-sys --missing
...
Transcript written on xelatex.log.
fmtutil: /usr/local/texlive/2009/texmf-var/web2c/xetex/xelatex.fmt installed.
\end{fverbatim}

Come puoi vedere, \prog{tlmgr} installa le dipendenze e si occupa di ogni
azione necessaria dopo l'installazione, incluso l'aggiornamento del
database dei nomi dei file e la generazione (o rigenerazione) dei formati.
Nell'esempio precedente, abbiamo generato nuovi formati per \XeTeX.
%As you can see, \prog{tlmgr} installs dependencies, and takes care of any
%necessary post-install actions, including updating the filename database
%and (re)generating formats.  In the above, we generated new formats for
%\XeTeX.

Per ottenere la descrizione di un pacchetto (o di una collezione o di uno
schema):
%To describe a package (or collection or scheme):
\begin{alltt}
> \Ucom{tlmgr show collection-latexextra}
\end{alltt}
che produce
%which produces
\begin{fverbatim}
package:    collection-latexextra
category:   Collection
shortdesc:  LaTeX supplementary packages
longdesc:   A large collection of add-on packages for LaTeX.
installed:  Yes
revision:   14675
\end{fverbatim}

Ultimo e più importante, per ottenere la documentazione completa, visita
\url{http://tug.org/texlive/tlmgr.html} oppure esegui:
%Last and most important, for full documentation see
%\url{http://tug.org/texlive/tlmgr.html}, or:
\begin{alltt}
> \Ucom{tlmgr -help}
\end{alltt}


\section{Note relative a Windows}
%\section{Notes on Windows}
\label{sec:windows}

\TL\ ha un singolo programma di installazione che è eseguito sia sotto
Windows che sotto Unix. L'unico modo per ottenere questo è stato
l'abbandono del supporto per le vecchie versioni di Windows, così adesso
\TL\ può essere installato solo su Windows 2000 e successivi.
%\TL\ has a single installer which runs on both Windows and Unix.  This
%was only possible by dropping support for older Windows versions, so
%\TL\ can now be installed only on Windows 2000 and later.


\subsection{Funzionalità specifiche per Windows}
%\subsection{Windows-specific features}
\label{sec:winfeatures}

Sotto Windows, l'installatore fa alcune cose in più:
%Under Windows, the installer does some extra things:
\begin{description}
\item[Menu e collegamenti.] Un nuovo sotto menu del menu Start, chiamato
  ``\TL{}'', viene creato e contiene le voci per alcuni programmi \GUI{}
  (\prog{tlmgr}, \prog{texdoctk}, il visualizzatore PostScript PS\_View
  \Dash\ \prog{psv}) e diversa documentazione. PS\_View ha anche un
  collegamento sul desktop sul quale trascinare i file PostScript da
  visualizzare.
%\item[Menus and shortcuts.] A new `\TL{}' submenu of the
%  Start menu is installed, which contains entries for some \GUI{}
%  programs (\prog{tlmgr}, \prog{texdoctk}, the PS\_View (\prog{psv})
%  PostScript previewer) and some documentation. PS\_View also gets a
%  shortcut on the desktop as a drag-and-drop target for PostScript
%  files.
\item[Impostazione automatica delle variabili d'ambiente.] Non sono
  richieste procedure di configurazione manuali.
%\item[Automatic setting of environment variables.] No manual
%  configuration steps are required.
\item[Disinstallatore.] L'installatore crea una voce sotto
  ``Installa/Rimuovi Programmi'' per \TL. Aprendo la pagina di
  disinstallazione di \prog{tlmgr} si viene informati di ciò.
%\item[Uninstaller.] The installer creates an entry under `Add/Remove
%  Programs' for \TL. The uninstall tab of \prog{tlmgr} refers to
%  this.
\end{description}

\subsection{Software aggiuntivo incluso sotto Windows}
%\subsection{Additional software included on Windows}

Un'installazione \TL{} per essere completa ha bisognod di alcuni pacchetti
di supporto che non si trovano abitualmente su una macchina Windows. \TL{}
fornisce questi pezzi mancanti:
%To be complete, a \TL installation needs support packages that are not
%commonly found on a Windows machine. \TL{} provides the missing
%pieces:
\begin{description}
\item[Perl e Ghostscript.] A causa dell'importanza di Perl e Ghostscript,
  \TL{} include una copia ``nascosta'' di questi programmi. I programmi di
  \TL{} che ne hanno bisogno sanno dove trovarli, ma non ne tradiscono la
  presenza tramite variabili d'ambiente o impostazioni sul registro di
  sistema. Non si tratta di installazioni complete e non dovrebbero
  interferire con installazioni nel sistema di Perl o Ghostscript.
%\item[Perl and Ghostscript.] Because of the importance of Perl and
%  Ghostscript, \TL{} includes `hidden' copies of these
%  programs. \TL{} programs that need them know where to find them,
%  but they don't betray their presence through environment variables
%  or registry settings. They aren't full-scale installations, and
%  shouldn't interfere with any system installations of Perl or
%  Ghostscript.

\item[PS\_View.] È installato anche PS\_View, un visualizzatore di file
  \PS{} e \acro{PDF}; vedi la figura~\ref{fig:psview}.
%\item[PS\_View.] Also installed is PS\_View, a \PS{} and \acro{PDF}
%  viewer; see figure~\ref{fig:psview}.

\begin{figure}[tb]
\tlpng{psview}{.6\linewidth}
\caption{PS\_View: ingrandimenti elevatissimi disponibili!}
\label{fig:psview}
%\caption{PS\_View: very high magnifications available!}\label{fig:psview}
\end{figure}

\item[dviout.] È installato anche \prog{dviout}, un visualizzatore di
  \acro{DVI}. La prima volta che visualizzi i file con \cmdname{dviout},
  saranno creati dei font in quanto quelli adatti allo schermo non sono
  installati. Dopo un po', avrai creato la maggior parte dei font che usi
  e raramente vedrai la finestra di creazione. Ulteriori informazioni
  possono essere trovate nella guida (caldamente raccomandata).
%\item[dviout.] Also installed is \prog{dviout}, a \acro{DVI} viewer.
%  At first, when you preview files with \cmdname{dviout}, it will create
%  fonts, because screen fonts were not installed. After a while, you
%  will have created most of the fonts you use, and you will rarely see
%  the font-creation window.  More information can be found in the
%  (highly recommended) on-line help.

\item[TeXworks.] \TeX{}works è un editor orientato a \TeX\ con un
  visualizzatore per \acro{PDF} incorporato. È fornito già preconfigurato
  per \TL.
%\item[TeXworks.]  \TeX{}works is a \TeX-oriented editor with
%  an integrated \acro{PDF} viewer. It comes already preconfigured for \TL.

\item[Strumenti a riga di comando.] Assieme ai soliti eseguibili per \TL{}
  vengono installate alcune versioni per Windows di tipici programmi Unix
  a riga di comando. Questi programmi includono \cmdname{gzip},
  \cmdname{unzip} e gli strumenti a riga di comando della suite
  \cmdname{xpdf}. Il visualizzatore \cmdname{xpdf} non è disponibile per
  Windows. Come alternativa, puoi scaricare Sumatra \acro{PDF}, che è
  basato su xpdf, da \url{http://blog.kowalczyk.info/software/sumatrapdf}.
%\item[Command-line tools.] A number of Windows ports of common Unix
%  command-line programs are installed along with the usual \TL{}
%  binaries. These include \cmdname{gzip}, \cmdname{unzip} and the
%  command-line utilities from the \cmdname{xpdf} suite.  The
%  \cmdname{xpdf} viewer itself is not available for
%  Windows. Instead, you can download the Sumatra \acro{PDF} viewer,
%  which is based on xpdf, from
%  \url{http://blog.kowalczyk.info/software/sumatrapdf}.

\item[fc-list, fc-cache ecc.] Gli strumenti dalla libreria fontconfig
  consentono a \XeTeX{} di gestire i font di sistema sotto Windows. Puoi
  usare \prog{fc-list} per trovare i nomi dei font da passare al comando
  esteso di Xe\TeX \cs{font}. Se è necessario, esegui per primo
  \prog{fc-cache} in modo da aggiornare le informazioni sui font.
%\item[fc-list, fc-cache et al.] The tools from the fontconfig library allow
%  \XeTeX{} to handle system fonts on Windows.  You can use
%  \prog{fc-list} to determine the font names to pass to Xe\TeX's
%  extended \cs{font} command. If necessary, run \prog{fc-cache}
%  first to update font information.

\end{description}


\subsection{Il profilo utente è home}
%\subsection{User Profile is Home}
\label{sec:winhome}

La controparte Windows di una directory di home di Unix è la directory
\verb|%USERPROFILE%|. Sotto Windows \acro{XP} e Windows 2000, di solito
è \verb|C:\Documents and Settings\<nomeutente>| e sotto Windows Vista è
\verb|C:\Users\<nomeutente>|. Nel file \filename{texmf.cnf} e in generale
in \KPS{}, \verb|~| verrà interpretato in modo appropriato sia sotto
Windows che sotto Unix.
%The Windows counterpart of a Unix home directory is the
%\verb|%USERPROFILE%| directory.  Under Windows \acro{XP} and Windows 2000, this
%is usually \verb|C:\Documents and Settings\<username>|, and under
%Windows Vista \verb|C:\Users\<username>|.  In the
%\filename{texmf.cnf} file, and \KPS{} in general, \verb|~| will expand
%appropriately on both Windows and Unix.


\subsection{Il registro di configurazione di Windows}
%\subsection{The Windows registry}
\label{sec:registry}

Windows memorizza quasi tutti i dati di configurazione nel suo registro.
Il registro contiene un insieme di chiavi organizzate gerarchicamente,
alcune delle quali sono le radici delle varie gerarchie. Le più importanti
per i programmi di installazione sono \path{HKEY_CURRENT_USER} e
\path{HKEY_LOCAL_MACHINE}, abbreviate in \path{HKCU} e \path{HKLM}. La
parte \path{HKCU} del registro si trova nella directory di home
dell'utente (vedi sezione~\ref{sec:winhome}). \path{HKLM} si trova di
solito in una sotto directory della cartella Windows.
%Windows stores nearly all configuration data in its registry.  The
%registry contains a set of hierarchically organized keys, with several
%root keys. The most important ones for installation programs are
%\path{HKEY_CURRENT_USER} and \path{HKEY_LOCAL_MACHINE}, \path{HKCU} and
%\path{HKLM} in short. The \path{HKCU} part of the registry is in the
%user's home directory (see section~\ref{sec:winhome}).  \path{HKLM} is
%normally in a subdirectory of the Windows directory.

In alcuni casi, le informazioni di sistema possono essere ottenute dalle
variabili d'ambiente, ma per altre informazioni, come la posizione dei
collegamenti, è necessario consultare il registro. Impostare
permanentemente le variabili d'ambiente richiede ugualmente l'accesso al
registro.
%In some cases, system information could be obtained from environment
%variables but for other information, for example the location of
%shortcuts, it is necessary to consult the registry.  Setting environment
%variables permanently also requires registry access.


\subsection{Permessi di Windows}
%\subsection{Windows permissions}
\label{sec:winpermissions}

Nelle ultime versioni di Windows si fa distinzione tra utenti normali ed
amministratori, dove solo questi ultimi hanno libero accesso all'intero
sistema operativo. In pratica, però, si potrebbero meglio descrivere
queste classi come ``utenti senza privilegi'' e ``utenti normali'': essere
un amministratore è la regola, non un'eccezione. Ciononostante, ci siamo
sforzati di rendere \TL{} installabile senza i privilegi amministrativi.
%In later versions of Windows, a distinction is made between regular
%users and administrators, where only the latter have free access to the
%entire operating system. In practice, though, you could better describe
%these classes of users as unprivileged users and normal users: being an
%administrator is the rule, not the exception. Nevertheless, we have made
%an effort to make \TL{} installable without administrative privileges.

Se l'utente è un amministratore, c'è l'opzione di installare per tutti gli
utenti. Se questa opzione è selezionata, i collegamenti sono creati per
tutti gli utenti e l'ambiente del sistema è modificato. Altrimenti, i
collegamenti e le voci nel menu sono creati solo per l'utente corrente e
l'ambiente dell'utente è modificato.
%If the user is an administrator, there is an option to install for all
%users.  If this option is chosen, shortcuts are created for all users,
%and the system environment is modified. Otherwise, shortcuts and menu
%entries are created for the current user, and the user environment is
%modified.

Indipendentemente dallo stato di amministratore, la cartella predefinita
di \TL{} proposta dall'installatore si trova sempre sotto
\verb|%SystemDrive%|. L'installatore verifica sempre se l'utente corrente
ha i permessi di scrittura in tale cartella.
%Regardless of administrator status, the default root of \TL{} proposed
%by the installer is always under \verb|%SystemDrive%|. The installer
%always tests whether the root is writable for the current user.

Possono sorgere problemi se l'utente non è un amministratore e \TeX{} già
esiste nel percorso di ricerca. Dato che il percorso di ricerca consulta
prima il percorso dell'intero sistema e dopo quello stabilito dall'utente,
la nuova \TL{} non avrebbe mai la precedenza. Come precauzione,
l'installatore crea un collegamento al prompt dei comandi in cui la nuova
directory degli eseguibili è posta prima del percorso di ricerca. La nuova
\TL{} sarà sempre accessibile usando questo prompt. Il collegamento per
TeX{}work, se viene installato, già antepone \TL{} al percorso di ricerca,
così da essere immune a questo problema.
%A problem may arise if the user is not an administrator and \TeX{}
%already exists in the search path.  Since the effective path
%consists of the system path followed by the user path, the new \TL{}
%would never get precedence.  As a safeguard, the installer creates a
%shortcut to the command-prompt in which the new \TL{} binary
%directory is prepended to the local search path.  The new \TL{} will
%be always usable from within such a command-prompt. The shortcut for
%\TeX{}works, if installed, also prepends \TL{} to the search path, so it
%should also be immune to this path problem.

Sotto Vista c'è un ulteriore colpo di scena: anche se hai eseguito
l'accesso come amministratore, devi chiedere esplicitamente i privilegi di
amministrazione. In pratica, non serve a molto entrare come
amministratore. Invece, facendo click con il tasto destro sul programma o
sul collegamento che vuoi eseguire solitamente di offre la scelta ``Esegui
come amministratore''.
%For Vista there is another twist: even if you are logged in as
%administrator, you need to explicitly ask for administrator
%privileges. In fact, there is not much point in logging in as
%administrator. Instead, right-clicking on the program or shortcut
%that you want to run usually gives you a choice `Run as
%administrator'.

\section{Una guida a Web2C}
%\section{A user's guide to Web2C}

\Webc{} è una collezione di programmi legati a \TeX: \TeX{} stesso, \MF{},
\MP, \BibTeX{}, ecc. È il cuore di \TL{}. Il sito web di \Webc{}, con il
manuale corrente e molto altro, è \url{http://tug.org/web2c}.
%\Webc{} is an integrated collection of \TeX-related programs: \TeX{}
%itself, \MF{}, \MP, \BibTeX{}, etc.  It is the heart of \TL{}.  The home
%page for \Webc{}, with the current manual and more, is
%\url{http://tug.org/web2c}.

Un po' di storia: l'implementazione originale fu realizzata da Tomas
Rokicki il quale, nel 1987, sviluppò un primo sistema \TeX{}-to-C
cambiando dei file sotto Unix, che erano principalmente un lavoro
originale di Howard Trickey e Pavel Curtis. Tim Morgan divenne il
manutentore del sstema e durante questo periodo il nome cambiò in
Web-to-C\@. Nel 1990, Karl Berry prese in mano il lavoro, con il
contributo di molte persone, e nel 1997 passò il testimone a Olaf Weber,
il quale lo restituì a Karl nel 2006.
%A bit of history: The original implementation was by Tomas Rokicki who,
%in 1987, developed a first \TeX{}-to-C system based on change files
%under Unix, which were primarily the original work of Howard Trickey and
%Pavel Curtis.  Tim Morgan became the maintainer of the system, and
%during this period the name changed to Web-to-C\@.  In 1990, Karl Berry
%took over the work, assisted by dozens of additional contributors, and
%in 1997 he handed the baton to Olaf Weber, who returned it to Karl in
%2006.

Il sistema \Webc{} funziona sotto Unix, sotto i sistemi Windows a 32 bit,
sotto MacOSX{} e sotto altri sistemi operativi. Usa i sorgenti di \TeX{}
originali di Knuth ed altri programmi scritti nel sistema di
programmazione letterata \web{} e li traduce in codice sorgente C. I
programmi basilari di \TeX{} gestiti in questo modo sono:
%The \Webc{} system runs on Unix, 32-bit Windows systems, \MacOSX{}, and
%other operating systems. It uses Knuth's original sources for \TeX{} and
%other basic programs written in the \web{} literate programming system
%and translates them into C source code.  The core \TeX{} programs
%handled in this way are:

\begin{cmddescription}
\item[bibtex]    Per mantenere le bibliografie.
%\item[bibtex]    Maintaining bibliographies.
\item[dvicopy]   Espande i riferimenti ai font virtuali nei file \dvi{}.
%\item[dvicopy]   Expands virtual font references in \dvi{} files.
\item[dvitomp]   Converte da \dvi{} a MPX (MetaPost picture).
%\item[dvitomp]   \dvi{} to MPX (MetaPost pictures).
\item[dvitype]   Converte i \dvi{} in testo leggibile
%\item[dvitype]   \dvi{} to human-readable text.
\item[gftodvi]   Bozze dei font generici.
%\item[gftodvi]   Generic font proofsheets.
\item[gftopk]    Converte da font generici ad impacchettati.
%\item[gftopk]    Generic to packed fonts.
\item[gftype]    Converte i font generici in testo leggibile.
%\item[gftype]    GF to human-readable text.
\item[mf]        Crea famiglie di caratteri tipografici.
%\item[mf]        Creating typeface families.
\item[mft]       Abbellisce la forma dei sorgenti \MF{}.
%\item[mft]       Prettyprinting \MF{} source.
\item[mpost]     Crea diagrammi tecnici.
%\item[mpost]     Creating technical diagrams.
\item[patgen]    Crea modelli di sillabazione.
%\item[patgen]    Creating hyphenation patterns.
\item[pktogf]    Converte da font impacchettati a generici.
%\item[pktogf]    Packed to generic fonts.
\item[pktype]    Converte i font impacchettati in testo leggibile.
%\item[pktype]    PK to human-readable text.
\item[pltotf]    Converte le liste di proprietà da testo leggibile in TFM.
%\item[pltotf]    Plain text property list to TFM.
\item[pooltype]  Mostra le riserve dei file di \web{}.
%\item[pooltype]  Display \web{} pool files.
\item[tangle]    Converte da \web{} al Pascal.
%\item[tangle]    \web{} to Pascal.
\item[tex]       Compone tipograficamente i documenti.
%\item[tex]       Typesetting.
\item[tftopl]    Converte le liste di proprietà da TFM in testo leggibile
%\item[tftopl]    TFM to plain text property list.
\item[vftovp]    Converte da font virtuali a liste di proprietà virtuali.
%\item[vftovp]    Virtual font to virtual property list.
\item[vptovf]    Converte da liste di proprietà virtuali a font virtuali.
%\item[vptovf]    Virtual property list to virtual font.
\item[weave]     Converte da \web{} a \TeX.
%\item[weave]     \web{} to \TeX.
\end{cmddescription}

\noindent Le funzioni e la sintassi esatte di questi programmi sono
descritte nella documentazione dei singoli pacchetti e di \Webc{} stesso.
Tuttavia, conoscere alcuni principi che governano l'intera famiglia di
programmi ti aiuterà a trarre vantaggio dalla tua installazione \Webc{}.
%\noindent The precise functions and syntax of these programs are
%described in the documentation of the individual packages and of \Webc{}
%itself.  However, knowing a few principles governing the whole family of
%programs will help you take advantage of your \Webc{} installation.

Tutti i programmi rispettano queste opzioni \GNU{} standard:
%All programs honor these standard \GNU options:
\begin{ttdescription}
\item[-{}-help] stampa un semplice sommario d'uso.
%\item[-{}-help] print basic usage summary.
\item[-{}-verbose] stampa un rapporto dettagliato sull'avanzamento.
%\item[-{}-verbose] print detailed progress report.
\item[-{}-version] stampa le informazioni sulla versione, quindi esce.
%\item[-{}-version] print version information, then exit.
\end{ttdescription}

Per cercare i file, i programmi \Webc{} usano la libreria di ricerca dei
percorsi \KPS{} (\url{http://tug.org/kpathsea}). Questa libreria usa una
combinazione di variabili d'ambiente e file di configurazione per
ottimizzare la ricerca nella (enorme) raccolta di file di \TeX{}. \Webc{}
può guardare simultaneamente all'interno di molti percorsi e questa
caratteristica è utile per poter mantenere la distribuzione standard
\TeX{}, le estensioni locali e quelle personali in directory distinte. Per
velocizzare le ricerche dei file, all'inizio di ogni gerarchia c'è un file
\file{ls-R} che contiene una voce che mostra il nome ed il percorso
relativo per tutti i file all'interno di quella gerarchia.
%For locating files the \Webc{} programs use the path searching library
%\KPS{} (\url{http://tug.org/kpathsea}).  This library uses a combination
%of environment variables and a configuration files to optimize searching
%the (huge) collection of \TeX{} files.  \Webc{} can look at many
%directory trees simultaneously, which is useful in maintaining \TeX's
%standard distribution and local and personal extensions in distinct
%trees.  To speed up file searches, the root of each tree has a file
%\file{ls-R}, containing an entry showing the name and relative pathname
%for all files under that root.


\subsection{Ricerca dei percorsi con Kpathsea}
%\subsection{Kpathsea path searching}
\label{sec:kpathsea}

Per prima cosa descriviamo il generico meccanismo di ricerca dei percorsi
usati dalla libreria \KPS{}.
%Let us first describe the generic path searching mechanism of the \KPS{}
%library.

Chiameremo \emph{percorso di ricerca} una lista separata da due punti o
punto e virgola di \emph{elementi di percorso}, che sono fondamentalmente
nomi di directory. Un percorso di ricerca può provendire da una
combinazione di molte fonti. Per cercare un file \samp{my-file} lungo il
percorso \samp{.:/dir}, \KPS{} controlla a turno ogni elemento del
percorso: prima \file{./my-file}, poi \file{/dir/my-file}, restituendo la
prima corrispondenza (o, eventualmente, tutte quelle trovate).
%We call a \emph{search path} a colon- or semicolon\hyph sepa\-rated list
%of \emph{path elements}, which are basically directory names.  A
%search path can come from (a combination of) many sources.  To look up
%a file \samp{my-file} along a path \samp{.:/dir}, \KPS{} checks each
%element of the path in turn: first \file{./my-file}, then
%\file{/dir/my-file}, returning the first match (or possibly all
%matches).

Al fine di adattarsi al meglio alle convenzioni di tutti i sistemi
operativi, sui sistemi non Unix \KPS{} può usare separatori di nomi di
file diversi dai due punti (\samp{:}) e barra (slash, \samp{/}).
%In order to adapt optimally to all operating systems' conventions, on
%non-Unix systems \KPS{} can use filename separators different from
%colon (\samp{:}) and slash (\samp{/}).

Per controllare un particolare elemento \var{p} di un percorso, \KPS{} per
prima cosa controlla se un archivio precostruito si applica a \var{p}
(consulta ``Archivi di file'' a pagina~\pageref{sec:filename-database}),
cioè se l'archivio è in una directory che è prefisso di \var{p}. Se è
così, la specifica del percorso è confrontata con il contenuto
dell'archivio.
%To check a particular path element \var{p}, \KPS{} first checks if a
%prebuilt database (see ``Filename data\-base'' on
%page~\pageref{sec:filename-database}) applies to \var{p}, i.e., if the
%database is in a directory that is a prefix of \var{p}.  If so, the path
%specification is matched against the contents of the database.

Se l'archivio non esiste o non si applica a questo elemento del percorso,
oppure se non contiene corrispondenze, viene eseguita una ricerca sul
disco (se non è stata proibita da una specifica che comincia con \samp{!!}\
e il file cercato deve esistere). \KPS{} costruisce una lista di directory
che corrispondono a questo elemento del percorso e verifica all'interno di
ciascuna se può individuare il file.
%If the database does not exist, or does not apply to this path
%element, or contains no matches, the filesystem is searched (if this
%was not forbidden by a specification starting with \samp{!!}\ and if
%the file being searched for must exist).  \KPS{} constructs the list
%of directories that correspond to this path element, and then checks
%in each for the file being sought.

La condizione ``il file deve esistere'' entra in gioco con i file
\samp{.vf} e i file letti tramite il comando \cs{openin} di \TeX. Questi
file potrebbero non esistere (ad esempio \file{cmr10.vf}) e dunque sarebbe
sbagliato cercarli sul disco. Quindi, se non sei riuscito ad aggiornare
correttamente il file \file{ls-R} quando hai installato un nuovo file
\samp{.vf}, ques'ultimo non sarà mai trovato. A turno, ogni elemento del
percorso è controllato: prima l'archivio, poiil disco. Se viene trovata
una corrispondenza, la ricerca si interrompe e il risultato è restituito. 
%The ``file must exist'' condition comes into play with \samp{.vf} files and
%input files read by \TeX's \cs{openin} command.  Such files may not
%exist (e.g., \file{cmr10.vf}), and so it would be wrong to search the
%disk for them.  Therefore, if you fail to update \file{ls-R} when you
%install a new \samp{.vf} file, it will never be found.
%Each path element is checked in turn: first the database, then the
%disk.  If a match is found, the search stops and the result is
%returned.

Per quanto l'elemento di un percorso più semplice e comune sia il nome di
una directory, \KPS{} supporta caratteristiche aggiuntive per la ricerca
di percorsi: valori predefiniti stratificati, nomi di variabili
d'ambiente, valori di file di configurazione, directory personali degli
utenti e ricerca ricorsiva di sotto directory. Dunque, diciamo che \KPS{}
\emph{espande} un elemento di percorso, intendendo che trasforma tutte le
specifiche in semplici nomi di directory. Questa trasformazione è
descritta nelle seguenti sezioni nello stesso ordine in cui avviene.
%Although the simplest and most common path element is a directory
%name, \KPS{} supports additional features in search paths: layered
%default values, environment variable names, config file values, users'
%home directories, and recursive subdirectory searching.  Thus, we say
%that \KPS{} \emph{expands} a path element, meaning it transforms all
%the specifications into basic directory name or names.  This is
%described in the following sections in the same order as it takes
%place.

Nota che se il nome del file cercato è assoluto o esplicitamente relativo,
ossia se comincia con \samp{/} o \samp{./} o \samp{../}, \KPS{} controlla
semplicemente se quel file esiste.
%Note that if the filename being searched for is absolute or explicitly
%relative, i.e., starts with \samp{/} or \samp{./} or \samp{../},
%\KPS{} simply checks if that file exists.

\ifSingleColumn
\else
\begin{figure*}
\verbatiminput{examples/ex5.tex}
\setlength{\abovecaptionskip}{0pt}
  \caption{Un esempio illustrativo di file di configurazione}
  %\caption{An illustrative configuration file sample}
  \label{fig:config-sample}
\end{figure*}
\fi

\subsubsection{Fonti dei percorsi}
%\subsubsection{Path sources}
\label{sec:path-sources}

Un percorso di ricerca può provenire da molte fonti. Nell'ordine in cui
\KPS{} le usa:
%A search path can come from many sources.  In the order in which
%\KPS{} uses them:

\begin{enumerate}
\item
  Una variabile d'ambiente impostata dall'utente, per esempio
  \envname{TEXINPUTS}\@. Le variabili d'ambiente a cui è stato aggiunto un
  punto ed il nome di un programma hanno la precedenza; per esempio, se
  \samp{latex} è il nome del programma in esecuzione, allora
  \envname{TEXINPUTS.latex} avrà la precedenza su \envname{TEXINPUTS}.
%\item
%  A user-set environment variable, for instance, \envname{TEXINPUTS}\@.
%  Environment variables with a period and a program name appended
%  override; e.g., if \samp{latex} is the name of the program being run,
%  then \envname{TEXINPUTS.latex} will override \envname{TEXINPUTS}.
\item
  Il file di configurazione specifico per un programma, per esempio, una
  linea \samp{S /a:/b} nel file di configurazione \file{config.ps} di
  \cmdname{dvips}.
%\item
%  A program-specific configuration file, for exam\-ple, a line
%  \samp{S /a:/b} in \cmdname{dvips}'s \file{config.ps}.
\item
  Un file di configurazione \file{texmf.cnf} di \KPS, contenente una linea
  del tipo \samp{TEXINPUTS=/c:/d} (vedi più avanti).
%\item   A \KPS{} configuration file \file{texmf.cnf}, containing a line like
%  \samp{TEXINPUTS=/c:/d} (see below).
\item Il valore predefinito alla compilazione del programma.
%\item The compile-time default.
\end{enumerate}
\noindent Puoi vedere ciascuno di questi valori per un dato percorso di
ricerca usando le opzioni di debug (vedi ``Risoluzione dei problemi'' a
pagina~\pageref{sec:debugging}).
%\noindent You can see each of these values for a given search path by
%using the debugging options (see ``Debugging actions'' on
%page~\pageref{sec:debugging}).

\subsubsection{File di configurazione}
%\subsubsection{Config files}

\begingroup\tolerance=3500
\KPS{} legge i \emph{file di configurazione a tempo di esecuzione}
chiamati \file{texmf.cnf} per i percorsi di ricerca ed altre definizioni.
Il percorso di ricerca usato per cercare questi file è chiamato
\envname{TEXMFCNF} (di base questo file si trova nella sotto directory
\file{texmf/web2c}). \emph{Tutti} i file \file{texmf.cnf} nel percorso di
ricerca saranno letti e le definizioni presenti nei file trovati per primi
avranno la precedenza nei confronti di quelle presenti nei file trovati
per ultimi. Quindi, con un percorso di ricerca \verb|.:$TEXMF|, i valori
provenienti da \file{./texmf.cnf} hanno la precedenza su quelli
provenienti da \verb|$TEXMF/texmf.cnf|.
\endgroup
%\begingroup\tolerance=3500
%\KPS{} reads \emph{runtime configuration files} named \file{texmf.cnf}
%for search path and other definitions.  The search path used to look
%for these files is named \envname{TEXMFCNF} (by default such a file lives
%in the \file{texmf/web2c} subdirectory).  \emph{All}
%\file{texmf.cnf} files in the search path will be read and definitions
%in earlier files override those in later files.  Thus, with a search
%path of \verb|.:$TEXMF|, values from \file{./texmf.cnf} override those
%from \verb|$TEXMF/texmf.cnf|.
%\endgroup

\begin{itemize}
\item
  I commenti iniziano con \code{\%} e continuano fino alla fine della
  linea.
%\item
%  Comments start with \code{\%} and continue to the end of the line.
\item
  Gli spazi vuoti sono ignorati.
%\item
%  Blank lines are ignored.
\item
  Un \bs{} alla fine di una linea funziona come un carattere di
  continuazione, cioè la linea successiva è come se fosse aggiunta alla
  fine. Gli spazi bianchi all'inizio delle linee di prosecuzione non sono
  ignorati.
%\item
%  A \bs{} at the end of a line acts as a continuation character,
%  i.e., the next line is appended.  Whitespace at the beginning of
%  continuation lines is not ignored.
\item
  Ogni linea rimanente ha la forma:
\begin{alltt}
  \var{variabile}[.\var{nomeprogramma}] [=] \var{valore}
\end{alltt}
  dove il carattere \samp{=} e gli spazi che lo circondano sono opzionali.
%\item
%  Each remaining line has the form:
%\begin{alltt}
%  \var{variable}[.\var{progname}] [=] \var{value}
%\end{alltt}%
%  where the \samp{=} and surrounding whitespace are optional.
\item
  Il nome \ttvar{variabile} può contenere un qualunque carattere diverso
  dallo spazio, da \samp{=} e da \samp{.}, ma usare solo i caratteri
  \samp{A-Za-z\_} è più sicuro.
%\item
%  The \ttvar{variable} name may contain any character other
%  than whitespace, \samp{=}, or \samp{.}, but sticking to
%  \samp{A-Za-z\_} is safest.
\item
  Se la parte \samp{.\var{nomeprogramma}} è presente, la definizione si
  applica solo se il programma che è in esecuzione ha il nome
  \texttt{\var{nomeprogramma}} oppure \texttt{\var{nomeprogramma}.exe}. In
  questo modo, ad esempio, le diverse varianti di \TeX{} possono avere
  diversi percorsi di ricerca.
%\item
%  If \samp{.\var{progname}} is present, the definition only
%  applies if the program that is running is named
%  \texttt{\var{progname}} or \texttt{\var{progname}.exe}.  This allows
%  different flavors of \TeX{} to have different search paths, for
%  example.
\item \var{valore} può contentere qualunque carattere tranne che \code{\%}
  e \samp{@}. La caratteristica \code{\$\var{variabile}.\var{programma}}
  non è disponibile come valore valido; in alternativa, devi usare una
  variabile aggiuntiva. Un \samp{;}\ in \var{valore} è tradotto
  automaticamente in \samp{:}\ se stai usando Unix; questa caratteristica
  è utile per avere un singlo \file{texmf.cnf} per Unix, \acro{MS-DOS} e
  Windows.
%\item \var{value} may contain any characters except
%  \code{\%} and \samp{@}.  The
%  \code{\$\var{var}.\var{prog}} feature is not available on the
%  right-hand side; instead, you must use an additional variable.  A
%  \samp{;}\ in \var{value} is translated to \samp{:}\ if
%  running under Unix; this is useful to be able to have a single
%  \file{texmf.cnf} for Unix, \acro{MS-DOS} and Windows systems.
\item
  Tutte le definizioni vengono lette prima di eseguire qualunque altra
  operazione, quindi si può fare riferimento alle variabili prima che
  siano definite.
%\item
%  All definitions are read before anything is expanded, so
%  variables can be referenced before they are defined.
\end{itemize}
Un frammento di file di configurazione che mostra la maggior parte di
questi punti è
%A configuration file fragment illustrating most of these points is
\ifSingleColumn
mostrato sotto:
%shown below:

\verbatiminput{examples/ex5.tex}
\else
mostrato nella figura~\ref{fig:config-sample}.
%shown in Figure~\ref{fig:config-sample}.
\fi

\subsubsection{Espansione dei percorsi}
%\subsubsection{Path expansion}
\label{sec:path-expansion}

\KPS{} riconosce nei percorsi di ricerca alcuni caratteri speciali e costrutti
simili a quelli disponibili nelle shell di Unix. Come esempio generale, il
percorso complesso \verb+~$USER/{foo,bar}//baz+ viene espanso in tutte le
sotto directory, contenute dentro \file{foo} e \file{bar} nella home
dell'utente \texttt{\$USER}, che contengono una directory o un file di
nome \file{baz}. Queste espansioni sono spiegate nelle prossime sezioni.
%\KPS{} recognizes certain special characters and constructions in
%search paths, similar to those available in Unix shells.  As a
%general example, the  complex path,
%\verb+~$USER/{foo,bar}//baz+, expands to all subdirectories under
%directories \file{foo} and \file{bar} in \texttt{\$USER}'s home
%directory that contain a directory or file \file{baz}.  These
%expansions are explained in the sections below.
%%$
\subsubsection{Espansione predefinita}
%\subsubsection{Default expansion}
\label{sec:default-expansion}

\tolerance=2500
Se il percorso di ricerca a più alta priorità (consulta ``Origini dei
percorsi'' a pagina~\pageref{sec:path-sources}) contiene un \emph{due
punti di troppo} (cioè all'inizio, alla fine oppure una coppia due punti),
\KPS{} inserisce in quel punto il percorso di ricerca con la seconda più
alta priorità che sia stato definito. Se questo percorso inserito ha a sua
vuolta un due punti aggiuntivo, accade lo stesso con il successivo in
ordine di priorità. Per esempio, data l'impostazione di una variabile
d'ambiente
%If the highest-priority search path (see ``Path sources'' on
%page~\pageref{sec:path-sources}) contains an \emph{extra colon} (i.e.,
%leading, trailing, or doubled), \KPS{} inserts at that point the
%next-highest-prio\-rity search path that is defined.  If that inserted
%path has an extra colon, the same happens with the next highest.  For
%example, given an environment variable setting

\tolerance=1500

\begin{alltt}
> \Ucom{setenv TEXINPUTS /home/karl:}
\end{alltt}
e un valore di \code{TEXINPUTS} proveniente da \file{texmf.cnf} di
%and a \code{TEXINPUTS} value from \file{texmf.cnf} of

\begin{alltt}
  .:\$TEXMF//tex
\end{alltt}
allora il valore finale usato nella ricerca sarà:
%then the final value used for searching will be:

\begin{alltt}
  /home/karl:.:\$TEXMF//tex
\end{alltt}

Dato che sarebbe inutile inserire lo stesso valore predefinito in più di
un posto, \KPS{} cambia solo uno dei \samp{:}\ di troppo e lascia gli
altri al loro posto. Per prima cosa cerca un \samp{:} all'inizio, quindi
cerca un \samp{:} alla fine, per ultimo cerca un \samp{:} doppio.
%Since it would be useless to insert the default value in more than one
%place, \KPS{} changes only one extra \samp{:}\ and leaves any others in
%place.  It checks first for a leading \samp{:}, then a trailing
%\samp{:}, then a doubled \samp{:}.

\subsubsection{Espansione delle parentesi graffe}
%\subsubsection{Brace expansion}

Una caratteristica utile è l'espansione delle parentesi graffe, che
significa che, per esempio, \verb+v{a,b}w+ viene espanso in
\verb+vaw:vbw+. L'annidamento delle parentesi è permesso. Tale
caratteristica è usata per implementare gerarchie \TeX{} multiple,
assegnando una lista tra graffe a \code{\$TEXMF}. Ad esempio, in
\file{texmf.cnf}, è fatta la seguente definizione (approssimativamente; in
pratica ci sono molte più gerarchie):
%A useful feature is brace expansion, which means that, for instance,
%\verb+v{a,b}w+ expands to \verb+vaw:vbw+. Nesting is allowed.
%This is used to implement multiple \TeX{} hierarchies, by
%assigning a brace list to \code{\$TEXMF}.
%For example, in \file{texmf.cnf}, the following definition
%(approximately; there are actually even more trees) is made:
\begin{verbatim}
  TEXMF = {$TEXMFHOME,$TEXMFLOCAL,!!$TEXMFVAR,!!$TEXMFMAIN}
\end{verbatim}
Usando questa variabile, puoi scrivere qualcosa come
%Using this you can then write something like
\begin{verbatim}
  TEXINPUTS = .;$TEXMF/tex//
\end{verbatim}
%$
che vuol dire che, dopo aver guardato nella directory attuale, la ricerca
proseguirà (\emph{soltanto}) nelle gerarchie \code{\$TEXMFHOME/tex},
\code{\$TEXMFLOCAL/tex}, \code{\$TEXMFVAR/tex} e \code{\$TEXMFMAIN/tex}
(le ultime due usando gli archivi \file{ls-R}). È una maniera conveniente
per adoperare due strutture \TeX{} parallele, una ``congelata'' (ad
esempio su un \CD) e l'altra aggiornata di continuo con nuove versioni non
appena diventino disponibili. Usando la variabile \code{\$TEXMF} in tutte
le definizion, si è sicuri di cercare innanzitutto nella gerarchia
aggiornata.
%which means that, after looking in the current directory, the
%\code{\$TEXMFHOME/tex}, \code{\$TEXMFLOCAL/tex}, \code{\$TEXMFVAR/tex}
%and \code{\$TEXMFMAIN/tex} trees \emph{only}) will be searched (the
%last two use using \file{ls-R} data base files). It is a convenient
%way for running two parallel \TeX{} structures, one ``frozen'' (on a
%\CD, for instance) and the other being continuously updated with new
%versions as they become available.  By using the \code{\$TEXMF}
%variable in all definitions, one is sure to always search the
%up-to-date tree first.

\subsubsection{Subdirectory expansion}
%\subsubsection{Espansione delle sotto directory}
\label{sec:subdirectory-expansion}

Due o più barre (slash) consecutivi in un elemento di percorso alla fine
di una directory \var{d} sono sostituite da tutte le sotto directory di
\var{d}: prima quelle che si trovano direttamente sotto \var{d}, quindi
quelle all'interno delle prime e così via. Ad ogni livello, l'ordine in
cui le directory sono cercate \emph{non è specificato}.
%Two or more consecutive slashes in a path element following a directory
%\var{d} is replaced by all subdirectories of \var{d}: first those
%subdirectories directly under \var{d}, then the subsubdirectories under
%those, and so on.  At each level, the order in which the directories are
%searched is \emph{unspecified}.

Se specifichi il nome di un file dopo il \samp{//}, solo le sotto
directory che lo conengono saranno incluse. Ad esempio, \samp{/a//b} si
espande nelle directory \file{/a/1/b}, \file{/a/2/b}, \file{/a/1/1/b} e
così via, ma non in \file{/a/b/c} o \file{/a/1}.
%If you specify any filename components after the \samp{//}, only
%subdirectories with matching components are included.  For example,
%\samp{/a//b} expands into directories \file{/a/1/b}, \file{/a/2/b},
%\file{/a/1/1/b}, and so on, but not \file{/a/b/c} or \file{/a/1}.

Costrutti \samp{//} multipli in un percorso sono possibili, ma ogni
\samp{//} posto all'inizio viene ignorato.
%Multiple \samp{//} constructs in a path are possible, but
%\samp{//} at the beginning of a path is ignored.

\subsubsection{Elenco dei caratteri speciali e loro significato: un
  riepilogo}
%\subsubsection{List of special characters and their meaning: a summary}

Il seguente elenco riassume i caratteri speciali nei file di
configurazione di \KPS{}.
%The following list summarizes the special characters in \KPS{}
%configuration files.

% need a wider space for the item labels here.
\newcommand{\CODE}[1]{\makebox[3em][l]{\code{#1}}}
\begin{ttdescription}
\item[\CODE{:}] Separatore nella specifica di un percorso; all'inizio o
    alla fine viene sostituito con l'espansione predefinita.\par
%\item[\CODE{:}] Separator in path specification; at the beginning or
%  the end of a path it substitutes the default path expansion.\par
\item[\CODE{;}] Separatore nei sistemi non Unix (si comporta come
    \code{:}).
%\item[\CODE{;}] Separator on non-Unix systems (acts like \code{:}).
\item[\CODE{\$}] Espansione di una variabile.
%\item[\CODE{\$}] Variable expansion.
\item[\CODE{\string~}] Rappresenta la directory di home di un utente.
%\item[\CODE{\string~}] Represents the user's home directory.
\item[\CODE{\char`\{...\char`\}}] Espansione di parentesi graffe.
%\item[\CODE{\char`\{...\char`\}}] Brace expansion.
\item[\CODE{//}] Espansione di sotto directory (può trovarsi ovunque in un
    percorso tranne che al suo inizio).
%\item[\CODE{//}] Subdirectory expansion (can occur anywhere in
%    a path, except at its beginning).
\item[\CODE{\%}] Inizio di un commento.
%\item[\CODE{\%}] Start of comment.
\item[\CODE{\bs}] Carattere di continuazione (consente alle voci di
    espandersi su più linee).
%\item[\CODE{\bs}] Continuation character (allows multi-line entries).
\item[\CODE{!!}] Cerca \emph{solo} negli archivi per individuare un file,
    \emph{non} cerca sul disco.
%\item[\CODE{!!}] Search \emph{only} database to locate file, \emph{do
%    not} search the disk.
\end{ttdescription}


\subsection{Archivi di nomi di file}
%\subsection{Filename databases}
\label{sec:filename-database}

% FIXME
\KPS{} si sforza di minimizzare gli accessi al disco per le ricerche. Ad
ogni modo, in installazioni con abbastanza directory, cercare un dato file
in ognuna di esse può richiedere una quantità di tempo eccessiva (vero in
special modo se devono essere attraversate molte centinaia di directory di
font). Quindi, \KPS{} può usare un file di ``archivio'' testuale costruito
esternamente, chiamato \file{ls-R}, che associa file a directory, evitando
in questo modo di cercare esaustivamente sul disco.
%\KPS{} goes to some lengths to minimize disk accesses for searches.
%Nevertheless, at installations with enough directories, searching each
%possible directory for a given file can take an excessively long time
%(this is especially true if many hundreds of font directories have to
%be traversed.)  Therefore, \KPS{} can use an externally-built plain text
%``database'' file named \file{ls-R} that maps files to directories,
%thus avoiding the need to exhaustively search the disk.

Un secondo file di archivio \file{aliases} ti permette di dare nomi
aggiuntivi ai file elencati in \file{ls-R}. Questa caratteristica può
essere utile per conformare i file sorgenti alla convenzione \acro{DOS}
di nomi 8.3 (8 caratteri di nome, più 3 per l'estensione).
%A second database file \file{aliases} allows you to give additional
%names to the files listed in \file{ls-R}.  This can be helpful to
%confirm to \acro{DOS} 8.3 filename conventions in source files.

\subsubsection{L'archivio di nomi dei file}
\subsubsection{The filename database}
\label{sec:ls-R}

Come spiegato in precedenza, il nome dell'archivio di nomi deil file
principale deve essere \file{ls-R}. Puoi metterne uno alla radice di ogni
gerarchia \TeX{} nella tua installazione che vuoi che sia oggetto di
ricerca (\code{\$TEXMF} di base). \KPS{} cerca i file \file{ls-R} nel
percorso \code{TEXMFDBS}.
%As explained above, the name of the main filename database must be
%\file{ls-R}.  You can put one at the root of each \TeX{} hierarchy in
%your installation that you wish to be searched (\code{\$TEXMF} by
%default).  \KPS{} looks for
%\file{ls-R} files along the \code{TEXMFDBS} path.

Il modo raccomandato per creare e mantenere \samp{ls-R} è quello di
eseguire lo script \code{mktexlsr} incluso nella distribuzione. Esso è
invocato da vari script \samp{mktex}\dots. In teoria, questo script esegue
semplicemente il comando
%The recommended way to create and maintain \samp{ls-R} is to run the
%\code{mktexlsr} script included with the distribution. It is invoked
%by the various \samp{mktex}\dots\ scripts.  In principle, this script
%just runs the command
\begin{alltt}
cd \var{/your/texmf/root} && \path|\|ls -1LAR ./ >ls-R
\end{alltt}
assumendo che il comando \code{ls} di sistema produca il giusto formato di
output (\code{ls} del progetto \GNU lo fa). Per garantire che
l'archivio sia sempre aggiornato, la cosa più semplice è rigenerarlo
regolarmente tramite \code{cron}, così che sia automaticamente ricostruito
quando i file installati cambiano, ad esempio dopo aver installato o
aggiornato un pacchetto \LaTeX.
%presuming your system's \code{ls} produces the right output format
%(\GNU \code{ls} is all right).  To ensure that the database is
%always up-to-date, it is easiest to rebuild it regularly via
%\code{cron}, so that it is automatically updated when the installed
%files change, such as after installing or updating a \LaTeX{} package.

Se un file non è trovato nell'archivio, di ripiego \KPS{} va avanti e
cerca sul disco. Se un particolare elemento di percorso comincia con
\samp{!!}, però, \emph{solo} l'archivo sarà cercato per quell'elemento,
mai il disco.
%If a file is not found in the database, by default \KPS{} goes ahead
%and searches the disk. If a particular path element begins with
%\samp{!!}, however, \emph{only} the database will be searched for that
%element, never the disk.


\subsubsection{kpsewhich: ricerca indipendente di percorsi}
%\subsubsection{kpsewhich: Standalone path searching}
\label{sec:invoking-kpsewhich}

Il programma \texttt{kpsewhich} compie la ricerca di percorsi
indipendentemente da una particolare applicazione. Può essere utile come
una sorta di programma \code{find} per individuare file nelle gerarchie
\TeX{} (è usato pesantemente negli script distribuiti \samp{mktex}\dots).
%The \texttt{kpsewhich} program exercises path searching independent of any
%particular application.  This can be useful as a sort of \code{find}
%program to locate files in \TeX{} hierarchies (this is used heavily in
%the distributed \samp{mktex}\dots\ scripts).

\begin{alltt}
> \Ucom{kpsewhich \var{opzioni}\dots{} \var{nomefile}\dots{}}
%> \Ucom{kpsewhich \var{option}\dots{} \var{filename}\dots{}}
\end{alltt}
Le opzioni specificate in \ttvar{option} cominciano con \samp{-} oppure
con \samp{-{}-} ed è accettata qualunque abbreviazione non ambigua.
%The options specified in \ttvar{option} start with either \samp{-}
%or \samp{-{}-}, and any unambiguous abbreviation is accepted.

\KPS{} ricerca ogni argomento su riga di comando che non sia un'opzione
come se fosse il nome di un file e ritorna il primo trovato. Non ci sono
opzioni per restituire tutti i file con un particolare nome (per fare ciò,
puoi eseguire il programma Unix \samp{find}).
%\KPS{} looks up each non-option argument on the command line as a
%filename, and returns the first file found. There is no option to
%return all the files with a particular name (you can run the Unix
%\samp{find} utility for that).

Le opzioni più comuni sono descritte in seguito.
%The most common options are described next.

\begin{ttdescription}
\item[\texttt{-{}-dpi=\var{num}}]\mbox{}
  Imposta la risoluzione a \ttvar{num}; questa opzione ha effetto solo
  nelle ricerche dei file \samp{gf} e \samp{pk}. \samp{-D} è un sinonimo,
  per compatibilità con \cmdname{dvips}. Il valore preimpostato è 600.
%\item[\texttt{-{}-dpi=\var{num}}]\mbox{}
%  Set the resolution to \ttvar{num}; this only affects \samp{gf}
%  and \samp{pk} lookups.  \samp{-D} is a synonym, for compatibility
%  with \cmdname{dvips}.  Default is 600.

\item[\texttt{-{}-format=\var{nome}}]\mbox{}\\
  Imposta il formato da cercare a \ttvar{nome}. Di base, il formato è
  ipotizzato a partire dal nome del file. Per i formati che non hanno
  associato un suffisso non ambiguo, come i file di supporto di \MP{} e i
  file di configurazione di \cmdname{dvips}, devi specificare il nome come
  è noto da \KPS{}, come \texttt{tex} o \texttt{enc files}. Esegui
  \texttt{kpsewhich -{}-help} per un elenco.
%\item[\texttt{-{}-format=\var{name}}]\mbox{}\\
%  Set the format for lookup to \ttvar{name}.  By default, the
%  format is guessed from the filename. For formats which do not have
%  an associated unambiguous suffix, such as \MP{} support files and
%  \cmdname{dvips} configuration files, you have to specify the name as
%  known to \KPS{}, such as \texttt{tex} or \texttt{enc files}.  Run
%  \texttt{kpsewhich -{}-help} for a list.

\item[\texttt{-{}-mode=\var{stringa}}]\mbox{}\\
  Imposta il nome della modalità a \ttvar{string}; questa opzione ha
  effetto solo sulle ricerche dei file \samp{gf} e \samp{pk}. Non c'è un
  valore predefinito: sarà trovata qualunque modalità.
%\item[\texttt{-{}-mode=\var{string}}]\mbox{}\\
%  Set the mode name to \ttvar{string}; this only affects \samp{gf}
%  and \samp{pk} lookups.  No default: any mode will be found.

\item[\texttt{-{}-must-exist}]\mbox{}\\
  Fa tutto ciò che è possibile per trovare il file, inclusa nello
  specifico la ricerca sul disco. Di base, solo l'archivio \file{ls-R} è
  controllato, per questioni di efficienza.
%\item[\texttt{-{}-must-exist}]\mbox{}\\
%  Do everything possible to find the files, notably including
%  searching the disk.  By default, only the \file{ls-R} database is
%  checked, in the interest of efficiency.

\item[\texttt{-{}-path=\var{stringa}}]\mbox{}\\
  Cerca nel percorso \ttvar{stringa} (separata da due punti, come al
  solito), invece di indovinarlo dal nome del file. \samp{//} e tutte le
  solite espansioni sono supportate. Le opzioni \samp{-{}-path} e
  \samp{-{}-format} si escludono a vicenda.
%\item[\texttt{-{}-path=\var{string}}]\mbox{}\\
%  Search along the path \ttvar{string} (colon-separated as usual),
%  instead of guessing the search path from the filename.  \samp{//} and
%  all the usual expansions are supported.  The options \samp{-{}-path}
%  and \samp{-{}-format} are mutually exclusive.

\item[\texttt{-{}-progname=\var{nome}}]\mbox{}\\
  Imposta il nome del programma a \texttt{\var{nome}}. Questa opzione può
  modificare i percorsi di ricerca per mezzo della funzionalità
  \texttt{.\var{nomeprogramma}}. Il valore predefinito è
  \cmdname{kpsewhich}.
%\item[\texttt{-{}-progname=\var{name}}]\mbox{}\\
%  Set the program name to \texttt{\var{name}}.
%  This can affect the search paths via the \texttt{.\var{progname}}
%  feature.
%  The default is \cmdname{kpsewhich}.

\item[\texttt{-{}-show-path=\var{nome}}]\mbox{}\\
  Mostra il percorso usato per la ricerca dei file il cui tipo sia
  \texttt{\var{nome}}. Può essere usata sia un'estensione (\code{.pk},
  \code{.vf}, ecc.) che un nome, proprio come per l'opzione
  \samp{-{}-format}.
%\item[\texttt{-{}-show-path=\var{name}}]\mbox{}\\
%  shows the path used for file lookups of file type \texttt{\var{name}}.
%  Either a filename extension (\code{.pk}, \code{.vf}, etc.) or a
%  name can be used, just as with \samp{-{}-format} option.

\item[\texttt{-{}-debug=\var{num}}]\mbox{}\\
  Imposta le opzioni per la ricerca degli errori a \texttt{\var{num}}.
%\item[\texttt{-{}-debug=\var{num}}]\mbox{}\\
%  sets the debugging options to \texttt{\var{num}}.
\end{ttdescription}


\subsubsection{Esempi d'uso}
%\subsubsection{Examples of use}
\label{sec:examples-of-use}

Diamo uno sguardo a \KPS{} in azione. Ecco una ricerca semplice:
%Let us now have a look at \KPS{} in action.  Here's a straightforward search:

\begin{alltt}
> \Ucom{kpsewhich article.cls}
   /usr/local/texmf-dist/tex/latex/base/article.cls
\end{alltt}
Stiamo cercando il file \file{article.cls}. Dato che il suffisso
\samp{.cls} non è ambiguo, non abbiamo bisogno di specificare che vogliamo
cercare un file di tipo \optname{tex} (file sorgente \TeX). Lo troviamo
nella sotto directory \file{tex/latex/base} sotto la directory di \TL\
\samp{texmf-dist}. In modo simile, tutti i file seguenti sono trovati
senza problemi grazia all'univocità dei loro suffissi.
%We are looking for the file \file{article.cls}. Since the \samp{.cls}
%suffix is unambiguous we do not need to specify that we want to look for a
%file of type \optname{tex} (\TeX{} source file directories). We find it in
%the subdirectory \file{tex/latex/base} below the \samp{texmf-dist} \TL\
%directory.  Similarly, all of the following are found without problems
%thanks to their unambiguous suffix.
\begin{alltt}
> \Ucom{kpsewhich array.sty}
   /usr/local/texmf-dist/tex/latex/tools/array.sty
> \Ucom{kpsewhich latin1.def}
   /usr/local/texmf-dist/tex/latex/base/latin1.def
> \Ucom{kpsewhich size10.clo}
   /usr/local/texmf-dist/tex/latex/base/size10.clo
> \Ucom{kpsewhich small2e.tex}
   /usr/local/texmf-dist/tex/latex/base/small2e.tex
> \Ucom{kpsewhich tugboat.bib}
   /usr/local/texmf-dist/bibtex/bib/beebe/tugboat.bib
\end{alltt}

Ad ogni modo, quest'ultimo è un registro bibliografico in formato
\BibTeX{} per gli articoli di \textsl{TUGboat}.
%By the way, that last is a \BibTeX{} bibliography database for
%\textsl{TUGboat} articles.

\begin{alltt}
> \Ucom{kpsewhich cmr10.pk}
\end{alltt}
I file di glifi dei font bitmat di tipo \file{.pk} sono usati dai
programmi di visualizzazione come \cmdname{dvips} e \cmdname{xdvi}. Non
viene restituito nulla in questo caso dato che in \TL\ non ci sono file
\samp{.pk} pregenerati per il carattere tipografico Computer Modern \Dash
come predefinite sono usate le varianti Type~1.
%Font bitmap glyph files of type \file{.pk} are used by display
%programs like \cmdname{dvips} and \cmdname{xdvi}.  Nothing is returned in
%this case since there are no pre-generated Computer Modern \samp{.pk}
%files in \TL{}\Dash the Type~1 variants are used by default.
\begin{alltt}
> \Ucom{kpsewhich wsuipa10.pk}
\ifSingleColumn   /usr/local/texmf-var/fonts/pk/ljfour/public/wsuipa/wsuipa10.600pk
\else /usr/local/texmf-var/fonts/pk/ljfour/public/
...                         wsuipa/wsuipa10.600pk
\fi\end{alltt}
Per questi font (un alfabeto fonetico creato dall'Università di
Washington) dobbiamo generare i file \samp{.pk} e dato che la modalità
predefinita di \MF{} nella nostra installazione è \texttt{ljfour} con una
risoluzione di base di 600\dpi{} (dots per inch, punti per pollice), viene
restituito questo valore.
%For these fonts (a phonetic alphabet from the University of Washington)
%we had to generate \samp{.pk} files, and since the default \MF{} mode on
%our installation is \texttt{ljfour} with a base resolution of 600\dpi{}
%(dots per inch), this instantiation is returned.
\begin{alltt}
> \Ucom{kpsewhich -dpi=300 wsuipa10.pk}
\end{alltt}
In questo caso, quando specifichiamo di essere interessati ad una
risoluzione di 300\dpi{} (\texttt{-dpi=300}) osserviamo che questo font
non è disponibile nel sistema. Un programma come \cmdname{dvips} o
\cmdname{xdvi} andrebbe avanti e genererebbe i file \texttt{.pk} richiesti
usando lo script \cmdname{mktexpk}.
%In this case, when specifying that we are interested in a resolution
%of 300\dpi{} (\texttt{-dpi=300}) we see that no such font is available on
%the system. A program like \cmdname{dvips} or \cmdname{xdvi} would
%go off and actually build the required \texttt{.pk} files
%using the script \cmdname{mktexpk}.

Adesso spostiamo la nostra attenzione sui file di intestazione e di
configurazione di \cmdname{dvips}. Cercheremo innanzitutto un file tra
quelli usati comunemente, il prologo generale \file{tex.pro} per il
supporto a \TeX, prima di spostare l'attenzione sul generico file di
configurazione (\file{config.ps}) e la mappa dei font \PS{}
\file{psfonts.map} \Dash\ a partire dal 2004, i file di mappatura e di
codifica hanno i propri percorsi di ricerca e una nuova posizione nelle
gerarchie \dirname{texmf}. Dato che il suffisso \samp{.ps} è ambiguo,
dobbiamo specificare esplicitamente quale tipo stiamo considerando
(\optname{dvips config}) per il file \texttt{config.ps}.
%Next we turn our attention to \cmdname{dvips}'s header and configuration
%files.  We first look at one of the commonly used files, the general
%prologue \file{tex.pro} for \TeX{} support, before turning our attention
%to the generic configuration file (\file{config.ps}) and the \PS{} font
%map \file{psfonts.map}\Dash as of 2004, map and encoding files have
%their own search paths and new location in \dirname{texmf} trees.  As
%the \samp{.ps} suffix is ambiguous we have to specify explicitly which
%type we are considering (\optname{dvips config}) for the file
\texttt{config.ps}.
\begin{alltt}
> \Ucom{kpsewhich tex.pro}
   /usr/local/texmf/dvips/base/tex.pro
> \Ucom{kpsewhich --format="dvips config" config.ps}
   /usr/local/texmf/dvips/config/config.ps
> \Ucom{kpsewhich psfonts.map}
   /usr/local/texmf/fonts/map/dvips/updmap/psfonts.map
\end{alltt}

Diamo ora uno sguardo ravvicinato ai file di supporto per il carattere
\PS{} \acro{URW} Times. Il prefisso per questi file nello schema dei nomi
dei font è \samp{utm}. Il primo file che cerchiamo è quello di
configurazione, che contiene il nome del file di mappatura:
%We now take a closer look at the \acro{URW} Times \PS{} support
%files.  The prefix for these in the standard font naming scheme is
%\samp{utm}.  The first file we look at is the configuration file,
%which contains the name of the map file:
\begin{alltt}
> \Ucom{kpsewhich --format="dvips config" config.utm}
   /usr/local/texmf-dist/dvips/psnfss/config.utm
\end{alltt}
Il contenuto di questo file è
%The contents of that file is
\begin{alltt}
  p +utm.map
\end{alltt}
che punta al file \file{utm.map}, che sarà il prossimo che cercheremo.
%which points to the file \file{utm.map}, which we want to
%locate next.
\begin{alltt}
> \Ucom{kpsewhich utm.map}
   /usr/local/texmf-dist/fonts/map/dvips/times/utm.map
\end{alltt}
Questo file di mappatura definisce i nomi dei file dei font \PS{} Type~1
nella collezione URW. Il suo contenuto è simile al seguente (mostriamo
solo una parte delle righe):
%This map file defines the file names of the Type~1 \PS{} fonts in
%the URW collection.  Its contents look like (we only show part of the
%lines):
\begin{alltt}
utmb8r  NimbusRomNo9L-Medi    ... <utmb8a.pfb
utmbi8r NimbusRomNo9L-MediItal... <utmbi8a.pfb
utmr8r  NimbusRomNo9L-Regu    ... <utmr8a.pfb
utmri8r NimbusRomNo9L-ReguItal... <utmri8a.pfb
utmbo8r NimbusRomNo9L-Medi    ... <utmb8a.pfb
utmro8r NimbusRomNo9L-Regu    ... <utmr8a.pfb
\end{alltt}
Prendiamo, ad esempio, il file per il Times Roman \file{utmr8a.pfb} e
cerchiamo la sua posizione nella gerarchia delle directory \file{texmf}
con una ricerca dei file di font Type~1:
%Let us, for instance, take the Times Roman instance
%\file{utmr8a.pfb} and find its position in the \file{texmf} directory
%tree with a search for Type~1 font files:
\begin{alltt}
> \Ucom{kpsewhich utmr8a.pfb}
\ifSingleColumn   /usr/local/texmf-dist/fonts/type1/urw/times/utmr8a.pfb
\else   /usr/local/texmf-dist/fonts/type1/
... urw/utm/utmr8a.pfb
\fi\end{alltt}

Dovrebbe essere chiaro da questi esempi come puoi individuare facilmente
dove sia un dato file. Tutto ciò è particolarmente importante se
sospetti che in qualche modo sia prelevata la versione sbagliata di un
file, in quanto \cmdname{kpsewhich} mostrerà il primo file incontrato.
%It should be evident from these examples how you can easily locate the
%whereabouts of a given file. This is especially important if you suspect
%that the wrong version of a file is picked up somehow, since
%\cmdname{kpsewhich} will show you the first file encountered.

\subsubsection{Risoluzione dei problemi}
%\subsubsection{Debugging actions}
\label{sec:debugging}

A volte è necessario investigare sul come un programma risolva i
riferimenti ad un file. Affinché sia pratico fare ciò, \KPS{} offre vari
livelli di messaggi diagnostici:
%Sometimes it is necessary to investigate how a program resolves file
%references. To make this practical, \KPS{} offers various levels of
%debugging output:

\begin{ttdescription}
\item[\texttt{\ 1}] accessi al disco. Quando si compie una ricerca avendo
  a disposizione un archivio \file{ls-R} aggiornato, questo livello non
  dovrebbe mostrare quasi nessun messaggio.
%\item[\texttt{\ 1}] \texttt{stat} calls (disk lookups). When running
%  with an up-to-date \file{ls-R} database this should almost give no
%  output.
\item[\texttt{\ 2}] Riferimenti alle tabelle dei dati (come gli archivi
  \file{ls-R}, i file di mappatura, quelli di configurazione).
%\item[\texttt{\ 2}] References to hash tables (such as \file{ls-R}
%  databases, map files, configuration files).
\item[\texttt{\ 4}] Operazioni di apertura e chiusura dei file.
%\item[\texttt{\ 4}] File open and close operations.
\item[\texttt{\ 8}] Informazioni generali sui percorsi per i tipi di file
  cercati da \KPS. Utile per scoprire dove è stato definito un particolare
  percorso per un file.
%\item[\texttt{\ 8}] General path information for file types
%  searched by \KPS. This is useful to find out where a particular
%  path for the file was defined.
\item[\texttt{16}] Elenco delle directory per ogni elemento di un percorso
  (rilevante solo per le ricerche su disco).
%\item[\texttt{16}] Directory list for each path element (only relevant
%  for searches on disk).
\item[\texttt{32}] Ricerche di file.
%\item[\texttt{32}] File searches.
\item[\texttt{64}] Valori delle variabili.
%\item[\texttt{64}] Variable values.
\end{ttdescription}
Un valore di \texttt{-1} attiverà tutte le opzioni precedenti; nell'uso
pratico, questo valore è di solito il più conveniente.
%A value of \texttt{-1} will set all the above options; in practice,
%this is usually the most convenient.

Analogamente, con il programma \cmdname{dvips}, impostando una
combinazione di opzioni di risoluzione dei problemi, è possibile seguire
in dettaglio le posizioni da cui i file sono prelevati. In alternativa,
quando un file non è trovato, la traccia dei messaggi mostra in quali
directory il programma ha cercato il file, così che si possa ottenere
un'indicazione sull'origine del problema.
%Similarly, with the \cmdname{dvips} program, by setting a combination of
%debug switches, one can follow in detail where files are being picked up
%from.  Alternatively, when a file is not found, the debug trace shows in
%which directories the program looks for the given file, so that one can
%get an indication what the problem~is.

In termini generali, dato che la maggior parte dei programmi invoca la
libreria \KPS{} internamente, è possibile selezionare il livello della
diagnostica usando la variabile d'ambiente \envname{KPATHSEA\_DEBUG} ed
impostandola ad una combinazione dei valori descritti nell'elenco di cui
sopra.
%Generally speaking, as most programs call the \KPS{} library
%internally, one can select a debug option by using the
%\envname{KPATHSEA\_DEBUG} environment variable, and setting it to (a
%combination of) values as described in the above list.

(Nota per gli utenti Windows: non è semplice redirigere tutti i messaggi
verso un file in questo sistema. Per scopi di diagnostica puoi impostare
temporaneamente \texttt{SET KPATHSEA\_DEBUG\_OUTPUT=err.log}).
%(Note for Windows users: it is not easy to redirect
%all messages to a file in this system. For diagnostic purposes
%you can temporarily \texttt{SET KPATHSEA\_DEBUG\_OUTPUT=err.log}).

Consideriamo, come esempio, un piccolo file sorgente di \LaTeX,
\file{hello-world.tex}, che contiene il seguente testo.
\begin{verbatim}
  \documentclass{article}
  \begin{document}
  Ciao Mondo!
  \end{document}
\end{verbatim}
%Let us consider, as an example, a small \LaTeX{} source file,
%\file{hello-world.tex}, which contains the following input.
%\begin{verbatim}
%  \documentclass{article}
%  \begin{document}
%  Hello World!
%  \end{document}
%\end{verbatim}
Questo piccolo file usa solo il font \file{cmr10}, quindi vediamo come
\cmdname{dvips} prepara il file \PS{} (vogliamo usare la versione Type~1
dei font Computer Modern, da qui l'opzione \texttt{-Pcms}).
%This little file only uses the font \file{cmr10}, so let us look at
%how \cmdname{dvips} prepares the \PS{} file (we want to use the Type~1
%version of the Computer Modern fonts, hence the option \texttt{-Pcms}).
\begin{alltt}
> \Ucom{dvips -d4100 hello-world -Pcms -o}
\end{alltt}
In questo caso abbiamo combinato la classe di diagnostica 4 di
\cmdname{dvips} (percorsi dei font) con l'espansione degli elementi di
percorso di \KPS{} (vedi il Manuale di Riferimento di \cmdname{dvips},
\OnCD{texmf/doc/dvips/dvips.pdf}).
Il risultato (leggermente riorganizzato) appare in
figura~\ref{fig:dvipsdbga}.
%In this case we have combined \cmdname{dvips}'s debug class 4 (font
%paths) with \KPS's path element expansion (see \cmdname{dvips} Reference
%Manual, \OnCD{texmf/doc/dvips/dvips.pdf}).
%The output (slightly rearranged) appears in
%Figure~\ref{fig:dvipsdbga}.
\begin{figure*}[tp]
\centering
\input{examples/ex6a.tex}
\caption{Ricerca dei file di configurazione}\label{fig:dvipsdbga}
%\caption{Finding configuration files}\label{fig:dvipsdbga}
% TODO figura

\bigskip

\input{examples/ex6b.tex}
\caption{Ricerca del file di prologo}\label{fig:dvipsdbgb}
%\caption{Finding the prolog file}\label{fig:dvipsdbgb}

\bigskip

\input{examples/ex6c.tex}
\caption{Ricerca del file del font}\label{fig:dvipsdbgc}
%\caption{Finding the font file}\label{fig:dvipsdbgc}
\end{figure*}

\cmdname{dvips} inizia individuando i file di lavoro. Per primo, viene
trovato \file{texmf.cnf}, che fornisce le definizioni dei percorsi di
ricerca per gli altri file, quindi il file di archivio \file{ls-R} (per
ottimizzare la ricerca) e il file \file{aliases}, che rende possibile
dichiarare diversi nomi (ad esempio, una versione breve in formato
\acro{DOS} 8.3 e una più lunga e più naturale) per lo stesso file. Quindi
\cmdname{dvips} va avanti nel cercare il file di configurazione generico
\file{config.ps} prima di guardare al file di personalizzazione
\file{.dvipsrc} (che, in questo caso \emph{non è trovato}). Infine,
\cmdname{dvips} individua il file di configurazione \file{config.cms} per
il font \PS{} Computer Modern (questo passo è stato attivato con l'opzione
\texttt{-Pcms} passata al comando \cmdname{dvips}). Questo file contiene
la lista delle mappature che definiscono la relazione tra il modo di
chiamare i font in \TeX{}, in \PS{} e sul disco.
%\cmdname{dvips} starts by locating its working files. First,
%\file{texmf.cnf} is found, which gives the definitions of the search
%paths for the other files, then the file database \file{ls-R} (to
%optimize file searching) and the file \file{aliases}, which makes it
%possible to declare several names (e.g., a short \acro{DOS}-like 8.3 and
%a more natural longer version) for the same file.  Then \cmdname{dvips}
%goes on to find the generic configuration file \file{config.ps}
%before looking for the customization file \file{.dvipsrc} (which, in
%this case is \emph{not found}).  Finally, \cmdname{dvips} locates the
%config file for the Computer Modern \PS{} fonts \file{config.cms}
%(this was initiated with the \texttt{-Pcms} option on the \cmdname{dvips}
%command).  This file contains the list of the map files which
%define the relation between the \TeX{}, \PS{} and file system
%names of the fonts.
\begin{alltt}
> \Ucom{more /usr/local/texmf/dvips/cms/config.cms}
   p +ams.map
   p +cms.map
   p +cmbkm.map
   p +amsbkm.map
\end{alltt}
\cmdname{dvips}, quindi, procede nel trovare tutti questi file, più il
generico file di mappatura \file{psfonts.map}, che è caricato sempre
(contiene le dichiarazioni per font \PS{} usati comunemente; consulta la
parte finale della sezione~\ref{sec:examples-of-use} per ulteriori
dettagli sulla gestione delle mappature \PS).
%\cmdname{dvips} thus goes on to find all these files, plus the generic
%map file \file{psfonts.map}, which is always loaded (it contains
%declarations for commonly used \PS{} fonts; see the last part of
%Section \ref{sec:examples-of-use} for more details about \PS{} map
%file handling).

A questo punto, \cmdname{dvips} si presenta all'utente:
%At this point \cmdname{dvips} identifies itself to the user:
\begin{alltt}
This is dvips(k) 5.92b Copyright 2002 Radical Eye Software (www.radicaleye.com)
\end{alltt}
\ifSingleColumn
Quindi prosegue nel cercare il file di prologo \file{texc.pro}:
%Then it goes on to look for the prolog file \file{texc.pro}:
\begin{alltt}\small
kdebug:start search(file=texc.pro, must\_exist=0, find\_all=0,
  path=.:~/tex/dvips//:!!/usr/local/texmf/dvips//:
       ~/tex/fonts/type1//:!!/usr/local/texmf/fonts/type1//).
kdebug:search(texc.pro) => /usr/local/texmf/dvips/base/texc.pro
\end{alltt}
\else
Quindi prosegue nel cercare il file di prologo \file{texc.pro} (vedi la
figura~\ref{fig:dvipsdbgb}).
%Then it goes on to look for the prolog file \file{texc.pro} (see
%Figure~\ref{fig:dvipsdbgb}).
\fi

Dopo aver trovato il file in questione, \cmdname{dvips} stampa la data e
l'ora, ci informa che genererà il file \file{hello-world.ps}, quindi
che avrà bisogno del file del font \file{cmr10} e che quest'ultimo è
dichiarato ``residente'' (non sono necessarie bitmap):
%After having found the file in question, \cmdname{dvips} outputs
%the date and time, and informs us that it will generate the
%file \file{hello-world.ps}, then that it needs the font file
%\file{cmr10}, and that the latter is declared as ``resident'' (no
%bitmaps needed):
\begin{alltt}\small
TeX output 1998.02.26:1204' -> hello-world.ps
Defining font () cmr10 at 10.0pt
Font cmr10 <CMR10> is resident.
\end{alltt}
Adesso la ricerca prosegue con il file \file{cmr10.tfm}, che viene
trovato, quindi viene fatto riferimento ad alcuni ulteriori file di
prologo (non mostrati) e, infine, l'istanza Type~1 del font,
\file{cmr10.pfb}, è individuata ed inclusa del file in uscita (vedi
l'ultima linea).
%Now the search is on for the file \file{cmr10.tfm}, which is found,
%then a few more prolog files (not shown) are referenced, and finally
%the Type~1 instance \file{cmr10.pfb} of the font is located and
%included in the output file (see last line).
\begin{alltt}\small
kdebug:start search(file=cmr10.tfm, must\_exist=1, find\_all=0,
  path=.:~/tex/fonts/tfm//:!!/usr/local/texmf/fonts/tfm//:
       /var/tex/fonts/tfm//).
kdebug:search(cmr10.tfm) => /usr/local/texmf/fonts/tfm/public/cm/cmr10.tfm
kdebug:start search(file=texps.pro, must\_exist=0, find\_all=0,
   ...
<texps.pro>
kdebug:start search(file=cmr10.pfb, must\_exist=0, find\_all=0,
  path=.:~/tex/dvips//:!!/usr/local/texmf/dvips//:
       ~/tex/fonts/type1//:!!/usr/local/texmf/fonts/type1//).
kdebug:search(cmr10.pfb) => /usr/local/texmf/fonts/type1/public/cm/cmr10.pfb
<cmr10.pfb>[1]
\end{alltt}

\subsection{Opzioni di esecuzione}
%\subsection{Runtime options}

% FIXME: va bene `array' come `registri'?
Un'altra funzionalità utile di \Webc{} è la sua possibilità di controllare
un certo numero di parametri relativi alla memoria (nello specifico la
dimensione dei registri) tramite il file \file{texmf.cnf} letto da \KPS{}
durante l'esecuzione. Le impostazioni della memoria possono essere trovate
nella Parte 3 di quel file nella distribuzione \TL. Le più importanti
sono:
%Another useful feature of \Webc{} is its possibility to control a number
%of memory parameters (in particular, array sizes) via the runtime file
%\file{texmf.cnf} read by \KPS{}.  The memory settings can be found in
%Part 3 of that file in the \TL{} distribution. The more important
%are:

\begin{ttdescription}
\item[\texttt{main\_memory}]
  La quantità complessiva di memoria disponibile per \TeX, \MF{} e \MP.
  Devi creare un nuovo file di formato per ogni impostazione differente.
  Per esempio, potresti generare una versione ``enorme'' di \TeX{} e
  chiamare il formato \texttt{hugetex.fmt}. Usando il modo normale di
  specificare il nome del programma usato da \KPS{}, l'opportuno valore
  della variabile \texttt{main\_memory} sarà letto da \file{texmf.cnf}.
%\item[\texttt{main\_memory}]
%  Total words of memory available, for
%  \TeX{}, \MF{} and \MP.  You must make a new format file for each
%  different setting. For instance, you could generate a ``huge''
%  version of \TeX{}, and call the format file \texttt{hugetex.fmt}.
%  Using the standard way of specifying the program name used by \KPS{},
%  the particular value of the \texttt{main\_memory} variable will then
%  be read from \file{texmf.cnf}.
\item[\texttt{extra\_mem\_bot}]
  Spazio aggiuntivo per le strutture dati ``grandi'' di \TeX: scatole,
  colle, interruzioni, ecc. Utile specialmente se usi \PiCTeX.
%\item[\texttt{extra\_mem\_bot}]
%  Extra space for ``large'' \TeX{} data structures:
%  boxes, glue, breakpoints, etc.  Especially useful if you use
%  \PiCTeX{}.
\item[\texttt{font\_mem\_size}]
  Numero di registri per le informazioni sui font disponibile in \TeX.
  Questo valore è più o meno pari alla dimensione totale di tutti i file
  \acro{TFM} che vengono letti.
%\item[\texttt{font\_mem\_size}]
%  Number of words for font information available for \TeX. This
%  is more or less the total size of all \acro{TFM} files read.
\item[\texttt{hash\_extra}]
  Spazio aggiuntivo per la tabella con i nomi delle sequenze di
  controllo. Nella tabella principale possono essere memorizzate solo
  approssimativamente 10.000 sequenze di controllo; se lavori su un libro
  di grandi dimensioni con numerosi riferimenti incrociati, questo valore
  potrebbe non essere sufficiente. Il valore predefinito per
  \texttt{hash\_extra} è \texttt{50000}.
%\item[\texttt{hash\_extra}]
%  Additional space for the hash table of control sequence names.
%  Only $\approx$10,000 control sequences can be stored in the main
%  hash table; if you have a large book with numerous cross-references,
%  this might not be enough.  The default value of
%  \texttt{hash\_extra} is \texttt{50000}.
\end{ttdescription}

\noindent Ovviamente questa funzionalità non è un sostituto per una vera
allocazione di memoria e tabelle dinamici, ma dato che questi sono
estremamente difficili da implementare negli attuali sorgenti di \TeX,
questi parametri di esecuzione forniscono un compromesso pratico che offre
un minimo di flessibilità.
%\noindent Of course, this facility is no substitute for truly dynamic
%arrays and memory allocation, but since these are extremely difficult to
%implement in the present \TeX{} source, these runtime parameters provide
%a practical compromise allowing some flexibility.


\begin{comment}
\section{Compilare su una nuova piattaforma Unix}
%\section{Building on a new Unix platform}

Se hai una piattaforma per la quale non sono inclusi gli eseguibili, avrai
bisogno di compilare \TeX{} e compagni. Questo passo non è difficile come
sembra. Quello di cui hai bisogno è tutto contenuto nella directory
\texttt{source} nella distribuzione.
%If you have a platform for which executables are not included, you will
%need to compile \TeX{} and friends. This is not as hard as it
%sounds. What you need is all in the directory \texttt{source} in the
%distribution.

\subsection{Prerequisiti}
%\subsection{Prerequisites}

Avrai bisogno di almeno 100 megabyte di spazio su disco per compilare
tutto \TeX{} e i suoi programmi di supporto. Avrai anche bisogno di un
compilatore \acro{ANSI} C, un'utilità \cmdname{make}, un analizzatore
lessicale e un generatore di parser. Raccomandiamo la versione \GNU{} di
questi programmi (\cmdname{gcc}, \GNU \cmdname{make}, \cmdname{m4},
\cmdname{flex}, \cmdname{bison}). Potresti essere in grado di lavorare con
altri compilatori C e programmi \cmdname{make}, ma avrai bisogno di una
buona comprensione della compilazione di programmi Unix per superare i
problemi.
%You will need at least 100 megabytes of disk space to compile all of
%\TeX{} and its support programs. You'll also need an \acro{ANSI} C
%compiler, a \cmdname{make} utility, a lexical scanner, and a parser
%generator. We recommend the \GNU version of these programs
%(\cmdname{gcc}, \GNU \cmdname{make}, \cmdname{m4}, \cmdname{flex},
%\cmdname{bison}).  You may be able to work with other C compilers and
%\cmdname{make} programs, but you will need a good understanding of
%building Unix programs to sort out problems.

Inoltre il comando \texttt{uname} deve restituire un valore significativo.
%Also, the command \texttt{uname} must return a sensible value.


\subsection{Configurazione}
%\subsection{Configuration}

Per cominciare, esegui una normale installazione di \TL{} sul tuo disco
(consulta la sezione~\ref{sec:install-disk} a
\p.\pageref{sec:install-disk}). Potresti voler saltare l'installazione di
tutti gli eseguibili pregenerati.
%To begin, perform a normal installation of \TL{} to your disk (see
%section~\ref{sec:install-disk} on
%\p.\pageref{sec:install-disk}).  You may wish to skip installing
%all of the prebuilt binaries.

Quindi, decomprimi i sorgenti dal file \texttt{tar} compresso nella
directory \dirname{source} sul tuo disco e sposta la directory corrente
dove hai messo il contenuto del file.
%Then, unpack the source from the compressed \texttt{tar} file in the
%directory \dirname{source} to your disk and change directory to where
%you placed it.

Successivamente, esegui \cmdname{configure} con una riga di comando come
questa:
%Next, run \cmdname{configure} with a command line like this:
\begin{alltt}
> \Ucom{sh configure -prefix=/usr/local/TeX}
\end{alltt}

La directory \optname{-prefix} è quella dove hai installato la gerarchia
di \TL; la strutturazione della directory sarà come segue (dove \$TEXDIR
sta per la directory che hai scelto):
%The \optname{-prefix} directory is the one where you installed the
%support tree; the directory layout will be as follows (where \$TEXDIR
%stands for the directory you chose):

\noindent
\begin{tabular}{>{\ttfamily}ll@{}}
  \dirname{$TEXDIR/share/texmf}          & gerarchia principale con
                                         & font, \qquad macro, ecc\\
%  \dirname{$TEXDIR/share/texmf}          & main tree with fonts,\\
%                                         & \qquad macros, etc\\
  \dirname{$TEXDIR/man}                  & pagine di manuale Unix\\
%  \dirname{$TEXDIR/man}                  & Unix manual pages\\
  \dirname{$TEXDIR/info}                 & manuali in stile \GNU Info\\
%  \dirname{$TEXDIR/info}                 & \GNU style Info manuals\\
  \dirname{$TEXDIR/bin/$PLATFORM}        & eseguibili\\
%  \dirname{$TEXDIR/bin/$PLATFORM}        & binaries\\
\end{tabular}
%$

Se vuoi lasciar fuori il livello di directory \dirname{$PLATFORM}, cioè
vuoi mettere gli eseguibili direttamente dentro \dirname{$TEXDIR/bin},
specifica a \cmdname{configure} l'opzione \verb|--disable-multiplatform|.
%If you want to leave out the \dirname{$PLATFORM} directory level,
%i.e., put the binaries directly into \dirname{$TEXDIR/bin}, specify
%the \verb|--disable-multiplatform| option to \cmdname{configure}.

Dai uno sguardo ai messaggi restituiti da \verb|./configure --help| per
ulteriori opzioni che puoi usare. Ad esempio, puoi saltare la compilazione
di \OMEGA{} ed \eTeX.
%Have a look at the output of \verb|./configure --help| for more
%options you can use.  For example, you can skip building of \OMEGA{} and
%\eTeX{}.


\subsection{Eseguire \cmdname{make}}
%\subsection{Running \cmdname{make}}

Sii sicuro che né la variabile d'ambiente o né l'opzione
\texttt{noclobber} siano impostate. Quindi, esegui il \cmdname{make}
principale in questo modo:
%Make sure the shell variable or option \texttt{noclobber} is not set.
5Then, run the main \cmdname{make} like this:
\begin{alltt}
> \Ucom{make world}
\end{alltt}
e rilassati\ldots
%and relax\ldots

In alternativa, puoi voler salvare in un registro tutti i messaggi
restituiti, come in:
%Alternatively, you want to log all the output, as in:
\begin{alltt}
> \Ucom{sh -c "make world >world.log 2>\&1" \&}
\end{alltt}

Prima di credere che tutto sia andato bene, controlla il file di registro
per gli errori: \GNU \cmdname{make} usa sempre la stringa \samp{***}
quando un comando fallisce. Inoltre, controlla se tutti i programmi sono
stati compilati:
%Before you believe that everything went ok, please check the log file
%for errors: \GNU \cmdname{make} always uses the string \samp{***}
%whenever a command fails.  Also, check if all the programs were built:

\begin{alltt}
> \Ucom{cd \var{TEXDIR}/bin/\var{archname}}
> \Ucom{ls | wc}
\end{alltt}
Il risultato dovrebbe essere oltre 200 (puoi verificare il numero esatto
con il contenuto della directory \dirname{bin} della distribuzione).
%The result should be over 200 (you can check the exact number with the
%\dirname{bin} directory contents in the distribution).

Se hai bisogno di privilegi speciali per \texttt{make install}, puoi
separare l'esecuzione di \samp{make world} in due parti, come segue:
%If you need special privileges for \texttt{make install}, you can
%separate the \samp{make world} into two different runs, like this:
\begin{alltt}
> \Ucom{make all}
> \Ucom{su}
> \Ucom{make install strip}
\end{alltt}

Dopo aver installato i tuoi nuovi eseguibili, dovresti eseguire le normali
procedure post installazione, fornite nella sezione~\ref{sec:postinstall}
(\p.\pageref{sec:postinstall}).
%After you've installed your new binaries, you should follow the normal
%post-installation procedures, given in section~\ref{sec:postinstall}
%(\p.\pageref{sec:postinstall}).

Inoltre, se desideri rendere disponibili i tuoi eseguibili ad altri,
contattaci. Saremo felici di inserirli nelle pagine web di \TL.
%Also, if you'd like to make your binaries available to others, please
%contact us.  We'll be happy to put them on the \TL\ web pages.
\end{comment}

\htmlanchor{ack}
\section{Ringraziamenti}
\section{Acknowledgements}

\TL{} è un lavoro congiunto di praticamente tutti i gruppi utenti \TeX.
Questa edizione di \TL{} è stata supervisionata da Karl Berry. Gli altri
contributori principali, passati e presenti, sono elencati qui sotto.
%\TL{} is a joint effort by virtually all of the \TeX{} user groups.
%This edition of \TL{} was overseen by Karl Berry.  The other principal
%contributors, past and present, are listed below.

\begin{itemize*}

\item I gruppi utenti \TeX{} Inglese, Tedesco, Olandese e Polacco
(\acro{TUG}, \acro{DANTE} e.V., \acro{NTG} e \acro{GUST},
rispettivamente), che forniscono la necessaria infrastruttura tecnica ed
amministrativa.  Unisciti al gruppo utenti \TeX\ più vicino a te (vedi
\url{http://tug.org/usergroups.html})!
%\item The English, German, Dutch, and Polish \TeX{} user groups
%(\acro{TUG}, \acro{DANTE} e.V., \acro{NTG}, and \acro{GUST},
%respectively), which provide the necessary technical and administrative
%infrastructure.  Please join the \TeX\ user group near you!  (See
%\url{http://tug.org/usergroups.html}.)

\item La squadra di \acro{CTAN}, in particolare Robin Fairbairns, Jim
Hef{}feron e Rainer Sch\"opf, che distribuiscono le immagini di \TL{} e
forniscono l'infrastruttura comune per gli aggiornamenti dei pacchetti,
sulla quale \TL{} dipende.
%\item The \acro{CTAN} team, notably Robin Fairbairns, Jim Hef{}feron,
%and Rainer Sch\"opf, which distributes the \TL{} images and provides the
%common infrastructure for package updates, upon which \TL{} depends.

\item Nelson Beebe per aver reso disponibili molte piattaforme agli
sviluppatori di \TL{} e per i suoi esaustivi collaudi.
%\item Nelson Beebe, for making many platforms available to \TL\
%developers, and his own comprehensive testing.

\item John Bowman per aver fatto molti cambiamenti al suo programma
avanzato di grafica Asymptote affinché funzionasse in \TL.
%\item John Bowman, for making many changes to his advanced graphics
%program Asymptote to make it work in \TL.

\item Peter Breitenlohner ed la squadra di \eTeX\ per le stabili
fondamenta del futuro di \TeX{} e Peter in particolare per l'aiuto
stellare con l'uso degli strumenti \GNU\ autotool in \TL.
%\item Peter Breitenlohner and the \eTeX\ team for the stable foundation
%of future \TeX's, and Peter specifically for stellar help with \GNU\
%autotools usage throughout \TL.

\item Jin-Hwan Cho e tutti i membri della squadra di DVIPDFM$x$ per il
loro driver eccellente e la reattività alle questioni di configurazione.
%\item Jin-Hwan Cho and all of the DVIPDFM$x$ team, for their
%excellent driver and responsiveness to configuration issues.

\item Thomas Esser, senza il cui meraviglioso pacchetto \teTeX{} \TL{} non
sarebbe mai esistito.
%\item Thomas Esser, without whose marvelous \teTeX{} package \TL{}
%would have never existed.

\item Michel Goossens, che è stato coautore della documentazione
originale.
%\item Michel Goossens, who co-authored the original documentation.

\item Eitan Gurari, il cui \TeX4ht è stato usato per creare la versione
\HTML{} di questa documentazione e che ha lavorato instancabilmente per
migliorarlo con brevi preavvisi. Eitan è scomparso prematuramente nel
giugno 2009 e dedichiamo questa documentazione alla sua memoria.
%\item Eitan Gurari, whose \TeX4ht was used to create the \HTML{}
%version of this documentation, and who worked tirelessly to improve
%it at short notice.  Eitan prematurely passed away in June 2009, and we
%dedicate this documentation to his memory.

\item Hans Hagen per la grande quantità di collaudi e per aver permesso al
suo formato \ConTeXt{} (\url{http://pragma-ade.com}) di lavorare
all'interno dell'infrastruttura di \TL.
%\item Hans Hagen, for much testing and making his \ConTeXt\ format
%(\url{http://pragma-ade.com}) work within \TL's framework.

\item \Thanh, Martin Schr\"oder e la squadra di pdf\TeX{}
(\url{http://pdftex.prg}) per i continui miglioramenti delle capacità di
\TeX.
%\item \Thanh, Martin Schr\"oder, and the pdf\TeX\ team
%(\url{http://pdftex.org}) for continuing enhancements of \TeX's
%abilities.

\item Hartmut Henkel per i significativi contributi allo sviluppo di
pdf\TeX, Lua\TeX{} e molto ancora.
%\item Hartmut Henkel, for significant development contributions to
%pdf\TeX\, Lua\TeX, and more.

\item Taco Hoekwater per gli importanti forzi nel rinnovato sviluppo di
MetaPost e (Lua)\TeX{} (\url{http://luatex.org}) stesso, per aver
\ConTeXt{} dentro \TL, per aver dato a Kpathsea le funzionalità
multi-thread e per molto altro ancora.
%\item Taco Hoekwater, for major renewed development efforts on MetaPost and
%(Lua)\TeX\ (\url{http://luatex.org}) itself, incorporating
%\ConTeXt\ into \TL, giving Kpathsea multi-threaded functionality, and
%much more.

\item Pawe{\l} Jackowski per l'installatore Windows \cmdname{tlpm} e
Tomasz {\L}uczak per \cmdname{tlpmgui}, usati nelle edizioni passate.
%\item Pawe{\l} Jackowski, for the Windows installer \cmdname{tlpm},
%and Tomasz {\L}uczak, for \cmdname{tlpmgui}, used in past releases.

\item Akira Kakuto per aver fornito gli eseguibili per Windows a partire
dalla sua distribuzione \acro{W32TEX} per \TeX{} in giapponese e per molti
altri contributi allo sviluppo.
%\item Akira Kakuto, for providing the Windows
%binaries from his \acro{W32TEX} distribution for Japanese \TeX\
%(\url{http://w32tex.org}), and many other development contributions.

\item Jonathan Kew per aver sviluppato il notevole motore \XeTeX{} e per
investito tempo e fatica per integrarlo in \TL, così come per la versione
iniziale dell'installatore di Mac\TeX{} e per il programma \TeX{}works,
che raccomandiamo.
%\item Jonathan Kew, for developing the remarkable \XeTeX{} engine and
%taking the time and trouble to integrate it in \TL{}, as well as the
%initial version of the Mac\TeX\ installer, and also for our recommended
%front-end \TeX{}works.

\item Dick Koch per mantenere Mac\TeX{} (\url{http://tug.org/mactex}) a
distanza molto ravvicinata da \TL{} e per il suo grande buon umore nel
farlo.
%\item Dick Koch, for maintaining Mac\TeX\ (\url{http://tug.org/mactex})
%in very close tandem with \TL{}, and for his great good cheer in doing
%so.

\item Reinhard Kotucha per i maggiori contributi all'infrastruttura e
all'installatore di \TL{} 2008, così come per gli sforzi di ricerca sotto
Windows, lo script \texttt{getnonfreefonts} e molto altro.
%\item Reinhard Kotucha, for major contributions to the \TL{} 2008
%infrastructure and installer, as well as Windows research efforts, the
%\texttt{getnonfreefonts} script, and more.

\item Anche Siep Kroonenberg per maggiori contributi all'infrastruttura e
all'installatore di \TL{} 2008, specialmente sotto Windows, e il grosso
del lavoro di aggiornamento di questo manuale per descrivere queste
funzionalità.
%\item Siep Kroonenberg, also for major contributions to the \TL{} 2008
%infrastructure and installer, especially on Windows, and for the bulk of
%work updating this manual describing those features.

\item Heiko Oberdiek per il pacchetto \pkgname{epstopdf} e molti altri,
per aver compresso gli enormi file di dati di \pkgname{pst-geo} cosi che
potessimo includerli e, più di tutto, per il suo notevole lavoro su
\pkgname{hyperref}.
%\item Heiko Oberdiek, for the \pkgname{epstopdf} package and many
%others, compressing the huge \pkgname{pst-geo} data files so we could
%include them, and most of all, for his remarkable work on
%\pkgname{hyperref}.

\item Petr Ok\v{s}ak che ha coordinato e controllato con grande attenzione
tutto il materiale ceco e slovacco.
%\item Petr Ol\v{s}ak, who coordinated and checked all the Czech and Slovak
%material very carefully.

\item Toshio Oshima per il suo visualizzatore per Windows
\cmdname{dviout}.
%\item Toshio Oshima, for his \cmdname{dviout} previewer for Windows.

\item Manuel Pégourié-Gonnard per aver aiutato con gli aggiornamenti dei
pacchetti, con i miglioramenti della documentazione e con lo sviluppo di
\cmdname{texdoc}.
%\item Manuel P\'egouri\'e-Gonnard, for helping with package updates,
%documentation improvements, and \cmdname{texdoc} development.

\item Fabrice Popineau per il supporto originale per Windows in \TL{} e
per il lavoro sulla documentazione in francese.
%\item Fabrice Popineau, for the original Windows support in \TL{} and
%work on the French documentation.

\item Norbert Preining, l'architetto principale dell'infrastruttura e
dell'installatore di \TL{} 2008, anche per aver coordinato la versione
Debian di \TL{} (insieme con Frank Küster), offrendo molti suggerimenti
lungo la via.
%\item Norbert Preining, the principal architect of the \TL{} 2008
%infrastructure and installer, and also for coordinating the Debian
%version of \TL{} (together with Frank K\"uster), making many suggestions
%along the way.

\item Sebastian Rahtz per aver originariamente creato \TL{} ed averne
curato la manutenzione per molti anni.
%\item Sebastian Rahtz, for originally creating \TL{} and maintaining it
%for many years.

\item Phil Taylor per aver allestito gli scaricamenti tramite BitTorrent.
%\item Phil Taylor, for setting up the BitTorrent downloads.

\item Tomasz Trzeciak per aiuti su vasta scala con Windows.
%\item Tomasz Trzeciak, for wide-ranging help with Windows.

\item Vladimir Volovich per il sostanziale aiuto nel porting e in altre
questioni di mantenimento e specialmente per aver reso possibile includere
\cmdname{xindy}.
%\item Vladimir Volovich, for substantial help with porting and other
%maintenance issues, and especially for making it feasible to include
%\cmdname{xindy}.

\item Staszek Wawrykiewicz, il collaudatore principale di tutta \TL{} e
coordinatore di molti dei maggiori contributori polacchi: font,
installazione sotto Windows e molto altro.
%\item Staszek Wawrykiewicz, the principal tester for all of \TL{},
%and coordinator of the many major Polish contributions: fonts, Windows
%installation, and more.

\item Olaf Weber per la sua paziente manutenzione a \Webc.
%\item Olaf Weber, for his patient maintenance of \Webc.

\item Gerben Wierda per aver creato e fatto manutenzione all'originale
supporto per \MacOSX{} e per molte integrazioni e collaudi.
%\item Gerben Wierda, for creating and maintaining the original \MacOSX\
%support, and much integration and testing.

\item Graham Williams, sul cui lavoro dipende il Catalogo \TeX{} dei
pacchetti.
%\item Graham Williams, on whose work the \TeX\ Catalogue of packages depends.

\end{itemize*}

Preparatori degli eseguibili:
%Builders of the binaries:
Peter Breitenlohner (\pkgname{x86\_64-linux}),
%Tim Arnold (\pkgname{hppa-hpux}),
%Randy Au (\pkgname{amd64-freebsd}),
%Edd Barrett (\pkgname{i386-openbsd}),
Karl Berry (\pkgname{i386-linux}, \pkgname{sparc-linux}),
Ken Brown (\pkgname{i386-cygwin}),
Akira Kakuto (\pkgname{win32}),
Dick Koch (\pkgname{universal-darwin}),
%Manfred Lotz (\pkgname{i386-freebsd}),
Norbert Preining (\pkgname{alpha-linux}),
%Arthur Reutenauer (\pkgname{sparc-linux}),
Jukka Salmi (\pkgname{i386-netbsd}),
Thomas Schmitz (\pkgname{powerpc-linux}),
Apostolos Syropoulos (\pkgname{i386-solaris}),
Vladimir Volovich (\pkgname{powerpc-aix}, \pkgname{sparc-solaris}),
Olaf Weber (\pkgname{mips-irix}).
Per informazioni sul processo di compilazione di \TL, visitate
\url{http://tug.org/texlive/build.html}.
%For information on the \TL{} build process, see
%\url{http://tug.org/texlive/build.html}.

Attuali traduttori della documentazione:
%Current documentation translators:
Jjgod Jiang, Jinsong Zhao, Yue Wang, \& Helin Gai (Cinese),
%Jjgod Jiang, Jinsong Zhao, Yue Wang, \& Helin Gai (Chinese),
Klaus H\"oppner (Tedesco),
%Klaus H\"oppner (German),
Manuel P\'egouri\'e-Gonnard (Francese),
%Manuel P\'egouri\'e-Gonnard (French),
Marco Pallante (Italiano),
%Marco Pallante (Italian),
Petr Sojka \& Jan Busa (Ceco\slash Slovacco),
%Petr Sojka \& Jan Busa (Czech\slash Slovak),
Boris Veytsman (Russo),
%Boris Veytsman (Russian),
Staszek Wawrykiewicz (Polacco).  La pagina web della documentazione di
\TL{} è \url{http://tug.org/texlive/doc.html}.
%Staszek Wawrykiewicz (Polish).  The \TL{} documentation web page
%is \url{http://tug.org/texlive/doc.html}.

Ovviamente il più importante ringraziamento deve andare a Donald Knuth,
innanzitutto per aver inventato \TeX{} e poi per averlo donato al mondo.
%Of course the most important acknowledgement must go to Donald Knuth,
%first for inventing \TeX, and then for giving it to the world.



\section{Storia delle edizioni}
%\section{Release history}
\label{sec:history}

\subsection{Passato}
%\subsection{Past}

La discussione iniziò nel tardo 1993 quando il gruppo utenti \TeX{}
olandese stava iniziando a lavorare al proprio \CD{} 4All\TeX{} per gli
utenti \acro{MS-DOS} e si sperava a quel tempo di rilasciare un solo
razionale \CD{} per tutti i sistemi. Questo era un obiettivo troppo
ambizioso per il tempo, ma non solo diede vita al \CD{} di grande successo
4All\TeX{}, ma spinse il gruppo di lavoro del \acro{TUG} Technical
Council verso una \emph{Struttura delle Directory \TeX{}} (\TDS{}, \TeX{}
Directory Structure, \url{http://tug.org/tds}), che specificò come creare
collezioni consistenti e gestibili di file di supporto a \TeX. Una bozza
completa della \TDS{} fu pubblicata nel numero di dicembre 1995 di
\textsl{TUGboat} e fu chiaro dalle fasi iniziali che un prodotto
desiderabile sarebbe stata una struttura di modello su \CD{}. La
distribuzione che hai ora è il risultato diretto delle decisioni del
gruppo di lavoro. Fu anche chiaro dal successo del \CD{} 4All\TeX{} che
gli utenti Unix avrebbero beneficiato da un simile semplice sistema e
questa è l'altro filone principale di \TL.
%Discussion began in late 1993 when the Dutch \TeX{} Users Group was
%starting work on its 4All\TeX{} \CD{} for \acro{MS-DOS} users, and it
%was hoped at that time to issue a single, rational, \CD{} for all
%systems. This was too ambitious a target for the time, but it did spawn
%not only the very successful 4All\TeX{} \CD{}, but also the \acro{TUG}
%Technical Council working group on a \emph{\TeX{} Directory Structure}
%(\url{http://tug.org/tds}), which specified how to create consistent and
%manageable collections of \TeX{} support files. A complete draft of the
%\TDS{} was published in the December 1995 issue of \textsl{TUGboat}, and
%it was clear from an early stage that one desirable product would be a
%model structure on \CD{}. The distribution you now have is a very direct
%result of the working group's deliberations. It was also clear that the
%success of the 4All\TeX{} \CD{} showed that Unix users would benefit
%from a similarly easy system, and this is the other main strand of
%\TL.

Per prima cosa ci mettemmo all'opera per realizare un nuovo \CD{} della
\TDS{} basato su Unix nell'autunno del 1995 e rapidamente identificammo il
\teTeX{} di Thomas Esser come l'impianto ideale, dato che già aveva il
supporto per più fiattaforme ed era costruito con la portabilità tra
diversi file system in mente. Thomas acconsentì ad aiutarci e il lavoro
cominciò seriamente all'inizio del 1996. La prima edizione fu rilasciata
nel maggio 1996. All'inizio del 1996, Karl Berry completò una nuova
versione di Web2c, che includeva praticamente tutte le funzionalità che
Thomas Esser aveva aggiunto in \teTeX{} e decidemmo di basare la seconda
edizione del \CD{} sul \Webc{} standard, con l'inclusione dello script
\texttt{texconfig} proveniente da \teTeX. La terza edizione del \CD{} fu
basata su una nuova grande revisione di \Webc, la 7.2, realizzata da Olaf
Weber; allo stesso tempo, era stata fatta una nuova revisione di \teTeX{}
e \TL{} incluse quasi tutte le sue funzionalità. La quarta edizione seguì
lo stesso modello, usando una nuova versione di \teTeX{} e di \Webc{}
(7.3). Il sistema adesso includeva anche un completo allestimento per
Windows.
%We first undertook to make a new Unix-based \TDS{} \CD{} in the autumn
%of 1995, and quickly identified Thomas Esser's \teTeX{} as the ideal
%setup, as it already had multi-platform support and was built with
%portability across file systems in mind. Thomas agreed to help, and work
%began seriously at the start of 1996. The first edition was released in
%May 1996. At the start of 1997, Karl Berry completed a major new release
%of Web2c, which included nearly all the features which Thomas Esser had
%added in \teTeX, and we decided to base the 2nd edition of the \CD{} on
%the standard \Webc, with the addition of \teTeX's \texttt{texconfig}
%script. The 3rd edition of the \CD{} was based on a major revision of
%\Webc, 7.2, by Olaf Weber; at the same time, a new revision of \teTeX
%was being made, and \TL{} included almost all of its features. The
%4th edition followed the same pattern, using a new version of \teTeX,
%and a new release of \Webc{} (7.3).  The system now included a complete
%Windows setup.

Per la quinta edizione (marzo 2000) furono riviste e controllate molte
parti del \CD, aggiornando centinaia di pacchetti. I dettagli sui
pacchetti furono memorizzati in file XML. Ma il cambiamento maggiore per
\TeX\ Live 5 fu che tutto il software non libero fu rimosso. Tutto in
\TL{} era pensato per essere compatibile con le Debian Free Software
Guidelines (linee guida Debian sul software libero,
\url{http://www.debian.org/intro/free}); abbiamo fatto del nostro meglio
per controllare le condizioni di licenza di tutti i pacchetti, ma
apprezzeremo tantissimo ogni segnalazione di errori.
%For the 5th edition (March 2000) many parts of the \CD{} were revised
%and checked, updating hundreds of packages. Package details were stored
%in XML files. But the major change for \TeX\ Live 5 was that all
%non-free software was removed. Everything in \TL{} is now intended
%to be compatible with the Debian Free Software Guidelines
%(\url{http://www.debian.org/intro/free}); we have done our best to check
%the license conditions of all packages, but we would very much
%appreciate hearing of any mistakes.

La sesta edizione (luglio 2001) aveva aggiornato ancora più materiale. Il
cambiamento più grande fu un nuovo concetto di installazione: l'utente
poteva selezionare un insieme più esatto delle collezioni desiderate. Le
collezioni relative alle lingue furono completamente riorganizzate, così
che, selezionandone una qualunque, non solo venissero installati le macro, i
font, ecc., ma fosse anche preparato un opportuno file
\texttt{language.dat}.
%The 6th edition (July 2001) had much more material updated. The major
%change was a new install concept: the user could select a more exact set
%of needed collections. Language-related collections were completely
%reorganized, so selecting any of them installs not only macros, fonts,
%etc., but also prepares an appropriate \texttt{language.dat}.

La settima edizione del 2002 ebbe la notevole aggiunta del supporto per
\MacOSX{} e la solita miriade di aggiornamenti ad ogni genere di pacchetto
e programma. Un traguardo importante fu l'integrazione dei sorgenti con
quelli di \teTeX{} per correggere l'allontanamento l'uno dall'altro
avvenuto nelle versioni~5 e~6.
%The 7th edition of 2002 had the notable addition of \MacOSX{} support,
%and the usual myriad of updates to all sorts of packages and
%programs. An important goal was integration of the source back with
%\teTeX, to correct the drift apart in versions~5 and~6.

\subsubsection{2003}

Nel 2003, con il continuo flusso di aggiornamenti ed aggiunte, trovammo
che \TL{} era cresciuta così tanto che non poteva più essere contenutoa in
un singolo \CD, quindi la dividemmo in tre diverse distribuzioni (consulta
la sezione~\ref{sec:tl-coll-dists}, \p.\pageref{sec:tl-coll-dists}). In
più:
%In 2003, with the continuing flood of updates and additions, we found
%that \TL{} had grown so large it could no longer be contained on a
%single \CD, so we split it into three different distributions (see
%section~\ref{sec:tl-coll-dists}, \p.\pageref{sec:tl-coll-dists}).  In
%addition:

\begin{itemize*}
\item Su richiesta della squadra di \LaTeX, cambiammo i comandi
      \cmdname{latex} e \cmdname{pdflatex} affinché usassero \eTeX{} (vedi
      \p.\pageref{text:etex}).
%\item At the request of the \LaTeX{} team, we changed the standard
%      \cmdname{latex} and \cmdname{pdflatex} commands to now use \eTeX{} (see
%      \p.\pageref{text:etex}).
\item I nuovi font Latin Modern furono inclusi (e sono raccomandati).
%\item The new Latin Modern fonts were included (and are recommended).
\item Il supporto per Alpha \acro{OSF} fu rimosso (il supporto per
      \acro{HPUX} era già stato rimosso in precedenza) dato che nessuno
      aveva (o voleva donare) l'hardware su cui compilare i nuovi
      eseguibili.
%\item Support for Alpha \acro{OSF} was removed
%      (\acro{HPUX} support was removed previously), since no one had (or
%      volunteered) hardware available on which to compile new binaries.
\item L'allestimento per Windows fu cambiato sostanzialment; per la prima
      volta fu introdotto un ambiente integrato basato su XEmacs.
%\item Windows setup was substantially changed; for the first time
%      an integrated environment based on XEmacs was introduced.
\item Importanti programmi aggiuntivi per Windows (Perl, Ghost\-script,
      Image\-Magic, Ispell) sono ora installati nelle directory \TL{}.
%\item Important supplementary programs for Windows
%      (Perl, Ghost\-script, Image\-Magick, Ispell) are now installed
%      in the \TL{} installation directory.
\item I file di mappatura per i font usati da \cmdname{dvips},
      \cmdname{dvipdfm} e \cmdname{pdftex} sono ora generati da un nuovo
      programma \cmdname{updmap} ed installati in
      \dirname{texmf/fonts/map}.
%\item Font map files used by \cmdname{dvips}, \cmdname{dvipdfm}
%      and \cmdname{pdftex} are now generated by the new program
%      \cmdname{updmap} and installed into \dirname{texmf/fonts/map}.
\item \TeX, \MF{} e \MP{} adesso mostrano la maggior parte dei caratteri
      in ingresso (dal numero 32 \acro{ASCII} in su) come se stessi nei
      file che vengono generati (ad esempio, con \verb|\write|), nei file
      di registro e sul terminale, ossia \emph{non sono più} tradotti
      usando la notazione \verb|^^|. In \TL{}~7 questa traduzione
      dipendeva dalle impostazioni sulla lingua del sistema; adesso,
      queste impostazioni non influenzano il comportamento dei programmi
      \TeX. Se per qualche ragione hai bisogno della notazione \verb|^^|,
      rinomina il file \verb|texmf/web2c/cp8bit.tcx| (le edizioni future
      avranno un modo più pulito per controllare questa opzione).
%\item \TeX{}, \MF{}, and \MP{} now, by default, output most input
%      characters (32 and above) as themselves in output (e.g.,
%      \verb|\write|) files,
%      log files, and the terminal, i.e., \emph{not} translated using the
%      \verb|^^| notation.  In \TL{}~7, this translation was
%      dependent on the system locale settings; now, locale settings do
%      not influence the \TeX{} programs' behavior.  If for some reason
%      you need the \verb|^^| output, rename the file
%      \verb|texmf/web2c/cp8bit.tcx|.  (Future releases will have cleaner
%      ways to control this.)
\item Questa documentazione fu revisionata sostanzialmente.
%\item This documentation was substantially revised.
\item Infine, dato che i numeri delle edizioni erano cresciuti in modo
      poco agevole, da adesso la versione è identificata semplicemente
      dall'anno: \TL{} 2003.
%\item Finally, since the edition numbers had grown unwieldy,
%      the version is now simply identified by the year: \TL{} 2003.
\end{itemize*}


\subsubsection{2004}

Il 2004 vide molti cambiamenti:
%2004 saw many changes:

\begin{itemize}

\item Se hai font installati localmente che usano i propri file di
supporto \filename{.map} o (molto meno probabilmente) \filename{.enc},
potresti aver bisogno di muovere questi file.
%\item If you have locally-installed fonts which use their own
%\filename{.map} or (much less likely) \filename{.enc} support files, you
%may need to move those support files.

I file \filename{.map} adesso sono cercati soltanto in sotto directory di
\dirname{fonts/map} (per ciascuna gerarchia \filename{texmf}, lungo il
percorso \envname{TEXFONTMAPS}. In modo simile, i file \filename{.enc}
sono cercati soltanto nelle sottodirectory di \dirname{font/enc}, lungo il
percorso \envname{ENCFONTS}. \cmdname{updmap} tenterà di avvisarti su file
che possono provocare problemi.
%\filename{.map} files are now searched for in subdirectories of
%\dirname{fonts/map} only (in each \filename{texmf} tree), along the
%\envname{TEXFONTMAPS} path.  Similarly, \filename{.enc} files are now
%searched for in subdirectories of \dirname{fonts/enc} only, along the
%\envname{ENCFONTS} path.  \cmdname{updmap} will attempt to warn about
%problematic files.

Per i metodi per gestire questa ed altre informazioni, visita la pagina
\url{http://tug.org/texlive/mapenc.html}.
%For methods of handling this and other information, please see
%\url{http://tug.org/texlive/mapenc.html}.

\item \TK{} è stata espansa con l'aggiunta di un \CD{} installabile basato
su \MIKTEX, per coloro che preferiscono quell'implementazione a Web2C.
Consulta la sezione~\ref{sec:overview-tl} (\p.\pageref{sec:overview-tl}).
%\item The \TK\ has been expanded with the addition of a \MIKTEX-based
%installable \CD, for those who prefer that implementation to Web2C.
%See section~\ref{sec:overview-tl} (\p.\pageref{sec:overview-tl}).

\item All'interno di \TL, la singola grande directory \dirname{texmf}
delle edizioni precedenti è stata sostituita da: \dirname{texmf},
\dirname{texmf-dist} e \dirname{texmf-doc}. Consulta la
sezione~\ref{sec:tld} (\p.\pageref{sec:tld} e il file \filename{README}
contenuto in ciascuna di esse.
%\item Within \TL, the single large \dirname{texmf} tree of previous
%releases has been replaced by three: \dirname{texmf},
%\dirname{texmf-dist}, and \dirname{texmf-doc}.  See
%section~\ref{sec:tld} (\p.\pageref{sec:tld}), and the \filename{README}
%files for each.

\item Tutti file in ingresso relativi a \TeX{} sono adesso raccolti nella
sotto directory \dirname{tex} delle varie \dirname{texmf*}, piuttosto che
avere le diverse posizioni \dirname{tex}, \dirname{etex},
\dirname{pdftex}, \dirname{pdfetex}, ecc. Vedi
\CDref{texmf-dist/doc/generic/tds/tds.html\#Extensions}
{\texttt{texmf-dist/doc/generic/tds/tds.html\#Extensions}}.
%\item All \TeX-related input files are now collected in
%the \dirname{tex} subdirectory of \dirname{texmf*} trees, rather than
%having separate sibling directories \dirname{tex}, \dirname{etex},
%\dirname{pdftex}, \dirname{pdfetex}, etc.  See
%\CDref{texmf-dist/doc/generic/tds/tds.html\#Extensions}
%{\texttt{texmf-dist/doc/generic/tds/tds.html\#Extensions}}.

\item Gli script di supporto (pensati per non essere invocati dagli
utenti) sono adesso posizionati in una nuova sotto directory delle varie
\dirname{texmf*} chiamata \dirname{scripts} e possono essere cercati
usando \verb|kpsewhich -format=texmfscripts|. Se quindi hai dei programmi
che richiamano questi script, allora dovranno essere corretti. Vedi
\CDref{texmf-dist/doc/generic/tds/tds.html\#Scripts}
{\texttt{texmf-dist/doc/generic/tds/tds.html\#Scripts}}.
%\item Helper scripts (not meant to be invoked by users) are now located
%in a new \dirname{scripts} subdirectory of \dirname{texmf*} trees, and
%can be searched for via \verb|kpsewhich -format=texmfscripts|.  So if you have
%programs which call such scripts, they'll need to be adjusted.  See
%\CDref{texmf-dist/doc/generic/tds/tds.html\#Scripts}
%{\texttt{texmf-dist/doc/generic/tds/tds.html\#Scripts}}.

\item Quasi tutti i formati lasciano la maggior parte dei caratteri
stampabili uguali a se stessi tramite il ``file di traduzione''
\filename{cp227.tcx}, piuttosto che trasformarli nella notazione
\verb|^^|. Nello specifico, i caratteri alle posizioni \acro{ASCII}
32--256, la tabulazione orizzontale, quella verticale e il ``form feed''
sono considerati stampabili e non vengono trasformati. Le eccezioni sono
plain \TeX{} (solo i caratteri 32--126 sono stampabili), \ConTeXt{}
(caratteri 0--255) e i formati legati ad \OMEGA. Questo comportamento
predefinito è quasi lo stesso che in \TL\,2003, ma è implementato in
maniera più pulita, con maggiori possibilità di personalizzazione. Vedi
See \CDref{texmf/doc/web2c/web2c.html\#TCX-files}
{\texttt{texmf/doc/web2c/web2c.html\#TCX-files}} (ad ogni modo, con
l'input in formato Unicode, \TeX{} potrebbe stampare sequenze parziali di
caratteri quando viene mostrato il contesto degli errori dato che legge
l'input come una sequenza di byte, non di caratteri).
%\item Almost all formats leave most characters printable as
%themselves via the ``translation file'' \filename{cp227.tcx}, instead of
%translating them with the \verb|^^| notation.  Specifically, characters
%at positions 32--256, plus tab, vertical tab, and form feed are
%considered printable and not translated.  The exceptions are plain \TeX\
%(only 32--126 printable), \ConTeXt\ (0--255 printable), and the
%\OMEGA-related formats.  This default behavior is almost the same as in
%\TL\,2003, but it's implemented more cleanly, with more possibilities
%for customization.  See \CDref{texmf/doc/web2c/web2c.html\#TCX-files}
%{\texttt{texmf/doc/web2c/web2c.html\#TCX-files}}.
%(By the way, with Unicode input, \TeX\ may output partial character
%sequences when showing error contexts, since it is byte-oriented.)

\item \textsf{pdfetex} è adesso il motore predefinito per tutti i formati
tranne (plain) \textsf{tex} stesso (ovviamente genera file \acro{DVI}
quando è eseguito come \textsf{latex}, ecc.). Questo significa, tra le
altre cose, che le caratteristiche di microtipografia di \textsf{pdftex}
sono disponibili in \LaTeX, \ConTeXt, ecc., così come le funzionalità di
\eTeX{} (\OnCD{texmf-dist/doc/etex/base/}).
%\item \textsf{pdfetex} is now the default engine for all formats
%except (plain) \textsf{tex} itself.  (Of course it generates \acro{DVI}
%when run as \textsf{latex}, etc.)  This means, among other things, that
%the microtypographic features of \textsf{pdftex} are available in
%\LaTeX, \ConTeXt, etc., as well as the \eTeX\ features
%(\OnCD{texmf-dist/doc/etex/base/}).

Significa anche che è \emph{più importante che mai} usare il pacchetto
\pkgname{ifpdf} (funziona sia con plain \TeX, che con \LaTeX) o del codice
equivalente, perché verificare semplicemente se \cs{pdfoutput} o qualche
altra primitiva sono definiti non è un modo affidabile per determinare se
si sta generando un file \acro{PDF}. Per quest'anno abbiamo cercato di
rendere questo aspetto compatibile con le edizioni passate al meglio delle
nostra capacità, ma a partire dal prossimo anno \cs{pdfoutput} potrebbe
essere definito anche se il file generato è un \acro{DVI}.
%It also means it's \emph{more important than ever} to use the
%\pkgname{ifpdf} package (works with both plain and \LaTeX) or equivalent
%code, because simply testing whether \cs{pdfoutput} or some other
%primitive is defined is not a reliable way to determine if \acro{PDF}
%output is being generated.  We made this backward compatible as best we
%could this year, but next year, \cs{pdfoutput} may be defined even when
%\acro{DVI} is being written.

\item pdf\TeX\ (\url{http://pdftex.org}) ha molte nuove funzionalità:
%\item pdf\TeX\ (\url{http://pdftex.org}) has many new features:

  \begin{itemize*}

  \item \cs{pdfmapfile} e \cs{pdfmapline} forniscono il supporto alle
  mappature dei font direttamente all'interno di un documento.
%  \item \cs{pdfmapfile} and \cs{pdfmapline} provide font map support
%  from within a document.

  \item L'espansione microtipografica dei font può essere usata più
  facilmente.\\
  \url{http://www.ntg.nl/pipermail/ntg-pdftex/2004-May/000504.html}
%  \item Microtypographic font expansion can be used more easily.\\
%  \url{http://www.ntg.nl/pipermail/ntg-pdftex/2004-May/000504.html}

  \item Tutti i parametri che prima erano impostati tramite lo speciale
  file di configurazione \filename{pdftex.cfg} devono essere adesso
  impostati tramite primitive, tipicamente in \filename{pdftexconfig.tex};
  \filename{pdftex.cfg} non è più supportato. Ogni file \filename{.fmt}
  deve essere rigenerato quando \filename{pdftexconfig.tex} viene
  modificato.
%  \item All parameters previously set through the special configuration
%  file \filename{pdftex.cfg} must now be set through primitives,
%  typically in \filename{pdftexconfig.tex}; \filename{pdftex.cfg} is no
%  longer supported.  Any extant \filename{.fmt} files must be redumped
%  when \filename{pdftexconfig.tex} is changed.

  \item Per saperne di più, consulta il manuale di pdf\TeX{}:
  \OnCD{texmf-dist/doc/pdftex/manual/pdftex-a.pdf}.
%  \item See the pdf\TeX\ manual for more: \OnCD{texmf-dist/doc/pdftex/manual/pdftex-a.pdf}.

  \end{itemize*}

\item La primitiva \cs{input} in \cmdname{tex} (e in \cmdname{mf} e
\cmdname{mpost}) adesso accetta nomi con spazi ed altri caratteri speciali
racchiusi tra doppi apici. Esempi tipici:
\begin{verbatim}
\input "file con spazi"   % plain
\input{"file con spazi"}  % latex
\end{verbatim}
%\item The \cs{input} primitive in \cmdname{tex} (and \cmdname{mf} and
%\cmdname{mpost}) now accepts double quotes containing spaces and other
%special characters.  Typical examples:
%\begin{verbatim}
%\input "filename with spaces"   % plain
%\input{"filename with spaces"}  % latex
%\end{verbatim}
Consulta il manuale di Web2C per saperne di più: \OnCD{texmf/doc/web2c}.
%See the Web2C manual for more: \OnCD{texmf/doc/web2c}.

\item Il supporto per enc\TeX{} è ora incluso in Web2C e, di conseguenza,
in tutti i programmi \TeX, per mezzo dell'opzione \optname{-enc} \Dash\
\emph{solo quando i formati sono stati generati}. enc\TeX{} supporta la
ricodifica generale dell'input e dell'output, permettendo il supporto
completo per l'Unicode (in \acro{UTF}-8). Consulta
\OnCD{texmf-dist/doc/generic/enctex/} e
\url{http://www.olsak.net/enctex.html}.
%\item enc\TeX\ support is now included within Web2C and consequently all
%\TeX\ programs, via the \optname{-enc} option\Dash \emph{only when
%formats are built}.  enc\TeX\ supports general re-encoding of input and
%output, enabling full support of Unicode (in \acro{UTF}-8).  See
%\OnCD{texmf-dist/doc/generic/enctex/} and
%\url{http://www.olsak.net/enctex.html}.

\item Aleph, un nuovo motore che combina \eTeX\ ed \OMEGA, è disponibile.
Alcune informazioni sono disponibili in \OnCD{texmf-dist/doc/aleph/base} e
su \url{http://www.tex.ac.uk/cgi-bin/texfaq2html?label=aleph}. Il formato
basato su \LaTeX{} per Aleph è chiamato \textsf{lamed}.
%\item Aleph, a new engine combining \eTeX\ and \OMEGA, is available.
%A little information is available in \OnCD{texmf-dist/doc/aleph/base}
%and \url{http://www.tex.ac.uk/cgi-bin/texfaq2html?label=aleph}.  The
%\LaTeX-based format for Aleph is named \textsf{lamed}.

\item L'ultimo aggiornamento di \LaTeX\ ha una nuova versione della
\acro{LPPL} \Dash\ adesso una licenza ufficialmente approvata da Debian.
Per altri aggiornamenti assortiti, consulta i file \filename{ltnews} in
\OnCD{texmf-dist/doc/latex/base}.
%\item The latest \LaTeX\ release has a new version of the
%\acro{LPPL}\Dash now officially a Debian-approved license.  Assorted
%other updates, see the \filename{ltnews} files in
%\OnCD{texmf-dist/doc/latex/base}.

\item \cmdname{dvipng}, un nuovo programma per convertire i \acro{DVI} in
immagini \acro{PNG}, è incluso. Vedi \OnCD{texmf/doc/man/man1/dvipng.1}.
%\item \cmdname{dvipng}, a new program for converting \acro{DVI} to
%\acro{PNG} image files, is included.  See \OnCD{texmf/doc/man/man1/dvipng.1}.

\item Abbiamo ridotto il pacchetto \pkgname{cbgreek} ad un insieme di font
di ``medie'' dimensioni, con il consenso e i suggerimenti dell'autore
(Claudio Beccari). I font omessi sono quelli invisibili, quelli profilati
e i trasparenti, che sono usati abbastanza raramente, mentre noi avevamo
bisogno di spazio. L'insieme completo è ovviamente disponibile su
\acro{CTAN} (\url{http://www.ctan.org/tex-archive/fonts/greek/cb}).
%\item We reduced the \pkgname{cbgreek} package to a ``medium'' sized set
%of fonts, with the assent and advice of the author (Claudio Beccari).
%The excised fonts are the invisible, outline, and transparency ones,
%which are relatively rarely used, and we needed the space.  The full set
%is of course available from \acro{CTAN}
%(\url{http://www.ctan.org/tex-archive/fonts/greek/cb}).

\item \cmdname{oxdvi} è stato rimosso; usa semplicemente \cmdname{xdvi}.
%\item \cmdname{oxdvi} has been removed; just use \cmdname{xdvi}.

\item I comandi (collegamenti) \cmdname{ini} e \cmdname{vir} per
\cmdname{tex}, \cmdname{mf} e \cmdname{mpost} non sono più creati, così
come \cmdname{initex}. Le funzionalità di \cmdname{ini} sono disponibili
oramai da anni tramite l'opzione su riga di comando \optname{-ini}.
%\item The \cmdname{ini} and \cmdname{vir} commands (links) for
%\cmdname{tex}, \cmdname{mf}, and \cmdname{mpost} are no longer created,
%such as \cmdname{initex}.  The \cmdname{ini} functionality has been
%available through the command-line option \optname{-ini} for years now.

\item Il supporto per la piattaforma \textsf{i386-openbsd} è stato
rimosso. Dato che è disponibile il pacchetto \pkgname{tetex} nel sistema
\acro{BSD} Port e gli eseguibili per \acro{GNU/}Linux e Free\acro{BSD}
erano disponibili, ci è sembrato che il tempo dedicato dai volontari
potesse essere meglio speso da altre parti.
%\item \textsf{i386-openbsd} platform support was removed.  Since the
%\pkgname{tetex} package in the \acro{BSD} Ports system is available, and
%\acro{GNU/}Linux and Free\acro{BSD} binaries were available, it seemed
%volunteer time could be better spent elsewhere.

\item Su \textsf{sparc-solaris} (almeno), potresti dover impostare la
variabile d'ambiente \envname{LD\_LIBRARY\_PATH} per eseguire i programmi
delle \pkgname{t1utils}. La ragione è che questi sono compilati con il C++
e non esiste una posizione comune per le librerie (questo problema non è
nuovo dell'edizione 2004, ma non era stato documentato in precedenza). In
modo simile, su \textsf{mips-irix}, sono richieste le librerie di runtime
del \acro{MIPS}pro 7.4.
%\item On \textsf{sparc-solaris} (at least), you may have to set the
%\envname{LD\_LIBRARY\_PATH} environment variable to run the
%\pkgname{t1utils} programs.  This is because they are compiled with C++,
%and there is no standard location for the runtime libraries.  (This is
%not new in 2004, but wasn't previously documented.)  Similarly, on
%\textsf{mips-irix}, the \acro{MIPS}pro 7.4 runtimes are required.

\end{itemize}

\subsubsection{2005}

L'edizione del 2005 ha visto il solito enorme numero di aggiornamenti ai
pacchetti ed ai programmi. L'infrastruttura è rimasta sostanzialmente
invariata dal 2004, ma inevitabilmente ci sono comunque stati dei
cambiamenti:
%2005 saw the usual huge number of updates to packages and programs.
%The infrastructure stayed relatively stable from 2004, but inevitably
%there were some changes there as well:

\begin{itemize}

\item Furono introdotti i nuovi script \cmdname{texconfig-sys},
      \cmdname{updmap-sys} e \cmdname{fmtutil-sys} che modificano la
      configurazione nei percorsi di sistema. Gli script
      \cmdname{texconfig}, \cmdname{updmap} e \cmdname{fmtutil} ora
      modificano i file specifici per i singoli utenti, sotto
      \dirname{$HOME/.texlive2005}.
%\item New scripts \cmdname{texconfig-sys}, \cmdname{updmap-sys}, and
%      \cmdname{fmtutil-sys} were introduced, which modify the
%      configuration in the system trees.  The \cmdname{texconfig},
%      \cmdname{updmap}, and \cmdname{fmtutil} scripts now modify
%      user-specific files, under \dirname{$HOME/.texlive2005}.

\item Le corrispondenti nuove variabili \envname{TEXMFCONFIG} e
      \envname{TEXMFSYSCONFIG} per specificare i percorsi dove trovare i
      file di configurazione (per il singolo utene e per l'intero sistema,
      rispettivamente). Quindi, puoi aver bisogno di spostare le versioni
      personali di \filename{fmtutil.cnf} e \filename{updmap.cfg} in
      questi posti; un'altra opzione è quella di ridefinire
      \envname{TEXMFCONFIG} o \envname{TEXMFSYSCONFIG} in
      \filename{texmf.cfg}. In ogni caso la posizione reale di questi file
      e il valore di \envname{TEXMFCONFIG} e \envname{TEXMFSYSCONFIG}
      devono concordare. Consulta la sezione~\ref{sec:texmftrees},
      \p.\pageref{sec:texmftrees}.
%\item Corresponding new variables \envname{TEXMFCONFIG} and
%      \envname{TEXMFSYSCONFIG} to specify the trees where configuration
%      files (user or system, respectively) are found.  Thus, you may
%      need to move personal versions of \filename{fmtutil.cnf} and
%      \filename{updmap.cfg} to these places; another option is to
%      redefine \envname{TEXMFCONFIG} or \envname{TEXMFSYSCONFIG} in
%      \filename{texmf.cnf}. In any case the real location of these files
%      and the values of \envname{TEXMFCONFIG} and \envname{TEXMFSYSCONFIG}
%      must agree.
%      See section~\ref{sec:texmftrees}, \p.\pageref{sec:texmftrees}.

\item Per l'ultimo anno, \verb|\pdfoutput| ed altre primitive non erano
      definite quando viene generato un \dvi, anche se veniva usato il
      programma \cmdname{pdfetex}. Quest'anno, come promesso, abbiamo
      annullato quella misura di compatibilità. Per cui se il tuo
      documento usa \verb|\ifx\pdfoutput\undefined| per verificare se
      l'output richiesto è in PDF, dovrà essere cambiato. Puoi usare il
      pacchetto \pkgname{ifpdf.sty} (che funziona sia sotto plain \TeX{}
      che sotto \LaTeX) per fare ciò, oppure copiarne la logica.
%\item Last year, we kept \verb|\pdfoutput| and other primitives undefined
%      for \dvi\ output, even though the \cmdname{pdfetex} program was
%      being used.  This year, as promised, we undid that compatibility
%      measure.  So if your document uses \verb|\ifx\pdfoutput\undefined|
%      to test if PDF is being output, it will need to be changed.  You
%      can use the package \pkgname{ifpdf.sty} (which works under both
%      plain \TeX\ and \LaTeX) to do this, or steal its logic.

\item Per l'ultimo anno, abbiamo cambiato la maggior parte dei formati per
      stampare i caratteri (a 8 bit) come se stessi (consulta la sezione
      precedente). Il nuovo file TCX \filename{empty.tcx} adesso offre un
      modo più semplice per ottenere l'originaria notazione \verb|^^| se
      la desideri, come in:
\begin{verbatim}
latex --translate-file=empty.tcx tuofile.tex
\end{verbatim}
%\item Last year, we changed most formats to output (8-bit) characters as
%      themselves (see previous section).  The new TCX file
%      \filename{empty.tcx} now provides an easier way to get the
%      original \verb|^^| notation if you so desire, as in:
%\begin{verbatim}
%latex --translate-file=empty.tcx yourfile.tex
%\end{verbatim}

\item È incluso il nuovo programma \cmdname{dvipdfmx} per la
      trasformazione dei DVI in PDF; si tratta di un aggiornamento
      attivamente mantenuto di \cmdname{dvipdfm} (che per ora è ancora
      disponibile, anche se non più raccomandato).
%\item The new program \cmdname{dvipdfmx} is included for translation of
%      DVI to PDF; this is an actively maintained update of
%      \cmdname{dvipdfm} (which is also still available for now, though
%      no longer recommended).

\item Sono inclusi i nuovi programmi \cmdname{pdfopen} e
      \cmdname{pdfclose} per consentire di ricaricare i file pdf nel
      lettore Adobe Acrobat Reader senza dover riavviare il programma
      (altri lettori pdf, in particolare \cmdname{xpdf}, \cmdname{gv} e
      \cmdname{gsview}, non hanno mai sofferto di questo problema).
%\item The new programs \cmdname{pdfopen} and \cmdname{pdfclose} are included
%      to allow reloading of pdf files in the Adobe Acrobat Reader without
%      restarting the program.  (Other pdf readers, notably \cmdname{xpdf},
%      \cmdname{gv}, and \cmdname{gsview}, have never suffered from this
%      problem.)

\item Per consistenza, le variabili \envname{HOMETEXMF} e
      \envname{VARTEXMF} sono state rinominate \envname{TEXMFHOME} e
      \envname{TEXMFVAR}, rispettivamente. C'è anche \envname{TEXMFVAR},
      che è specifica per ogni utente. Consulta il primo punto
      dell'elenco.
%\item For consistency, the variables \envname{HOMETEXMF} and
%      \envname{VARTEXMF} have been renamed to \envname{TEXMFHOME} and
%      \envname{TEXMFSYSVAR}, respectively.  There is also
%      \envname{TEXMFVAR}, which is by default user-specific.  See the
%      first point above.

\end{itemize}


\subsubsection{2006--2007}

Nell'edizione 2006--2007, la nuova più grande aggiunta a \TL{} è stato il
programma \XeTeX, disponibile tramite i comandi \texttt{xetex} e
\texttt{xelatex}; visita il sito \url{http://scripts.sil.org/xetex}.
%In 2006--2007, the major new addition to \TL{} was the \XeTeX{} program,
%available as the \texttt{xetex} and \texttt{xelatex} programs; see
%\url{http://scripts.sil.org/xetex}.

Anche MetaPost ha ricevuto un notevole aggiornamento, mentre altri ne sono
stati pianificati per il futuro (\url{http://tug.org/metapost/articles}),
così come per pdf\TeX{} (\url{http://tug.org/applications/pdftex}).
%MetaPost also received a notable update, with more planned for the
%future (\url{http://tug.org/metapost/articles}), likewise pdf\TeX{}
%(\url{http://tug.org/applications/pdftex}).

I \filename{.fmt} di \TeX{} (formati ad alta velocità) e i file simili per
MetaPost e \MF{} adesso sono posizionati in sotto directory di
\dirname{texmf/web2c}, invece che nella directory stessa (sebbene la
directory è ancora visitata durante la ricerca, per motivi relativi ai
\filename{.fmt} esisitenti). Le sotto directory sono chiamate in base al
``motore'' usato, come \filename{tex} o \filename{pdftex} o
\filename{xetex}. Questo cambiamento dovrebbe essere invisibile nell'uso
normale.
%The \TeX\ \filename{.fmt} (high-speed format) and the similar files for
%MetaPost and \MF\ are now stored in subdirectories of \dirname{texmf/web2c},
%instead of in the directory itself (although the directory is still
%searched, for the sake of existing \filename{.fmt}'s).  The
%subdirectories are named for the `engine' in use, such as \filename{tex}
%or \filename{pdftex} or \filename{xetex}.  This change should be
%invisible in normal use.

Il programma (plain) \texttt{tex} non legge più le prime linee
identificate da \texttt{\%\&} per determinare quale formato adoperare; è
il puro \TeX{} Knuthiano (\LaTeX{} e tutto il resto leggono ancora le
linee \texttt{\%\&}).
%The (plain) \texttt{tex} program no longer reads \texttt{\%\&} first
%lines to determine what format to run; it is the pure Knuthian \TeX.
%(\LaTeX\ and everything else do still read \texttt{\%\&} lines).

Ovviamente anche quest'anno ha visto (i soliti) centinaia di altri
aggiornamenti ai pacchetti ed ai programmi. Come sempre, visita
\acro{CTAN} (\url{http://www.ctan.org}) per tutti gli aggiornamenti.
%Of course the year also saw (the usual) hundreds of other updates to
%packages and programs.  As usual, please check \acro{CTAN}
%(\url{http://www.ctan.org}) for updates.

Internamente, i sorgenti sono memorizzato tramite Subversion, con
un'interfaccia web standard per visitarli; il collegamento è sulla nostra
pagina home. Sebbene invisibile nella distribuzione finale, ci aspettiamo
che questo fornisca un fondamento stabile per lo sviluppo negli anni a
venire.
%Internally, the source tree is now stored in Subversion, with a standard
%web interface for viewing the tree, as linked from our home page.
%Although not visible in the final distribution, we expect this will
%provide a stable development foundation for future years.

Infine, nel maggio 2006 Thomas Esser ha annunciato che non avrebbe più
aggiornato te\TeX{} (\url{http://tug.org/tetex}). Come risultato, c'è
stata una nascita di interesse verso \TL, specialmente tra i distributori
di \GNU/Linux (c'è un nuovo schema di installazione in \TL{} chiamato
\texttt{tetex}, che fornisce approssimativamente qualcosa di equivalente).
Speriamo che questo evento si trasformi alla fine in miglioramenti per
tutti all'ambiente \TeX.
%Finally, in May 2006 Thomas Esser announced that he would no longer be
%updating te\TeX{} (\url{http://tug.org/tetex}).  As a result, there was
%been a surge of interest in \TL{}, especially among \GNU/Linux
%distributors.  (There is a new \texttt{tetex} installation scheme in
%\TL{}, which provides an approximate equivalent.)  We hope this will
%eventually translate to improvements in the \TeX\ environment for
%everyone.

\subsubsection{2008}

Nell'edizione del 2008, l'intera infrastruttura di \TL{} è stata
riprogettata e reimplementata. Le informazioni complete su
un'installazione sono adesso memorizzate in un file di testo puro
\filename{tlpkg/texlive.tlpdb}.
%In 2008, the entire \TL{} infrastructure was redesigned and
%reimplemented.  Complete information about an installation is now stored
%in a plain text file \filename{tlpkg/texlive.tlpdb}.

Tra le altre cose, ciò rende finalmente possibile aggiornare
un'installazione \TL{} tramite Internet dopo l'installazione iniziale, una
caratteristica che Mik\TeX{} ha offerto per anni. Ci attendiamo di
aggiornare regolarmente i nuovi pacchetti così come diventano disponibili
su \CTAN.
%Among other things, this finally makes possible upgrading a \TL{}
%installation over the Internet after the initial installation, a feature
%MiK\TeX\ has provided for many years.  We expect to regularly update new
%packages as they are released to \CTAN.

È stato incluso Lua\TeX{} (\url{http://luatex.org}) il nuovo maggior
motore; affianco ad un nuovo livello di flessibilità nella composizione
tipografica, questo fornisce un eccellente linguaggio di scripting da
usare sia dentro che fuori i documenti \TeX.
%The major new engine Lua\TeX\ (\url{http://luatex.org}) is included;
%besides a new level of flexibility in typesetting, this provides an
%excellent scripting language for use both inside and outside of \TeX\
%documents.

Il supporto tra le piattaforme Windows e basate su Unix è ora molto più
uniforme. In particolare, la maggior parte degli script Perl e Lua sono
ora disponibili sotto Windows tramite l'interprete Perl distribuito
all'interno di \TL.
%Support among Windows and the Unix-based platforms is now much more
%uniform.  In particular, most Perl and Lua scripts are now available on
%Windows, using the Perl internally distributed with \TL.

Il nuovo script \cmdname{tlmgr} (sezione~\ref{sec:tlmgr}) è l'interfaccia
generale per amministrare \TL{} dopo l'installazione iniziale. Esso
gestice l'aggiornamento dei pacchetti e la conseguente rigenerazione dei
formati, dei file di mappatura e dei file delle lingue, incluse
opzionalmente le aggiunte locali.
%The new \cmdname{tlmgr} script (section~\ref{sec:tlmgr}) is the
%general interface for managing \TL{} after the initial installation.
%It handles package updates and consequent regeneration of formats, map
%files, and language files, optionally including local additions.

Con l'avvento di \cmdname{tlmgr}, è ora disabilitata l'azione di modifica
dei file di configurazione dei formati e delle sillabazioni operata da
\cmdname{texconfig}.
%With the advent of \cmdname{tlmgr}, the \cmdname{texconfig} actions to
%edit the format and hyphenation configuration files are now disabled.

Il programma per la creazione degli indici \cmdname{xindy}
(\url{http://xindy.sourceforge.net/}) è ora incluso nella maggior parte
delle piattaforme.
%The \cmdname{xindy} indexing program
%(\url{http://xindy.sourceforge.net/}) is now included on most platforms.

Lo strumento \cmdname{kpsewhich} può ora riportare tutte le corrispondenze
per un dato file (opzione \optname{--all}) oppure limitarle ad una data
sotto directory (opzione \optname{--subdir}).
%The \cmdname{kpsewhich} tool can now report all matches for a given file
%(option \optname{--all}) and limit matches to a given subdirectory
%(option \optname{--subdir}).

Il programma \cmdname{dvipdfmx} adesso include la funzionalità di estrarre
le informazioni sulla bounding box attraverso il comando
\cmdname{extractbb}; si tratta di una delle ultime caratteristiche che
erano fornite da \cmdname{dvipdfm} ma non ancora incluse in
\cmdname{dvipdfmx}.
%The \cmdname{dvipdfmx} program now includes functionality to extract
%bounding box information, via the command name \cmdname{extractbb}; this
%was one of the last features provided by \cmdname{dvipdfm} not in
%\cmdname{dvipdfmx}.

Gli alias dei font \filename{Times-Roman}, \filename{Helvetica} e così via
sono stati rimossi. Diversi pacchetti si aspettavano che funzionassero in
modo diverso (in particolare, che avessero codifiche differenti) e non
c'era un buon modo per risolvere il problema.
%The font aliases \filename{Times-Roman}, \filename{Helvetica}, and so on
%have been removed.  Different packages expected them to behave
%differently (in particular, to have different encodings), and there was
%no good way to resolve this.

Il formato \pkgname{platex} è stato rimosso per risolvere un conflitto di
nome con un completamente diverso \pkgname{platex} giapponese; il
pacchetto \pkgname{polski} contiene ora il principale supporto per la
lingua polacca.
%The \pkgname{platex} format has been removed, to resolve a name conflict
%with a completely different Japanese \pkgname{platex}; the
%\pkgname{polski} package is now the main Polish support.

Internamente, i file con le riserve di stringhe di \web{} sono compilati
dentro gli eseguibili, per semplificare gli aggiornamenti.
%Internally, the \web\ string pool files are now compiled into the
%binaries, to ease upgrades.

Infine, in questa edizione sono stati inclusi i cambiamenti fatti da
Donald Knuth nella sua ``messa a punto di \TeX\ del 2008''. Consulta
\url{http://tug.org/TUGboat/Articles/tb29-2/tb92knut.pdf}.
%Finally, the changes made by Donald Knuth in his `\TeX\ tuneup of 2008'
%are included in this release.  See
%\url{http://tug.org/TUGboat/Articles/tb29-2/tb92knut.pdf}.


% 
\htmlanchor{news}
\subsection{Presente}
%\subsection{Present}
\label{sec:tlcurrent}
\label{sec:2009news} % keep with 2009

\begin{comment}
Nel 2009, il cambiamento più visibile è pdf\AllTeX\ adesso converte
\emph{automaticamente} un file Encapsulated PostScript (EPS) richiesto in
PDF tramite il pacchetto \pkgname{epstopdf}, quando e se il file di
configurazione di \LaTeX{} \code{graphics.cfg} è caricato e il PDF è il
formato di uscita. Le opzioni predefinite sono pensate per eliminare ogni
possibilità che i file PDF creati manualmente siano sovrascritti, ma puoi
anche impedire che \code{epstopdf} sia caricato del tutto inserendo
|\newcommand{\DoNotLoadEpstopdf}{}| (o |\def...|) prima della
dichiarazione \cs{documentclass}. Per maggiori dettagli, consulta la
documentazione del pacchetto epstopdf
(\url{http://ctan.org/pkg/epstopdf-pkg}).
%In 2009, the most visible change is that pdf\AllTeX\ now
%\emph{automatically} converts a requested Encapsulated PostScript (EPS)
%file to PDF, via the \pkgname{epstopdf} package, when and if the \LaTeX\
%\code{graphics.cfg} configuration file is loaded, and PDF is being
%output.  The default options are intended to eliminate any chance of
%hand-created PDF files being overwritten, but you can also prevent
%\code{epstopdf} from being loaded at all by putting
%|\newcommand{\DoNotLoadEpstopdf}{}| (or |\def...|) before the
%\cs{documentclass} declaration.  For details, see the epstopdf package
%documentation (\url{http://ctan.org/pkg/epstopdf-pkg}).

Un importante cambiamento correlato è che l'esecuzione di alcuni comandi
esterni, attraverso la funzionalità \cs{write18}, è ora abilitata di base
\Dash\ per esempio, \code{epstopdf}, \code{makeindex} e \code{bibtex}. La
lista esatta dei comandi è definita nel file \code{texmf.cnf}. Per gli
ambienti che devono disabilitare tutti questi comandi esterni, è possibile
deselezionare questa opzione nell'installatore (consulta la
sezione~\ref{sec:options}) oppure sostituire il valore in \code{texmf.cnf}
dopo l'installazione.
%A related important change is that execution of a few external commands,
%via the \cs{write18} feature, is now enabled by default\Dash for
%example, \code{epstopdf}, \code{makeindex}, and \code{bibtex}.  The
%exact list of commands is defined in the \code{texmf.cnf} file.
%Environments which must disallow all such external commands can deselect
%this option in the installer (see section~\ref{sec:options}), or
%override the value in \code{texmf.cnf} after installation.
\end{comment}

Il formato di output predefinito per Lua\AllTeX{} è ora il PDF, in modo
tale da avvalersi del supporto OpenType di Lua\TeX, ecc. I nuovi
eseguibili chiamati \code{dviluatex} e \code{dvilualatex} eseguono
Lua\TeX{} abilitando la generazione del DVI. La pagina web di Lua\TeX{} è
\url{http://luatex.org}.
%The default output format for Lua\AllTeX\ is now PDF, to take advantage
%of Lua\TeX's OpenType support, et al.  New executables named
%\code{dviluatex} and \code{dvilualatex} run Lua\TeX\ with DVI output.
%The Lua\TeX\ home page is \url{http://luatex.org}.

L'originale motore Omega e il formato Lambda sono stati rimossi, dopo
averne discusso con gli autori di Omega. Restano gli aggiornati Aleph e
Lamed, così come le utilità di Omega.
%The original Omega engine and Lambda format have been excised, after
%discussions with the Omega authors.  The updated Aleph and Lamed remain,
%as do the Omega utilities.

È stata inclusa una nuova versione dei font AMS \TypeI, incluso il
Computer Modern: sono stati integrati alcuni cambiamenti nelle forme fatti
da Knuth nei sorgenti Metafont nel corso degli anni e l'\emph{hinting} è
stato aggiornato. I font Euler sono stati ridisegnati a fondo da Hermann
Zapf (visita
\url{http://tug.org/TUGboat/Articles/tb29-2/tb92hagen-euler.pdf}). In ogni
caso, le metriche non hanno subito modifiche. La pagina dei font AMS è
\url{http://www.ams.org/tex/amsfonts.html}.
%A new release of the AMS \TypeI\ fonts is included, including Computer
%Modern: a few shape changes made over the years by Knuth in the Metafont
%sources have been integrated, and the hinting has been updated.  The
%Euler fonts have been thoroughly reshaped by Hermann Zapf (see
%\url{http://tug.org/TUGboat/Articles/tb29-2/tb92hagen-euler.pdf}).  In
%all cases, the metrics remain unchanged.  The AMS fonts home page is
%\url{http://www.ams.org/tex/amsfonts.html}.

È stato incluso, anche in Mac\TeX, il nuovo editor per Windows
\TeX{}works. Per le altre piattaforme, e per ulteriori informazioni,
visita il sito di \TeX{}works, \url{http://tug.org/texworks}. Si tratta di
un programma multi piattaforma ispirato al TeXShop per \MacOSX, che ha
come obiettivo la facilità d'uso.
%The new GUI front end \TeX{}works is included for Windows, and also in
%Mac\TeX.  For other platforms, and more information, see the \TeX{}works
%home page, \url{http://tug.org/texworks}.  It is a cross-platform front
%end inspired by the \MacOSX\ TeXShop editor, aiming at ease-of-use.

È stato incluso il programma per la grafica Asymptote per diverse
piattaforme. Questo programma implementa un linguaggio testuale
descrittivo per la grafica vagamente simile a MetaPost, ma con il supporto
avanzato per il 3D ed altre caratteristiche. Il suo sito web è
\url{http://asymptote.sourceforge.net}.
%The graphics program Asymptote is included for several platforms.  This
%implements a text-based graphics description language vaguely akin to
%MetaPost, but with advanced 3D support and other features.  Its home
%page is \url{http://asymptote.sourceforge.net}.

Il programma \code{dvipdfm} è stato sostituito da \code{dvipdfmx}, che
opera in una speciale modalità di compatibilità quando viene invocato con
il primo dei due nomi. \code{dvipdfmx} include il supporto per \acro{CJK}
(Chinese-Japanese-Korean, Cinese-Giapponese-Coreano) ed ha accumulato
molte correzione negli anni passati dall'ultima versione di
\code{dvipdfm}. Il sito di DVIPDFMx è
\url{http://project.ktug.or.kr/dvipdfmx}.
%The separate \code{dvipdfm} program has been replaced by
%\code{dvipdfmx}, which operates in a special compatibility mode under
%that name.  \code{dvipdfmx} includes \acro{CJK} support and has
%accumulated many other fixes over the years since the last
%\code{dvipdfm} release.  The DVIPDFMx home page is
%\url{http://project.ktug.or.kr/dvipdfmx}.

Sono stati inclusi gli eseguibili per le piattaforme \pkgname{cygwin} e
\pkgname{i386-netbsd}, mentre sono state abbandonate le altre
distribuzioni BSD. Siamo stati avvisati che gli utenti di OpenBSD e
FreeBSD ottengono \TeX{} tramite i propri sistemi di pacchetti, e in più
c'erano difficoltà nel compilare degli eseguibili che avevano la
possibilità di lavorare su più di una versione.
%Executables for the \pkgname{cygwin} and \pkgname{i386-netbsd} platforms
%are now included, while the other BSD distributions have been dropped;
%we were advised that OpenBSD and FreeBSD users get \TeX\ through their
%package systems, plus there were difficulties in making binaries that
%have a chance of working on more than one version.

Una varietà di cambiamenti più piccoli: adesso usiamo la compressione
\pkgname{xz}, un rimpiazzo stabile per \pkgname{lzma}
(\url{http://tukaani.org/xz/}); un carattere |$| è ammesso nei nomi dei
file quando non introduce un nome di variabile noto; la libreria Kpathsea
è ora multi-thread (lo sfrutta in MetaPost); l'intera compilazione di
\TL{} è ora basata su Automake.
%A miscellany of smaller changes: we now use \pkgname{xz} compression,
%the stable replacement for \pkgname{lzma}
%(\url{http://tukaani.org/xz/}); a literal |$| is allowed in filenames
%when it does not introduce a known variable name; the Kpathsea library
%is now multi-threaded (made use of in MetaPost); the entire \TL{} build
%is now based on Automake.

Nota finale sul passato: tutte le edizioni di \TL{}, assiame al materiale
ausiliario come le etichette dei \CD, sono disponibili all'indirizzo
\url{ftp://tug.org/historic/systems/texlive}.
%Final note on the past: all releases of \TL{}, along with ancillary
%material such as \CD\ labels, are available at
%\url{ftp://tug.org/historic/systems/texlive}.


\subsection{Futuro}
%\subsection{Future}

\emph{\TL{} non è perfetto!} (e mai lo sarà). Intendiamo di continuare a
fornire nuove versione e vorremmo fornire ulteriore materiale d'aiuto, più
programmi di utilità, più programmi di installazione e (ovviamente) un
ancor più migliorato e meglio controllato insieme di macro e font. Questo
lavoro è fatto completamente da volontari sovraccaricati nel loro
limitato tempo libero e resta ancora tanto da fare. Visita il sito
\url{http://tug.org/texlive/contribute.html}.
%\emph{\TL{} is not perfect!}  (And never will be.)  We intend to
%continue to release new versions, and would like to provide more help
%material, more utilities, more installation programs, and (of course) an
%ever-improved and better-checked tree of macros and fonts. This work is
%all done by hard-pressed volunteers in their limited spare time, and a
%great deal remains to be done.  Please see
%\url{http://tug.org/texlive/contribute.html}.

Puoi inviare correzioni, suggerimenti e offerte d'aiuto a:
%Please send corrections, suggestions, and offers of help to:
\begin{quote}
\email{tex-live@tug.org} \\
\url{http://tug.org/texlive}
\end{quote}

\medskip
\noindent \textsl{Buon lavoro con \TeX!}
%\noindent \textsl{Happy \TeX ing!}

\end{document}

% vim:tw=74:ts=2:sw=2:autoindent:expandtab
